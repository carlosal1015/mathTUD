\chapter{Numerik von Hamilton-Systemen}

\section{Hamilton-Systeme}

Extrem wichtige Klasse von Differentialgleichungen entstammen der klassischen Mechanik, Quantenmechanik und relativistische Mechanik. Dazu gehören auch spezielle numerische Verfahren --- eine \enquote{schöne Mathematik}.

\enquote{Vereinigendes Prinzip}: Bringt ganz unterschiedliche Gleichungen auf eine gemeinsame Form.


\begin{bsp}
	Mathematisches Pendel -- Fadenpendel
	\begin{itemize}
		\item Koordinate: Winkel $\alpha$
		\item Masse $m$, Fadenlänge $l$, Erdbeschleunigung $g$
	\end{itemize}
	Bewegungsgleichungen: $\displaystyle \ddot \alpha + \frac{g}{l} \sin \alpha = 0$
\end{bsp}

\begin{bsp}
	Teilchen in einem Kraftfeld $F(x)$
	\begin{equation}
	m \ddot x = F(x) \tag{Newtons Gesetz}
	\end{equation}
\end{bsp}

\begin{bsp}
	1d-Wellengleichung --- Longitudinale Auslenkung einer elastischen Schnur
	\begin{alignat*}{2}
	\frac{\partial^2 u}{\partial x^2} & = \frac{\partial^2 u}{\partial t^2} & \qquad & x \in [a, b], t \ge 0 \\
	u(a, t) = u(b, t) & = 0 && \forall t \ge 0
	\end{alignat*}
\end{bsp}


\subsection{Die Lagrange-Gleichungen}
\label{sec:lagrange_gleichung}

Wir betrachten ein mechanisches System mit $d$ Freiheitsgraden $q = (q_1, \dots, q_d)$.
\begin{itemize}
	\item Kinetische Energie: $T = T(q, \dot q)$
	(häufig: $T(q, \dot{q}) = \frac12 \dot{q}^T M(q) \dot{q}$ mit $M(q)$ s.p.d.)
	\item Potentielle Energie: $U = U(q)$
\end{itemize}

\begin{definition}
	Die Lagrange-Funktion eines mechanischen Systems ist $L = T - U$.
\end{definition}

Das mechanische System löst die Lagrange-Gleichungen
\begin{equation*}
\frac{d}{dt} \left( \frac{\partial L}{\partial \dot q} \right) = \frac{\partial L}{\partial q}
\end{equation*}

\emph{Warum?} Es gilt das Prinzip der stationären Wirkung.

\begin{definition}[Prinzip der stationären Wirkung / Hamilton'sches Prinzip]
	Sei $q: [t_0, t_1] \to \R^d$ eine Trajektorie eines mechanischen Systems.
	Für die in der Natur vorkommenden Trajektorien ist die \textit{Wirkung}
	\begin{equation*}
	W \colonequals \int_{t_0}^{t_1} L(q(t) \ \dot q(t)) \dt
	\end{equation*}
	stationär.
\end{definition}

Sei $q$ eine Trajektorie, und $\delta q$ eine Variation davon, die die Endpunkte fest lässt, also $\delta q(t_0)=\delta q(t_1) = 0$.
Stationarität von $q$ heißt dann, dass für alle solche $\delta q$
\begin{equation*}
\frac{d}{d\epsilon} S(q+\epsilon\delta q)\vert_{\epsilon=0} = 0.
\end{equation*}
gilt. Ausrechnen:
\begin{align*}
\frac{d}{d\epsilon} \int_{t_0}^{t_1} L(q+ \epsilon\delta q,\dot q + \epsilon\delta \dot q)\dt\vert_{\epsilon=0} &= \int_{t_0}^{t_1} \frac{\partial L}{\partial q}\delta q + \frac{\partial L}{\partial\dot q} \delta\dot q\, dt \\
&= \int_{t_0}^{t_1} \Bigg(\frac{\partial L}{\partial q} \delta q - \frac{d}{dt} \frac{\partial L}{\partial \dot q} \delta q\Bigg)\dt \tag{partielle Integration} \\
&= \int_{t_0}^{t_1} \Bigg(\frac{\partial L}{\partial q} - \frac{d}{dt} \frac{\partial L}{\partial \dot q}  \Bigg)\delta q\dt
\end{align*}
Da dieser Ausdruck für alle hinreichend glatten Funktionen $\delta q$ gleich Null sein muss, erhält man die Lagrange-Gleichung
\begin{equation*}
\delta W = 0 = \frac{d}{dt} \left( \frac{\partial L}{\partial \dot q} \right) - \frac{\partial L}{\partial q}
\end{equation*}

\begin{bsp}[Pendel]
	\begin{itemize}
		\item Kinetische Energie
		\begin{equation*}
		T = \frac12 m (\dot x^2 + \dot y^2) = \frac12 m l^2 \dot\alpha^2
		\end{equation*}
		\item Potentielle Energie
		\begin{equation*}
		U = mgy = -mgl \cos \alpha
		\end{equation*}
		\item Lagrange-Funktion
		\begin{equation*}
		L(\alpha, \dot\alpha) = \frac12 m l^2 \dot\alpha^2 + mgl \cos \alpha
		\end{equation*}
		\item Lagrange-Gleichung
		\begin{equation*}
		0
		= \frac{d}{dt} \left( \frac{\partial L}{\partial \dot q} \right) - \frac{\partial L}{\partial q}
		= \frac{d}{dt}( ml^2\dot\alpha ) + mgl \sin \alpha
		= ml^2\ddot\alpha + mgl \sin \alpha.
		\end{equation*}
	\end{itemize}
\end{bsp}

\begin{bsp}[Teilchen in einem Kraftfeld]
	Angenommen das Kraftfeld ist \emph{konservativ}, d.h. es gibt ein $U: \R^3 \to \R$, so dass $F(x) = -\nabla U(x)$.
	\begin{itemize}
		\item Kinetische Energie
		\begin{equation*}
		T(x, \dot x) = \frac12 m \langle \dot x, \dot x \rangle
		\end{equation*}
		\item Potentielle Energie
		\begin{equation*}
		U
		\end{equation*}
		\item Lagrange-Gleichung
		\begin{equation*}
		0
		= \frac{d}{dt} \left( \frac{\partial L}{\partial \dot q} \right) - \frac{\partial L}{\partial q}
		= \frac{d}{dt}(m\dot x) + \nabla U(x) = m\ddot x - F(x)
		\end{equation*}
	\end{itemize}
\end{bsp}

\begin{bsp}[Eindimensionale Wellengleichung]
	Ein unendlich-dimensionales System wird nicht beschrieben durch $d$
	Freiheitsgrade $(q_1, \dots, q_d)$, sondern durch die Funktion $u: [a, b] \to \R$. Diese beschreibt die transversale Auslenkung einer Saite.
	\begin{itemize}
		\item Kinetische Energie
		\begin{equation*}
		T(u, \dot u) = \frac12 \int_a^b m \dot u(x)^2 dx
		\end{equation*}
		Dabei ist $m$ die Massendichte.
		\item Potentielle Energie
		\begin{equation*}
		U(u)
		=
		\int_a^b S \Big[ \sqrt{1+ u'(x)^2} -1 \Big] dx
		\approx
		\int_a^b S\frac{u'(x)^2}{2} dx
		\end{equation*}
		Dabei ist $S$ die Zugsteifigkeit.
		\item Lagrange-Funktion
		\begin{equation*}
		L(u, \dot u) = T(u, \dot u) - U(u)
		\end{equation*}
		\item Lagrange-Gleichung
		\begin{equation*}
		\frac{\partial L}{\partial u}
		=
		\frac{\partial}{\partial x} \frac{\partial L}{\partial u'}
		+ \frac{\partial}{\partial t} \frac{\partial L}{\partial \dot{u}}
		\end{equation*}
		Einsetzen:
		\begin{equation*}
		0 =
		\frac{\partial}{\partial x} \Big(- S u'(x)\Big)
		+ \frac{\partial}{\partial t}m\dot{u}
		\end{equation*}
		Umstellen:
		\begin{equation*}
		\frac{\partial^2 u}{\partial t^2} = \frac{S}{m} \frac{\partial^2 u}{\partial x^2}
		\end{equation*}
		Das ist die eindimensionale Wellengleichung.
	\end{itemize}
\end{bsp}


%\subsection{Die Hamiltonschen Gleichungen}
%
%Eine Transformation der Lagrange-Gleichung;
%quasi \glqq die andere Seite der Medaille\grqq
%
%\begin{itemize}
%\item Definiere die Impulse
%  \begin{equation*}
%    p_k \colonequals \frac{\partial L}{\partial \dot q_k}(q, \dot q) \qquad \text{für } k = 1, \dots, d
%  \end{equation*}
%  Diese Abbildung heißt \emph{Legendre-Transformation}.
%\end{itemize}
%
%\begin{definition}
%  Die Hamilton-Funktion ist
%  \begin{equation*}
%    H(p, q) \colonequals p^T \dot q - L(q, \dot q).
%  \end{equation*}
%\end{definition}
%
%Dabei geht man natürlich davon aus, dass die Legendre-Transformation eine $C^1$-Bijektion $\dot q \leftrightarrow p$ darstellt.
%
%\emph{Beispiel}: kinetische Energie ist quadratisch:
%
%\begin{equation*}
%  T = \frac 12 \dot q^T M \dot q \qquad \text{ mit $M$ s.p.d.}
%\end{equation*}
%
%\begin{itemize}
%\item Legendre-Transformation: Für festes $q$ hat man
%  \begin{equation*}
%    p = M \dot q.
%  \end{equation*}
%  Transformation ist also tatsächlich glatte Bijektion.
%\item Hamilton-Funktion
%  \begin{align*}
%    H(p, q)
%    & = p^T \dot q - L(q, \dot q) \\
%    & = p^T M^{-1} p - L(q, M^{-1} p) \\
%    & = p^T M^{-1} p - T(q, M^{-1} p) + U(q) \\
%    & = p^T M^{-1} p - \frac12 (M^{-1}p)^T M (M^{-1} p) + U(q) \\
%    & = \frac12 (M^{-1}p)^T M (M^{-1}p) + U(q) \\
%    & = T + U
%  \end{align*}
%  Die Hamilton-Funktion ist die Gesamtenergie!
%\end{itemize}
%
%Auch mit Hilfe der Hamilton-Funktion kann man das Verhalten des mechanischen Systems einfach ausdrücken.
%
%\begin{satz}[{\citet[Thm.\,VI.1.3]{hairer_lubich_wanner:2006}}]
%  Die Lagrange-Gleichung ist äquivalent zu den Hamilton-Gleichungen
%  \begin{equation*}
%    \dot p_k = -\frac{\partial H}{\partial q_k}(p, q),
%    \quad
%    \dot q_k = \frac{\partial H}{\partial p_k}(p, q),
%    \qquad
%    k = 1, \dots, d.
%  \end{equation*}
%\end{satz}
%
%\begin{proof}
%  Lagrange $\implies$ Hamilton (die andere Richtung ist ähnlich)
%  \begin{align*}
%    \frac{\partial H}{\partial q}
%    & = \frac\partial{\partial q}\left( p^T \dot q - L(q, \dot q) \right) & \text{(Def. von $H$)} \\
%    & = p^T \frac{\partial \dot q}{\partial q} - \frac{\partial L}{\partial q} - \underbrace{\frac{\partial L}{\partial \dot q}}_{= p^T} \frac{\partial \dot q}{\partial q} & \text{(Kettenregel)} \\
%    & = - \frac{\partial L}{\partial q} & \text{(Def. von $p = \frac{\partial L}{\partial \dot q}$)} \\
%    & = - \frac{d}{dt}\left( \frac{\partial L}{\partial \dot q} \right) & \text{(Lagrange-Gleichung)} \\
%    & = - \dot p & \text{(Def. von $p$)}
%  \end{align*}
%  Und:
%  \begin{align*}
%    \frac{\partial H}{\partial p}
%    & = \frac{\partial}{\partial p}\left( p^T \dot q - L(q, \dot q) \right) \\
%    & = \dot q + p^T \frac{\partial \dot q}{\partial p} - \underbrace{\frac{\partial L}{\partial \dot q}}_{=p^T} \frac{\partial \dot q}{\partial p} & \text{(Produktregel; $q$ hängt nicht von $p$ ab)} \\
%    & = \dot q \qedhere
%  \end{align*}
%\end{proof}
%
%Sowohl die Lagrangesche als auch die Hamiltonsche Formulierungen haben ihre Daseinsberechtigung.
%\begin{itemize}
%\item Die Lagrange-Formulierung ist besonders fundamental: sie beruht auf Variationsprinzipien
%\item Die Hamilton-Formulierung ist besonders fundamental: sie beruht auf der Gesamtenergie des Systems.
%\end{itemize}
%
%\emph{Beispiel}: Pendel (mit $q = \alpha$)
%\begin{itemize}
%\item Kinetische Energie
%  \begin{equation*}
%    T = \frac12 m l^2 \dot q^2
%  \end{equation*}
%\item Potentielle Energie
%  \begin{equation*}
%    U = -mgl \cos q
%  \end{equation*}
%
%\item Impuls
%  \begin{equation*}
%    p
%    \colonequals \frac{\partial L}{\partial\dot q}
%    = \frac{\partial}{\partial\dot q}\Big( \frac12 m l^2 \dot q^2 + mgl \cos q \Big)
%    = m l^2 \dot q
%  \end{equation*}
%\item Kinetische Energie ist quadratisch, also
%  \begin{align*}
%    H(p, q)
%    & = T(p, q) + U(q) \\
%    & = \frac12 m l^2 \dot q^2 - m g l \cos q \\
%    & = \frac12 \frac{1}{m l^2} p^2 - m g l \cos q
%  \end{align*}
%\item Bewegungsgleichungen:
%  \begin{eqnarray*}
%    \dot p = -\frac{\partial H}{\partial q} & \Leftrightarrow & \dot p = - m g l \sin q \\
%    \dot q = \frac{\partial H}{\partial p} & \Leftrightarrow & \dot q = \frac{1}{m l^2} p
%  \end{eqnarray*}
%\end{itemize}
%
%In der letzten Vorlesung hatten wir gesehen, dass das Pendel die Größe
%\begin{equation*}
%  \frac12 \frac{1}{ml^2} p^2 - mgl \cos q = T(p, q) + U(p)
%\end{equation*}
%erhält.
%Das ist kein Zufall.
%
%\begin{satz}
%  Die Hamilton-Funktion $H$ ist Invariante des Flusses der Hamiltonschen Gleichung.
%\end{satz}
%
%\begin{proof}
%  \begin{align*}
%    \frac{d}{dt} H(p, q)
%    & = \frac{\partial H}{\partial p} \dot p + \frac{\partial H}{\partial q} \dot q & \text{(Kettenregel)} \\
%    & = \frac{\partial H}{\partial p} \Big(-\frac{\partial H}{\partial q}\Big)
%      + \frac{\partial H}{\partial q} \Big(\frac{\partial H}{\partial p}\Big) & \text{(Hamiltonsche Gl.)} \\
%    & = 0 \qedhere
%  \end{align*}
%\end{proof}
%
%Diese sehr allgemeine Erhaltungseigenschaft wollen wir natürlich ins Diskrete übertragen!

%\section{Symplektizität}
%
%Flüsse von Hamiltonschen Systemen haben eine weitere wichtige Erhaltungseigenschaft,
%\begin{itemize}
%\item die sog. Symplektizität
%\item ähnlich wie Volumenerhaltung im Phasenraum
%\end{itemize}
%
%\emph{Beispiel}: Volumenerhaltung beim mathematischen Pendel
%
%  \bigskip
%  \todoannot{1.5\baselineskip}{Bild}
%
%Betrachte die Hamiltonschen Gleichungen
%\begin{equation*}
%  \dot p = - \frac{\partial H}{\partial q}(p, q),
%  \quad
%  \dot q = \frac{\partial H}{\partial p}(p, q)
%\end{equation*}
%
%Umschreiben:
%\begin{equation*}
%  \begin{pmatrix} \dot p \\ \dot q \end{pmatrix}
%  =
%  \begin{pmatrix} 0 & I \\ -I & 0 \end{pmatrix}^{-1}
%  \begin{pmatrix} \frac{\partial H}{\partial p} \\ \frac{\partial H}{\partial q} \end{pmatrix}
%\end{equation*}
%
%Diese Beziehung wollen wir jetzt abstrakter betrachten.
%
%Bemerke: die harmlos aussehende Matrix $\begin{pmatrix} 0 & I \\ -I & 0 \end{pmatrix}$ hat eine besondere Eigenschaft.
%Es gilt nämlich
%\begin{equation*}
%  \begin{pmatrix} 0 & I \\ -I & 0 \end{pmatrix}^2
%  = - \begin{pmatrix} I & 0 \\ 0 & I \end{pmatrix}.
%\end{equation*}
%\begin{itemize}
%\item Verhält sich also wie die imaginäre Einheit $i$.
%\item Erzeugt eine komplexe Struktur auf $\R^{2d}$.
%\end{itemize}
%
%Wir betrachten 2-dimensionale Parallelogramme in $\R^{2d}$.
%\begin{itemize}
%\item Aufgespannt durch Vektoren
%  \begin{equation*}
%    \xi = \begin{pmatrix} \xi^p \\ \xi^q \end{pmatrix},
%    \quad
%    \eta = \begin{pmatrix} \eta^p \\ \eta^q \end{pmatrix}.
%  \end{equation*}
%\end{itemize}
%
%Falls $d = 1$, so ist die orientierte Fläche des Parallelogramms gerade
%\begin{equation*}
%  \det\begin{pmatrix} \xi^p & \eta^p \\ \xi^q & \eta^q \end{pmatrix}
%  =
%  \xi^p \eta^q - \xi^q \eta^p
%  =
%  (\xi^p \quad \xi^q)
%  \begin{pmatrix} 0 & 1 \\ -1 & 0 \end{pmatrix}
%  \begin{pmatrix} \eta^p \\ \eta^q \end{pmatrix}.
%\end{equation*}
%Das verallgemeinern wir jetzt für höhere Dimensionen.
%
%\begin{definition}[Symplektische Form]
%  Die symplektische Form $\omega: \R^{2d} \times \R^{2d} \to \R$ ist
%  \begin{equation*}
%    \omega(\xi, \eta)
%    \colonequals \sum_{i=1}^d \det\begin{pmatrix} \xi_i^p & \eta_i^p \\ \xi_i^q & \eta_i^q \end{pmatrix}
%    = \sum_{i=1}^d \left( \xi_i^p \eta_i^q - \xi_i^q \eta_i^p \right).
%  \end{equation*}
%\end{definition}
%
%\begin{itemize}
%\item Bilineare Form
%\item Interpretation: Summe der orientierten Flächen der Projektionen auf die Koordinatenebenen $(p_i, q_i)$.
%\item Matrixdarstellung
%  \begin{equation*}
%    \omega(\xi, \eta)
%    =
%    \begin{pmatrix} {\xi^p}^T & {\xi^q}^T \end{pmatrix}
%    \begin{pmatrix} 0 & I \\ -I & 0 \end{pmatrix}
%    \begin{pmatrix} \eta^p \\ \eta^q \end{pmatrix}.
%  \end{equation*}
%\end{itemize}
%
%Da die Matrix $\begin{pmatrix} 0 & I \\ -I & 0 \end{pmatrix}$ wichtig zu sein scheint geben wir ihr den Namen $J$.
%
%Eine wichtige Eigenschaft von Hamiltonschen Systemen ist nun, dass ihre Flüsse
%\begin{equation*}
%  \Phi^t: \R^{2d} \to \R^{2d}
%\end{equation*}
%die symplektische Form erhalten.
%
%Das muss man natürlich erklären.
%
%\begin{definition}[Lineare symplektische Abbildung]
%  Eine lineare Abbildung $A: \R^{2d} \to \R^{2d}$ heißt \emph{symplektisch}, wenn
%  \begin{equation*}
%    \omega(A\xi, A\eta) = \omega(\xi, \eta) \qquad \forall \xi, \eta \in \R^{2d}.
%  \end{equation*}
%  Alternativ: Wenn
%  \begin{equation*}
%    A^T J A = J.
%  \end{equation*}
%\end{definition}
%
%\begin{itemize}
%\item Für $d=1$ bedeutet das gerade, dass $A$ flächenerhaltend ist.
%\end{itemize}
%
%\begin{definition}[Differenzierbare symplektische Abbildung]
%  Sei $U$ eine offene Teilmenge von $\R^{2d}$.
%  Eine differenzierbare Abbildung $g: U \to \R^{2d}$ heißt \emph{symplektisch}, falls die Jacobi-Matrix $\nabla g(p,q)$ für alle $(p, q) \in U$ symplektisch ist.
%\end{definition}
%
%Jetzt der zentrale Satz:
%die Flüsse $\Phi^t$ von Hamiltonschen Systemen erhalten die symplektische Form:
%
%\begin{satz}[Poincaré, 1899]
%  Sei $H(p, q)$ zweimal stetig differenzierbar auf $U \subset \R^{2d}$.
%  Sei $\Phi^t$ der Phasenfluss der Differentialgleichung
%  \begin{equation*}
%    \dot y = J^{-1} \nabla H(y)
%  \end{equation*}
%  mit $y = (p, q)$.
%
%  Für jedes feste $t$ ist $\Phi^t$ eine symplektische Abbildung.
%\end{satz}
%
%\begin{proof}
% Der Beweis erfolgt in zwei Schritten:
%  \begin{enumerate}
%  \item $\Phi^0$ ist symplektisch.
%  \item Die \glqq Abweichung von der Symplektizität\grqq{} hängt nicht von $t$ ab.
%  \end{enumerate}
%
%  Zu 1):
%  \begin{itemize}
%  \item $\Phi^0$ ist symplektisch, wenn seine erste Ableitung an jedem Punkt $y_0 = (p_0, q_0)$ symplektisch ist.
%  \item Da $\Phi^0 y_0 = y_0$ gilt
%    \begin{equation*}
%      \left( \frac{\partial \Phi^0 y_0}{\partial y_0} \right)^T J \left( \frac{\partial \Phi^0 y_0}{\partial y_0} \right)
%      = I^T J I
%      = J.
%    \end{equation*}
%  \item Also ist $\Phi^0$ symplektisch.
%  \end{itemize}
%
%  Zu 2):
%  \begin{itemize}
%  \item Wir müssen die Ableitung $\frac{\partial \Phi^t y_0}{\partial y_0}$ untersuchen.
%  \item Linearisierte Störung der Lösung bei einer Störung des Startwerts.
%  \item Also gerade die Wronski-Matrix $\Xi$
%  \item Löst die Gleichung
%    \begin{equation*}
%      \dot \Xi = J^{-1} \underbrace{\nabla^2 H( \Phi^t(y_0) )}_{\text{Hesse-Matrix von $H$}} \Xi
%    \end{equation*}
%
%  \item Konkret heißt das hier
%    \begin{equation}
%     \label{eq:wronski_symplectic_form}
%      \frac{d}{dt} \frac{\partial \Phi^t}{\partial y_0}
%      = J^{-1} \nabla^2 H(\Phi^t y_0) \frac{\partial \Phi^t}{\partial y_0}
%    \end{equation}
%
%
%  \item Produktregel:
%   \begin{equation*}
%    \frac{d}{dt}\bigg[ \bigg(\frac{\partial \Phi^t}{\partial y_0}\bigg)^T J \bigg(\frac{\partial \Phi^t}{\partial y_0}\bigg) \bigg]
%    = \bigg( \frac{d}{dt} \frac{\partial \Phi^t}{\partial y_0} \bigg)^T J \bigg( \frac{\partial \Phi^t}{\partial y_0} \bigg)
%    + \bigg( \frac{\partial \Phi^t}{\partial y_0} \bigg)^T J \bigg( \frac{d}{dt} \frac{\partial \Phi^t}{\partial y_0} \bigg)
%    \end{equation*}
%
%   \item Dort wird jetzt \eqref{eq:wronski_symplectic_form} eingesetzt:
%    \begin{equation*}
%      \frac{d}{dt}\bigg[ \bigg(\frac{\partial \Phi^t}{\partial y_0}\bigg)^T J \bigg(\frac{\partial \Phi^t}{\partial y_0}\bigg) \bigg]
%      =
%      \bigg(\frac{\partial \Phi^t}{\partial y_0}\bigg)^T \nabla^2 H(\Phi^t y_0)^T J^{-T} J \bigg(\frac{\partial \Phi^t}{\partial y_0}\bigg)
%      + \bigg(\frac{\partial \Phi^t}{\partial y_0}\bigg)^T J J^{-1} \nabla^2 H(\Phi^t y_0) \bigg(\frac{\partial \Phi^t}{\partial y_0}\bigg)
%    \end{equation*}
%  \item Aber $J^T = -J$, also $J^{-T} J = -I$, und $\nabla^ H$ ist symmetrisch.
%  \item Deshalb ist
%    \begin{equation*}
%      \frac{d}{dt}\bigg[ \bigg(\frac{\partial \Phi^t}{\partial y_0}\bigg)^T
%        J
%        \bigg(\frac{\partial \Phi^t}{\partial y_0}\bigg) \bigg]
%      = 0.
%      \qedhere
%    \end{equation*}
%  \end{itemize}
%\end{proof}
%
%Es gilt sogar die Umkehrung des Satzes:
%\emph{nur} Hamiltonsche Systeme haben symplektische Flüsse!
%
%\begin{definition}[lokal Hamiltonsch]
%  Eine Differentialgleichung $x' = f(x)$ heißt \emph{lokal Hamiltonsch}, wenn für jedes $x_0 \in U$ eine Nachbarschaft existiert, in der
%  \begin{equation*}
%    f(x) = J^{-1} \nabla H(x)
%  \end{equation*}
%  für eine Funktion $H$.
%\end{definition}
%
%\begin{satz}[{\citet[Satz~VI.2.6]{hairer_lubich_wanner:2006}}]
%  Sei $f: U \to \R^{2d}$ stetig differenzierbar.
%  Dann ist $x' = f(x)$ genau dann lokal Hamiltonsch, wenn der Fluss $\Phi^t x$ für alle $x \in U$ und alle $t$ hinreichend klein symplektisch ist.
%\end{satz}
%
%\section{Symplektische Verfahren}
%
%Wir wollen Verfahren entwickeln, die die Symplektizität von Hamiltonschen Flüssen erben.
%
%\begin{definition}
%	Ein Einschrittverfahren heißt symplektisch, falls der diskrete Fluss
%	\begin{equation*}
%		\Psi^t\colon \R^{2d}\to\R^{2d}
%	\end{equation*}
%	symplektisch ist, wenn das Verfahren auf ein Hamiltonsches System angewendet wird.
%\end{definition}
%
%Die einfachsten symplektischen Verfahren sind die symplektischen Euler-Verfahren
%\begin{align*}
%	p_{k+1} &= p_k - \tau H_q(p_{k+1},q_k) \\
%	q_{k+1} &= q_k + \tau H_p(p_{k+1},q_k)
%\end{align*}
%und
%\begin{align*}
%	p_{k+1} &= p_k - \tau H_q(p_k,q_{k+1}) \\
%	q_{k+1} &= q_k + \tau H_p(p_k,q_{k+1}).
%\end{align*}
%
%\begin{satz}[{\citet[Satz~VI.3.3]{hairer_lubich_wanner:2006}}]
%	Die symplektischen Euler-Verfahren sind symplektisch.
%\end{satz}
%\begin{proof}
%	Beweis für die erste Methode:
%	\begin{itemize}
%		\item Methode ist symplektisch, wenn
%		\begin{equation*}
%			\frac{\partial\Psi^\tau y}{\partial y}\in \R^{2d\times 2d}
%		\end{equation*}
%		für alle $y=(p,q)$ die symplektisch Form erhält, wenn also
%		\begin{equation}\label{eq:symplektische_form}
%			\Big(\frac{\partial\Psi^\tau y}{\partial y}\Big)^TJ\Big(\frac{\partial\Psi^\tau y}{\partial y}\Big) = J.
%		\end{equation}
%		\item Wir bestimmen die vier Komponenten von $\frac{\partial\Psi^\tau y}{\partial y}$:
%		\item [1)] Erste Gleichung des Verfahrens:
%			\begin{equation*}
%				p_{k+1} = p_k - \tau H_q(p_{k+1},q_k)
%			\end{equation*}
%			Ableiten nach $p_k$:
%			\begin{equation*}
%				\frac{\partial p_{k+1}}{\partial p_k} = I - \tau H_{qp}(p_{k+1},q_k)\cdot \frac{\partial p_{k+1}}{\partial p_k}
%			\end{equation*}
%			$\Leftrightarrow$
%			\begin{equation*}
%				\frac{\partial p_{k+1}}{\partial p_k}\Big(I + \tau H_{qp}\Big) = I
%			\end{equation*}
%			Ebenso z.B.
%			\begin{equation*}
%				\frac{\partial p_{k+1}}{\partial q_k}\Big(I + \tau H_{qp}\Big) = -\tau H_{qq}
%			\end{equation*}
%			etc.
%	\end{itemize}
%Zusammen erhält man
%\begin{equation*}
%	\begin{pmatrix}
%		I + \tau H_{qp}^T & 0 \\ -\tau H_{pp} & I
%	\end{pmatrix}
%	\begin{pmatrix}
%		\frac{\partial p_{k+1}}{\partial p_k} & 	\frac{\partial p_{k+1}}{\partial q_k} \\
%			\frac{\partial q_{k+1}}{\partial p_k}  & 	\frac{\partial q_{k+1}}{\partial q_k}
%	\end{pmatrix}
%	=
%	\begin{pmatrix}
%		I & -\tau H_{qq} \\ 0 & I+\tau H_{qp}
%	\end{pmatrix},
%\end{equation*}
%also
%\begin{equation*}
%	\frac{\partial \Psi^\tau y}{\partial y} = \begin{pmatrix}
%	I + \tau H_{qp}^T & 0 \\ -\tau H_{pp} & I
%	\end{pmatrix}^{-1} \begin{pmatrix}
%	I & -\tau H_{qq} \\ 0 & I+\tau H_{qp}
%	\end{pmatrix}.
%\end{equation*}
%Damit kann man die Erhaltungseigenschaft \eqref{eq:symplektische_form} direkt nachrechnen.
%\end{proof}
%
%\medskip
%
%Die symplektischen Euler-Verfahren sind \emph{keine} RK-Verfahren.\\
%Stattdessen gehören sie zu den sog.\ \emph{partitionierten} RK-Verfahren.\\
%Betrachte Differentialgleichungen der Form
%\begin{equation*}
%	y'=f(y,z),\qquad z'=g(y,z),
%\end{equation*}
%wobei $y\in\R^{n_1}$ und $z\in\R^{n_2}$\\
%\emph{Idee:} Nimm für $y$ und $z$ zwei verschiedene RK-Verfahren.\\
%Details bei \citet[Kapitel II.2]{hairer_lubich_wanner:2006}
%
%\medskip
%
%Es gibt auch ein \glqq einfaches\grqq\ Verfahren zweiter Ordnung, das symplektisch ist.
%\begin{satz}[{\citet[Satz~VI.3.5]{hairer_lubich_wanner:2006}}]
%	Die implizite Mittelpunktsregel
%	\begin{equation*}
%		y_{k+1} = y_k + \tau J^{-1}\nabla H\Big( \frac{y_{k+1}+y_k}{2}\Big)
%	\end{equation*}
%	ist symplektisch.
%\end{satz}
%\begin{proof}
%	Wir leiten wieder ab
%	\begin{align*}
%		\frac{\partial \Psi^\tau y_k}{\partial y_k} &= \frac{\partial y_{k+1}}{\partial y_k} \\
%		&= I + \tau J^{-1} \nabla^2 H\Big( \frac{y_{k+1} + y_k}{2} \Big) \cdot \Big( \frac{1}{2} \frac{\partial y_{k+1}}{y_k} + \frac{1}{2}\Big).
%	\end{align*}
%Umformen ergibt
%\begin{align*}
% \frac{\partial y_{k+1}}{\partial y_k}
% &=
% \Big( I - \frac{\tau}{2} J^{-1}\nabla^2H\Big)^{-1}\Big(I+\frac{\tau}{2}J^{-1}\nabla^2 H\Big).
%\end{align*}
%
%	Dann kann man direkt nachrechnen dass
%	$\Big(\frac{\partial y_{k+1}}{\partial y_k}\Big)^T J \frac{\partial y_{k+1}}{\partial y_k} = J$.
%
% \bigskip
%\todoannot{1.5\baselineskip}{Ein paar Details zu dieser Rechnung einfügen!}
%\end{proof}
%
%
%\subsection{Symplektische RK-Verfahren}
%
%\begin{itemize}
%	\item Relativ neue Verfahren, wurden erst Ende der 1980er Jahre systematisch untersucht.
%\end{itemize}
%
%Wir interessieren uns wieder für die Ableitung
%\begin{equation*}
%\Xi(t) = \frac{\partial \Phi^t y_0}{\partial y_0}.
%\end{equation*}
%Diese löst bekanntlich eine lineare Differentialgleichung.
%
%\begin{lemma}[{\citet[Lemma VI.4.1]{hairer_lubich_wanner:2006}}]
%	
%	Das folgende Diagramm kommutiert für alle Runge-Kutta-Verfahren und alle partitionierten Runge-Kutta-Verfahren:\\
%	\begin{center}
%	    \begin{tikzpicture}
%	    	\draw (0,0)  node{$\{y_k\}$};
%	    	\draw[->] (1,0) -- node[above]{$\frac{\partial}{\partial y_0}$} (3,0); 
%	    	\draw (5,0) node{$\{y_k,\Xi_k\}$};
%	    	\draw[->] (0,2)-- node[left]{RK-Verfahren}(0,1);
%	    	\draw[->] (5,2)-- node[right]{RK-Verfahren}(5,1);
%	    	\draw[->] (1,3)-- node[above]{$\frac{\partial}{\partial y_0}$}(3,3);
%	    	\draw (-1,3)  node{$\dot y = f(y),\ y(0)=y_0$};
%	    	\draw (5.5,3.2)  node{$\dot y = f(y),\quad y(0)=y_0$};
%	    	\draw (5.5,2.7)  node{$\dot\Xi = f'(y)\Xi,\quad \Xi(0)=I$};
%	    \end{tikzpicture}
%	\end{center}
%\end{lemma}
%\begin{proof}[Beweisidee:] Betrachte exemplarisch das explizite Euler-Verfahren
%\begin{equation*}
%	y_{k+1} = y_k +\tau f(y_k).
%\end{equation*}
%Ableiten nach $y_0$ ergibt
%\begin{equation*}
%	\Xi_{k+1} = \Xi_k + \tau f'(y_k)\Xi_k.
%\end{equation*}
%Das ist gerade das explizite Euler-Verfahren für die Gleichung
%\begin{equation*}
%	\dot \Xi = f'(y)\Xi.
%\end{equation*}
%Startwert passt auch, denn $I=\frac{\partial y_0}{\partial y_0} = \Xi_0$.
%\end{proof}
%
%\bigskip
%
%\emph{Idee:} Die Symplektizitätsbedingung ist eine quadratische invariante des
%erweiterten Systems für die Variablen $y$, $\Xi$.
%
%
%\begin{satz}
%	Alle Verfahren die quadratische Invarianten erhalten sind symplektisch.
%\end{satz}
%\begin{proof}
%	Der quadratische Ausdruck $\Xi^T J\Xi$ is Invariante der Gleichung
%	\begin{equation*}
%		\dot \Xi = J^{-1}\nabla^2 H(y)\Xi,
%	\end{equation*}
%	denn
%	\begin{align*}
%		\frac{d}{dt}\Big(\Xi^T J\Xi\Big) & = \dot \Xi^T J\Xi + \Xi^T J\dot\Xi \\
%		                                 & = (J^{-1}\nabla^2 H\Xi)^T J\Xi + \Xi^T JJ^{-1}\nabla^2 H \Xi \\
%		                                 & = \Xi^T \nabla^2 H \underbrace{J^{-T}J}_{=-I} + \Xi^T\underbrace{ JJ^{-1}}_{=I} \nabla^2 H \Xi \\
%		                                 & = 0.  \qedhere
%	\end{align*}
%\end{proof}
%
%\begin{kor}
%	Gauß-Verfahren sind symplektisch.
%\end{kor}
%
%\subsection{Reversibilität vs.\ Symplektizität}
%
%Es gibt reversible Verfahren, die nicht symplektisch sind.
%
%\medskip
%
%Es gibt symplektische Verfahren, die nicht reversibel sind.
%
%\medskip
%
%Für quadratische Hamilton-Funktionen ist das anders.
%
%\begin{satz}[{\citet[Satz VI.4.9.]{hairer_lubich_wanner:2006}}]
%	Für RK-Verfahren sind die folgenden Aussagen äquivalent:
%	\begin{itemize}
%		\item [i)] Die Methode ist reversibel für lineare Probleme
%		\begin{equation*}
%			\dot y = Ly
%		\end{equation*}
%		\item [ii)] Die Methode ist symplektisch für Hamilton-Gleichungen mit quadratischer Hamilton-Funktion
%		\begin{equation*}
%			H(y) = \frac{1}{2} y^TCy.\qquad C\ s.p.d.
%		\end{equation*}
%	\end{itemize}
%\end{satz}
%\begin{proof}
%	$ii) \to i)$\\
%	\begin{itemize}
%		\item Die Hamilton-Gleichungen haben die Form
%		\begin{equation*}
%			\dot y = J^{-1} \nabla H(y) = J^{-1}Cy,
%		\end{equation*}
%		sind also linear.
%
%		\item Das Runge-Kutta-Verfahren dafür hat also die Form
%		\begin{equation*}
%			\Psi^\tau y = R(\tau J^{-1} C) y
%		\end{equation*}
%		wobei $R$ die Stabilitätsfunktion ist.
%		\item Da das Verfahren symplektisch ist, gilt
%		\begin{equation*}
%			R(\tau J^{-1} C)^T J R(\tau J^{-1} C) = J.
%		\end{equation*}
%
%		\item Da $R= PQ^{-1}$ für Polynome $P,Q$ erhält man
%		\begin{equation}\label{eq:beweis_symplektisch_symmetrisch}
%			P(\tau J^{-1} C)^T JP(\tau J^{-1} C) = Q(\tau J^{-1} C)^T J Q(\tau J^{-1} C).
%		\end{equation}
%
%\todoannot{1.5\baselineskip}{Die folgende Rechnung muss nochmal im Detail überprüft werden!}
%
%		\item Betrachte das Produkt \glqq Polynom in $J^{-1} C$\grqq\ mit $J$.
%
%		\item Für jedes Monom $(J^{-1} C)^k$ gilt ($C$ ist symmetrisch)
%		\begin{equation*}
%			((J^{-1} C)^k)^T J
%			=
%			(C^T J^{-T})^k J
%			=
%			J(-(J^{-1} C)^k)\quad \forall k=0,1,2,\hdots
%		\end{equation*}
%		\item Also folgt aus \eqref{eq:beweis_symplektisch_symmetrisch}
%		\begin{equation*}
%			P(-\tau J^{-1} C)\cdot P(\tau J^{-1} C) =  Q(-\tau J^{-1} C)\cdot Q(\tau J^{-1} C)
%		\end{equation*}
%		bzw.
%		\begin{equation*}
%			R(-\tau J^{-1} C)\cdot R(\tau J^{-1} C) = I.
%		\end{equation*}
%		\item  Das ist gerade die Symmetrie des Verfahrens.
%		\qedhere
%	\end{itemize}
%\end{proof}
%
%\section{Energieerhaltung}
%
%\begin{itemize}
%\item Wir haben einige Mühe in Verständnis und Erhaltung der Symplektizität gesteckt.
%\item Aber Symplektizität ist eine sehr abstrakte Eigenschaft.
%  Wozu soll die gut sein?
%\item Hier komme eine etwas konkretere Rechtfertigung.
%\end{itemize}
%
%Betrachte das mathematische Pendel.
%\begin{itemize}
%\item Kinetische Energie:
%  \begin{equation*}
%    T(q, \dot q) = \frac{m l^2}{2} \dot q
%  \end{equation*}
%  ($q$ ist der Winkel)
%\item Potentielle Energie:
%  \begin{equation*}
%    U(q) = -mgl \cos q
%  \end{equation*}
%\item Bewegungsgleichungen:
%  \begin{equation*}
%    \ddot q + \frac gl \sin q = 0
%  \end{equation*}
%\item Gesamtenergie:
%  \begin{equation*}
%    E = \frac{m l^2}2 \dot q^2 - mgl \cos q
%  \end{equation*}
%  Dies entspricht der Hamilton-Funktion
%  \begin{equation*}
%    H(p, q) = \frac{1}{2ml^2} p^2 - mgl \cos q
%  \end{equation*}
%  Damit ist die Gesamtenergie eine Erhaltungsgröße!
%\item Aber: weder linear noch quadratisch.
%  Wird daher nicht automatisch von z.B. Gauß-Verfahren erhalten.
%\end{itemize}
%
%Wird die Energie von symplektischen Verfahren erhalten?
%
%Nein! Aber fast...
%
%\begin{satz}[{\citet{benettin_giorgilli:1994}; \citet[Thm.\,IX.8.1]{hairer_lubich_wanner:2006}}]
%  Betrachte ein Hamilton-System mit analytischer Hamilton-Funktion $H: D \to R$, ($D \subset \R^{2d}$),
%  und wende ein symplektisches Verfahren $\Psi^\tau$ mit Schrittweite $\tau$ an.
%  Wenn die numerische Lösung in einer kompakten Menge $K\subset D$ bleibt, dann existiert ein $\tau_0$, so dass
%  \begin{equation*}
%    H(y_n) = H(y_0) + O(\tau^p)
%  \end{equation*}
%  für exponentiell lange Zeitintervalle $n\tau \le e^{\frac{\tau_0}{2 \tau}}$.
%\end{satz}
%
%Symplektische Verfahren erhalten also \emph{nicht} die Hamilton-Funktion bzw.\ die Gesamtenergie.
%Aber die numerische Energie bleibt \glqq in der Nähe\grqq{} der exakten Energie!
%
%
%
%\section{Variationelle Integratoren}
%
%Mit dem jetzt Gelernten können wir Zeitschrittverfahren auf eine ganz neue Art konstruieren.\\
%Siehe \citet{marsden_west:2001} für eine detailliertere Übersicht.\\
%
%Wir erinnern an das Prinzip der stationären Wirkung (auch Hamiltonsches Prinzip genannt)
%\begin{itemize}
%	\item Lagrange-Funktion
%	\begin{equation*}
%		L(q,\dot q) = T(q,\dot q) - U(q)
%	\end{equation*}
%\end{itemize}
%
%\begin{definition}
% Die \emph{Wirkung} eine Trajektorie $q\colon t\mapsto (q(t),\dot q(t))$ ist
%	\begin{equation*}
%		S(q)\coloneqq \int_{t_0}^{t_1} L(q(t),\dot q(t))\,dt.
%	\end{equation*}
%\end{definition}
%
%\medskip
%
% Wir betrachten nur Trajektorien mit gegebenem festen Start- und Endpunkt
%	\begin{equation*}
%		q(t_0) = q_0,\qquad q(t_1) = q_1.
%	\end{equation*}
%
%\begin{definition}[Hamiltonsches Prinzip]
%Die tatsächlich vorkommenden Trajektorien sind die, die die Wirkung stationär machen.
%\end{definition}
%
%Sei $q$ eine Trajektorie, und $\delta q$ eine Variation davon, die die Endpunkte fest lässt, also $\delta q(t_0)=\delta(t_1) = 0$.
%\begin{itemize}
%	\item Stationarität von $q$ heißt dann, dass für alle solche $\delta q$
%	\begin{equation*}
%		\frac{d}{d\epsilon} S(q+\epsilon\delta q)\Big\vert_{\epsilon=0} = 0.
%	\end{equation*}
%\end{itemize}
%Wie schon in Kapitel~\ref{sec:lagrange_gleichung} gezeigt ist dies äquivalent zur Euler--Lagrange-Gleichung
%\begin{equation*}
%	\frac{d}{dt}\frac{\partial L}{\partial \dot q} = \frac{\partial L}{\partial q}.
%\end{equation*}
%
%\bigskip
%
%Wir betrachten jetzt das Wirkungsintegral $S$ als Funktion der Start- und Endposition
%\begin{equation*}
%	S(q_0,q_1) = \int_{t_0}^{t_1} L(q(t),\dot q(t))\,dt.
%\end{equation*}
%Dabei ist $q$ die zu $q_0,q_1$ gehörige Lösung der Lagrange-Gleichung.\\
%
%\subsubsection*{Exkurs Anfang: Erzeugendenfunktionen}
%
%Wir brauchen ein weiteres Kriterium für Symplektizität:
%\begin{itemize}
%	\item Betrachte ein gegebenes Hamiltonsches System $H$ auf einem festen Zeitintervall $[t_0,t_1]$
%	\item Seien $p_0 \in \R^d$ und $q_0 \in \R^d$ die Startwerte zur Zeit $t_0$
%	\item Bezeichne die Werte zur Zeit $t_1$ mit  $p_1 \in \R^d$ und $q_1 \in \R^d$
%	\item Es gibt eine Abbildung $\Phi^{t_0,t_1}(p_0,q_0) = (p_1,q_1)$.
%\end{itemize}
%
%Wie wir wissen, ist diese symplektisch.\\
%{[Achtung: der folgende Satz enthält überdurchschnittlich viel didaktische Reduktion]}
%
%\begin{satz}[{\citet[Satz VI.5.1]{hairer_lubich_wanner:2006}}]
%\label{thm:erzeugendenfunktion}
%	Eine Abbildung $\varphi\colon (p_0,q_0) \mapsto (p_1,q_1)$ ist genau dann symplektisch, wenn lokal eine Funktion
%	\begin{equation*}
%	S\colon (q_0,q_1)\mapsto S(q_0,q_1)\in\R
%	\end{equation*}
%	existiert, so dass
%	\begin{equation}\label{eq:abbildung_symplektisch}
%	\nabla S
%	=
%	\begin{pmatrix}
%		\frac{\partial S}{\partial q_0}  \\
%		\frac{\partial S}{\partial q_1}
%	\end{pmatrix} = \begin{pmatrix}
%		-p_0 \\ p_1
%	\end{pmatrix}.
%	\end{equation}
%\end{satz}
%
%\begin{itemize}
%	\item Wenn man eine symplektische Abbildung $(p_0,q_0)\mapsto(p_1,q_1)$ hat, dann sie durch \eqref{eq:abbildung_symplektisch} aus der Funktion $S$ rekonstruiert werden.
%	\item Aber der obige Satz ist \glqq genau dann, wenn\grqq\ . Es gilt also auch die Umkehrung:
%	\begin{itemize}
%		\item Jede hinreichen glatte (und in einem gewissen Sinne nicht degenerierte)
%		 Funktion $S$ \emph{erzeugt} via \eqref{eq:abbildung_symplektisch} eine symplektische Abbildung $(p_0,q_0)\mapsto (p_1,q_1)$!
%	\end{itemize}
%	\item Man kann also auf systematische Art symplektische Abbildungen erzeugen. Die Funktion $S$ heißt deshalb Erzeugendenfunktion.
%\end{itemize}
%
%
%\subsubsection*{Exkurs Ende}
%
%Große Überraschung: Diese Funktion $S$ ist gerade die Erzeugendenfunktion einer symplektischen Abbildung!
%\begin{itemize}
% \item Berechne partielle Ableitung:
%  \begin{align*}
%   \frac{\partial S}{\partial q_0}
%   & =
%   \int_{t_0}^{t_1} \bigg( \frac{\partial L}{\partial q} \frac{\partial q}{\partial q_0} + \frac{\partial L}{\partial \dot q}\frac{\partial\dot q}{\partial q_0}\bigg)\,dt \\
%   %
%   \intertext{Partielle Integration des zweiten Terms in der Klammer liefert}
%   &=
%   \frac{\partial L}{\partial \dot q}\frac{\partial q}{\partial q_0}\Bigg\vert_{t_0}^{t_1} + \int_{t_0}^{t_1} \underbrace{\bigg( \frac{\partial L}{\partial q}  - \frac{d}{dt} \frac{\partial L}{\partial \dot q}\bigg)}_{=0} \frac{\partial q}{\partial q_0}\,dt \\
%		&= \frac{\partial L(q_1,\dot q_1)}{\partial \dot q} \cdot \underbrace{\frac{\partial q_1}{\partial q_0}}_{=0}  - \frac{\partial L(q_0,\dot q_0)}{\partial \dot q} \cdot \underbrace{\frac{\partial q_0}{\partial q_0}}_{=1}  \\
%		&= - \frac{\partial L(q_0,\dot q_0)}{\partial \dot q} \\
%		&= -p_0 \qquad\text{(Def. des Impulses)}.
%	\end{align*}
%\end{itemize}
%Ebenso berechnet man
%\begin{equation*}
%	\frac{\partial S}{\partial q_1} = p_1.
%\end{equation*}
%Wir erhalten also
%\begin{equation}
%\nabla S =
%\begin{pmatrix}
%\frac{\partial S}{\partial q_0}  \\
%\frac{\partial S}{\partial q_1}
%\end{pmatrix}
%=
%\begin{pmatrix}
%-p_0 \\ p_1
%\end{pmatrix}.
%\end{equation}
%Dies ist gerade die Formel \eqref{eq:abbildung_symplektisch} für Erzeugendenfunktionen von symplektischen Abbildungen.
%
%Daraus folgt dass die entsprechende Abbildung $(p_0,q_0) \mapsto (p_1,q_1)$ symplektisch ist.
%
%\subsection{Idee der variationellen Integratoren}
%
%Wir ersetzen das Integral im Hamiltonschen Prinzip durch eine diskrete Approximation:
%
%\medskip
%
%\begin{itemize}
%	\item Führe ein Zeitgitter ein
%	\begin{equation*}
%	t_0<t_1<\hdots < t_N = T.
%	\end{equation*}
%	\item  Führe die approximative Wirkung ein
%	\begin{equation*}
%		L_h(q_k,q_{k+1})\approx \int_{t_k}^{t_{k+1}} L(q(t),\dot q(t))\,dt
%	\end{equation*}
%	(z.B.\ durch eine Quadraturformel)\\
%
%	Hier könnte man denken dass $L_h$ ein schlechtes Symbol ist, weil es sich ja schließlich
%	um eine Wirkung handelt.  Andererseits fungiert $L_h$ später bei der Definition der
%	diskreten Impulse wie eine Lagrange-Funktion (siehe~\eqref{eq:diskrete_lagrange_transformation}).
%
%	$q$ ist hier die Lösung der Lagrange-Gleichung auf $[t_k,t_{k+1}]$ mit gegebenen Start- und Endwerten $q_k,q_{k+1}$.
%	\item Definiere das diskrete Wirkungsfunktional
%	\begin{equation*}
%		S_h\Big(\big\{q_k\big\}_{k=0}^N\Big) \colonequals \sum_{k=0}^{N-1} L_h(q_k,q_{k+1}).
%	\end{equation*}
%\end{itemize}
%
%\begin{definition}[Diskretes Hamilton-Prinzip]
%Finde $\{q_k\}^N_{k=0}$  mit gegebenen $q_0, q_N$, so dass $S_h$ stationär wird.
%\end{definition}
%
%\medskip
%
%Wie kann man $L_h$ wählen?
%
%\medskip
%
%\emph{Beispiel:} (\citet{mackay:1992}, 1992)
%\begin{itemize}
%\item Approximiere $q$ auf $[t_k, t_{k+1}]$ als linear Interpolierende von $q_k$ und $q_{k+1}$.
%\item Approximiere das Integral durch die Trapezregel
%\begin{equation*}
%L_h(q_k, q_{k+1})
%=
%\tau \cdot \frac{1}{2} \Big[L\Big(q_k, \frac{q_{k+1} - q_k}{\tau}\Big) +  L\Big(q_{k+1}, \frac{q_{k+1} - q_k}{\tau}\Big)\Big].
%\end{equation*}
%\end{itemize}
%
%Beispiel: (\citet{wendlandt_marsden:1997}, 1997)
%\begin{itemize}
%\item Nimm statt der Trapezregel die Mittelpunktsregel
%\begin{equation*}
%L_h(q_k, q_{k+1}) = \tau \Big[L\Big(\frac{q_{k+1} + q_k}{2}, \frac{q_{k+1} - q_k}{\tau}\Big)\Big].
%\end{equation*}
%\end{itemize}
%
%Wie kommen wir an stationäre Punkte von $S_h$?
%\begin{itemize}
%\item Ableitung ausrechnen und gleich Null setzen!
%\newline Partielle Ableitung für $k=1, \dots, N-1$:
%\begin{equation*}
%\frac{\partial S_h}{\partial q_k}
%=
%\frac{\partial}{\partial q_k} \sum_{i=0}^{N-1} L_h (q_i, q_{i+1})
%=
%\frac{\partial}{\partial q_k} L_h(q_{k-1}, q_k) +  \frac{\partial}{\partial q_k} L_h(q_k, q_{k+1}).
%\end{equation*}
%\item Dieser Ausdruck $=0$ sind die \emph{diskreten Euler--Lagrange-Gleichungen}.
%\item Ein System von algebraischen Gleichungen (mit Bandstruktur).
%\end{itemize}
%
%\medskip
%
%\textbf{Fragen:}
%\begin{itemize}
%\item Wann kriege ich mit diesem Ansatz symplektische Integratoren?
%\item Kann ich das Verfahren so umschreiben dass ich wieder einen Zeitschritt nach dem anderen berechnen kann?
%\end{itemize}
%
%\subsection{Variationelle Integratoren sind symplektisch}
%
%Wir schreiben die diskrete Wirkung jetzt wieder als Funktion von Anfangs- und Endzustand
%\begin{equation*}
%S_h(q_0, q_N) = \sum_{k=0}^{N-1} L_h (q_k, q_{k+1}).
%\end{equation*}
%Dabei ist $\{q_k\}$ die dazugehörige Lösung des variationellen Integrators.
%\begin{itemize}
%\item Wir rechnen wieder die partiellen Ableitungen aus. Es bezeichne $\frac{\partial L_h}{\partial x}$,
%   $\frac{\partial L_h}{\partial y}$ die partiellen Ableitungen von $L_h$ nach dem ersten bzw.\ zweiten Argument:
%\begin{align*}
%\frac{\partial S_h}{\partial q_0}
%& = \sum_{k=0}^{N-1} \Bigg[ \frac{\partial L_h}{\partial x} \cdot\frac{\partial q_k}{\partial q_0} +  \frac{\partial L_h}{\partial y} \cdot\frac{\partial q_{k+1}}{\partial q_0}\Bigg] \\
%%
%& = \frac{\partial L}{\partial x} (q_0,q_1)\cdot \underbrace{\frac{\partial q_0}{\partial q_0}}_{=1} \\
%& \qquad  + \sum_{k=1}^{N-1} \Bigg[ \underbrace{\frac{\partial L_h}{\partial y}(q_{k-1},q_k) \cdot\frac{\partial q_k}{\partial q_0} +  \frac{\partial L_h}{\partial x} (q_k,q_{k+1})\cdot\frac{\partial q_k}{\partial q_0}}_{\text{$=0$, wg.\ diskreter Lagrange-Gleichung}}\Bigg] \\
%& \qquad \qquad  + \frac{\partial L_h}{\partial y} (q_{N-1},q_N)\cdot \underbrace{\frac{\partial q_N}{\partial q_0}}_{=0} \\
%%
%& =
%\frac{\partial L}{\partial x} (q_0, q_1).
%\end{align*}
%\item Ebenso 
%\begin{equation*}
%\frac{\partial S_h}{\partial q_N}  = \frac{\partial L_h}{\partial y} (q_{N-1}, q_N).
%\end{equation*}
%\end{itemize}
%Jetzt führen wir die \emph{diskreten Impulse} durch eine \emph{diskrete Legendre-Transformation} ein:
%\begin{equation}
%\label{eq:diskrete_lagrange_transformation}
% p_k
% \colonequals
% - \frac{\partial L_h}{\partial x} (q_k, q_{k+1})
% =
% \frac{\partial L_h}{\partial y} (q_{k-1}, q_k)
%\end{equation}
%Die Gleichheit ist gerade die diskrete Euler--Lagrange-Gleichung.
%
%[Vergleiche: $p = \frac{\partial L}{\partial \dot{q}}(q,\dot{q})$]
%
%\begin{itemize}
%\item Für $k=N$ erhalten wir
%\begin{equation*}
%p_N = \frac{\partial L_h}{\partial y} (q_{N-1}, q_N).
%\end{equation*}
%\end{itemize}
%
%\bigskip
%
%Zusammen also:
%\begin{equation*}
%\nabla S_h = \begin{pmatrix}
%\frac{\partial S_h}{\partial q_0} \\
%\frac{\partial S_h}{\partial q_N}
%\end{pmatrix} 
%=
%\begin{pmatrix}
%\frac{\partial L_h}{\partial x} (q_0, q_1) \\
%\frac{\partial L_h}{\partial y} (q_{N-1},q_N)
%\end{pmatrix}
%=
%\begin{pmatrix}
%- p_0 \\ p_N
%\end{pmatrix}
%\end{equation*}
%
%Nach Satz~\ref{thm:erzeugendenfunktion} ist $S_h$ also eine Erzeugendenfunktion für die symplektische Abbildung
%\begin{equation*}
%(p_0, q_0) \mapsto (p_N, q_N).
%\end{equation*} 
%
%\subsection{Variationelle Integratoren als klassische Einschrittverfahren}
%
%Jetzt bauen wir uns ein klassisches Zeitschrittverfahren:
%
%\medskip
%
%Angenommen die diskrete Legendre-Transformation \eqref{eq:diskrete_lagrange_transformation} sei eine Bijektion zwischen $p_k$ und $q_{k+1}$.
%
%Einschrittverfahren:
%\begin{center}
%    \begin{tikzpicture}
%      \draw (0,0)  node{$(p_k, q_k)$};
%      \draw[->] (0,-1) -- node[right]{inv. disk. Legendre-Transformation} (0,-2);
%      \draw (0,-3) node{$(q_k, q_{k+1})$};
%      \draw[->] (0,-4)-- node[right]{diskrete Euler--Lagrange-Gleichung}(0,-5);
%      \draw (0,-6) node{$(q_{k+1}, q_{k+2})$};
%      \draw[->] (0,-7)-- node[right]{diskrete Legendre-Transformation}(0,-8);
%      \draw (0,-9) node{$(p_{k+1}, q_{k+1})$};
%    \end{tikzpicture}
%\end{center}
%Schritt 2 und 3 schreibt man als
%\begin{equation*}
%p_{k+1} = \frac{\partial L_h}{\partial y} (q_k, q_{k+1}).
%\end{equation*}
%
%\begin{satz}
%Das diskrete Hamilton Prinzip erzeugt das Zeitschrittverfahren
%\begin{equation*}
%(p_k, q_k) \mapsto (p_{k+1}, q_{k+1}),
%\qquad
%p_k = - \frac{\partial L_h}{\partial x} (q_k, q_{k+1}),
%\qquad
%p_{k+1}  = \frac{\partial L_h}{\partial y} (q_k, q_{k+1}).
%\end{equation*}
%(Die erste Gleichung ist dabei als implizite Gleichung für $q_{k+1}$ zu verstehen.)
%\begin{itemize}
%\item[i)] Dieses Verfahren ist symplektisch.
%\item[ii)] Jedes symplektische Verfahren lässt sich auf diese Art darstellen.
%\end{itemize}
%\end{satz}
%\begin{proof}\mbox{} % Force line break
%\begin{itemize}
%  \item [i)] $L_h$ ist Erzeugendenfunktion für die symplektische Abbildung $(p_k, q_k) \mapsto (p_{k+1}, q_{k+1})$
%  \item [ii)] Jede symplektische Abbildung hat eine Erzeugendenfunktion. Wähle diese als diskrete Lagrange-Funktion.
%  \qedhere
%\end{itemize}
%\end{proof}
%
%\bigskip
%
%\emph{Beispiel:} Das Verfahren von \citet{mackay:1992}:
%\begin{equation*}
%L_h (q_k, q_{k+1}) = \frac{\tau}{2} L(q_k, v_{k + \frac{1}{2}}) + \frac{\tau}{2} L(q_{k+1}, v_{k + \frac{1}{2}})
%\end{equation*}
%mit $v_{k + \frac{1}{2}} \colonequals \frac{1}{\tau} (q_{k+1} - q_k)$.
%\newline
%\newline
%Man erhält das Verfahren:
%\begin{flalign*}
%p_k &= - \frac{\partial L_h}{\partial x} (q_k, q_{k+1}) \\
%&= - \frac{\tau}{2} \Big[ \frac{\partial L}{\partial q}(q_k, v_{k + \frac{1}{2}}) + \frac{\partial L}{\partial \dot{q}}(q_k, v_{k + \frac{1}{2}}) \cdot \Big(-\frac{1}{\tau}\Big) \\
%& \quad + \frac{\partial L}{\partial q}(q_{k+1}, v_{k + \frac{1}{2}}) \cdot \underbrace{\frac{\partial q_{k+1}}{\partial q_n}}_{=0} + \frac{\partial L}{\partial \dot{q}}(q_{k+1}, v_{k + \frac{1}{2}}) \cdot \Big(-\frac{1}{\tau}\Big) \Big] \\
%&= - \frac{\tau}{2}\frac{\partial L}{\partial q}(q_k, v_{k + \frac{1}{2}}) + \frac{1}{2} \frac{\partial L}{\partial \dot{q}}(q_k, v_{k + \frac{1}{2}}) +  \frac{1}{2}\frac{\partial L}{\partial \dot{q}}(q_{k+1}, v_{k + \frac{1}{2}}),
%\end{flalign*}
%sowie
%\begin{flalign*}
%p_{k+1}
%& =
%\frac{\partial L_h}{\partial y} (q_k, q_{k+1}) \\
%%
%&= \frac{\tau}{2}\frac{\partial L}{\partial q}(q_{k+1}, v_{k + \frac{1}{2}}) + \frac{1}{2} \frac{\partial L}{\partial \dot{q}}(q_k, v_{k + \frac{1}{2}}) +  \frac{1}{2}\frac{\partial L}{\partial \dot{q}}(q_{k+1}, v_{k + \frac{1}{2}}) .
%\end{flalign*}
%
%\medskip
%
%Wir betrachten das mechanische System
%\begin{equation*}
%L (q, \dot{q}) = \frac{1}{2} \dot{q}^T M \dot{q} - U(q) \quad\text{mit $M$ s.p.d}
%\end{equation*}
%
%Damit ist
%\begin{align*}
%\frac{\partial L}{\partial \dot{q}}(q_k, v_{k + \frac{1}{2}}) & = Mv_{k + \frac{1}{2}} = \frac{\partial L}{\partial \dot{q}} (q_{k+1}, v_{k + \frac{1}{2}}) \\
%%
%\frac{\partial L}{\partial q}(q_k, v_{k + \frac{1}{2}}) & = - \nabla U(q_k) = F(q_k) \quad\text{(Kraftfeld an der Stelle $q_k$)}
%\end{align*}
%
%Das Verfahren wird also zu:
%\begin{align*}
%p_k & = - \frac{\tau}{2} F(q_k) + \frac{1}{2} Mv_{k+\frac{1}{2}}  + \frac{1}{2} Mv_{k+\frac{1}{2}} \\
%p_{k + 1} & =  \frac{\tau}{2} F(q_{k+1}) + \frac{1}{2} Mv_{k+\frac{1}{2}}  + \frac{1}{2} Mv_{k+\frac{1}{2}}
%\end{align*}
%
%Umschreiben:
%\begin{align*}
%Mv_{k+\frac{1}{2}} & = p_k + \frac{\tau}{2} F(q_k) &\quad\text{(erste Gleichung)}\\
%q_{k + 1} & = q_k + \tau v_{k+\frac{1}{2}} &\quad\text{(Definition von $v_{k+ \frac{1}{2}}$)}\\
%p_{k + 1} & = Mv_{k+\frac{1}{2}}  + \frac{\tau}{2} F(q_{k+1}) &\quad\text{(zweite Gleichung)}\\
%\end{align*}
%
%Das ist gerade das Störmer-Verlet-Verfahren!
%
%\medskip
%
%Diskrete Lagrange-Gleichung in diesem Fall:
%\begin{align*}
%0 & =  \frac{\partial L_h}{\partial y} (q_{k-1}, q_k) + \frac{\partial L_h}{\partial x} (q_k, q_{k+1}) \\
%\Leftrightarrow M\underbrace{\frac{(q_{k+1} - 2q_k + q_{k-1})}{\tau^2}}_{\approx \ddot{q}_k} & = F(q_k)
%\end{align*}
%
%Kann also als direkte Diskretisierung der Bewegungsgleichung $M \ddot{q} = F(q)$ interpretiert werden!
%
%\subsection{Variationelle Integratoren höherer Ordnung}
%
%Wie können wir variationelle Integratoren höherer Ordnung konstruieren?
%
%\bigskip
%
%Bessere Approximation von
%\begin{equation*}
% L_h (q_k, q_{k+1})
% \colonapprox
% \int_{t_k}^{t_{k+1}} L(q(t),\dot{q}(t))\,dt
%\end{equation*}
%heißt:
%\begin{itemize}
% \item Approximation von $q$ höherer Ordnung,
% \item Quadraturformel höherer Ordnung.
%\end{itemize}
%
%\bigskip
%
%\emph{Idee:} (\citet{marsden_west:2001})
%\begin{equation}
%\label{eq:diskrete_lagrange_fkt_hoher_ordnung}
% L_h (q_k, q_{k+1})
% \colonequals
% \tau \sum_{i=1}^s b_i L(u(c_i \tau), \dot{u}(c_i\tau))
%\end{equation}
%\begin{itemize}
% \item Quadraturformel mit $s$ Stützstellen $c_1, \dots, c_s$, Gewichten $b_1, \dots,b_s$
%
% \item $u$ ist Polynom vom Grad höchstens $s$ mit
% \begin{itemize}
%  \item $u(0) = q_k$, \qquad $u(\tau) = q_{k+1}$
%  \item $u$ macht die rechte Seite von~\eqref{eq:diskrete_lagrange_fkt_hoher_ordnung} stationär
%    im Raum aller Polynome vom Grad höchstens $s$.
% \end{itemize}
%\end{itemize}
%
%\bigskip
%
%Tatsächlich werden von $u$ nur die Werte und Ableitungen an den Stützstellen $c_i \tau$ gebraucht.
%
%\medskip
%
%Definiere deshalb:
%\begin{equation*}
%Q_i \colonequals u(c_i \tau), \quad\quad \dot{Q}_i \colonequals \dot{u}(c_i \tau).
%\end{equation*}
%
%\medskip
%
%Die $Q_i$ können durch die $\dot{Q}_i$ ausgedrückt werden:
%\begin{align}
%\nonumber
%Q_i &= u(c_i \tau) = u(0) + \int_0^{c_i} \dot{u}(\sigma\tau)\, d\sigma\\
%\nonumber
% &=
% q_k + \tau\int_0^{c_i}\sum_{j=1}^s L_j(\sigma)\dot{u}(c_j\tau)\,d\sigma \quad\text{(Lagrange--Darstellung)}\\
%\label{eq:vi_darstellung_der_Q}
% &=
% q_k + \tau\sum_{j=1}^s a_{ij}\dot{Q}_i \quad \text{mit}\quad a_{ij} = \int_0^{c_i} L_j(\sigma)\,d\sigma.
%\end{align}
%Die $b_i$ sind Quadraturgewichte.  Wähle deshalb
%\begin{equation*}
%b_i = \int_0^1 L_i(\sigma)\,d\sigma.
%\end{equation*}
%
%\bigskip
%
%Wir wählen die $\dot{Q}_i$ so, dass der Ausdruck
%\begin{equation*}
%L_h(q_k, q_{k+1}) = \tau\sum_{i=1}^s b_iL(Q_i(\dot{Q}_1,\dots,\dot{Q}_s),\dot{Q}_i)
%\end{equation*}
%stationär wird.
%
%\medskip
%
%Allerdings brauchen wir zusätzlich die Nebenbedingung
%\begin{equation}
%\label{eq:vi_hoher_ordnung_nebenbedingung}
% q_{k+1}
% =
% u(\tau)
% =
% u(0) + \tau \int_0^1 \dot{u}(\tau \sigma)\,d\sigma
% =
% q_k+\tau\sum_{i=1}^s b_i \dot{Q}_i.
%\end{equation}
%
%
%\subsubsection{Exkurs: Stationärität unter Gleichheitsnebenbedingungen}
%
%Betrachte eine Funktion $f:\R^d \to \R$.
%\begin{itemize}
%\item Wir suchen einen stationären Punkt von $f$.
%\item D.h., einen Punkt $x$, in dem die Richtungsableitung in alle Richtungen $v$ verschwindet
%\begin{equation*}
%\frac{df}{dv} = 0 \quad\forall v\in\R^d,\,v\neq 0.
%\end{equation*}
%\begin{center}
%\begin{tikzpicture}[scale=0.8]
%\draw[->] (0.5,1)--(5,1);
%\draw[->] (1,0.5)--(1,5);
%\draw[thick, ->] (3,3)--(4,3.5);
%\draw[thick, ->] (3,3)--(3,4);
%\draw[thick, ->] (3,3)--(2,3.5);
%\draw[thick, ->] (3,3)--(2,2.5);
%\draw[thick, ->] (3,3)--(3,2);
%\draw[thick, ->] (3,3)--(4,2.5);
%\draw (3.4,3) node{$x$};
%\draw (4.5,4.5) node{$\R^d$};
%\end{tikzpicture}
%\end{center}
%
%\item D.h. $\nabla f(x) = 0$.
%\end{itemize}
%Betrachte jetzt eine weitere Funktion $g:\R^d\rightarrow\R$.
%\begin{itemize}
%\item Wir suchen ein $x$ mit $g(x)=0$, so dass $f$ in $x$ stationär ist bzgl. der Menge $M=\{y\in\R^d \; :\; g(y)=0\}$.
%\item D.h. $\frac{df}{dv}=0$ for alle Richtungen $v$, die tangential zu $M$ sind.
%\end{itemize}
%\begin{center}
%\begin{tikzpicture}[scale=0.8]
%\draw[->] (0.5,1)--(5,1);
%\draw[->] (1,0.5)--(1,5);
%\draw[thick] (3,3) circle (1);
%\draw[thick, ->] (2.3,2.3)--(1.3,3.3);
%\draw[thick, ->] (2.3,2.3)--(3.3,1.3);
%\draw (2.6,2.6) node{$x$};
%\draw (4.5,4.5) node{$\R^d$};
%\draw (6.3,2.6) node{$M = \{y : g(y) = 0\}$};
%\end{tikzpicture}
%\end{center}
%\begin{itemize}
%\item $\nabla f(x)$ ist nicht zwangsläufig null!
%\item Aber $\nabla f(x)$ steht senkrecht auf $M$.
%\item $\nabla g(x)$ steht ebenfalls senkrecht auf $M$.
%\item Gesucht werden $x\in\R^d$, $\lambda \in\R$, so dass
%\begin{equation*}
%  \nabla f(x) = \lambda \nabla g(x).
%\end{equation*}
%Solch eine Variable $\lambda$ heißt \emph{Lagrange--Multiplikator}.
%\end{itemize}
%
%\bigskip
%
%Umschreiben: Definiere die Lagrange--Funktion
%\begin{equation*}
%\mathcal{L}(x,\lambda) \colonequals f(x) - \lambda g(x).
%\end{equation*}
%\begin{itemize}
%\item Der Gradient davon ist
%\begin{equation*}
%\nabla\mathcal{L}(x,y)
%=
%\begin{pmatrix}
% \frac{\partial\mathcal{L}}{\partial x} \\
% \frac{\partial\mathcal{L}}{\partial \lambda}
%\end{pmatrix}
%=
%\begin{pmatrix}
%\nabla f(x) - \lambda \nabla g(x) \\ -g(x)
%\end{pmatrix}.
%\end{equation*}
%\item Die gesuchten Punkte sind also gerade die stationären Punkte von $\mathcal{L}$ (ohne Nebenbedingungen).
%\end{itemize}
%
%\subsubsection{Exkurs Ende}
%
%Diese Technik wenden wir auf die Nebenbedingung
%\begin{equation*}
%q_{k+1} = q_k + \tau\sum_{i=1}^s b_i\dot{Q}_i
%\end{equation*}
%an. Da diese Nebenbedingung $d$-wertig ist, ist auch der Lagrange--Multiplikator $\lambda$ aus $\R^d$.
%
%\medskip
%
%Wir suchen also stationäre Punkte von
%\begin{equation*}
%\mathcal{L}(\dot{Q}_1,\hdots,\dot{Q}_d,\lambda)
%=
%\tau \sum_{i=1}^s b_iL(Q_i,\dot{Q}_i)
% - \Big\langle \lambda, \Big( q_k - q_{k+1} + \tau\sum_{i=1}^s b_i\dot{Q}_i \Big) \Big\rangle.
%\end{equation*}
%Wir berechnen hiervon die partiellen Ableitungen nach den $\dot{Q}_j$ (die part. Ableitungen nach $\lambda$ sind klar).
%\begin{equation*}
%\frac{\partial\mathcal{L}}{\partial\dot{Q}_j}
% = \tau\sum_{i=1}^s b_i \left[\frac{\partial L}{\partial q}\cdot \frac{\partial Q_i}{\partial\dot{Q}_j}
%   +  \frac{\partial L}{\partial \dot q}\cdot \frac{\partial \dot Q_i}{\partial\dot{Q}_j}\right]
%   - \underbrace{\bigg\langle \lambda, \left( \tau\sum_{i=1}^s b_i \frac{\partial\dot{Q}_i}{\partial\dot{Q}_j}\right) \bigg\rangle}_{= \tau b_j \lambda}.
%\end{equation*}
%Da
%\begin{equation*}
%\frac{\partial Q_i}{\partial \dot{Q}_j}
%=
%\frac{\partial}{\partial \dot{Q}_j}\left( q_k+\tau \sum_{l=1}^s a_{il}\dot{Q}_l\right)
%=
%\tau a_{ij} I_{d\times d}
%\end{equation*}
%folgt
%\begin{equation*}
%\frac{\partial\mathcal{L}}{\dot{Q}_j} = \tau\sum_{i=1}^s b_i \frac{\partial L}{\partial q}\cdot \tau a_{ij} + \tau b_j\frac{\partial L}{\partial\dot{q}} - \tau b_j\lambda.
%\end{equation*}
%Stationäre Punkte von $\mathcal{L}$ erfüllen also
%\begin{equation}
% \label{eq:bedingung_stationäre_punkte}
%\sum_{i=1}^s b_i\frac{\partial L}{\partial q}(Q_i,\dot{Q}_i)\cdot\tau a_{ij} + b_j\frac{\partial L}{\partial \dot{q}}(Q_j,\dot{Q}_j) = b_j\lambda.
%\end{equation}
%Wir führen wieder die konjugierten Impulse ein:
%\begin{equation*}
%P_i = \frac{\partial L}{\partial\dot{q}}(Q_i,\dot{Q}_i),
%\end{equation*}
%und schreiben außerdem formal
%\begin{equation*}
%\dot{P}_i = \frac{\partial L}{\partial q}(Q_i,\dot{Q}_i).
%\end{equation*}
%Damit vereinfacht sich die Bedingung~\eqref{eq:bedingung_stationäre_punkte} zu
%\begin{equation}
%\label{eq:bedingung_stationäre_punkte_einfach}
%\tau\sum_{i=1}^s b_i\dot{P}_i a_{ij} + b_jP_j = b_j\lambda.
%\end{equation}
%
%\bigskip
%
%Das allgemeine variationelle Integrationsverfahren hatte die Form
%\begin{equation*}
%p_k = -\frac{\partial L_h}{\partial x}(q_k,q_{k+1}),
%\qquad
%p_{k+1} = \frac{\partial L_h}{\partial y}(q_k,q_{k+1}).
%\end{equation*}
%Das rechnen wir jetzt für den konkreten Fall aus.
%\begin{align*}
%p_k &= -\frac{\partial L_h}{\partial q_k}(q_k,q_{k+1})\\
%%
%    &= -\tau\sum_{i=1}^s b_i\frac{\partial}{\partial q_k} L(Q_i,\dot{Q}_i)\\
%%
%    &= -\tau\sum_{i=1}^s b_i\Bigg[ \underbrace{\frac{\partial}{\partial x} L(Q_i,\dot{Q}_i)}_{=\dot{P}_i} \cdot \frac{\partial Q_i}{\partial q_k}
%       +  \underbrace{\frac{\partial}{\partial y} L(Q_i,\dot{Q}_i)}_{=P_i} \cdot \frac{\partial\dot Q_i}{\partial q_k}     \Bigg] \\
%%
%    &= -\tau \sum_{i=1}^s b_i \left[\dot{P}_i\bigg( I +\tau\sum_{j=1}^s a_{ij}\frac{\partial\dot{Q}_j}{\partial q_k} \bigg) + P_i\frac{\partial\dot{Q}_i}{\partial q_k}\right]
%    \quad \text{(wg. } Q_i = q_k+\tau\sum_{j=1}^s a_{ij}\dot{Q}_j\text{)}\\
%%
%    &= -\tau \sum_{i=1}^s b_i\dot{P}_i -\tau\sum_{j=1}^s\underbrace{\tau\sum_{i=1}^s b_i\dot{P}_i a_{ij}}_{=b_j\lambda -b_jP_j}\frac{\partial\dot{Q}_j}{\partial q_k}
%      - \tau\sum_{i=1}^s b_i P_i\frac{\partial\dot{Q}_i}{\partial q_k}\\
%%
%    &= -\tau \sum_{i=1}^s b_i\dot{P}_i -\tau\sum_{j=1}^s(b_j\lambda -b_j P_j) \frac{\partial\dot{Q}_j}{\partial q_k}
%      - \tau\sum_{i=1}^s b_i P_i\frac{\partial\dot{Q}_i}{\partial q_k}\\
%%
%    &= -\tau \sum_{i=1}^s b_i\dot{P}_i - \tau\sum_{j=1}^s b_j\lambda  \frac{\partial\dot{Q}_j}{\partial q_k}.
%\end{align*}
%Differenzieren der Nebenbedingung $q_{k+1} = q_k + \tau\sum_{i=1}^s b_i\dot{Q}_i$ ergibt
%\begin{equation*}
%0 = I + \tau\sum_{i=1}^s b_i\frac{\partial \dot{Q}_i}{\partial q_k}.
%\end{equation*}
%Deshalb ist
%\begin{equation}
%\label{eq:allgemeines_vi_hoher_ordnung_konkret_1}
%p_k = -\tau\sum_{i=1}^s b_i\dot{P}_i + \lambda.
%\end{equation}
%Ganz ähnlich erhält man
%\begin{equation}
%\label{eq:allgemeines_vi_hoher_ordnung_konkret_2}
%p_{k+1} = \frac{\partial L_h}{\partial y}(q_k,q_{k+1}) = \lambda.
%\end{equation}
%\begin{lemma}
%Es gilt
%\begin{enumerate}
%\item $\displaystyle p_{k+1} = p_k + \tau\sum_{i=1}^s b_i\dot{P}_i$
%\item $\displaystyle q_{k+1} = q_k + \tau\sum_{i=1}^s b_i\dot{Q}_i$
%\item $\displaystyle P_i = p_k + \tau\sum_{j=1}^s \underbrace{( b_j - b_ja_{ji}/b_i)}_{\equalscolon \hat{a}_{ij}} \dot{P}_j$
%\item $\displaystyle Q_i = q_k + \tau\sum_{j=1}^s a_{ij}\dot{Q}_j$
%\end{enumerate}
%\end{lemma}
%
%\begin{proof}
%\begin{enumerate}
%\item ist \eqref{eq:allgemeines_vi_hoher_ordnung_konkret_1} mit \eqref{eq:allgemeines_vi_hoher_ordnung_konkret_2}
%\item ist gerade die Nebenbedingung \eqref{eq:vi_hoher_ordnung_nebenbedingung}, d.h. $u(\tau)=q_{k+1}$.
%\item Aus $1.$ und $p_{k+1}=\lambda$ folgt
%\begin{equation*}
%0 = p_k + \tau\sum_{j=1}^s b_j\dot{P}_j - \lambda.
%\end{equation*}
%Multiplizieren mit einem $b_i\ (\neq 0)$:
%\begin{equation*}
%0 = b_ip_k + \tau\sum_{j=1}^s b_ib_j\dot{P}_j - b_i\lambda.
%\end{equation*}
%Addiere $b_iP_i$ auf beiden Seiten:
%\begin{equation*}
%b_i P_i
%=
%b_ip_0 + \tau\sum_{j=1}^s b_ib_j\dot{P}_j
%  + \underbrace{b_iP_i - b_i\lambda}_{\substack{=-\tau\sum_{j=1}^sb_ja_{ji}\dot{P}_j\\\text{ (wg.~\eqref{eq:bedingung_stationäre_punkte_einfach})}}}
%\end{equation*}
%\begin{equation*}
%\implies b_iP_i = b_ip_0 + \tau\sum_{j=1}^s ( b_j - b_ja_{ji}) \dot{P}_j
%\end{equation*}
%
%\item ist \eqref{eq:vi_darstellung_der_Q} (Darstellung der Werte $Q$ über Hauptsatz und Lagrange-Darstellung).
%\qedhere
%\end{enumerate}
%\end{proof}
%
%\bigskip
%
%Die vier Gleichungen aus dem Lemma bilden ein partitioniertes Runge--Kutta-Verfahren
%$(p_k,q_k)\mapsto(p_{k+1},q_{k+1})$ für die Gleichungen
%\begin{equation*}
%\dot{p}=\frac{\partial L}{\partial q}(q,\dot{q}),
%\qquad
%p=\frac{\partial L}{\partial \dot{q}}(q,\dot q)
%\qquad
%\left(\text{bzw.\ } \frac{d}{dt}\left(\frac{\partial L}{\partial \dot q}\right) = \frac{\partial L}{\partial q}\right).
%\end{equation*}
%
%\emph{Erinnerung:} Partitionierte RK-Verfahren für ein System
%\begin{align*}
%\dot{y}=f(y,z)\quad &\dot{z}=g(y,z) \\
%y_{k+1} = y_k + \tau\sum_{i=1}^s\hat{b}_i k_i \quad &z_{k+1} = z_k + \tau\sum_{i=1}^sb_i l_i \\
% k_i = f(y_k+\tau\sum_{j=1}^s\hat{a}_{ij} k_j,z_k+\tau\sum_{j=1}^s a_{ij}l_j)
%  \quad
% &l_i = g(y_k+\tau\sum_{j=1}^s\hat{a}_{ij} k_j,z_k+\tau\sum_{j=1}^s a_{ij}l_j)
%\end{align*}
%Symmetrische Form
%\begin{align*}
%y_{k+1} = y_k + \tau\sum_{i=1}^s\hat{b}_i f(g_i,h_i)\quad &z_{k+1} = z_k + \tau\sum_{i=1}^sb_i g(g_i,h_i) \\
%g_i = y_k + \tau\sum_{i=1}^s\hat{a}_{ij} f(g_i,h_i)\quad &h_i = z_k + \tau\sum_{i=1}^sa_{ij} g(g_i,h_i)
%\end{align*}
%Wir haben ein Verfahren dieser Bauart, mit
%\begin{align*}
%\dot{P}_i=f(g_i,h_i)\quad &\dot{Q}_i=g(g_i,h_i)\\
%P_i = g_i\quad & Q_i = h_i\\
%   &\hat{b}_i = b_i.
%\end{align*}
%und insbesondere
%\begin{equation*}
% \hat{a}_{ij}
% =
% b_j - b_j a_{ji} / b_i.
%\end{equation*}
%
%\bigskip
%
%Diese letzte Relation hat eine besondere Eigenschaft:
%\begin{satz}[{\citet[Thm.\,VI.4.6]{hairer_lubich_wanner:2006}}]
% Wenn für die Koeffizienten eines partitionierten Runge--Kutta-Verfahrens gilt:
% \begin{alignat*}{2}
%  b_i \hat{a}_{ij} + \hat{b}_j a_{ji} & = b_i \hat{b}_j  & \qquad & i,j = 1,\dots,s \\
%  b_i & = \hat{b}_i   && i = 1,\dots,s,
% \end{alignat*}
% dann ist das Verfahren symplektisch.
%\end{satz}
%
%
%\section{Mechanische Systeme mit Nebenbedingungen}
%
%\begin{itemize}
%\item auch: Zwangsbedingungen
%\item auch: differential-algebraische Gleichungen (DAEs)
%\end{itemize}
%
%Betrachte System mit Positionskoordinaten $q_1, \dots, q_d$.
%\begin{itemize}
%\item Nebenbedingung: sei $g: \R^d \to \R^m$ mit $m < d$.
%\item Nur Positionen $q \in \R^d$ mit $g(q) = 0$ sind erlaubt.
%\end{itemize}
%
%Wie sehen Bewegungsgleichungen aus, wenn es solche eine Nebenbedingung gibt?
%
%Formulierung dieser Nebenbedingung wieder mit Lagrange-Multiplikator.
%
%\begin{itemize}
%\item Kinetische Energie: $T(q, \dot q) = \frac12 \dot q^T M \dot q$
%\item Potentielle Energie: $U(q)$
%\item Lagrange-Funktion
%  \begin{equation*}
%    L(q, \dot q) = T(q, \dot q) - U(q) - g(q)^T \lambda
%  \end{equation*}
%  mit Lagrange-Multiplikatoren $\lambda_1, \dots, \lambda_m$.
%\end{itemize}
%
%Stationäre Punkte von
%\begin{equation*}
%  S(q, \lambda) := \int L(q(t), \dot q(t), \lambda(t)) dt
%\end{equation*}
%erfüllen dann
%\begin{align*}
%  \frac{d}{dt}\big( \frac{\partial L}{\partial \dot q} \big) - \frac{\partial L}{\partial q} & = 0 \\
%  g(q(t)) & = 0 \quad \forall t
%\end{align*}
%
%Bewegungsgleichungen:
%\begin{equation*}
%  \frac{\partial L}{\partial \dot q} = M \dot q,
%  \qquad
%  \frac{\partial L}{\partial q} = - \nabla U - (\nabla g)^T \lambda
%\end{equation*}
%
%\begin{align*}
%  M \ddot q + \nabla U(q) + (\nabla g(q))^T \lambda & = 0 \\
%  g(q) & = 0
%\end{align*}
%
%System erster Ordnung:
%\begin{align*}
%  v & = \dot q \\
%  M \dot v & = - \nabla U(q) - (\nabla g(q))^T \lambda \\
%  g(q) & = 0
%\end{align*}
%
%Beispiel: Das Kugelpendel
%\begin{itemize}
%\item Wieder ein Fadenpendel, aber das Pendel darf sich in drei Dimensionen bewegen
%\item Beschreibung mit \emph{zwei} Winkeln:
%  Möglich, aber technisch
%\item Alternative: Kartesische Koordinaten $q_1$, $q_2$, $q_3$
%\item Nebenbedingung: Die Länge des Pendels ist fest:
%  \begin{equation*}
%    g(q) = q_1^2 + q_2^2 + q_3^2 - l^2 = 0
%  \end{equation*}
%\item Kinetische Energie:
%  \begin{equation*}
%    T = \frac m2 (\dot q_1^2 + \dot q_2^2 + \dot q 3^3)
%  \end{equation*}
%\item Potentielle Energie:
%  \begin{equation*}
%    U = m g q_3
%  \end{equation*}
%\end{itemize}
%
%Bewegungsgleichungen:
%\begin{align*}
%  v_1 = \dot q_1, \quad
%  v_2 = \dot q_2, \quad
%  v_3 = \dot q_3
%  \\
%  m \dot v_1 = -2 q_1 \lambda, \quad
%  m \dot v_2 = -2 q_2 \lambda, \quad
%  m \dot v_3 = - m g - 2 q_3 \lambda
%  \\
%  0 = q_1^2 + q_2^2 + q_3^2 - l^2
%\end{align*}
%
%Der Lagrange-Multiplikator $\lambda$ kann als die Spannung im Faden interpretiert werden.
%
%Betrachte ein mechanisches System mit Positionskoordinaten $q_1,..., q_d$.
%\begin{itemize}
%\item Nebenbedingung: Sei $g: \R^d \rightarrow \R^m$ mit $m < d$.
%\item Es sind nur solche Positionen $q \in \R^d$ erlaubt, für die $g(p) = 0$ gilt.
%\end{itemize}
%Betrachte ein mechanisches System mit
%\begin{itemize}
%\item Kinetischer Energie: $T(q,\dot{q}) = \frac{1}{2} \dot{q}^T M(q) \dot{q}$.
%\item Potenzieller Energie $U(q)$
%\end{itemize}
%Lagrange-Funktion $L(q, \dot{q}, \lambda) = T (q, \dot{q}) - U(q) - g(q)^T \lambda $ mit Lagrange--Multiplikatoren $\lambda_1, ..., \lambda_m$.
%
%Wir suchen Lösungen der Lagrange--Gleichung:
%\begin{equation*}
%\frac{d}{dt}\Bigg(\frac{\partial L}{\partial \dot{q}}\Bigg) -  \frac{\partial L}{\partial q} = 0
%\end{equation*}
%
%Dazu rechnen wir:
%\begin{align*}
%\frac{\partial L}{\partial \dot{q}} & = \frac{\partial}{\partial \dot{q}}\Bigg( \frac{1}{2}  \dot{q} M(q) \dot{q} \Bigg) = M(q) \dot{q}
%\\
%\frac{\partial}{\partial t} \Bigg(\frac{\partial L}{\partial \dot{q}}\Bigg) & = \dot{q}^T\frac{\partial M}{\partial q} \dot{q} + M(q) \ddot{q}
%\\
%\frac{\partial L}{\partial q} & = \frac{\partial }{\partial q} T(q,\dot{q}) - \frac{\partial U}{\partial q} - \frac{\partial g}{\partial q} \lambda
%\end{align*}
%
%Schreiben als System erster Ordnung:
%\begin{align*}
%\dot{q} & = v
%\\
%M(q) \dot{v} & = \underbrace{-v^T\frac{\partial M}{\partial q} v + \frac{\partial }{\partial q} T(q,v) - \frac{\partial U}{\partial q}}_{\colonequals f(q,v)} - \underbrace{\frac{\partial g^T}{\partial q}}_{\colonequals G(q)} \lambda
%\\
%0 & = g(q)
%\end{align*}
%
%Geometrische Interpretation:
%Wir betrachten jetzt die Menge
%\begin{equation*}
%M = \{q \in \R^d : g(q) = 0 \}
%\end{equation*}
%als geometrisches Objekt.
%
%Wenn $g$ hinreichend freundlich ist, dann ist $M$ eine glatte, gekrümmte, geschlossense $(d-m)$-dimensionale Fläche in $\R^d$.
%
%Der Profi sagt: $M$ ist eine $(d-m)$-dimensionale Mannigfaltigkeit.
%
%$M$ ist genau die Menge aller Werte, die $q$ annehmen darf.
%
%\begin{itemize}
%\item Man könnte jetzt also auf die Idee kommen, die Differenzialgleichung für $q$ nicht mehr \glqq in $\R^d$\grqq{} sondern \glqq auf $M$\grqq{} zu definieren.
%\item Und die restlichen Punkte $\R^d \backslash M$ komplett vergessen!
%\end{itemize}
%
%Aber: in welchem Raum leben die Geschwindigkeiten v?
%\begin{itemize}
%\item Sie leben \emph{nicht} in $M$!
%\end{itemize}
%
%\begin{definition}
%Der Tangentialraum $T_q M$ von $M$ in einem Punkt $q$ ist die Menge aller Tangentialvektoren zu $M$ im Punkt $q$.
%\begin{itemize}
%\item Ein Vektorraum! (für festes $q$)
%\end{itemize}
%\end{definition}
%
%Tangential zu $M$ heißt in unserem Fall gerade $d \perp M \mid_q \quad \Leftrightarrow \quad \nabla g(q)^T d = 0$.
%Betrachte jetzt die Nebenbedingung
%\begin{equation*}
%g(q) = 0
%\end{equation*}
%Sei $q \colonequals [-\epsilon, \epsilon] \rightarrow M$ ein Pfad in $M$. Ableiten von
%\begin{equation*}
%g(q) = 0
%\end{equation*}
%ergibt
%\begin{equation*}
%\nabla g(q)^T \dot{q} = 0
%\end{equation*}
%Ergo: $\dot{q} = v$ ist immer tangential zu $M$. Soweit galt das nur an einem festen Punkt auf $M$. Jetzt betrachten wir alle Punkte.
%
%\begin{definition}
%Das Tangentialbündel $TM$ von $M$ ist die disjunkte Vereinigung aller Tangentialräume von $M$:
%\begin{equation*}
%TM \colonequals \{ (q,v) : q \in M, v \in T_q M\}
%\end{equation*}
%\end{definition}
%
%\begin{itemize} 
%\item Kein linearer Raum!
%\item Aber eine $2(d-m)$-dimensionale Mannigfaltigkeit.
%\end{itemize}
%
%Wir können also die Bewegungsgleichungen als ein System von gewöhnlichen Differenzialgleichungen erster Ordnung \emph{auf der Mannigfaltigkeit TM} auffassen.
%
%\subsection*{Hamilton-Formulierung}
%\begin{itemize}
%\item Definiere Impulse (wie gehabt)
%\begin{equation*}
%p = \frac{\partial L}{\partial \dot{q}} = M(q) \dot{q}
%\end{equation*}
%\item Hamilton-Funktion
%\begin{align*}
%\tilde H(p,q) & \colonequals  p^T \dot{q} - L(q,\dot{q})
%\\
%& = p^T \dot{q} - \frac{1}{2} \dot{q}^T M(q) \dot{q} + U(q) + g(q)^T \lambda
%\\
%& = \underbrace{\frac{1}{2} \dot{q}^T M(q) \dot{q} + U(q)}_{= H (p,q)} + g(q)^T \lambda
%\end{align*}
%\item Hamilton-System
%\begin{align*}
%\dot{q} & = \frac{\partial H}{\partial p}
%\\
%\dot{q} & = - \frac{\partial H}{\partial q} - \nabla g(q)^T \lambda
%\\
%0 & = g(q)
%\end{align*}
%\end{itemize}
%
%Statt der Tangentialitätsbedingung $\nabla g(q)^T v = 0$ haben wir dismal
%\begin{equation*}
%\nabla g(q)^T \frac{\partial H}{\partial p} = 0
%\end{equation*}
%
%\emph{Ausblick:} Dies sind Gleichungen auf dem Kotangentialbündel $T^*M$ von $M$:
%\begin{equation*}
%T^*M \colonequals \{ (q,l) : q \in M, l \text{ ist lineares Funktional auf } T_qM\}
%\end{equation*}
%
%\begin{itemize}
%\item Die Hamilton-Funktion $H$ wird weiterhin erhalten.
%\item Die Flüsse sind weiterhin symplektisch.
%\end{itemize}
%
%\subsection*{Einfaches Verfahren erster Ordnung}
%Basis: symplektisches Euler-Verfahren
%\begin{itemize}
%\item $p$ implizit, $q$ explizit
%\end{itemize}
%\begin{align*}
%\hat p_{k+1} & = p_k - \tau \Bigg( \frac{\partial H}{\partial q} \Big( \hat p_{k+1}, q_k\Big) + \nabla g\big(q_k\big)^T \lambda_{k+1}\Bigg)
%\\
%q_{k+1} & = q_k + \tau \frac{\partial H}{\partial p} \Big(\hat p_{k+1}, q_k\Big)
%\\
%0 & = g(q_{k+1})
%\end{align*}
%
%Der neue Wert $(\hat p_{k+1}, q_{k+1})$ erfüllt $0 = g(q_{k+1})$, aber nicht $\nabla g(q_{k+1}) \frac{\partial H}{\partial p}(p_{k+1}, q_{k+1})$.
%
%Deshalb: Projektionsschritt
%\begin{align*}
%p_{k+1} & = \hat p_{k+1} - \tau \nabla g \big(q_{k+1}\big)^T \mu_{k+1}\text{  TODO Handgeschrieben steht hier eindeutig $\mu$...}
%\\
%0 & = \nabla g (q_{k+1}) \frac{\partial H}{\partial p} \Big(p_{k+1}, q_{k+1}\Big)
%\end{align*}
%\begin{itemize}
%\item Wohldefiniert für kleine $\tau$
%\item Konsistent erster Ordnung
%\item Symplektisch
%\end{itemize}
