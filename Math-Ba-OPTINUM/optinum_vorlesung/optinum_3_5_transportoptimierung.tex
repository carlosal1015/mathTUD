\section{Transportoptimierung}

\subsection{Problemstellung}

\textbf{Zur Erinnerung:}
Es gebe Erzeuger $i \in I = \menge{0, \dots, r}$ und Verbraucher $k \in K = \menge{1, \dots, s}$. Weiterhin seien die Kosten $c_{ik}$ für den Transport einer Einheit von $i$ nach $k$ sowie der Vorrat $a_i > 0$ und der Bedarf $b_k > 0$ für alle $i \in I$ und $k \in K$ bekannt. Wie ist der gesamte Transport kostenminimal zu gestalten.

Als Variablen verwenden wir die Transportmenge $x_{ik}$ von $i$ nach $k$.
\begin{equation}
\begin{alignedat}{3}
	z = \sum_{i \in I} \sum_{k \in K} c_{ik} x_{ik} \to \min \quad \bei \quad 
	&\sum_{k \in K} x_{ik} &= a_i \quad &(i \in I) \\
	&\sum_{i \in I} x_{ik} &= b_k \quad &(k \in K) \\
	& x_{ik} &\ge 0 \quad &(i,k) \in I \times K \\
\end{alignedat}
\label{eq: 3.13}
\end{equation}

Mit   
\begin{equation*}
	\begin{aligned}
		x &= \transpose{x_{11}, x_{12}, \dots, x_{1s}, x_{21}, \dots, x_{rs}} \\
		c &= \transpose{c_{11}, c_{12}, \dots, c_{1s}, c_{21}, \dots, c_{rs}} \\
		\quer{b} &= \transpose{a_1, \dots, a_r, b_1, \dots, b_s}
	\end{aligned}
\end{equation*}
hat \eqref{eq: 3.13} die Form
\begin{equation*}
	z = \trans{c} x \to \min \bei Ax = \quer{b}, x \ge 0
\end{equation*}

\begin{bemerkung} %3.6
	Das Transportproblem ist eine sehr spezielle Optimierungsaufgabe. Die Koeffizientenmatrix 
	\begin{equation*}
		A = 
		\left( \begin{array}{cccc|cccc|cccc}
		1 & 1 & \cdots & 1 &   &   &        &   &   &   &        &\\
		  &   &        &   & 1 & 1 & \cdots & 1 &   &   &        &\\
		  &   &        &   &   &   &        &   & 1 & 1 & \cdots & 1  \\
		\hline
		1 &   &        &   & 1 &   &        &   & 1 &   &        &  \\
		  & \multicolumn{2}{c}{\ddots} & & & \multicolumn{2}{c}{\ddots} & & & \multicolumn{2}{c}{\ddots} & \\
		  &   &        & 1 &   &   &        & 1 &   &   &        & 1  \\	  
		\end{array} \right) \in \R^{(r+s) \times (r*s)}
	\end{equation*}
	ist schwach besetzt. Insbesondere hat die Spalte von $A$, die zur Variablen $x_{ik}$ gehört, die Gestalt $A^{ik} = \begin{psmallmatrix} e^i \\ e^k \end{psmallmatrix} \in \R^{r+s}$.
\end{bemerkung}

\begin{satz} %3.13
	Das Transportproblem ist genau dann lösbar, wenn die Sättigungsbedingung
	\begin{equation}
		\sum_{i \in I} a_i = \sum_{k \in K} b_k
		\label{eq: 3.15}
	\end{equation}
	gilt.
\end{satz}
\begin{proof}
	Wir zeigen zuerst, dass \eqref{eq: 3.15}  äquivalent zu $G \neq \emptyset$ ist.
	\begin{itemize}
		\item Einerseits folgt aus $x \in G \neq \emptyset$ durch Summation der Gleichungsnebenbedingungen
		\begin{equation*}
			\sum_{i \in I} a_i = \sum_{i \in I} \sum_{k \in K} x_{ik} = \sum_{k \in K} \sum_{i \in I} x_{ik} = \sum_{k \in K} b_k
		\end{equation*}
		\item  Gilt hingegen \eqref{eq: 3.15}, so ist mit $\sigma \defeq \sum_{i \in I} a_i = \sum_{k \in K} b_k$ ein zulässiger Punkt $x = (x_{ik})$ wie folgt definiert:
		\begin{equation*}
			x_{ik} \defeq \frac{a_i b_k}{\sigma}
		\end{equation*}
	\end{itemize}
	Der zulässige Bereich $G$ ist polyedrisch (und damit abgeschlossen) und ferner wegen $0 \le x_{ik} \le \min\menge{a_i, b_k}$ beschränkt und somit kompakt.
	Mit dem Satz von Weierstraß\footnote{Die Zielfunktion ist linear, d.h. stetig, auf einer komapkten Menge $G$.} folgt dann die Lösbarkeit des Transportproblems.
\end{proof}

Die Systemmatrix $A$ besitzt für praxisrelevante Problemgrößen eine sehr große Anzahl an Einträgen, sodass die Anwendung des Simplexverfahrens im Allgemeinen nicht empfehlenswert ist; insbesondere deshalb, weil dieses die Struktur von $A$ nicht mit einbezieht.

Zur Lösung des Transportproblems hat sich daher ein Verfahren etabliert, das auch die duale Aufgabe 
\begin{equation}
	w \defeq \trans{a} u + \trans{b} v \to \max \bei \trans{A} \begin{psmallmatrix} u \\ v \end{psmallmatrix} \le c, u \in \R^r, v \in \R^s
	\label{eq: 3.16}
\end{equation}
bzw.
\begin{equation}
	w \defeq \sum_{i \in I} a_i u_i + \sum_{k \in K} b_k v_k \to \max \bei u_i + v_k \le c_{ik} \quad (u_i, v_k \in \R, i \in I, k \in K)
	\label{eq: 3.17}
\end{equation}
löst.

\begin{satz}[Optimalitätskriterium] %3.14
	\label{satz: 3.14}
	Sei $x \in G$, d.h. $x$ ist zulässiger Transportplan, dann gilt
	\begin{equation*}
		\begin{aligned}
			x \text{ optimal} \equivalent \exists u \in \R^r, v \in \R^s \mit u_i + v_k \le c_{ik}, x_{ik} * (c_{ik} - u_i - v_k) = 0 \quad \forall i \in I, k \in K
		\end{aligned}
	\end{equation*}
\end{satz}
\begin{proof}
	Nach dem Charakterisierungssatz gilt: $x \in G$ ist genau dann optimal, wenn duale Variablen $u \in \R^r$ und $v \in \R^s$ existieren, sodass $(u,v)$ dual zulässig ist und die Komplementaritätsbedingungen gelten.
\end{proof}

\begin{aussage} %3.15
	Der Rang von $A$ ist $\card{I} + \card{K} - 1 = r+s-1$.
\end{aussage}
\begin{proof}
	Einerseits sind die Spalten $A^{11}, \dots, A^{1s}, A^{21}, A^{31}, \dots, A^{r1}$ von $A$ linear unabhängig, d.h. $\rg(A) \ge r + s - 1$.
	Andererseits ist die Summe der ersten $r$ Zeilen identisch mit der Summe der letzten $s$ Zeilen. Somit ist der Rang von oben beschränkt mit $\rg(A) \le r + s -1$.
\end{proof}

\begin{folgerung} %3.16
	Jede Ecke des zulässigen Bereichs $G$ hat höchstens $r+s-1$ positive Komponenten.
\end{folgerung}

\begin{definition} %3.4
	Eine Folge von Zellen (Indexpaaren) $(i_1, k_1)$, $(i_2, k_1)$, $(i_2, k_2)$ $\dots$ $(i_\ell, k_\ell)$, $(i_1, k_\ell)$, $(i_1, k_1)$ mit $i_\nu \neq i_\mu$ und $k_\nu \neq k_\mu$ für $\nu \neq \mu$ heißt \begriff{Zyklus} (der Länge $2 \ell$).
\end{definition}

\begin{beispiel} %3.5
	Im folgenden Schema ist ein Zyklus der Länge $2 \ell = 8$ abgebildet:
	\begin{center}
		\begin{tabular}{l|ccccc}
			$i / k$ & 1 & 2 & 3 & 4 & 5 \\ \hline
			1       & $\ast$ & & $\ast$ & & \\
			2 & & $\ast$ & & & $\ast$ \\
			3 & $\ast$ & & & & $\ast$ \\
			4 & & $\ast$ & $\ast$ & & 
		\end{tabular}
	\end{center}
	Dieses Beispiel spielt eine wichtige Rolle bei der Feststellung, ob ein gegebener Transportplan eine Ecke von $G$ ist.
\end{beispiel}

\begin{aussage} %3.17
	\begin{enumerate}[label=(\roman*), nolistsep, topsep=-\parskip]
		\item Sei $J$ eine Menge von Zellen. Gilt $\card{J} \ge r+s$, so enthält $J$ mindestens einen Zyklus.
		\item Sei $x = (x_{ik})$ ein zulässiger Transportplan. $x$ ist genau dann eine Ecke von $G$, wenn $J_+ \defeq \menge{(i,k) : x_{ik} > 0}$ keinen Zyklus enthält.
	\end{enumerate}
\end{aussage}
\begin{proof}
	vielleicht in der Übung --- oder auch nicht.
\end{proof}

\subsection{Erzeugung eines ersten Transportplans}
Dieser Teil entspricht der ersten Phase des Simplexverfahrens, d.h. also der Bestimmung einer Startecke.
Gemäß der vorherigen Beobachtungen genügt es einen zyklenfreien zulässigen Transportplan zu finden. Hierfür können unterschiedliche Methoden genutzt werden.

\begin{description}
	\item[Nordwest-Ecken-Regel:] Die jeweilige noch nicht belegte Nordwest-Zelle wird mit maximaler Transportmenge belegt.
	\item[Regel der minimalen Kosten:] In jedem Schritt wird eine noch nicht belegte Zelle, die minimale Kosten hat, mit maximaler Transportmenge belegt.
	\item[Methode von Vogel:] Bestimme in jeder Zeile und Spalte die Differenz der zwei kleinsten Kostenkoeffizienten der noch freien Zellen. Wähle dann eine Zeile/Spalte mit maximaler Differenz und belege die Zelle mit kleinsten Kosten.
\end{description}

Darstellung der Inputdaten oder zulässigen Punkte in folgenden Schemata:

\begin{minipage}{\dimexpr0.5\linewidth-\fboxrule-\fboxsep}
	\centering
	\begin{tabular}{c|cccc}
		C & $b_1$ & $b_2$ & $\dots$ & $b_s$ \\ \hline
		$a_1$ & $c_{11}$ & $c_{12}$ & $\dots$ & $c_{1s}$ \\
		$\vdots$  & $\vdots$ & $\vdots$ & $\ddots$ & $\vdots$ \\
		$a_r$ & $c_{r1}$ & $c_{r2}$ & $\dots$ & $c_{rs}$
	\end{tabular}
	\captionof{table}{Inputdaten}
\end{minipage}
\begin{minipage}{\dimexpr0.5\linewidth-\fboxrule-\fboxsep}
	\centering
	\begin{tabular}{c|cccc}
		X & $b_1$ & $b_2$ & $\dots$ & $b_s$ \\ \hline
		$a_1$ & $x_{11}$ & $x_{12}$ & $\dots$ & $x_{1s}$ \\
		$\vdots$  & $\vdots$ & $\vdots$ & $\ddots$ & $\vdots$ \\
		$a_r$ & $x_{r1}$ & $x_{r2}$ & $\dots$ & $x_{rs}$
	\end{tabular}
	\captionof{table}{Transportplan}
\end{minipage}




\begin{beispiel} %3.6
	\label{beispie: 3.6}
	Gegeben Sei das folgende Transportproblem:
	\begin{center}
		\begin{tabular}{r|rrrrr}
			C & 12 &  5 &  6 &  7 &  7 \\ \hline
			4 & 12 &  6 & 10 &  9 &  5 \\
			19 & 10 & 16 & 17 &  3 &  7 \\
			14 &  4 & 11 &  5 &  8 & 10
		\end{tabular}
	\end{center}
	
	Man erhält folgende Startecken:
	\begin{center}
		\begin{tabular}{r|rrrrr}
			$\text{X}_{\text{NW}}$ & \cancel{12} \cancel{8} 0 &  \cancel{5} 0 &  \cancel{6} 0 &  \cancel{7} 0 &  \cancel{7} 9 \\ \hline
			0 &  4 &  0 &  0 &  0 &  0 \\
			0 \cancel{6} \cancel{11} \cancel{19} &  8 & 5 & 6 & 0 & 0 \\
			0 \cancel{7} \cancel{14} &  0 & 0 &  0 &  7 & 7
		\end{tabular}
	\end{center}

	\begin{center}
		\begin{tabular}{r|rrrrr}
			$\text{X}_{\text{NW}}$ & \cancel{12} \cancel{8} 0 &  \cancel{5} 0 &  \cancel{6} 0 &  \cancel{7} 0 &  \cancel{7} 9 \\ \hline
			0 &  4 &  0 &  0 &  0 &  0 \\
			0 \cancel{6} \cancel{11} \cancel{19} &  8 & 5 & 6 & 0 & 0 \\
			0 \cancel{7} \cancel{14} &  0 & 0 &  0 &  7 & 7
		\end{tabular}
	\end{center}

	Zielfunktionswert: $z(x_{\text{NW}}) = 4 * 12 + 8 * 10 + 5 * 16 + 6 * 17 + 7 * 8 + 7 * 10 = 436$
	
	\vspace{\parskip}
	
	\begin{center}
		\begin{tabular}{r|rrrrr}
			$\text{X}_{\text{MK}}$ & 12 &  5 &  6 & \cancel{7} 0 &  \cancel{7} 3 \\ \hline
			0 \cancel{4} & 0 &  0 & 0 &  0 &  $\fbox{4}^3$ \\
			0 \cancel{12} \cancel{19} & 0 & $\fbox{5}$ & $\fbox{4}$ &  $\fbox{7}^1$ &  $\fbox{3}$ \\
			0 \cancel{2} \cancel{14} &  $\fbox{12}^2$ & 0 &  $\fbox{2}^4$ &  0 & 0
		\end{tabular}
	\end{center}

	Zielfunktionswert: $z(x_{\text{MK}}) = 268$

	\vspace{\parskip}
	
	\begin{center}
		\begin{tabular}{r|rrrrr}
			X & 12 &  5 &  6 &  7 &  7 \\ \hline
			4 &  &  4 &  &   &   \\
			19 &  & 1 & 4 &  7 &  7 \\
			14 &  12 &  &  2 &   & 
		\end{tabular}
	\end{center}
	Zielfunktionswert: $z(x_{\text{V}}) = 236$
\end{beispiel}

Je nach Qualität der Startlösung können unterschiedlich viele Iterationen des Transportalgorithmus vonnöten sein.

\subsection{Der Transportalgorithmus}

Ausgehend von einer Startlösung berechnet der Algorithmus zunächst ein Paar $(u,v)$ dualer Variablen und prüft dann die Optimalität mit \cref{satz: 3.14}. Liegt keine Optimalität vor, wird ein neuer Plan erzeugt.

\textbf{Vorgehensweise:}

\begin{enumerate}
	\item Bestimme einen zulässigen, zyklenfreien Transportplan $X_0$ mit genau $r+s-1$ markierten Basiszellen. (Diese bilden dann die Menge $J_B = J_B(X_0)$.)
	\item Bestimme für den aktuellen Plan $X$ die zugehörigen dualen Variablen $u_i$ und $v_k$ aus dem Gleichungssystem 
	\begin{equation}
		u_i + v_k = c_{ik} \qquad (i,k) \in J_B = J_B(X)
		\label{eq: 3.18}
	\end{equation}
	\item Berechne für alle $(i,k) \notin J_B$ die Koeffizienten $w_{ik} = c_{ik} - u_i - v_k$. Falls $w_{ik} \ge 0$ für alle Zellen ist, dann ist $X$ optimal. Andernfalls wähle man eine Zelle $(p,q)$ mit $w_{pq} = \min\menge{w_{ik} : i \in I, k \in K} < 0$.
	\item Markiere $(p,q)$ im Schema von $X$, bestimme den  (eindeutigen) Zyklus $J_{pq}$ in $J_B \cup \menge{(p,q)}$ und markiere abwechselnd die Zellen in $J_{pq}$ mit ''+`` und ''-``. Sei $J_{pq}^-$ die Menge der mit ''-`` gekennzeichneten Zellen.
	\item Ermittle $\delta = x_{gh} \defeq \min\menge{x_{ik} : (i,k) \in J_{pq}^-}$ und aktualisiere den Plan $X$ gemäß
	\begin{equation*}
		X^{\text{neu}} \defeq \brackets{x_{ik}^{\text{neu}}} \quad \mit \quad  
		x_{ik}^{\text{neu}} \defeq \begin{cases}
		x_{ik} + \delta & \text{ falls }(i,k) \in \brackets{J_{pq} \cup \menge{(g,h)}} \setminus J_{pq}^-\\
		x_{ik} - \delta & \text{ falls }(i,k) \in J_{pq}^-\\
		x_{ik} & \text{ sonst}
		\end{cases}
	\end{equation*}
	Aktualisere die Menge der Basiszellen
	\begin{equation*}
		J_B^{\text{neu}} \defeq \brackets{J_B \cup \menge{(p,q)}} \setminus \menge{(g,h)}
	\end{equation*}
	und gehe zu Schritt 2.
\end{enumerate}

\begin{bemerkung} %3.7
	Aufgrund von $\card{J_B} = r + s - 1$ ist das in Schritt 2 zu lösende Gleichungssystem (zur Ermittlung von $u$ und $v$) unterbestimmt. Eine der Variablen kann also beliebig festgelegt werden. Die in Schritt 3 bestimmten Werte $w_{ik}$ sind jedoch unabhängig von dieser Wahl.
\end{bemerkung}

Im gesamten Algorithmus gilt stets $w_{ik} = 0$ für die aktuellen Basiszellen $(i,k) \in J_B$ (per Konstruktion in Schritt 2), diese beeinflussen den Optimalitätstest in Schritt 3 also nicht. Daher kann zur Darstellung das folgende komprimierte Schema genutzt werden:

\begin{center}
	\begin{tabular}{c|cccc}
		T & $v_1$ & $v_2$ & $\dots$ & $v_s$ \\ \hline
		$u_1$ & \fbox{$x_{11}$} & $w_{12}$  & $\dots$ & $w_{1s}$ \\
		$u_2$ & $w_{21}$ & $w_{22}$ & \fbox{$x_{23}$} & $w_{2s}$ \\
		$\vdots$ &  $\vdots$ & $\vdots$ & $\vdots$ & $\vdots$ \\
		$u_r$ & $w_{r1}$ & \fbox{$x_{r2}$} & $\dots$ & $w_{rs}$ 
	\end{tabular}
\end{center}


\begin{beispiel}[Fortsetzung von \cref{beispie: 3.6}] %3.7
	Gegeben Sei das Problem
	\begin{center}
		\begin{tabular}{r|rrrrr}
			C & 12 &  5 &  6 &  7 &  7 \\ \hline
			4 & 12 &  6 & 10 &  9 &  5 \\
			19 & 10 & 16 & 17 &  3 &  7 \\
			14 &  4 & 11 &  5 &  8 & 10
		\end{tabular}
	\end{center}
	Mit der Minimale-Kosten-Regel haben wir bereits den Plan $X_0$ bestimmt:
	\begin{center}
		\begin{tabular}{r|rrrrr}
			$X_0$ & 12 &  5 &  6 & 7 & 7 \\ \hline
			4 & 0 &  0 & 0 &  0 &  \fbox{4} \\
			19 & 0 & \fbox{5} & \fbox{4} &  \fbox{7} &  \fbox{3} \\
			14 &  \fbox{12} & 0 &  \fbox{2} &  0 & 0
		\end{tabular}
	\end{center}
	Bestimmung der dualen Variablen (Potenziale) $u_i$ und $v_k$
	\begin{center}
		\begin{tabular}{l|ccccc}
			$T_0$ & $v_1=16$ & $v_2 = 16$ & $v_3=17$ & $v_4=3$ & $v_5=7$ \\ \hline
			$u_1 = -2$ & $-2$ & $-8$  & $-5$ & $8$ & \fbox{$4$} \\
			$u_2 = 0$ & $-6$ & \fcolorbox{black}{cdgray!10}{$5$} & \fbox{$4$} &\fbox{$7$} & \fbox{$3$} \\
			$u_3 = \fcolorbox{cdgreen!10}{cdgreen!10}{-12}$ &  \fbox{$12$} & $7$ & \fbox{$2$} & $17$ & $15$ \\
		\end{tabular}
	\end{center}
	\fcolorbox{black}{cdgray!10}{$5$} : $\underbrace{u_2}_{=0} + v_2 = c_{22} = 16 \follows v_2 = 16$
	
	\fcolorbox{cdgreen!10}{cdgreen!10}{$-12$} : $u_3 + \underbrace{v_3}_{=17} = c_{33} = 5 \follows u_3 = c_{33} - 17 = -12$
	
	Dieser Plan ist nicht optimal, da negative Einträge $w_{ik}$ existieren.
	Wir wählen also den eindeutig bestimmten Zyklus $-8 \to 4 \to 3 \to 5 \to 8$ und geben alternierende ''Vorzeichen``, d.h. $-8^+ \to 4^- \to 3^+ \to 5^- \to 8^+$. 	
	Wir wählen das kleinste mit einem ''$-$`` markierte Zellenelement des Zyklus: $\delta = \min\menge{x_{ik} : (i,k) \in J_{pq}^-} = 4$.
	
	Ein neuer Plan ergibt sich nun mit
	\begin{center}
		\begin{tabular}{l|ccccc}
			$X_1$ & \\ \hline
			&  & \fcolorbox{black}{cdgray!10}{$4$}  & & & $0$ \\
			&  & \fbox{$1$} & \fbox{$4$} &\fbox{$7$} & \fbox{$7$} \\
			&  \fbox{$12$} &  & \fbox{$2$} & &  \\
		\end{tabular}
	\end{center}

	Achtung: $x_{21} = \fcolorbox{black}{cdgray!10}{4}$ war vorher nicht in der Basis (also $x_{21} = 0$) und somit $x_{21}^{\text{neu}} = x_{21} + \delta = 4$.
	
	Bestimmung der Potenziale $u_i$ und $v_k$ für $X_1$:
	\begin{center}
		\begin{tabular}{l|ccccc}
			$T_1$ & $v_1=16$ & $v_2 = 16$ & $v_3=17$ & $v_4=3$ & $v_5=7$ \\ \hline
			$u_1 = -10$ & $6$ & \fbox{$4$}  & $4$ & $16$ & $8$ \\
			$u_2 = 0$ & $-6$ & \fbox{$1$} & \fbox{$4$} &\fbox{$7$} & \fbox{$7$} \\
			$u_3 = -12$ &  \fbox{$12$} & $7$ & \fbox{$2$} & $17$ & $15$ \\
		\end{tabular}
	\end{center}
	Auch dieses Tableau ist noch nicht optimal. Wir erkennen den Zyklus $-6^+ \to 12^- \to 2^+ \to 4^-$.
	
	Ein neuer Plan ergibt sich zu
	\begin{center}
		\begin{tabular}{l|ccccc}
			$X_2$ & \\ \hline
			&  & \fbox{$4$}  & & &  \\
			& \fbox{$4$} & \fbox{$1$} & &\fbox{$7$} & \fbox{$7$} \\
			&  \fbox{$8$} &  & \fbox{$6$} & &  \\
		\end{tabular}
	\end{center}
	mit den Potenzialen
	\begin{center}
		\begin{tabular}{l|ccccc}
			$T_2$ & $v_1=10$ & $v_2 = 16$ & $v_3=11$ & $v_4=3$ & $v_5=7$ \\ \hline
			$u_1 = -10$ & $12$ & \fbox{$4$}  & $9$ & $16$ & $8$ \\
			$u_2 = 0$ & \fbox{$4$} & \fbox{$1$} & $6$ &\fbox{$7$} & \fbox{$7$} \\
			$u_3 = -6$ & \fbox{$8$} & $1$ & \fbox{$6$} & $11$ & $9$ \\
		\end{tabular}
	\end{center}

	Alle $w_{ik} \ge 0$, d.h. der das Tableau ist optimal und $X_2$ ist eine Lösung der gegebenen Optimierungsaufgabe. Es gilt $z(X_2) = 212$.
\end{beispiel}