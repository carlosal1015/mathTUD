\begin{exercisePage}
%
% Aufgabe 1
\begin{exercise}
	Untersuchen Sie in den folgenden Teilaufgaben jeweils, ob eine Umgebung vom $x=1$ existiert, in der die angegebene Funktion $f = f(x)$ Lipschitz-stetig ist. In welchen Teilaufgaben kann der Satz von Picard und Lindelöf angewendet werden, um zu folgern, dass das Anfangswertproblem
	\begin{equation*}
		x' = f(x) \qquad x(0) = 1
	\end{equation*}
	eine eindeutige Lösung besitzt?
	\begin{enumerate}[nolistsep]
		\item $f(x) = 1 - x^2$
		\item $f(x) = \abs{1 - x^2}$
		\item $f(x) = \sqrt{\abs{ 1 - x^2}}$
	\end{enumerate}
\end{exercise}

\begin{enumerate}[label=(zu \alph*), leftmargin=*]
	\item Sei $\epsilon > 0$ und $x,y, \in B_\epsilon(1)$. Definiere $L \defeq 2\epsilon < \infty$. Dann gilt
	\begin{equation*}
		\begin{aligned}
			\abs{f(x) - f(y)} 
			= \abs{1-x^2 - (1-y^2)} 
			= \abs{y^2 - x^2} 
			= \abs{(y-x)(y+x)} 
			&= \abs{x+y} * \abs{x-y} \\
			&\le L * \abs{x-y}
		\end{aligned}
	\end{equation*}
	Damit ist $f$ lokal Lipschitz-stetig in $x_0 = 1$. Damit liefert der Satz von Picard-Lindelöf die Existenz einer eindeutigen, bis zum Rand fortsetzbaren Lösung $x$.
	%
	\item Sei wiederum $\epsilon > 0$ und $x,y, \in B_\epsilon(1)$. Definiere $L \defeq 2\epsilon < \infty$, sodass schließlich gilt
	\begin{equation*}
		\begin{aligned}
			\abs{f(x) - f(y)} = \abs{\abs{1-x^2} - \abs{1-y^2}} \le \abs{1 + x^2 - 1 - y^2} = \abs{x^2 - y^2} \le L * \abs{x-y}
		\end{aligned}
	\end{equation*}
	Damit ist $f$ lokal Lipschitz-stetig in $x_0 = 1$. Nach Picard-Lindelöf existiert eine eindeutige Lösung, die bis zum Rand fortsetzbar ist.
	%
	\item Sei $f$ lokal Lipschitz-stetig in $x_0 = 1$. Dann existieren $\epsilon > 0$ und $0 < L < \infty$ mit
	\begin{equation*}
		\abs{f(x) - f(y)} \le L * \abs{x-y} 
		\tag{$\star$} \label{eq: uebung1c}
	\end{equation*}
	für alle $x,y \in B_\epsilon(1)$.
	Sei nun $\epsilon > 0$ beliebig, dann existiert ein $n \in \N_+$ sodass $x_n \defeq 1 + \frac{1}{n} \in B_\epsilon(1)$. Setze $y \defeq 1$. Es ist $x_n \neq y$, d.h. $\abs{x_n - y} \neq 0$ für alle $n \in \N_+$. Somit gilt mit \eqref{eq: uebung1c} 
	\begin{equation*}
		\begin{aligned}
			L \ge \frac{\abs{f(x_n) - f(y)}}{\abs{x_n - y}}
			= \frac{\abs{\sqrt{\abs{1 - \brackets{1+\frac{1}{n}}^2}}}}{\abs{1 + \frac{1}{n} - 1}}
			&= \frac{\sqrt{\abs{1 - 1 - \frac{2}{n} - \frac{1}{n^2}}}}{\frac{1}{n}} \\
			&= n * \sqrt{\frac{2n + 1}{n^2}} \\
			&= \sqrt{2n + 1} \enskip \longrightarrow \infty \qquad (n \to \infty)
		\end{aligned}
	\end{equation*}
	Dies steht jedoch im Widerspruch zur Existenz einer gültigen Lipschitzkonstante. Somit ist $f$ nicht lokal Lipschitz-stetig in $x_0 = 1$ und der Satz von Picard-Lindelöf kann nicht angewendet werden.
\end{enumerate}




\begin{exercise}
	Sei $\abb{\Phi}{\R \times \Rd}{\Rd}, (t,x) \mapsto \Phi^t x$ ein Phasenfluss, d.h. $\Phi$ erfülle die Eigenschaften
	\begin{enumerate}[label=(\roman*), nolistsep, topsep=-\parskip]
		\item $\Phi(0,x) = x$ bzw. $\Phi^0 = \id$
		\item $\Phi(s+t,x) = \Phi(s,\Phi(t,x))$ bzw. $\Phi^{s+t} x = \Phi^s \Phi^t x$
	\end{enumerate}
	Sei $\Phi$ ferner in $t$ und $x$ stetig differenzierbar und sei $f(x) \defeq \partial_t \Phi(0,x)$. Zeigen Sie, dass $x(t) \defeq \Phi(t,x_0)$ das autonome Anfangswertproblem
	\begin{equation*}
		x'(t) = f(x(t)) \qquad x(0) = x_0
	\end{equation*}
	zum Anfangswert $x_0 \in \Rd$ erfüllt.
\end{exercise}

\begin{description}
	\item[Differentialgleichung:] Es gilt
	\begin{equation*}
			f(x(t)) = \partial_t \Phi(0,x) = \partial_t \Phi(0, \Phi(t,x_0)) \overset{\text{(ii)}}{=} \partial_t \Phi(0+t, x_0) = \partial_t \Phi(t,x_0) = x'(t)
	\end{equation*}
	\item[Anfangswert:]
	\begin{equation*}
		x(0) = \Phi(0,x_0) \overset{\text{(i)}}{=} x_0
	\end{equation*}
\end{description}

\end{exercisePage}