\section{Die Methode Branch \& Bound}

Branch \& Bound (B\&B) ist eine sehr flexible Technik, um exakte Lösungsverfahren für Probleme der diskreten Optimierung zu entwickeln. Anschaulich betrachtet wird dabei eine schwierige Optimierungsaufgabe sukzessiv in Teilprobleme zerlegt, die wiederum \enquote{leicht} (näherungsweise) gelöst werden können und somit zur Lösung des Gesamtproblems beitragen. Näherungslösungen erhält man dabei oftmals mithilfe geeigneter Relaxationen.

\subsection{Grundlagen}

Wir betrachten das Anfangsproblem
\begin{equation*}
	f(x) \to \min \bei x \in E \cap D
	\tag{$P_0$}
	\label{eq: p_0}
\end{equation*}
und eine zugehörige Relaxation
\begin{equation*}
	g(x) \to \min \bei x \in E
	\tag{Q}
	\label{eq: q}
\end{equation*}
wobei $g(x) \le f(x)$ auf $D \cap E$ gilt.

\vspace{\parskip}
\fbox{\textbf{Prinzip der B\&B-Methode}}

Die Menge $E$ wird durch Separation in Teilmengen $E_i$ mit $i \in I$ zerlegt. Dadurch entstehen \begriff{Teilprobleme} 
\begin{equation*}
	f(x) \to \min \bei x \in D \cap E_i
	\tag{$P_i$}
	\label{eq: p_i}
\end{equation*}
Jedem dieser Teilprobleme \eqref{eq: p_i} soll nun eine Zahl $b(P_i)$, genannt \begriff{untere Schranke}, zugeordnet werden, sodass gilt
\begin{enumerate}[label=(\alph*), nolistsep]
	\item $b(P_i) \le \min\menge{f(x) : x \in D \cap E_i}$
	\item $b(P_i) = f(\dach{x})$ falls $D \cap E = \menge{\dach{x}}$
	\item $b(P_i) \le b(P_j)$ falls $E_j \subset E_i$
\end{enumerate}

Eine geeignete Möglichkeit besteht darin, z.B. die stetige Relaxation der Teilprobleme \eqref{eq: p_i} zu betrachten, d.h. 
\begin{equation*}
	b(P_i) \defeq \begin{cases}
	\min\menge{g(x) : x \in E_i} & \falls \card{E_i \cap D} > 1 \\
	f(\dach{x}) & \falls \card{E_i \cap D} = 1 \\
	+ \infty & \falls E_i \cap D = \emptyset
	\end{cases}
\end{equation*}

\pagebreak

\subsection{Allgemeiner B\&B-Algorithmus}
Bezeichne mit $R$ die Menge der noch zu bearbeitenden Teilprobleme (''Restmenge``) und mit $\quer{z}$ den Zielfunktionswert der bisher besten gefundenen zulässigen Lösung $\quer{x} \in D \cap E$.

\begin{enumerate}[label=\underline{\textbf{Schritt \arabic*:}}, leftmargin=*]
	\setcounter{enumi}{-1}
	\item \textbf{Initialisierung} --- Bestimme $b(P_0)$.
	\begin{enumerate}[label=(\alph*), noitemsep]
		\item Falls $\quer{x} \in D \cap E$ bekannt ist mit $f(\quer{x}) = b(P_0)$, dann \texttt{STOP}.
		\item Setze $R \defeq \menge{P_0}$ und $\quer{z} \defeq + \infty$ oder $\quer{z} = f(x)$, wenn ein $x \in D \cap E$ bekannt ist.
	\end{enumerate}	
	\item \textbf{Abbruchtest} --- Falls $R \neq \emptyset$, dann \texttt{STOP}. Falls $\quer{z} = +\infty$, dann ist \eqref{eq: p_0} nicht lösbar (leerer zulässiger Bereich), andernfalls ist $\quer{x}$ Lösung von \eqref{eq: p_0}
	\item \textbf{Strategie} --- Wähle entsprechend einer Auswahlstrategie ein $P_i \in R$ und setze $R \defeq R \setminus \menge{P_i}$.
	\item \textbf{Zerlegung (''branch``)} --- Zerlege $P_i$ durch Separation in endlich viele Teilprobleme $P_{i,1}, \dots, P_{i,k_i}$. Setze $j \defeq 1$.
	\item \textbf{Schranken- und Dominanztests (''bound``)}
	\begin{enumerate}[label=(\alph*), noitemsep]
		\item Berechne $b(P_{i,j})$. Falls dabei ein $\schlange{x} \in D \cap E$ gefunden wurde mit $f(\schlange{x}) < \quer{z}$, setze $\quer{x} \defeq \schlange{x}$ und $\quer{z} \defeq f(\schlange{x})$.
		\item Falls $b(P_{i,j}) < \quer{z}$, dann setze $R \defeq R \cup \menge{P_{i,j}}$. Falls $j < k_i$, setze $j \defeq j + 1$ und gehe zu (a).
		\item Setzte $R \defeq R \setminus \menge{P_k}$ für alle $P_k \in R$ mit $b(P_k) \ge \quer{z}$. 
	\end{enumerate}	
	Gehe zu Schritt 1.
\end{enumerate}

\begin{*bemerkung}
	\begin{enumerate}
		\item Die Endlichkeit des Verfahrens ist zu sichern, z.B. durch $\card{E_{i,j} \cap D} \le \card{E_i \cap D}$ für alle $j$ (falls $E_i \cap D$ endlich ist) oder durch $b(P_{i,j}) > b(P_i) + \epsilon$ mit $\epsilon > 0$ für alle $j$ und $i$.
		\item Das B\&B-Verfahren kann mithilfe eines Verzweigungsbaumes veranschaulicht werden.
		\item In Schritt 2 können verschiedene Auswahlstrategien gewählt werden, z.B. 
		\begin{itemize}[noitemsep]
			\item \begriff{Minimalsuche} (best bound search): wähle $P_i \in R$ mit $b(P_i) \le b(P_k)$ für alle $P_k \in R$.
			\item \begriff{Tiefensuche} (depth-first search, LIFO): wähle $P_i \in R$ mit kleinstem Schrankenwert unter allen Teilproblemen mit maximaler Verzweigungstiefe.
			\item \begriff{Breitensuche} (breadth-first-search, FIFO): wähle $P_i \in R$ mit kleinstem Schrankenwert unter allen Teilproblemen mit minimaler Verzweigungstiefe.
		\end{itemize}
	\end{enumerate}
\end{*bemerkung}

\subsection{Beispiele für B\&B-Verfahren}

\subsubsection{Das 0/1-Rucksackproblem}

Zur Erinnerung: Gegeben seien ganze Zahlen $c_i > 0$, $0 < a_i < b$ für $i \in I \defeq \menge{1, \dots, n}$
\begin{align*}
	f(x) &= \trans{c} x \to \max \bei \trans{a} x \le b, \enskip x \in \menge{0,1}^n 
	\tag{P} \\
	f(x) &= \trans{c} x \to \max \bei \trans{a} x \le b, \enskip x \in [0,1]^n 
	\tag{Q} \label{eq: rucksack-q}
\end{align*}
Hier gilt also $E = \menge{x \in \Rn : \trans{a} x \le b, 0 \le x \le 1, i \in I}$ und $D = \mathbb{B}^n = \menge{0,1}^n$.

\begin{bemerkung}
	Die stetige Relaxation \eqref{eq: rucksack-q} besitzt unter der Voraussetzung
	\begin{equation*}
		\frac{c_i}{a_i} \ge \frac{c_{i+1}}{a_{i+1}} \quad \text{ für } 1 \le i \le n
	\end{equation*}
	die Lösung
	\begin{equation*}
		\dach{x_i} = 1 \quad (i = 1, \dots, k) \qquad \dach{x}_{k+1} = \frac{b - \sum_{i=1}^k a_i}{a_{k+1}} \qquad \dach{x}_i = 0 \enskip i = k+2, \dots, n 
		\tag{R} \label{eq: rucksack-r}
	\end{equation*}
	mit $k \defeq \max\menge{j \in I : \sum_{i=1}^j a_i \le b}$.
\end{bemerkung}

\begin{*bemerkung_inline}
	Wir sortieren also abfallend nach Nutzen pro Volumen und packen dann soviel wie möglich in den Rucksack ($k$ Elemente). Dann füllen wir den Restplatz noch mit (einem Anteil von) dem nächsten Element auf ($k+1$-tes Element), alles danach können wir nicht mehr mitnehmen.
\end{*bemerkung_inline}

Innerhalb des Verzweigungsbaums ergeben sich die folgenden Teilprobleme (für bereits fixierte Variablen $\quer{x}_i$, $i \in I_k \subseteq I$).
\begin{align*}
	\sum_{i \in I_k} c_i \quer{x}_i + \sum_{i \notin I_k} c_i x_i &\to \max \bei \sum_{i \notin I_k} a_i x_i \le b - \sum_{i \in I_k} a_i \quer{x}_i, \quad i \notin I_k, x_i \in \mathbb{B}
	\tag{$P_k(\quer{x})$} \label{eq: rucksack-pk}\\
	\sum_{i \in I_k} c_i \quer{x}_i + \sum_{i \notin I_k} c_i x_i &\to \max \bei \sum_{i \notin I_k} a_i x_i \le b - \sum_{i \in I_k} a_i \quer{x}_i, \quad i \notin I_k, x_i \in [0,1]  \tag{$Q_k(\quer{x})$} \label{eq: rucksack-qk}
\end{align*}

\textbf{Beachte:} Da es sich um eine Maximierungsaufgabe handelt, werden im B\&B-Algorithmus \textit{obere} Schranken $b(P_k(\quer{x}))$ benötigt. Diese gewinnen wir aus den Optimalwerten der stetigen Relaxation $Q_k(\quer{x})$.

\begin{beispiel}
	Gegeben sei das binäre Rucksackproblem 
	\begin{equation*}
	\begin{split}
	\setlength{\arraycolsep}{2pt}
	\begin{array}{rcrcrcrcrcrcrcrl}
	z = \trans{c} x &= &8x_1 &+ &16x_2 &+ &20x_3 &+ &12x_4 &+ &6x_5 &+ &10x_6 &+ &4x_7 &\to \max \\
	\bei \trans{a} x &= &3x_1 &+ &7x_2 &+ &9x_3 &+ &6x_4 &+ &3x_5 &+ &5x_6 &+ &2x_7 &\le 17 = b, 
	\end{array}
	\\
	x_i \in \menge{0,1}, i = 1, \dots, 7
	\end{split}
	\tag{P} \label{eq: rucksack-bsp-p}
	\end{equation*}
	
	\begin{itemize}[leftmargin=*]
		\item Die korrekte Sortierung liegt bereits vor.
		\item Wurzelknoten \eqref{eq: rucksack-bsp-p}$=(\text{P}_0)$: Die stetige Relaxation besitzt die Lösung $x_1 = x_2 = 1$ und $x_3 = \frac{7}{9}$ sowie $x_4 = \dots = x_7 = 0$. Somit ist $b(\text{P}_0) = \lfloor 8 + 16 + 20 * \frac{7}{9} \rfloor = 39$. Die Abrundung ist dabei erlaubt, da $\trans{c} x \in \Z$ für alle zulässigen Punkte von \eqref{eq: rucksack-bsp-p}. Ein zulässiger Punkt für \eqref{eq: rucksack-bsp-p} ist gegeben durch $\quer{x}_1 = \quer{x}_2 = \quer{x}_4 = 1$, $\quer{x}_i = 0$ sonst. Dabei ist $z(\quer{x}) = 8 + 16 + 12 = 36 \defqe \quer{z}$. Wegen $36 < 39$ muss weiter verzweigt werden.
		\item Verzweigung: $x_3 = 0$ vs. $x_3 = 1$.
		\begin{itemize}
			\item Teilproblem: $(P_1) = (P_{0,1})$. Setzt man $x_3 = 0$, so hat die stetige Relaxation die Lösung $x_1 = x_2 = x_4 = 1$, $x_5 = \frac{1}{3}$ und $x_6 = x_7 = 0$. Somit ist $b(P_1) = \lfloor 8 + 16 + 12 + 6 * \frac{1}{3} \rfloor = 38$. Ein daraus ableitbarer Punkt ist gegeben durch $\quer{x}_1 = \quer{x}_2 = \quer{x}_4 = 1$ mit $z = 36$, d.h. $\quer{z}$ muss nicht aktualisiert werden.
			\item $(P_2) = (P_{0,2})$. Setzt man $x_3 = 1$, so erhält man $x_1 = 1$, $x_2 = \frac{5}{7}$ mit $b = \lfloor 8 + 20 + 16 * \frac{5}{7} \rfloor = 39$ und einen zulässigen Punkt für \eqref{eq: rucksack-bsp-p} durch $\quer{x}_1 = \quer{x}_3 = \quer{x}_5 = \quer{x}_7 = 1$ mit $\quer{z} = 38$, d.h. $\quer{z}$ wurde verbessert.
			\item Damit kann ($P_1$) abgeschlossen werden, da maximal noch der Zielfunktionswert $38$ möglich ist, der in ($P_2$) bereits erreicht worden ist.
		\end{itemize}
		%
		\item Verzweigung: $x_2 = 0$ vs. $x_2 = 1$.
		\begin{itemize}
			\item Teilproblem $(P_3) = (P_{2,1})$. Setzt man $x_2 = 0$ (und $x_3 = 1$ von oben), dann hat die stetige Relaxation die Lösung $x_1 = 1$, $x_4 = \frac{5}{6}$ mit $b(P_3) = 38 = \quer{z}$, also sind wir hier fertig.
			\item Teilproblem $(P_4) = (P_{2,2})$. Setzt man $x_2 = 1$ (und $x_3 = 1$) so folgt $x_1 = \frac{1}{3}$, also $b(P_4) = 38 = \quer{z}$, d.h. wir sind hier wieder fertig.
		\end{itemize}
		%
		\item Nun ist $R = \emptyset$ und wir haben eine Lösung gefunden:
		\begin{equation*}
			x_1 = x_3 = x_5 = x_7 = 1 \quad \mit \quad z = 38
		\end{equation*}
	\end{itemize}
\end{beispiel}


%%%%%%%%%%%%%%%%%%%%%%%%%%%%%%%%%%%%%%%%%%%%%%%%%%%%%%%%%%%%%%%%%%%%%%%%%%%%%%%%%%
%%%%%%%%%%%%%%%%%%%%%%%%%%%%%%%%%%%%%%%%%%%%%%%%%%%%%%%%%%%%%%%%%%%%%%%%%%%%%%%%%%


\subsubsection{Ganzzahlige lineare Optimierung nach Land/Doig/Dahin}

Wir betrachten die ganzzahlige Optimierungsaufgabe 
\begin{equation*}
	\trans{c}x  \to \min \bei x \in D \cap E
	\tag{P} \label{eq: ldl-p}
\end{equation*}
mit $D = \Z^n$ und $E = E_0 = \menge{x \in \Rn: Ax = b, x \ge 0}$, wobei alle Inputdaten $(A,b,c)$ ganzzahlig sind. Die im Verlauf des Verfahrens zu betrachtenden Teilprobleme \eqref{eq: ldl-pi} haben die Form
\begin{equation*}
	\trans{c} x \to \min \bei x \in D \cap E_i
	\tag{P${}_\text{i}$} \label{eq: ldl-pi}
\end{equation*}
wobei $E_i$ durch eine oder mehrere zusätzliche Ungleichungen aus $E_0$ entsteht. Sei $x^{\text{LP}}$ eine Lösung der zu \eqref{eq: ldl-pi} gehörenden stetigen Relaxation
\begin{equation*}
	z(Q_i) = \min \menge{\trans{c} x : x \in E_i} 
	\tag{Q${}_\text{i}$} \label{eq: ldl-qi}
\end{equation*}
mit $x^{\text{LP}}_j \notin \Z$ für mindestens einen Index $j$. Dann kann der (gerundete) Optimalwert als Schranke $b(P_i)$ genutzt werden.
\begin{align*}
	E_{i,1} &= \menge{x \in E_i : x_j \le \left\lfloor x^{\text{LP}} \right\rfloor}
	\tag{P${}_{i,1}$} \\
	E_{i,2} &= \menge{x \in E_i : x_j \ge \left\lfloor x^{\text{LP}} \right\rfloor + 1 = \left\lceil x^{\text{LP}} \right\rceil} 
	\tag{P${}_{i,2}$}	
\end{align*}

\begin{bemerkung}
	Das Runden des Optimalwertes einer Relaxation zur Schrankenbestimmung ist nur für $c \in \Z^n$ zulässig.
\end{bemerkung}

\begin{beispiel}
	Betrachte 
	\begin{equation*}
	\begin{alignedat}{4}
		z = -7x_1 - 2x_2 \to \min \bei &&-x_1 &+ &2x_2 &+ &x_4 &= 4 \\
		&&5x_1 &+ &x_2 &+ &x_4 &= 20, \quad x_1, \dots, x_n \in \Z_+
	\end{alignedat}
	\end{equation*}
	Die Relaxation ergibt sich durch $x_1, \dots, x_n \ge 0$.
	\begin{itemize}[leftmargin=*]
		\item Wurzelknoten:
		\begin{center}
			\begin{tabular}{r|rr|r}
				$T_1$ & $x_1$ & $x_2$ & $1$ \\ \hline
				$x_3 = $ & $1$ & $-2$ & $4$ \\
				$x_4 = $ & $-5$ & $-1$ & $20$ \\ \hline
				$z = $   & $-7$ & $-2$ & $0$
			\end{tabular}
			$\qquad \overset{x_1 \leftrightarrow x_4}{\longrightarrow} \quad \dots \quad \overset{x_2 \leftrightarrow x_3}{\longrightarrow} \qquad$
			\begin{tabular}{r|rr|r}
				$T_3$ & $x_4$ & $x_3$ & $1$ \\ \hline
				$x_2 = $ & $-\frac{1}{11}$ & $-\frac{5}{11}$ & $\frac{40}{11}$ \\
				$x_1 = $ & $\frac{-2}{11}$ & $\frac{1}{11}$ & $\frac{36}{11}$ \\ \hline
				$z = $   & $\frac{16}{11}$ & $\frac{3}{11}$ & $-\frac{332}{11}$
			\end{tabular}
		\end{center}	
		%
		\item Verzweigung: nach $x_2$, da $\min \menge{4-\frac{40}{11}, \frac{40}{11} - 3} > \min \menge{4 - \frac{36}{11}, \frac{36}{11} - 3}$. 
		\begin{itemize}
			\item 1. Teilproblem: $x_2 \ge 4$ (führe $s_2 \ge 0$ ein)
			\begin{equation*}
				\begin{aligned}
					\follows s_2 &= x_2 - 4 
					\overset{\text{aus T}_3}{=} \brackets{-\frac{1}{11}x_4 - \frac{5}{11} x_3 + \frac{40}{11}} - 4 \\
					&= -\frac{1}{11} x_4 - \frac{5}{11} x_3 - \frac{4}{11} < 0
				\end{aligned}
			\end{equation*}
			Damit ist $x_2 \ge 4$ nicht möglich (leerer zulässiger Bereich) und dieser Fall muss nicht betrachtet werden.
			%
			\item 2. Teilproblem: $x_2 \le 3$ (führe $s_2 \ge 0$ ein)
			\begin{equation*}
				\begin{aligned}
					s_2 &= 3 - x_2 = 3 - (-\frac{1}{11} x_4 - \frac{5}{11} x_3 + \frac{40}{11}) \\
					&= \frac{1}{11} x_4 + \frac{5}{11} x_3 - \frac{7}{11}
				\end{aligned}
			\end{equation*}
			Füge dies ist $T_3$ ein:
			
			\begin{center}
				\begin{tabular}{r|rr|r}
					$T_3'$ & $x_4$ & $x_3$ & $1$ \\ \hline
					$x_2 = $ & $-\frac{1}{11}$ & $-\frac{5}{11}$ & $\frac{40}{11}$ \\
					$x_1 = $ & $\frac{-2}{11}$ & $\frac{1}{11}$ & $\frac{26}{11}$ \\
					$s_2 = $ & $\frac{1}{11}$ & \fbox{$\frac{5}{11}$} & $-\frac{7}{11}$ \\ \hline
					$z = $   & $\frac{16}{11}$ & $\frac{3}{11}$ & $-\frac{332}{11}$ \\ \hline
					Keller & $-\frac{1}{5}$ & $\ast$ & $\frac{7}{5}$
				\end{tabular}	
				$\quad \overset{s_2 \leftrightarrow x_3}{\longrightarrow} \quad$		
				\begin{tabular}{r|rr|r}
					$T_4$ & $x_4$ & $s_2$ & $1$ \\ \hline
					$x_2 = $ & $0$ & $-1$ & $3$ \\
					$x_1 = $ & $-\frac{1}{5}$ & $\frac{1}{5}$ & $\frac{17}{5}$ \\
					$x_3 = $ & $-\frac{1}{5}$ & $\frac{11}{5}$ & $\frac{7}{5}$ \\ \hline
					$z = $   & $\frac{7}{5}$ & $\frac{3}{5}$ & $-\frac{149}{5}$
				\end{tabular}
			\end{center}
		\end{itemize}
		%
		\item Verzweigung: $x_1$ ($x_3$ ebenso möglich)
		\begin{itemize}
			\item 3. Teilproblem: $x_1 \ge 4 \follows s_1 = x_1 - 4 = -\frac{1}{5}x_4 + \frac{1}{5} s_2 - \frac{3}{5}$
			
			\begin{center}
				\begin{tabular}{r|rr|r}
					$T_4'$ & $x_4$ & $s_2$ & $1$ \\ \hline
					$x_2 = $ & $0$ & $-1$ & $3$ \\
					$x_1 = $ & $-\frac{1}{5}$ & $\frac{1}{5}$ & $\frac{17}{5}$ \\
					$x_3 = $ & $-\frac{1}{5}$ & $\frac{11}{5}$ & $\frac{7}{5}$ \\
					$s_1 = $ & $-\frac{1}{5}$ & \fbox{$\frac{1}{5}$} & $\frac{3}{5}$ \\ \hline
					$z = $   & $\frac{7}{5}$ & $\frac{3}{5}$ & $-\frac{149}{5}$
				\end{tabular}
				$\quad \overset{s_1 \leftrightarrow s_2}{\longrightarrow} \quad$	
				\begin{tabular}{r|rr|r}
					$T_5$ & $x_4$ & $s_1$ & $1$ \\ \hline
					$x_2 = $ &  &  & $0$ \\
					$x_1 = $ &  &  & $4$ \\
					$x_3 = $ &  &  & $3$ \\
					$s_2 = $ &  &  & $8$ \\ \hline
					$z = $   & $2$ & $3$ & $-28$
				\end{tabular}
			\end{center}
			Damit haben wir zumindest \textit{eine} ganzzahlige Lösung $z = -28$. Wir wissen jedoch noch nicht, ob es tatsächlich die optimale Lösung ist.
			%
			\item 4. Teilproblem: $x_1 \le 3 \follows s_1 = 3-x_1 = \frac{1}{5} x_4 - \frac{1}{5} s_2 - \frac{2}{5}$. Dies fügt man zu $T_4$ hinzu. Ein dualer Simplexschritt führt dann zu
			\begin{center}
				\begin{tabular}{r|rr|r}
					$T_6$ & $s_1$ & $s_2$ & $1$ \\ \hline
					$x_2 = $ &  &  & $3$ \\
					$x_1 = $ &  &  & $3$ \\
					$x_3 = $ &  &  & $2$ \\
					$x_3 = $ &  &  & $1$ \\ \hline
					$z = $   & $2$ & $3$ & $-27$
				\end{tabular}
			\end{center}
			Dabei haben wir also eine ganzzahlige Lösung mit $z = -27$.
		\end{itemize}
	\end{itemize}
	Insgesamt wissen wir also $x^\ast = (4,0,3,0)$ mit $z^\ast = -28$ aus dem dritten Teilproblem.
\end{beispiel}


%%%%%%%%%%%%%%%%%%%%%%%%%%%%%%%%%%%%%%%%%%%%%%%%%%%%%%%%%%%%%%%%%%%%%%%%%%%%%%%%%%
%%%%%%%%%%%%%%%%%%%%%%%%%%%%%%%%%%%%%%%%%%%%%%%%%%%%%%%%%%%%%%%%%%%%%%%%%%%%%%%%%%


\subsubsection{Das Rundreiseproblem (Traveling Salesman Problem, TSP)}

Gegeben seien $n$ Orte, eine Kostenmatrix $C = (c_{ik}) \in \R^{n \times n}$, wobei $c_{ik}$ die Entfernung (Zeit, Kosten, ...) von $i$ nach $k$ beschreibt. \\
\textit{Annahme:} $c_{ii} = + \infty$ für alle $i \in I = \menge{1, \dots, n}$. In einigen Anwendungen ist $C$ auch symmetrisch.

\textbf{Variablen:} $x_{ik} \in \menge{0,1}$, $(i,k) \in I \times I$ mit $x_{ik} = 1 \equivalent $ man reist von $i$ nach $k$

\textbf{Optimierungsproblem:}
\begin{align}
	z = \sum_{i \in I} \sum_{k \in I} c_{ik} x_{ik} \to \min 
	\label{eq: 4.1} \\
	\bei \sum_{i \in I} x_{ik} &= 1 \quad (k \in I) 
	\label{eq: 4.2} \\
	\sum_{k \in I} x_{ik} &= 1 \quad (i \in I)
	\label{eq: 4.3} \\
	x_{ik} &\in \menge{0,1} \quad (i,k) \in I \times I 
	\label{eq: 4.4} \\
	\sum_{i,k\in S} x_{ik} &\le \card{S} - 1 \quad (S \subset I, 0 < \card{S})
	\label{eq: 4.5} 
\end{align}

Die Bedingung \eqref{eq: 4.5} wird auch \begriff{Subtoureliminationsbedingung} (SEB) genannt.
Die Bedingungen \eqref{eq: 4.1} bis \eqref{eq: 4.4} modellieren ein spezielles Transportproblem (das \begriff{Zuordnungsproblem}).
Für das Problem \eqref{eq: 4.1} -- \eqref{eq: 4.5} kann man auf folgende Weise Schranken erhalten:
\begin{itemize}
	\item Lösung des (ganzzahligen) Zuordnungsproblems
	\item stetige Relaxation von \eqref{eq: 4.1} -- \eqref{eq: 4.5}: löse zunächst das Zuordnungsproblem. Falls dabei Subtouren entstehen, verbiete man diese mit geeigneten Bedingungen vom Typ \eqref{eq: 4.5} und löse danach das um diese Bedingung erweiterte Zuordnungsproblem.
	\item Zeilen- und Spaltenreduktion: Ermittlung einer zulässigen Lösung $(u,v)$ des dualen Problems (der stetigen Relaxation des Zuordnungsproblems) mithilfe der folgenden Idee:
	\begin{equation*}
		\begin{alignedat}{3}
			&u_i &&\defeq \min\menge{c_{ik} : k \in I} \qquad &&(i \in I) \\
			&v_k &&\defeq \min\menge{c_{ik} - u_i : i \in I} \qquad &&(k \in I) \\
			\follows &w_{ik} &&\phantom{:}= c_{ik} - u_i - v_k \ge 0 \qquad &&\forall (i,k) \in I \times I \text{ (duale Zulässigkeit)}
		\end{alignedat}
	\end{equation*}
	Somit gilt
	\begin{equation*}
		\begin{aligned}
			z &= \sum_{i \in I} \sum_{k \in I} c_{ik} x_{ik} 
			\overset{(\star)}{=} \sum_{i \in I} \sum_{k \in I} \underbrace{w_{ik}}_{\ge 0} \underbrace{x_{ik}}_{\ge 0} + \sum_{i \in I} u_i + \sum_{k \in I} v_k \\
			&\ge \sum_{i \in I} u_i + \sum_{k \in I} v_k \qquad \text{ (untere Schranke für $z$) }
		\end{aligned}
	\end{equation*}
	Im Schritt ($\star$) haben wir dabei $w_{ik} = c_{ik} + u_i + v_i$ und die Bedingungen \eqref{eq: 4.2} sowie \eqref{eq: 4.3} verwendet.
	
	Die oben genannten Schrankenwerte unterscheiden sich (mitunter stark) hinsichtlich des numerischen Aufwands und der Güte der erhaltenen Näherungen. In dieser Vorlesung betrachten wir die dritte Variante.
\end{itemize}

\begin{beispiel} \label{beispiel: 4.3}
	Wir betrachten die Kostenmatrix
	
	\begin{center}
		\begin{tabular}{r|ccccc|r}
			$C = (c_{ik})$ &          &    &    &    &    & $u_i$ \\ \hline 
			& $\infty$ & 32 & \underline{22} & 30 & 24 & 22 \\
			& 10       & $\infty$ & \underline{3} & 18 & $\infty$ & 3\\
			& $\infty$ & \underline{9} & $\infty$ & 14 & 12 & 9\\
			& 16 & 10 & 7 & $\infty$ & \underline{6} & 6 \\
			& 15 & 19 & 15 & \underline{12} & $\infty$ & 12 \\ \hline
			$v_k$ & 3 & 0 & 0 & 0 & 0 & $b = 55$ \\ 
		\end{tabular}
	\end{center}

	Damit ist $b = \sum_i u_i + \sum_k v_k = 55$ eine untere Schranke für den Optimalwert.
\end{beispiel}

\begin{bemerkung}
	Die Schranke, die aus Zeilen- und anschließender Spaltenreduktion erhalten wird, weicht im Allgemeinen von der Schranke ab, die aus Spalten- und anschließender Zeilenreduktion gewonnen wird.
\end{bemerkung}

Unter Einbeziehung der reduzierten Kostenmatrix $D = (w_{ik})$ mit $w_{ik} = c_{ik} - u_i - v_k$ kann eine Verzweigungs- und Auswahlstrategie formuliert werden:
Dabei betrachten wir die Elemente mit $w_{ik} = 0$ und verzweigen gemäß $x_{ik} = 0$ vs. $x_{ik} = 1$.
\begin{itemize}
	\item Die Belegung $x_{ik} = 0$ ist gleichbedeutend mit der Änderung des aktuellen Kostenwertes auf $+\infty$.
	\item Eingedenk\footnote{Alternativer Vorschlag: im Lichte der Bedingungen \dots} der Bedingungen \eqref{eq: 4.2} und \eqref{eq: 4.3} impliziert die Wahl $x_{ik} = 1$, dass $x_{jk} = 0$ für alle $j \neq i$ und $x_{i\ell} = 0$ für alle $\ell \neq k$ gelten muss. Für diese Indexpaare kann man die Kosten auf $+\infty$ erhöhen. Weiterhin muss die Subtour $i \to k \to i$ verhindert werden, d.h. es gilt $x_{ki} = 0$ und damit $w_{ki} = +\infty$. (Analog verfahre man gegebenenfalls mit längeren Subtouren.)
	Infolge dieser Änderung der Kostenmatrix können im Anschluss weitere Zeilen- und Spaltenreduktionen ermöglicht werden, die zu verbesserte Schranken (in den entsprechenden Teilproblemen) führen können.
\end{itemize}

Um wenigstens in einem der beiden Fälle ($x_{ik} = 0$, $x_{ik}=1$) einen möglichst guten Anstieg der unteren Schranke zu erreichen, ist folgende Auswahlregel empfehlenswert. Für jedes $(i,k)$ mit $w_{ik} = 0$ überlege man im Vorfeld bereits, welcher Zuwachs der Schranke im Fall $x_{ik} = 0$ (d.h. wenn $w_{ik} = \infty$ gesetzt wird) zu erwarten ist. Dieses Gewicht $\schlange{w}_{ik}$ bestimmt sich zu 
\begin{equation*}
	\schlange{w}_{ik} \defeq \min_{q \neq k} w_{iq} + \min_{p \neq i} w_{pk}
\end{equation*}
und kann an jeder Nullzelle vorab notiert werden. Für die Verzweigung ist dann eine solche Zelle zu wählen, deren Wert $\schlange{w}_{ik}$ am größten ist.

\begin{beispiel}[Fortsetzung]
	Die reduzierte Matrix aus \cref{beispiel: 4.3} ergibt sich zu
	\begin{center}
		\begin{tabular}{r|ccccc}
			$w_{ik}$ &          &    &    &    &   \\ \hline 
			& $\infty$ & 10 & $0^{\textcolor{cdpurple}{2}}$ & 8 & 2 \\
			& 4       & $\infty$ & $0^{\textcolor{cdpurple}{4}}$ & 15 & $\infty$\\
			& $\infty$ & $0^{\textcolor{cdpurple}{7}}$ & $\infty$ & 5 & 3 \\
			& 7 & 4 & 1 & $\infty$ & $0^{\textcolor{cdpurple}{3}}$ \\
			& $0^{\textcolor{cdpurple}{4}}$ & 7 & 3 & $0^{\textcolor{cdpurple}{5}}$ & $\infty$ \\ 
		\end{tabular}
	\end{center}
	Die Hilfsgröße $\textcolor{cdpurple}{\schlange{w}_{ik}}$ ist für $(i,k) = (3,2)$ maximal. Die nächste Verzweigung wäre also $x_{32} = 0$ vs. $x_{32} = 1$.
	
	$x_{32} = 0$ liefert
	\begin{center}
		\begin{tabular}{r|ccccc|r}
			$w_{ik}$ &          &    &    &    &  & $u_i$  \\ \hline 
			& $\infty$ & 10 & $0^{\textcolor{cdpurple}{2}}$ & 8 & 2 & 0 \\
			& 4       & $\infty$ & $0^{\textcolor{cdpurple}{4}}$ & 15 & $\infty$ & 0 \\
			& $\infty$ & $\infty$ & $\infty$ & 5 & 3& 3 \\
			& 7 & 4 & 1 & $\infty$ & $0^{\textcolor{cdpurple}{3}}$ & 0\\
			& $0^{\textcolor{cdpurple}{4}}$ & 7 & 3 & $0^{\textcolor{cdpurple}{5}}$ & $\infty$ & 0 \\ \hline
			$v_k$ & 0 & 4 & 0 & 0 & 0 & $b = 55 + 7 = 62$
		\end{tabular}
		$\overset{\text{Reduktion}}{\longrightarrow}$
		\begin{tabular}{r|ccccc}
			$w_{ik}$ &          &    &    &    &    \\ \hline 
			& $\infty$ & 6 & 0 & 8 & 2  \\
			& 4       & $\infty$ & 0 & 15 & $\infty$ \\
			& $\infty$ & $\infty$ & $\infty$ & 2 & 0 \\
			& 7 & 4 & 1 & $\infty$ & 0 \\
			& 0 & 3 & 3 & 0 & $\infty$ \\ 
		\end{tabular}
	\end{center}
	
	$x_{32} = 1$ liefert
	\begin{center}
		\begin{tabular}{r|ccccc|r}
			$w_{ik}$ &          &    &    &    &  & $u_i$  \\ \hline 
			& $\infty$ & $\textcolor{cdpurple}{\infty}$ & 0 & 8 & 2 & 0 \\
			& 4       & $\textcolor{cdpurple}{\infty}$ & $\textcolor{cdpurple}{\infty}$ & 15 & $\infty$ & 4 \\
			& $\infty$ & 0 & $\textcolor{cdpurple}{\infty}$ & $\textcolor{cdpurple}{\infty}$ & $\textcolor{cdpurple}{\infty}$ & 0 \\
			& 7 & $\textcolor{cdpurple}{\infty}$ & 1 & $\infty$ & 0 & 0 \\
			& $0$ & $\textcolor{cdpurple}{\infty}$ & 3 & 0 & $\infty$ & 0 \\ \hline
			$v_k$ & 0 & 0 & 0 & 0 & 0 & $b = 55 + 4 = 59$
		\end{tabular}
		$\overset{\text{Reduktion}}{\longrightarrow}$
		\begin{tabular}{r|ccccc}
			$w_{ik}$ &          &    &    &    &  \\ \hline 
			& $\infty$ & $\textcolor{cdpurple}{\infty}$ & 0 & 8 & 2\\
			& 0       & $\textcolor{cdpurple}{\infty}$ & $\textcolor{cdpurple}{\infty}$ & 11 & $\infty$ \\
			& $\infty$ & 0 & $\textcolor{cdpurple}{\infty}$ & $\textcolor{cdpurple}{\infty}$ & $\textcolor{cdpurple}{\infty}$ \\
			& 7 & $\textcolor{cdpurple}{\infty}$ & 1 & $\infty$ & 0 \\
			& $0$ & $\textcolor{cdpurple}{\infty}$ & 3 & 0 & $\infty$ \\
		\end{tabular}
	\end{center}
Fazit: 
\begin{itemize}[nolistsep, topsep=-\parskip]
	\item $z \ge 59$ für jede Rundreise mit $x_{32} = 1$
	\item $z \ge 62$ für jede Rundreise mit $x_{32} = 0$
\end{itemize}
Ein vollständiges Beispiel befindet sich auf dem Übungsblatt.
\end{beispiel}