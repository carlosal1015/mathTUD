\section{Beispiele zur kontinuierlichen Optimierung}

\subsection{Transportoptimierung}

Die Transportoptimierung ist ein Beispiel einer linearen Optimierung.

Es gebe Erzeuger $i \in I = \menge{1, \dots , n}$ und Verbraucher $j \in J = \menge{1, \dots , n}$. Weiterhin seien die Kosten $c_{ij}$ für den Transport einer Einheit von $i$ nach $j$ sowie der Vorrat $a_i > 0$ und der Bedarf $b_j > 0$ für alle $i$ und $j$ gegeben. Wie muss der Transport organisiert werden, damit die Gesamtkosten minimal sind?

Für jedes mathematische Modell einer OA braucht man
\begin{itemize}[nolistsep, topsep=-\parskip]
	\item geeignete Variablen ($\to x$)
	\item Zielfuntkion ($\to f$)
	\item Nebenbedingungen ($\to G$)
\end{itemize}

\begin{description}
	\item[Variablen] $x_{ij} \ge 0$ für alle $i \in I$ und $j \in J$ beschreibe die Einheiten, die von $i$ nach $j$ transportiert werden.
	\item[Zielfunktion] $f(x) = \sum\limits_{i \in I} \sum\limits_{j \in J} c_{ij} x_{ij} \to \min$
	\item[Nebenbedingungen] \leavevmode
%	\begin{equation*}
%		\begin{alignedat}{2}
%		\text{Kapazitätsbeschränkung der Erzeuger: } \qquad  \sum\limits_{j \in J} x_{ij} &\le a_i & \qquad & (i \in I) \\
%		\text{Bedarfserfüllung von Verbrauchern: } \qquad \sum\limits_{i \in I} x_{ij} &\ge b_j & \qquad & (j \in J)
%		\end{alignedat}
%	\end{equation*}
	\begin{itemize}[nolistsep, topsep=-\parskip]
		\item Kapazitätsbeschränkung der Erzeuger $i \in I$: $\sum\limits_{j \in J} x_{ij} \le a_i \quad (i \in I)$
		\item Bedarfserfüllung von Verbrauchern $j \in J$: $\sum\limits_{i \in I} x_{ij} \ge b_j \quad (j \in J) $
	\end{itemize}
\end{description}

Somit können wir als Modell formulieren:
\begin{equation*}
	\begin{alignedat}{2}
	f(x) = \sum_{i \in I} \sum_{j \in J} c_{ij} x_{ij} \to \min \quad \bei \quad \sum_{j \in J} x_{ij} &\le a_i &\quad &(i \in I), \\
	\sum_{i \in I} x_{ij} &\ge b_j &\quad & (j \in J), \\
	 x_{ij} &\ge 0 & \quad & ((i,j) \in I \times J)
	\end{alignedat}
\end{equation*}

\subsection{Kürzeste euklidische Entfernung}

Die Optimierung der kürzesten euklidischen Entfernung ist nichtlinear.

Gegeben seien ein Punkt $\schlange{x} \in \Rn$ und eine Menge $G \subseteq \Rn$ mit $\schlange{x} \notin G$. Wir betrachten die folgende OA:
\begin{equation*}
	f(x) = \norm{x - \schlange{x}}_2^2 \to \min \quad \bei x \in G
\end{equation*}
Ist $G \neq \emptyset$ und abgeschlossen, so existiert eine Lösung. Ist $G$ zusätzlich konvex, so ist die Lösung sogar eindeutig.

Weitere Beispiele und Theorie sind in der Vorlesung \enquote{Kontinuierliche Optimierung} im Master Mathematik zu erfahren.