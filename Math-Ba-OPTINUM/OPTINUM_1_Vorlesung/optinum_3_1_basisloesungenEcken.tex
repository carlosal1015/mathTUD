\section{Basislösungen und Ecken}

Sei $I \defeq \menge{1, \dots , n}$. Da $\rg(A) = m \le n$, existiert eine Indexmenge $I_B \subseteq I$ mit $\card{I_B} = m$ derart, dass alle Spalten $A^i$ ($i \in I_B$) linear unabhängig sind. $I_B$ wird \begriff{Basis-Indexmenge} genannt. Mit $I_N \defeq I \setminus I_B$ (Nichtbasis) definieren wir
\begin{align*}
A_B &= (A^i)_{i \in I_B} & A_N &= (A^i)_{i \in I_N} \\
c_B &= (c_i)_{i \in I_B} & c_N &= (c_i)_{i \in I_N} \\
x_B &= (x_i)_{i \in I_B} & x_N &= (x_i)_{i \in I_N}
\end{align*}
Dann lässt sich \eqref{eq: 3.1} schreiben als
\begin{equation}
	z = \trans{c_B} x_B + \trans{c_N} x_N \to \min \bei A_B x_B + A_N x_N = b, x_B \ge 0, x_N \ge 0 \label{eq: 3.2}
\end{equation}
bzw. durch Auflösen der Gleichung nach $x_B$ (beachte: $A_B$ hat Vollrang) als
\begin{equation}
	z = \brackets{\trans{c_N} - \trans{c_B} A_B^{-1}A_N} x_N + \trans{c_B} A_B^{-1} b \to \min \bei x_B = -A_B^{-1} A_N x_N + A_B^{-1}b, x_B, x_N \ge 0 \label{eq: 3.3}
\end{equation}

\begin{definition} %3.1
	Der Punkt 
	\begin{equation*}
		x = \left(\begin{matrix} x_B \\ x_N \end{matrix} \right) = \left( \begin{matrix} A_B^{-1} b \\ 0 \end{matrix} \right)
	\end{equation*}
	heißt \begriff{Basislösung} zu  $I_B$. Gilt zusätzlich $A_B^{-1}b \ge 0$, dann heißt $x=(x_B,x_N)$ \begriff{zulässige Basislösung}.
\end{definition}

\begin{definition} %3.2
	Der Punkt $x \in G$ heißt \begriff{Ecke} (von $G$), falls aus $x = \frac{1}{2} \brackets{x^1 + x^2}$ mit $x^1,x^2 \in G$ stets $x = x^1 = x^2$ folgt.
\end{definition}

Ecken des zulässigen Bereichs können also nicht durch andere zulässige Punkte linear kombiniert werden.

Zur Wiederholung benennen wir im Folgenden (ohne Beweis) einige Eigenschaften von Ecken und zulässigen Basislösungen.

\begin{satz} %3.1
	Sei $\rg(A) = m$. Dann ist jede zulässige Basislösung auch Ecke von $G$. Umgekehrt gibt es zu jeder Ecke mindestens eine zulässige Basislösung.
\end{satz}

Häufig unterscheidet man zwischen 
\begin{itemize}[nolistsep, topsep=-\parskip]
	\item \begriff{degenerierten} (oder entarteten) Ecken, die mehrere zulässige Basislösungen besitzen
	\item \begriff{nicht-degenerierten} (oder nicht-entarteten) Ecken, die genau eine zulässige Basislösung besitzen.
\end{itemize}
Dabei gilt: Eine Ecke $x \in G$ ist genau dann degeneriert, wenn ein $i \in I_B$ mit $x_i=0$ existiert.

\begin{beispiel} %3.1
	Sei
	\begin{equation*}
		G \defeq \menge{x \in \R^4 \colon x_1+x_2+x_3=1, \enskip 2x_1+x_2+x_4=2, \enskip x_1,\dots , x_4 \ge 0}
	\end{equation*}
	Hierbei ist die Ecke $E_1 = \transpose{0,1,0,1}$ nicht degeneriert, da sie nur die Zerlegung $I_B = \menge{2,4}$ und $I_N = \menge{1,3}$ gestattet.
	Die Ecke $E_2 = \transpose{1,0,0,0}$ ist degeneriert, weil ein $i \in I_B$ zwangsläufig $x_i = 0$ erfüllen muss.
\end{beispiel}

\begin{satz} %3.2
	Sei $G \neq \emptyset$. Dann besitzt $G$
	\begin{enumerate}[nolistsep, topsep=-\parskip]
		\item mindestens eine Ecke
		\item höchstens endlich viele Ecken.
	\end{enumerate}
\end{satz}
\begin{proof}
	siehe Übung
\end{proof}

\begin{satz} %3.3
	Ist \eqref{eq: 3.1} lösbar, dann gibt es eine Ecke von $G$, die \eqref{eq: 3.1} löst.
\end{satz}

Bei linearen Optimierungsaufgaben genügt es daher die Ecken von $G$ zu betrachten. Ist die Aufgabe lösbar, so findet man durch systematisches Abschreiten der Ecken eine Lösung.
Um dabei zu erkennen, ob Optimalität vorliegt, hilft folgendes Resultat:

\begin{aussage}[Optimalitätskriterium] %3.4
	\label{aussage: 3.4}
	Gilt für die zulässige Basislösung $x = (x_B, x_N) = (A_B^{-1}b, 0)$ die Bedingung 
	\begin{equation}
		\trans{c_N} - \trans{c_B} A_B^{-1} A_N \ge 0 \label{eq: 3.4}
	\end{equation}
	dann ist $x$ Lösung von \eqref{eq: 3.1}.
\end{aussage}
\begin{proof}
	Sei $x = (x_B, x_N)$ eine zulässige Basislösung. Wir zeigen zunächst:
	\begin{equation*}
		Z(x) \subseteq \menge{d \in \Rn \colon Ad = 0, d_N \ge 0}
	\end{equation*}
	Sei $d \in Z(x)$. Dann existiert $t > 0$ mit $A(x+td) \overset{!}{=} b$ (beachte die Definition von $G$ mit Gleichheitsrestriktionen). Es gilt 
	\begin{equation*}
		Ax + t Ad = b  \equivalent b + t Ad = b \overset{t > 0}{\equivalent} Ad = 0
	\end{equation*}
	Wegen $x_N = 0$ ergibt sich aus $x + td \overset{!}{\ge} 0$ (nach Definition von $G$) sofort $d_N \ge 0$. Insbesondere gilt
	\begin{equation*}
		Ad = 0 \equivalent A_B d_B + A_N d_N = 0 \equivalent d_B = -A_B^{-1} A_N d_N \qquad \forall d \in Z(x)
	\end{equation*}
	Damit folgt unter Berücksichtigung von \eqref{eq: 3.4}
	\begin{align*}
		\nabla \trans{f(x)} d 
		&= \trans{c} d \\
		&= \trans{c_B} d_B + \trans{c_N} d_N \\
		&= -\trans{c_B} A_B^{-1} A_N d_N + \trans{c_N} d_N \\
		&= \underbrace{\brackets{\trans{c_N} - \trans{c_B} A_B^{-1} A_N}}_{\ge 0} \underbrace{d_N}_{\ge 0} \ge 0 \qquad \forall d \in Z(x)
	\end{align*}
	d.h. $x$ genügt der notwendigen Optimalitätsbedingung \eqref{eq: 2.2}, die hier (im konvexen Fall) auch hinreichend ist.
\end{proof}

Eine entsprechende Systematik zum Abschreiten der Ecken wird im Folgenden Abschnitt behandelt.