\begin{exercisePage}[Kompakte Konvergenz, Integralberechnung, Nullhomologie \& --homotopie]
	
	\begin{task}
		Sei $\Omega \subseteq \Rn$ offen, $(f_n)$ eine Folge von stetigen Funktionen $\abb{f_n}{\Omega}{\CC}$ und $\abb{f}{\Omega}{\CC}$. Zeigen Sie die Äquivalenz folgender Aussagen:
		\begin{enumerate}[label=(\roman*), leftmargin=*, nolistsep]
			\item $(f_n)$ ist kompakt konvergent gegen $f$, d.h. gleichmäßig konvergent auf kompakten Teilmengen von $\Omega$: Für alle $K \subseteq \Omega$ kompakt mit $K \neq \emptyset$ gilt
			\begin{equation*}
				\sup\menge{\abs{f(\zeta) - f_n(\zeta)} : \zeta \in K} \to 0 \qquad (n \to \infty)
			\end{equation*}
			\item $(f_n)$ ist lokal gleichmäßig konvergent gegen $f$, d.h. für alle $z \in \Omega$ gibt es eine Umgebung $U \subseteq \Omega$ von $z$, sodass
			\begin{equation*}
				\sup\menge{\abs{f(\zeta) - f_n(\zeta)} : \zeta \in U} \to 0 \qquad (n \to \infty)
			\end{equation*}
		\end{enumerate}
	\end{task}
	
	\begin{equivalence}
		\hinrichtung Sei $f_n \to f$ kompakt und $z \in \Omega$. Da $\Omega$ offen ist, existiert eine Umgebung $B_\epsilon(z) \subseteq \Omega$ mit $\quer{B_\epsilon(z)} \subseteq \Omega$. Da $\quer{B_\epsilon(z)}$ beschränkt und abgeschlossen ist, ist $\quer{B_\epsilon(z)}$ kompakt. Nach Voraussetzung konvergiert also $f_n \to f$ gleichmäßig auf $\quer{B_\epsilon(z)}$, d.h. $\sup\menge{\abs{f(\zeta) - f_n(\zeta)} : \zeta \in \quer{B_\epsilon(z)}} \to 0$. Somit ist auch $\sup\menge{\abs{f(\zeta) - f_n(\zeta)} : \zeta \in B_\epsilon(z)} \to 0$ für alle $z \in \Omega$ und daher $f_n \to f$ lokal gleichmäßig.
		\rueckrichtung Sei $f_n \to f$ lokal gleichmäßig, d.h. für jedes $z \in \Omega$ existiert ein $\epsilon > 0$, sodass $B_\epsilon(z) \subseteq \Omega$ und $\sup\menge{\abs{f(\zeta) - f_n(\zeta)} : \zeta \in B_\epsilon(z)} \to 0$. Sei nun $K \subseteq \Omega$ eine beliebige kompakte Menge und $\folge{B_\epsilon(z)}{z \in K}$ eine offene Überdeckung von $K$ mit $f_n \to f$ gleichmäßig auf jeder Umgebung $B_\epsilon(z)$ ($z \in K$). Aufgrund der Kompaktheit von $K$ existiert eine endliche Teilüberdeckung $\menge{B_{\epsilon_i}(z_i)}_{i=1}^n$. Sei $\epsilon > 0$. Aufgrund der gleichmäßigen Konvergenz existiert dann für $1 \le i \le n$ ein $N_i$, sodass $\sup\menge{\abs{f(\zeta) - f_n(\zeta)} : \zeta \in B_\epsilon(z)} < \epsilon$ für alle $n > N_i$. Setze nun $N \defeq \max\menge{N_i : 1 \le i \le n}$ und $B \defeq \bigcup_{i=1}^n B_{\epsilon_i}(z_i)$. Dann ist $\sup\menge{\abs{f(\zeta) - f_n(\zeta)} : \zeta \in B} < \epsilon$ für alle $n > N$, d.h. $\sup\menge{\abs{f(\zeta) - f_n(\zeta)} : \zeta \in B} \to 0$ für $n \to \infty$. Somit konvergiert $f_n \to f$ gleichmäßig auf $B$, also auch auf dem beliebigen Kompaktum $K$ und daher ist $f_n$ kompakt konvergent gegen $f$.
	\end{equivalence}

	
	\begin{task}
		Sei $\abb{f}{\CC}{\CC}$ stetig und die Einschränkung von $f$ auf $\menge{z \in \CC : \Re(z) \neq 0}$ sei holomorph. Zeigen Sie, dass
		\begin{equation*}
			\int_{\triangle(a,b,c)} f(z) \dz = 0
		\end{equation*} 
		für alle $a,b,c \in \CC$. Mit dem Satz von Morera ist dann $f$ holomorph. 
		
		Hinweis: Zeigen Sie zuerst, dass $\int_\gamma f(z) \dz = 0$ für jeden geschlossenen Weg $\abb{\gamma}{[a,b]}{\menge{z \in \CC : \Re(z) \ge 0}}$.
	\end{task}


	\begin{task}
		Berechnen Sie das Integral
		\begin{equation*}
			\int_{-\infty}^\infty \frac{1}{1 + x^2} \dx
		\end{equation*}
		auf folgende Weise: Sei $R > 2$ und $\gamma_R$ der nebenstehende geschlossene Weg. Ermitteln Sie $\int_{\gamma_R} \frac{1}{1 + x^2} \dx$ mithilfe der Cauchyschen Integralformel und benutze $1 + z^2 = (z+\i) (z - \i)$. Dann Grenzübergang $R \to \infty$. 
		
		Hier darf folgende Variante des Zentrierungslemmas ohne Beweis verwendet werden: Ist $\abb{f}{\menge{z \in \CC : \Im(z) > -\frac{1}{2}} \setminus \menge{\i}}{\CC}$ holomorph, so gilt $\int_{\gamma_R} f(z) \dz = \int_{\abs{z-\i} = 1} f(z) \dz$.
		
		Berechnen Sie obiges Integral zur Probe auch mithilfe einer Stammfunktion von $x \mapsto \frac{1}{1 + x^2}$.
	\end{task}

	Parametrisiere den Weg $\gamma_R = \gamma_R^1 \dot{+} \gamma_R^2$ wie folgt: $\gamma_R^1 \defeq \id_{[-R,R]}$ und $\gamma_R^2(t) \defeq R*e^{\i t}$ für $t \in [0,\pi]$. Dann gilt 
	\begin{equation*}
		\abs{\int_{\gamma_R^2} \frac{1}{1+x^2} \dx} \le \pi R * \sup\menge{\abs{\frac{1}{1+x^2}} : \abs{x}=R, \Im(x) \ge 0} \le \frac{\pi R}{1 + R^2} \to 0 \tag{$R \to \infty$}
	\end{equation*}
	Mit der Cauchyschen Integralformel (CIF) erhält man dann
	\begin{align*}
		\lim_{R \to \infty} \int_{-R}^R \frac{1}{1 + x^2} \dx 
		&= \lim_{R \to \infty} \int_{\gamma_R^1} \frac{1}{1 + z^2} \dz \\
		&= \lim_{R \to \infty} \int_{\gamma_R} \frac{1}{1 + z^2} \dz \\
		&= \lim_{R \to \infty} \int_{\abs{z - \i} = 1} \frac{1}{(z- \i)(z + \i)} \dz \\
		&= \lim_{R \to \infty} 2\pi \i * \frac{1}{2\i}  \tag{CIF} \\
		&= \pi
	\end{align*}
	Zur Probe: Für $f(x) = \frac{1}{1 + x^2}$ ist eine Stammfunktion gegeben durch $F(x) = \arctan(x)$. Dann ist
	\begin{equation*}
		\int_{-\infty}^\infty f(x) \dx = \lim_{R \to \infty} \int_{-R}^R \frac{1}{1 + x^2} \dx = \lim_{R \to \infty} \brackets{\arctan(R) - \arctan(-R)} = \frac{\pi}{2} - \brackets{-\frac{\pi}{2}} = \pi
	\end{equation*}

	
	\begin{task}
		Sei $\abb{\gamma}{[0,1]}{\CC}$ ein geschlossener Weg. Zeigen Sie:
		\begin{equation*}
			\gamma \text{ nullhomotop} \follows \gamma \text{ nullhomolog}
		\end{equation*}		
	\end{task}

	Sei $\Omega \subseteq \CC$ offen und $\gamma$ nullhomotop in $\Omega$, d.h. homotop zu einem konstanten Weg $\abb{c}{[0,1]}{\CC}$. Dann gilt nach einer Bemerkung der Vorlesung für alle $\abb{f}{\Omega}{CC}$ holomorph, dass $\int_\gamma f(z) \dz = \int_c f(z) \dz$. Da $c$ konstant ist, ist 
	\begin{equation*}
		\int_c f(z) \dz = \int_0^1 f(c(t)) * \underbrace{c'(t)}_{=0} \dt = 0
	\end{equation*}
	und somit also $\int_\gamma f(z) \dz = 0$ für alle holomorphen $\abb{f}{\Omega}{\CC}$. Nach Folgerung 9.4 ist dies äquivalent zur Nullhomologie von $\gamma$.
	
	
	\begin{task}
		Beweisen Sie das \textit{Schwarzsche Lemma}: Sei $\abb{f}{E}{E}$ eine holomorphe Abbildung der Einheitskreisscheibe $E$ in sich selbst mit $f(0) = 0$. Dann gilt $\abs{f'(0)} \le 1$ und $\abs{f(z)} \le \abs{z}$ für alle $z \in E$. Gibt es ein $z_0 \neq 0$ mit $\abs{f(z_0)} = \abs{z_0}$ oder gilt $\abs{f'(0)} = 1$, so ist $f$ eine Drehung mit $f(z) = e^{\i \theta} * z$ für ein $\theta \in \R$ und alle $z \in E$.
	\end{task}

	Wir definieren uns eine Funktion 
	\begin{equation*}
		\abb{g}{E}{\CC} \mit g(z) \defeq \begin{cases}
		\frac{f(z)}{z} & \text{falls } z \neq 0 \\
		f'(0)		   & \text{falls } z = 0
		\end{cases}
	\end{equation*}
	Damit ist $g$ stetig, denn
	\begin{equation*}
		g(0) = f'(0) = \lim_{z \to 0} \frac{f(z) - f(0)}{z - 0} = \lim_{z \to 0} \frac{f(z)}{z} = \lim_{\substack{z \to 0 \\ z \neq 0}} g(z)
	\end{equation*}
	Damit ist dann auch $g$ holomorph auf $E$. Für $r < 1$ gilt mit dem Maximumprinzip für $\abs{z} \le r$ 
	\begin{equation*}
		\abs{g(z)} \le \max_{\abs{z} = r} \abs{g(z)} = \frac{1}{r} * \max_{\abs{z} = r} \abs{f(z)} \le \frac{1}{r} \le 1
	\end{equation*}
	Somit ist also $\abs{g(z)} \le 1$ für alle $z \in E$, d.h. $\abs{\frac{f(z)}{z}} = \frac{\abs{f(z)}}{\abs{z}} \le 1 \follows \abs{f(z)} \le \abs{z}$ und $\abs{f'(0)} \le 1$.
	Ist $\abs{f(z_0)} = \abs{z_0}$ für ein $z_0 \neq 0$ oder $\abs{f'(0)} = 1$, so hat $\abs{g}$ im Inneren von $E$ ein lokales Maximum, was nach dem Maximumprinzip bedeutet, dass $g$ konstant ist, d.h. $g \equiv \lambda$ für ein $\lambda \in \CC$ mit $\abs{\lambda} = 1$ bzw. in Polardarstellung von $\lambda = e^{\i \theta}$ mit $\theta \in \R$ geschrieben als $f(z) = e^{\i \theta} * z$ für alle $z \in E$.
\end{exercisePage}