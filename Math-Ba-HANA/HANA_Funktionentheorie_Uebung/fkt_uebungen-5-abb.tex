\documentclass[margin=5mm]{standalone}
\usepackage{tudscrcolor} 
\usepackage{pgfplots}
\usepgfplotslibrary{fillbetween} 
\pgfplotsset{
  compat=1.10,% mit writeLaTeX bisher noch nicht möglich
  flaeche/.style={draw=none,fill=black,fill opacity=0.2},
  every axis/.append style={
  	axis x line=middle,    % put the x axis in the middle
  	axis y line=middle,    % put the y axis in the middle
  	axis line style={->}, % arrows on the axis
  	xlabel={$\Re(z)$},          % default put x on x-axis
  	ylabel={$\Im(z)$},          % default put y on y-axis
  	enlargelimits=0.05,
  	grid,
  	domain=-5:5,
  	ymin = -5,
  	ymax = 5,
  	xmin = -5,
  	xmax = 5,
  }
}
\usetikzlibrary{decorations.pathmorphing}

\renewcommand{\Re}{\mathrm{Re}}
\renewcommand{\Im}{\mathrm{Im}}



\begin{document} 
	


\pgfdeclaredecoration{penciline}{initial}{
	\state{initial}[width=+\pgfdecoratedinputsegmentremainingdistance,auto corner on length=1mm,]{
		\pgfpathcurveto%
		{% From
			\pgfqpoint{\pgfdecoratedinputsegmentremainingdistance}
			{\pgfdecorationsegmentamplitude}
		}
		{%  Control 1
			\pgfmathrand
			\pgfpointadd{\pgfqpoint{\pgfdecoratedinputsegmentremainingdistance}{0pt}}
			{\pgfqpoint{-\pgfdecorationsegmentaspect\pgfdecoratedinputsegmentremainingdistance}%
				{\pgfmathresult\pgfdecorationsegmentamplitude}
			}
		}
		{%TO 
			\pgfpointadd{\pgfpointdecoratedinputsegmentlast}{\pgfpoint{1pt}{1pt}}
		}
	}
	\state{final}{}
}

\begin{tikzpicture}
\begin{axis}[%
	axis x line=middle,    % put the x axis in the middle
	axis y line=middle,    % put the y axis in the middle
	axis line style={->},  % arrows on the axis
	xlabel={$\Re(z)$}, 			ylabel={$\Im(z)$},
	enlargelimits=0.05,
	ymin = -5,
	ymax = 5,
	xmin = -5,
	xmax = 5,
	axis equal,
	ticks=none,
	grid=none
]
 
\node[anchor=east](a) at (axis cs:-4, 2){$a$};
\node[anchor=west](b) at (axis cs: 4, 4){$b$};
\node[anchor=north](c) at (axis cs: 3,-2){$c$};
\node[anchor=south east](d) at (axis cs:0,3){$d$};
\node[anchor=north east](e) at (axis cs:0,-0.286){$e$};

\draw[line width=2pt] 
   (axis cs:-4, 2) node[anchor=east]{}
-- (axis cs: 4, 4) node[anchor=west]{}
-- (axis cs: 3,-2) node[anchor=north]{}
-- cycle;

\draw[cdorange, line width=1pt] (axis cs:-4, 2) -- (axis cs:-0.05,3) -- (axis cs:-0.05,-0.286) -- cycle;
\draw[cdpurple, line width=1pt] (axis cs:4, 4) -- (axis cs:0.05,-0.286) -- (axis cs:0.05,3) -- cycle;
\draw[cdgreen, line width=1pt] (axis cs:4, 3.95) -- (axis cs:0.04,-0.3) -- (axis cs: 3,-2) -- cycle;
\end{axis} 
\end{tikzpicture} 





%\node[above,font=\large\bfseries] at (current bounding box.north) {Teil (f): $z = t e^{\mathrm{i}t} \enskip (t \ge 0)$};

\end{document}