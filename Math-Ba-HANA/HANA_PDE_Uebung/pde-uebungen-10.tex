\begin{exercisePage}
	
	\setcounter{taskcount}{27}
	
	\begin{task}
		Es sei $u\colon (0,1) \to \mathbb R$. Zeigen Sie:
		\begin{enumerate}
			\item Ist $u$ stetig und stückweise stetige differenzierbar mit gleichmäßig beschränkter Ableitung, so hat $u$ eine stückweise stetige schwache Ableitung.
			\item Ist $u$ stückweise gleichmäßig stetig und ist $x_0 \in (0,1)$ eine Sprungstelle von $u$, so hat $u$ \textit{keine} schwache Ableitung.
		\end{enumerate} 
	\end{task}
	
	\begin{enumerate}[label=(zu \alph*), leftmargin=*]
		\item Sei $0 = x_1 < x_2 < \dots < x_n = 1$ eine Zerlegung des Intervalls $(0,1)$, sodass $\left. u \right\vert_{(x_i, x_{i+1})} \defqe u_i$ für alle $i = 0, \dots, n-1$ stetig differenzierbar ist. Aufgrund der gleichmäßigen Beschränktheit von $u'$ existiert ein $0 < C < \infty$, sodass $u_i'(x) \le C$ für alle $x \in (x_i, x_{i+1})$ und alle $i \in \menge{0, 1, \dots, n-1}$. Dann gilt
		\begin{equation*}
			\int_0^1 u'(x) \dx = \sum_{i=0}^{n-1} \int_{x_i}^{x_{i+1}} u_i'(x) \dx \le \sum_{i=0}^{n-1} \int_{x_i}^{x_{i+1}} C \dx = \sum_{i=0}^{n-1} C * \underbrace{\abs{x_{i+1} - x_i}}_{\le 1} \le (n+1) C < \infty
		\end{equation*}
		und somit $u' \in L^1_{\text{loc}}(0,1)$. Außerdem gilt mit stückweiser partieller Integration für alle $\phi \in C_0^\infty(0,1)$
		\begin{equation*}
			\begin{aligned}
				\int_0^1 u * \phi' \dx 
				= \sum_{i=0}^{n-1} \int_{x_i}^{x_{i+1}} u_i * \phi' \dx 
				&= \sum_{i=0}^{n-1} \sqbrackets{u_i * \phi}_{x_i}^{x_{i+1}} - \int_{x_i}^{x_{i+1}} u_i' * \phi \dx \\
				&= \sqbrackets{u * \phi}_0^1 - \sum_{i=0}^{n-1} \int_{x_i}^{x_{i+1}} u_i' * \phi \dx \\ 
				&= - \sum_{i=0}^{n-1} \int_{x_i}^{x_{i+1}} u_i' * \phi \dx \\
				&= - \int_0^1 u' * \phi \dx
			\end{aligned}
		\end{equation*}
		wobei $\sqbrackets{u \phi}_0^1 = 0$, da $\phi(0) = \phi(1) = 0$ wegen $\phi \in C_0^\infty(0,1)$.
		Somit ist $u' \in L_{\text{loc}}^1(0,1)$ und es gilt $\int_0^1 u \phi' \dx = - \int_0^1 u' \phi \dx$ für alle $\phi \in C_0^\infty(0,1)$, d.h. $u'$ ist schwache Ableitung von $u$ auf $(0,1)$. Die stückweise Stetigkeit folgt dabei aus der stetigen Differenzierbarkeit von $u_i$ für alle $i \in \menge{0, \dots, n-1}$.
		
		\item ---
	\end{enumerate}


	\begin{task}
		Es seien $U \subset \mathbb R^n$ offen und beschränkt und $u,v \in W^{k,p}(U)$. Zeigen Sie:
		\begin{enumerate}
			\item $D^\beta(D^\alpha u) = D^{\alpha + \beta} u$, für alle $\alpha, \beta$ mit $|\alpha| + |\beta| \leq k$. 
			\item $\lambda u + \mu v \in W^{k,p}(U)$ mit $D^\alpha(\lambda u + \mu v) = \lambda D^\alpha u + \mu D^\alpha v$ für alle $\lambda, \mu \in \mathbb{R}$ und $|\alpha|\leq k$.
			\item $u|_V \in W^{k,p}(V)$ für jede offene Menge $V\subset U$.
		\end{enumerate}	
	\end{task}
	\begin{enumerate}[label=(zu \alph*), leftmargin=*]
		\item Sei $v \defeq D^\alpha u$. Dann ist
		\begin{align*}
			(-1)^{\abs{\alpha}} \int_U v D^\beta \phi \dx 
			&= (-1)^{\abs{\alpha}}  \int_U D^\alpha u * D^\beta \phi \dx \\
			&= \int_U u * D^\alpha (D^\beta \phi) \dx \tag{partielle Integration}\\
			&= \int_U u * D^{\alpha + \beta} \phi \dx \tag{klassische Ableitung vertauschbar} \\
			&= (-1)^{\abs{\alpha + \beta}} \int_U D^{\alpha + \beta} u * \phi \dx
			\tag{partielle Integration} \\
			&= (-1){\abs{\alpha} + \abs{\beta}} \int_U D^{\alpha + \beta} u * \phi \dx \tag{$\alpha, \beta \ge 0$}
		\end{align*}
		Und somit ist $D^\beta v = D^{\alpha + \beta} u$, also gerade die Behauptung.
		
		\item Der Sobolev-Raum $W^{k,p}(U)$ ist ein Banachraum, d.h. insbesondere ein Vektorraum über dem Körper $\R$. Somit ist er abgeschlossen unter Linearkombination und aus $u,v \in W^{k,p}(U)$ folgt auch, dass für alle $\lambda, \mu \in \R$ schon $\lambda u + \mu v \in W^{k,p}(U)$. Weiterhin ist wegen $u,v \in W^{k,p}(U)$ für $\abs{\alpha} \le k$ auch
		\begin{equation*}
			\begin{aligned}
				\int_U (\lambda u + \mu v) * D^\alpha \phi \dx 
				&= \int_U \lambda u * D^\alpha \phi + \mu v * D^\alpha \phi \dx \\
				&= \lambda * \int_U u * D^\alpha \phi \dx + \mu \int_U v * D^\alpha \phi \dx \\
				&= \lambda * (-1)^{\abs{\alpha}} * \int_U D^\alpha u * \phi \dx + \mu * (-1)^{\abs{\alpha}} \int_U D^\alpha v * \phi \dx \\
				&= (-1)^{\abs{\alpha}} \int_U \underbrace{\brackets{\lambda D^\alpha u + \mu D^\alpha v}}_{= D^\alpha(\lambda u + \mu v)} * \phi \dx
			\end{aligned}
		\end{equation*}
		und daher $D^\alpha (\lambda u + \mu v) = \lambda D^\alpha u + \mu D^\alpha v$ für alle $u,v \in W^{k,p}(U)$ und alle $\lambda, \mu \in \R$ sowie $\abs{\alpha} \le k$.
		
		\item ---
	\end{enumerate}
\end{exercisePage}