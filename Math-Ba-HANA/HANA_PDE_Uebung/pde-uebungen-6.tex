\begin{exercisePage}
	\begin{task}
		Es sei $U \subset \Rn$ ein beschränktes Gebiet und $u \in C^2(U)\cap C(\overline U)$ eine Lösung von
		$$-\Delta u = f(u,x)\quad\text{in }U,\qquad u = g \quad\text{auf }\partial U,$$
		wobei $f$ und $g$ im Folgenden spezifisch gewählt werden.
		\begin{enumerate}
			\item Es sei $f(u,x) = u-u^3$ und $g=0$. Beweisen Sie \emph{mit elementaren Methoden} die Abschätzung $-1\leq u(x)\leq 1$ für alle $x\in \overline{U}$.
			
			\item Es sei $U = (-a,a)^n$ für ein $a>0$, $f(u,x) = -1$ und $g=0$. Finden Sie möglichst gute obere und untere Schranken für $u(0)$, indem Sie eine harmonische Funktion der Form $v=u+w$ betrachten, wobei $w$ geeignet zu wählen ist
			
			\item Es sei $U = B_1(0)$, $f(u,x) = h(x)$ mit $h,g \in C^1(\Rn,\R)$. Zeigen Sie, dass es eine von $n$, $h$, $g$ und $u$ unabhängige Konstante $c>0$ gibt, für die gilt:
			\begin{equation*}
				\max_{\quer{B_1(0)}} |u| \leq c * \left( \max_{\rand B_1(0)} |g| +\max_{\quer{B_1(0)}} |h| \right)
			\end{equation*}		
			\emph{Hinweis}: Betrachten sie $u-v$, wobei
			$v(x)=\max\limits_{\rand B_1(0)} |g| + \left(e^2 - e^{x_1+1} \right) \max\limits_{\quer{B_1(0)}} |h|$.
		\end{enumerate}
	\end{task}

	\begin{enumerate}[label=(zu \alph*), leftmargin=*]
		\item Es sei $f(u) = u - u^3$ und $g \equiv 0$. Da $U$ beschränktes Gebiet ist und $\quer{U}$ eine abgeschlossene Menge, ist also $\quer{U}$ kompakt. Da $u$ stetig auf $\quer{U}$ ist, nimmt $u$ ein Maximum in einem $x_0 \in \quer{U}$ an. Nehmen wir an es sei $u(x_0) > 1$. Es gilt $ \equiv g \equiv 0$ auf $\rand U$, d.h. um ein Maximum $u(x_0 > 1$ zu besitzen, muss $x_0 \in \inn U$ sein. Da $u(x_0)$ Maximum ist mit $u \in C^2(U)$, gilt $u_{x_i x_i}(x_0) \le 0$ für alle $i=1, \dots, n$. Nach Differentialgleichung ist dann also
		\begin{equation*}
			0 \le - \Delta u(x_0) =  - \sum_{i=1}^n u_{x_i x_i} (x_0) = u(x_0) - u^3(x_0) = u(x_0) \brackets{1 - u(x_0)^2} \overset{u(x_0) > 1}{<} 0
		\end{equation*}
		ein Widerspruch. Somit ist dann $u(x_0) \le 1$ und aufgrund der Maximalität von $u(x_0)$ auch $u(x) \le u(x_0) \le 1$ für alle $x \in \quer U$.
		
		Analog dazu nimmt $u$ auf $\quer U$ ein globales Minimum in $x_0 \in \quer U$ an, für welches nach gleicher Argumentation wie oben $x_0 \in \inn U$ gilt. Nehmen wir an, es sei $u(x_0) < -1$. Als Minimalstelle gilt $u_{x_i  x_i} (x_0) \ge 0$ für alle $i = 1, \dots, n$. Nach PDE gilt dann
		\begin{equation*}
			0 \ge - \Delta u(x_0) =  - \sum_{i=1}^n u_{x_i x_i} (x_0) = u(x_0) - u^3(x_0) = u(x_0) \brackets{1 - u(x_0)^2} \overset{u(x_0) < -1}{<} 0
		\end{equation*} 
		ein Widerspruch. Somit ist $u(x) \ge u(x_0) \ge -1$ für alle $x \in \quer U$ und schließlich gilt die Einschließung $-1 \le u(x) \le 1$ für alle $x \in \quer U$.
		
		\item Sei $a > 0$ und $U = (-a , a)^n$ ein $n$-dimensionaler (offener) Quader. Weiter sei $f \equiv -1$ und $g \equiv 0$. Gesucht sind \enquote{gute} (obere und untere) Schranken von $u(0)$.
		Wir betrachten eine harmonische Funktion $v$ der Form $v = u + w$, d.h. $0 = \Delta v = \Delta u + \Delta w = - f(u,x) + \Delta w = 1 + \Delta w$. Somit suchen wir nun eine Funktion $w$ mit $\Delta w = -1$. Eine Lösung dieser PDE erhalten wir beispielsweise mit $w(x) = - \frac{1}{2n} \abs{x}^2 = - \frac{1}{2n}\sum_{i=1}^n x_i^2$. 
		
		Wenden wir das Maximumsprinzip auf $v$ an, dann nimmt damit $v$ sein Maximum in einem $x_0 \in \rand U$ an. Aufgrund der Randwertbedingung gilt dort $u \equiv 0$. Wir betrachten oBdA den Randpunkt $x_0 = (a, 0, \dots, 0) \in \rand U$. Dieser minimiert $\sum_{i=1}^n x_i^2$, da jeder andere Randpunkt auch mindestens eine Koordinate $j \in \menge{1, \dots, n}$ mit $x_j = \pm a$ besitzt. Somit gilt dann schlussendlich
		\begin{equation*}
			u(0) = v(0) - w(0) = v(0) \le \max_{x \in U} v(x) \le \max_{x \in \rand U} \brackets{u(x) + w(x)} \le w(x_0) = - \frac{1}{2n} a^2
		\end{equation*}
		
		Analog liefert das Minimumsprinzip die Existenz des Minimum in $x_0 \in \rand U$. Nach Randwertbedingung gilt dort wieder $u(x_0) = 0$. Die Funktion $w$ wird auf dem Rand minimiert durch den Punkt $x_0 = (a, \dots, a) \in \rand U$ mit Minimalwert $w(x_0) = - \frac{1}{2n} \sum_{i=1}^n a^2 = - \frac{1}{2} a^2$. Analog zu oben gilt nun
		\begin{equation*}
			u(0) = v(0) - w(0) = v(0) \ge \min_{x \in U} v(x) \ge \min_{x \in \rand U} \brackets{u(x) + w(x)} \ge w(x_0) = - \frac{1}{2} a^2
		\end{equation*}
		Somit ist schließlich $- \frac{1}{2} a^2 \le u(0) \le - \frac{1}{2n} a^2$.
		
		\item Sei $U = B_1(0)$ und $f(u,x) = h(x))$ für $g,h \in C^1(\Rn,\R)$. Gemäß Hinweis betrachten wir $u-v$ mit $v(x) \defeq \max_{\rand B_1(0)} \abs{g} + \brackets{e^- e^{x_1 + 1}} * \max_{\quer{B_1(0)}} \abs{h}$. Es ist $\Delta v(x) = -e^{x_1 + 1} * \max_{\quer{B_1(0)}} \abs{h} \le - \max_{\quer{B_1(0)}} \abs{h}$. Somit ist
		\begin{equation*}
			- \Delta (u-v) (x) = - \Delta u(x) + \Delta v(x) = f(u,x) - \Delta v(x) \le h(x) - \max_{\quer{B_1(0)}} \abs{h} \le 0 \quad \forall x \in B_1(0)
		\end{equation*}
		
		Damit ist $u-v$ subharmonisch und mit dem Maximumsprinzip gilt auch hier, dass das Maximum in einem $x_0 \in \rand B_1(0)$ angenommen wird. Dementsprechend gilt
		\begin{equation*}
			\max_{\quer{B_1(0)}} (u-v) \le \max_{\rand B_1(0)} (u-v) \le \max_{\rand B_1(0)} u + \max_{\rand B_1(0)} (-v) = \max_{\rand B_1(0)} \abs{g} - \max_{\rand B_1(0)} \abs{g} = 0
		\end{equation*}
		sowie daraus folgend
		\begin{equation*}
			\begin{aligned}
				\max_{\quer{B_1(0)}} u 
				\le \max_{\quer{B_1(0)}}(u-v) + \max_{\quer{B_1(0)}} v 
				&= \max_{\quer{B_1(0)}} v \\
				&= \max_{\rand B_1(0)} \abs{g} + \max_{\quer{B_1(0)}} \abs{h} * \max_{\quer{B_1(0)}} \brackets{e^2 - e^{x_1 + 1}} \\
				&= \max_{\rand B_1(0)} \abs{g} + \large( e^2 - \underbrace{e^0}_{=1} \large) * \max_{\quer{B_1(0)}} \abs{h}
			\end{aligned}
		\end{equation*}
		Die gleiche Rechnung funktioniert auch mit $-u$ bzw. $f(u,x) = -h(x)$, sodass $c \defeq e^2 - 1$ vollständig unabhängig ist.
	\end{enumerate} 
	
	
	\begin{task}
		Für $t>0$ ist die Fundamentallösung der Wärmeleitungsgleichung auf $\Rn$ gegeben durch $u(x,t) = (c t)^{-n/2} e^{-\frac{|x|^2}{4t}}$ mit einer Konstante $c>0$. Zeigen Sie:
		\begin{enumerate}
			\item $u_t = \Delta u$ auf $\Rn \times \R_+$.
			\item $\lim_{t\to 0+} u(x,t)= 0$ für $x\neq 0$, $\lim_{t\to 0+} u(0,t) = \infty$ und $\lim_{t\to\infty} u(x,t)=0$.
			\item $\int_{\Rn} u(x,t)dx = 1$ für alle $t>0$ und $c= 4\pi$.
		\end{enumerate}
	\end{task}

	\begin{enumerate}[label=(zu \alph*), leftmargin=*]
		\item Es sei $u(x,t) = (ct)^{-\frac{n}{2}} * e^{- \frac{\abs{x}^2}{4t}}$. Dann ist
		\begin{align*}
			u_t(x, t) &= c^{-\frac{n}{2}} \brackets{- \frac{n}{2}} t^{-\frac{n}{2} - 1} e^{-\frac{\abs{x}^2}{4t}} + (ct)^{-\frac{n}{2}} e^{- \frac{\abs{x}^2}{4t}} * \frac{4 \abs{x}^2}{16 t^2} \\
			&= (ct)^{-\frac{n}{2}} * e^{- \frac{\abs{x}^2}{4t}} * \brackets{- \frac{n}{2} t^{-1} + \frac{\abs{x}^2}{4t^2}} \\
			&= (ct)^{-\frac{n}{2}} * e^{- \frac{\abs{x}^2}{4t}} * \brackets{\frac{\abs{x}^2 - 2nt}{4t^2}} \\
			%
			u_{x_i}(x,t) &= (ct)^{-\frac{n}{2}} * e^{- \frac{\abs{x}^2}{4t}} * \brackets{-\frac{2x_i}{4t}} = - (ct)^{-\frac{n}{2}} * e^{- \frac{\abs{x}^2}{4t}} * \frac{x_i}{2t} \\
			u_{x_i x_i}(x) &= (ct)^{-\frac{n}{2}} \brackets{e^{- \frac{\abs{x}^2}{4t}} * \frac{x_i}{2t} * \frac{x_i}{2t} - \frac{1}{2t} e^{- \frac{\abs{x}^2}{4t}}} \\
			&= (ct)^{-\frac{n}{2}} e^{- \frac{\abs{x}^2}{4t}} \brackets{ \frac{x_i^2}{4t^2} - \frac{1}{2t}} \\
			%
			\Delta_x u(x,t) &= \sum_{i=1}^n x_{x_i x_i}(x)  
			= (ct)^{-\frac{n}{2}} e^{- \frac{\abs{x}^2}{4t}} \brackets{\frac{1}{4t^2} * \sum_{i=1}^n x_i^2 - \frac{n}{2t}} \\
			&= \sum_{i=1}^n x_{x_i x_i}(x)  
			= (ct)^{-\frac{n}{2}} e^{- \frac{\abs{x}^2}{4t}} \brackets{\frac{\abs{x}^2 - 2nt}{4t^2}}
		\end{align*}
		Somit gilt also $u_t = \Delta u$ auf $\Rn \times \R_+$.
		
		\item \begin{itemize}
			\item Sei $t_k \to  0+$ für $k \to \infty$ und $x \neq 0$.
			
			\item Sei $t_k \to 0+$ für $k \to \infty$. Dann ist 
			\begin{equation*}
				\lim_{t \to 0+} u(0,t) = \lim_{k \to \infty} (c \ t_k)^{- \frac{n}{2}} = \lim_{k \to \infty} \frac{1}{(c \ t_k)^{\frac{n}{2}}} = \infty
			\end{equation*} 
			
			\item Sei nun $t_k \to \infty$ für $k \to \infty$. Dann ist
			\begin{equation*}
				\lim_{t \to \infty} (c \ t)^{- \frac{n}{2}} e^{- \frac{\abs{x}^2}{4t}} = \lim_{k \to \infty} \frac{1}{(c \ t_k)^{\frac{n}{2}}} * \exp\brackets{\lim_{k \to \infty} - \frac{\abs{x}^2}{4t_k}} = 0 * 1 = 0
			\end{equation*}
		\end{itemize}
		
		\item Wir wollen die Substitution $y_i = \frac{x_i}{2\sqrt{t}}$ verwenden. Dabei ist $\ableitung{x_i} y_i = \frac{1}{2 \sqrt{t}}$. Es gilt
		\begin{align*}
			\int_{\Rn} (ct)^{-\frac{n}{2}} e^{- \frac{\abs{x}^2}{4t}} \dx
			&= (ct)^{-\frac{n}{2}} \int_{\Rn} e^{- \frac{\abs{x}^2}{4t}} \dx \\
			&= (ct)^{-\frac{n}{2}} \int_\R \dots \int_\R e^{\frac{-x_1^2 - \dots - x_n^2}{4t}} \diffskip{x_1} \dots \diff{x_n} \\
			&= (ct)^{-\frac{n}{2}} \int_\R \dots \int_\R \prod_{i=1}^n e^{\frac{-x_i^2}{4t}} \diffskip{x_1} \dots \diff{x_n} \\
			&= (4\pi t)^{- \frac{n}{2}} \prod_{i=1}^n \int_\R e^{-\frac{x_i^2}{4t}} \diffskip{x_i} \\
			&= \pi^{- \frac{n}{2}} \prod_{i=1}^n \int_\R e^{-\frac{x_i^2}{4t}} * \frac{1}{2 \sqrt{t}} \diffskip{x_i} \\
			\overset{\text{subst.}}&{=} \pi^{- \frac{n}{2}} \prod_{i=1}^n \int_\R e^{-y_i^2} \diffskip{y_i} \\
			&= \prod_{i=1}^n \underbrace{\frac{1}{\sqrt{\pi}} \int_\R e^{-y_i^2} \diffskip{y_i}}_{=1} \\
			&= 1
		\end{align*}
	\end{enumerate}
\end{exercisePage}