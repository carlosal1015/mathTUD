\begin{exercisePage}
\begin{task}
	Wir betrachten die \textit{Burgers-Gleichung}
	\begin{equation*}
	\begin{aligned}
	u_t + uu_x &= 0 \qquad \text{für } (x,t) \in \R \times (0,\infty)\\
	u(x,0) &= g(x) \qquad\text{für alle } x \in \R
	\end{aligned}
	\end{equation*}
	\begin{enumerate}
		\item Es sei zuerst $g$ gegeben durch
		\begin{equation*}
			g(x) = \begin{cases} 1 &\text{für } x \geq 0,\\ 0 & \text{für } x < 0 \end{cases}
		\end{equation*}
		Zeigen Sie, dass die Funktionen
		\begin{equation*}
			v(x,t) = \begin{cases} 1 & \text{für } x \geq \tfrac12 t,\\ 0 &\text{für }x < \tfrac12t \end{cases}
			\qquad \und \qquad
			w(x,t) = \begin{cases} 1 & \text{für } x \geq t,\\ \frac{x}{t} & \text{für } x\in [0,t),\\ 0 & \text{für } x < 0 \end{cases}
		\end{equation*}
		schwache Lösungen sind.
		
		\textit{Hinweis}: Es genügt, die Randwerte, die Rankine-Hugoniot-Bedingung auf Sprungkurven und die Differentialgleichung abseits dieser Kurven zu überprüfen.
		
		\item Es sei nun $g$ gegeben durch
		\begin{equation*}
			g(x) = \begin{cases} 1 & \text{für } x\leq 0 \\ 1-x & \text{für } 0 \leq x \leq 1 \\ 0 & \text{für } x \geq 1 \end{cases}
		\end{equation*}
		\begin{enumerate}
			\item Offenbar ist $g \notin C^1(\R)$. Stellen Sie dennoch die charakteristischen Gleichungen auf und ermitteln Sie, welche Lösung man formal für $t<1$ erwarten würde.
			\item In welchen Punkten $(x,t)\in \R \times (0,1)$ gilt $u_t + u u_x = 0$?
			\item Setzen Sie $u$ derart auf $\R \times (0,\infty)$ fort, dass nur eine Sprungkurve existiert und dass entlang dieser die Rankine-Hugoniot-Bedingung erfüllt ist.
			
			\textit{Hinweis}: Es bietet sich an, dass $u$ für $t \geq 1$ nur die Werte $0$ und $1$ annimmt.
		\end{enumerate}
	\end{enumerate}
\end{task}

Für die Burgers-Gleichung gilt in Anlehnung an die Notation der Vorlesung $F(u)_x = u u_x$ und damit $F(u) = \frac{1}{2} u^2$.
\begin{enumerate}[label=(zu \alph*), leftmargin=*]
	\item Wir betrachten die Funktion $v$. Für die Randwerte, d.h. für $y=0$ gilt 
	\begin{equation*}
		v(x,0) = \begin{cases} 1 &\text{für } x \geq 0 \\ 0 & \text{für } x < 0 \end{cases} \enskip = g(x) \qquad \text{für alle } x \in \R
	\end{equation*}
	Die Sprungkurve ist gegeben durch $s(t) = \frac{1}{2} t$ mit $\dot{s} = \frac{1}{2}$. Es gilt $\dsq{v} = -1$ und $\dsq{F(v)} = 0 - \frac{1}{2} = \frac{1}{2}$, also ist mit $\dsq{F(v)} = \dot{s} * \dsq{v}$ die Rankine-Hugoniot-Bedingung erfüllt.
	Betrachte die Differentialgleichung abseits der Sprungkurve:
	\begin{itemize}
		\item Sei $x > \frac{1}{2} t$. Dann ist $v$ gegeben durch $v(x,t) = 1$ und somit $v_t = v_x \equiv 0$. Eingesetzt in die PDE ergibt dies $v_t + v v_x = 0 + 1 * 0 = 0$. $\checkmark$ 
		\item Sei $x < \frac{1}{2} t$. Dann ist $v$ gegeben durch $v(x,t) = 0$ und somit $v_t = v_x \equiv 0$. Eingesetzt in die PDE ergibt dies $v_t + v v_x = 0 + 0 * 0 = 0$. $\checkmark$ 
	\end{itemize}

	Betrachten wir nun die Funktion $w$. Die Randwerte werde wegen
	\begin{equation*}
		w(x,0) = \begin{cases} 1 &\text{für } x \geq 0 \\ 0 & \text{für } x < 0 \end{cases} \enskip = g(x) \qquad \text{für alle } x \in \R
	\end{equation*}
	erfüllt. Wir erhalten hier zwei Sprungkurven:
	\begin{itemize}
		\item Die Winkelhalbierende können wir durch $s_1(t) = t$ parametrisieren, also ist $\dot{s} = 1$. Dann ergibt sich für die Differenzen
		\begin{equation*}
			\dsq{F(w)} = \frac{1}{2} - \frac{1}{2} = 0  \quad \und \quad \dsq{w} = 1 - \frac{t}{t} = 0 \quad \follows \quad \dsq{F(w)} = \dot{s_1} * \dsq{w}
		\end{equation*}
		\item Für die andere Sprungkurve, die wir mit $s_2(t) = 0$ parametrisieren ($\dot{s} = 0$), erhalten wir 
		\begin{equation*}
		\dsq{F(w)} = 0 - 0 = 0  \quad \und \quad \dsq{w} = \frac{0}{t} - 0 = 0 \quad \follows \quad \dsq{F(w)} = \dot{s_2} * \dsq{w}
		\end{equation*}
	\end{itemize}
	Die Differentialgleichung wird abseits der Sprungkurven auch erfüllt:
	\begin{itemize}
		\item Sei $x > t$. Dann ist $w$ gegeben durch $w(x,t) = 1$ und somit $w_t = w_x \equiv 0$. Eingesetzt in die PDE ergibt dies $w_t + w w_x = 0 + 1 * 0 = 0$. $\checkmark$ 
		\item Sei $x \in [0,t)$. Dann ist $w$ gegeben durch $w(x,t) = \frac{x}{t}$ und somit $w_t = \frac{- x}{t^2}$ und $w_x = \frac{1}{t}$. Eingesetzt in die PDE ergibt dies $w_t + w w_x = \frac{- x}{t^2} + \frac{x}{t} \frac{1}{t} = 0$. $\checkmark$ 
		\item Sei $x < 0$. Dann ist $w$ gegeben durch $w(x,t) = 0$ und somit $w_t = w_x \equiv 0$. Eingesetzt in die PDE ergibt dies $w_t + w w_x = 0 + 0 * 0 = 0$. $\checkmark$ 
	\end{itemize}

	Damit sind also $v$ und $w$ schwache Lösungen der partiellen Differentialgleichung.
	
	\item Mit $y \defeq (x,t)$, $a(u,y) = \left( \begin{smallmatrix} u \\ 1 \end{smallmatrix} \right)$ und $b(u,y) = 0$ hat die Burgergsgleichung die quasilineare Form der Vorlesung. Dann sind die charakteristischen Gleichungen gegeben durch
	\begin{equation*}
		\begin{aligned}
			\dot{y}(\tau, \sigma) &= \left( \begin{smallmatrix} \dot{x} \\ \dot{t} \end{smallmatrix} \right) = a(\alpha, y) = \left( \begin{smallmatrix} \alpha \\ 1 \end{smallmatrix} \right) \\
			\dot{\alpha}(\tau, \sigma) &= a(\alpha, y) * Du = - b(u,y) = 0 \enskip \mit \enskip \alpha(0,\sigma) = g(\sigma)
		\end{aligned}
	\end{equation*}
	Aus der zweiten Gleichung erhalten wir die Lösung $\alpha(\tau, \sigma) = g(\sigma)$, d.h. $u$ ist entlang der Charakteristiken konstant. Aus der ersten Gleichung erhalten wir zum einen die Identifizierung $\tau = t$, d.h. die Charakteristiken können durch die Zeit $t$ parametrisiert werden. Zum anderen die gewöhnliche Differentialgleichung $\dot{x} = \alpha$ mit der Lösung $x(\tau, \sigma) = \alpha(\tau,\sigma) * \tau + s = g(\sigma) * \tau + s$ bzw. mit der Identifizierung dann $x(t) = g(\sigma) * t + \sigma$. Setzen wir die Definition von $g$ ein, so erhalten wir
	\begin{equation*}
		x(t) = \begin{cases}
			t + s       & s \le 0 \\
			(1-s) t + s & 0 \le s \le 1 \\
			s           & s \ge 1
		\end{cases}
	\end{equation*}
	Für $t \le 1$ können wir jeden Fall umstellen und erhalten
	\begin{equation*}
		s = \begin{cases}
			x - t & x - t \le 0\\
			\frac{x-t}{1-t} & x-t \ge 0 \land x \le 1 \\
			x & x \ge 1
		\end{cases}
	\end{equation*}
	Nach Konstruktion erhalten wir dann die Lösung
	\begin{equation*}
		u(x,t) = \alpha(\tau,\sigma) = g(s(t,x)) = \begin{cases}
			1                   & x - t \le 0 \\
			1 - \frac{x-t}{1-t} & x-t \ge 0 \land x \le 1 \\
			0                   & x \ge 1
		\end{cases} 
	\end{equation*}
	Sei nun $t \in (0,1)$. Wir prüfen die Gültigkeit der Differentialgleichung:
	\begin{itemize}
		\item Sei $x - t \le 0$.  Dann ist also $u(x,t) = 1$ und $u_x = u_t \equiv 0$ und die Differentialgleichung erfüllt. 
		\item Ist $x-t \ge 0$ und $x \le 1$, dann ist $u(x,t) = 1 - \frac{x-t}{1-t}$ und $u_x(x,t) = \frac{1}{1-t}$ sowie $u_t(x,t) = \frac{(t-1)+(x-t)}{(1-t)^2}$. In die Differentialgleichung eingesetzt ergibt dies
		\begin{equation*}
			u_t + u u_x = \frac{(t-1)+(x-t)}{(1-t)^2} + \frac{1}{1-t} - \frac{x-t}{(1-t)^2} = 0 \quad \checkmark
		\end{equation*}
		\item Sei $x \ge 1$. Dann ist $u = u_x = u_t \equiv 0$ und die Differentialgleichung damit erfüllt.
	\end{itemize}
	Somit ist $u$ Lösung der Differentialgleichung für alle $x \in \R$ und $t \in (0,1)$.
\end{enumerate}
\end{exercisePage}