\documentclass[a4paper, 12pt]{article}

\usepackage[english]{babel}
\usepackage[utf8]{inputenc}

\usepackage{lipsum}

\usepackage[top=2.5cm,bottom=2.5cm,left=2.5cm,right=2.5cm]{geometry}
\usepackage{parskip}  	% split paragraphs by vspace instead of intendations
\usepackage[onehalfspacing]{setspace} % increase row-space
%\usepackage[title,titletoc]{appendix}
%
%\usepackage{chngcntr}
%\usepackage{eufrak}
%
\usepackage{lmodern}
%\usepackage[default, scale=0.9]{opensans}
\usepackage[normalem]{ulem} 
\usepackage{csquotes}

\usepackage{graphicx}

%\usepackage{fancyhdr} 	% customize header / footer
%\usepackage{tocloft}
%%\renewcommand{\cfttoctitlefont}{\titlefont\Huge\bfseries}
%%\renewcommand{\cftbeforetoctitleskip}{0pt}
%%\renewcommand{\cftchapnumwidth}{2em}
%%\renewcommand{\cftsecindent}{2em}
%%\renewcommand{\cftsecnumwidth}{2em}
%%\renewcommand{\cftsubsecindent}{4em}
%
%\usepackage{amsmath,amssymb,amsfonts,mathtools}
%\usepackage{blkarray}
%\usepackage{latexsym}
%\usepackage{marvosym} 	% lightning (contradiction)
%\usepackage{stmaryrd} 	% Lightning symbol
%\usepackage{bbm} 		% unitary matrix
%\usepackage{wasysym}	% add some symbols
%
%\usepackage{systeme}	% easy typesetting systems of equations
%\usepackage{witharrows} % arrows from one equation to another
%
%% further support for different equation setting
%\usepackage{cancel}
%\usepackage{xfrac}		% sfrac -> fractions e.g. 3/4
%\usepackage{units}		% units and fractions
%\usepackage{diagbox}
%
%\usepackage{../texmf/tex/latex/mathoperatorsMathTUD}
%
%\usepackage[table,dvipsnames]{tudscrcolor}
%\usepackage{tabularx} 	% tabularx-environment (explicitly set width of columns)
%\usepackage{longtable} 	% Tabellen mit Seitenumbrüchen
%\usepackage{multirow}
%\usepackage{booktabs}	% improved rules
%\usepackage{colortbl}
%
\newcommand{\begriff}[1]{\textit{#1}}
\newcommand{\person}[1]{\textsc{#1}}
%
%%%%%%%%%%%%%%%%%%%%%%%%%%%%%%%%%%%%%%%%%%%%%%%%%%%%%%%%%%%%%%%%%%%%
%%                             COUNTER                             %
%%%%%%%%%%%%%%%%%%%%%%%%%%%%%%%%%%%%%%%%%%%%%%%%%%%%%%%%%%%%%%%%%%%%
%\pretocmd{\chapter}{\setcounter{section}{0}}{}{}
%\pretocmd{\chapter}{\setcounter{equation}{0}}{}{}
%
%\usepackage{enumerate}
%\usepackage[inline]{enumitem} 		%customize label
%
%\renewcommand{\labelitemi}{\raisebox{2pt}{\scalebox{.4}{$\blacksquare$}}}
%\renewcommand{\labelitemii}{$\vartriangleright$}
%\renewcommand{\labelitemiii}{--}
%% Variantionen des Dreiecks als Aufzählungszeichen $\blacktriangleright$ / $\vartriangleright$ / $\triangleright$
%
%\renewcommand{\labelenumi}{(\arabic{enumi})}
%\renewcommand{\labelenumii}{\alph{enumii}.}
%\renewcommand{\labelenumiii}{\roman{enumiii}.}
%
%%%%%%%%%%%%%%%%%%%%%%%%%%%%%%%%%%%%%%%%%%%%%%%%%%%%%%%%%%%%%%%%%%%%
\usepackage{titlesec}   % change title headings look
\usepackage{chngcntr}   % modify counters
\usepackage{relsize}    % relative font size (smaller[i], larger[i], ...)
%
%
%%%%%%%%%%%%%%%%%%%%%%%%%%%%%%%%%%%%%%%%%%%%%%%%%%%%%%%%%%%%%%%%%%%%
%% headings
%%%%%%%%%%%%%%%%%%%%%%%%%%%%%%%%%%%%%%%%%%%%%%%%%%%%%%%%%%%%%%%%%%%%
%\newcommand{\titlefont}{\osfamily}
%\newcommand{\chaptersize}{\huge}
%\newcommand{\sectionsize}{\LARGE}
%
%\renewcommand{\thepart}{\Alph{part}}
%
%% \titleformat{<command>}[<shape>]{<format>}{<label>}{<sep>}{<before-code>}[<after-code>]
%% \titlespacing*{<command>}{<left>}{<before-sep>}{<after-sep>}[<right-sep>]
%
%%%%%%%%%% section
%%\titlelabel{\thetitle \quad} % no "." behind section/sub... (3 instead of 3.)
\titleformat{\section}[hang]{\centering\bfseries\normalsize\MakeUppercase}{\thesection}{8pt}{}%
\titleformat*{\section}{\centering\bfseries\normalsize\MakeUppercase}
\titlespacing*{\section}{0pt}{\parskip}{0pt}
%%\titleformat*{\subsection}{\bfseries\titlefont\sectionsize\smaller}
%
%%%%%%%%%%%%%%%%%%%%%%%%%%%%%%%%%%%%%%%%%%%%%%%%%%%%%%%%%%%%%%%%%%%%
%
%\usepackage{listings}
%
%%%%%%%%%%%%%%%%%%%%%%%%%%%%%%%%%%%%%%%%%%%%%%%%%%%%%%%%%%%%%%%%%%%%
%%                           REFERENCES                            %
%%%%%%%%%%%%%%%%%%%%%%%%%%%%%%%%%%%%%%%%%%%%%%%%%%%%%%%%%%%%%%%%%%%%
%
%\usepackage{titlesec}   % change title headings look
%\usepackage{relsize}    % relative font size (smaller[i], larger[i], ...)
%
%\usepackage{titling}
%%\pretitle{\begin{center}\Huge\bfseries\sffamily}
%%\posttitle{\par}
%%\preauthor{\par \normalfont \large \scshape}
%%\postauthor{\par}
%%\postdate{\end{center}
%
%\DeclareMathSymbol{*}{\mathbin}{symbols}{"01}
%
%\counterwithin{equation}{section}
%\newcounter{themcount}
%\counterwithin{themcount}{section}
%\usepackage{ntheorem}
%
%\newcommand{\skiparound}{10pt}
%\theorempreskip{\skiparound}
%\theorempostskip{\skiparound}
%
%\theoremstyle{plain}
%\theoremseparator{.}
%\theorembodyfont{}
%\newtheorem{definition}[themcount]{Definition}
%\newtheorem{lemma}[themcount]{Lemma}
%\newtheorem{satz}[themcount]{Satz}
%
%\newtheorem{beispiel}[themcount]{Beispiel}
%
%\theorembodyfont{\itshape}
%\newtheorem{bemerkung}[themcount]{Bemerkung}
%
%%\newtheoremstyle{proofstyle}%
%%{\item[\hskip\labelsep {\theorem@headerfont ##1}\theorem@separator]}%
%%{\item[\hskip\labelsep {\theorem@headerfont ##1}\ (##3)\theorem@separator]}
%
%\theoremstyle{nonumberplain}
%\theoremheaderfont{\normalsize\slshape}
%\theorembodyfont{}
%\theoremseparator{.}
%\theorempreskip{5pt}
%\theorempostskip{5pt}
%\theoremsymbol{$\square$}
%\newtheorem{proof}{Beweis}
%
%\usepackage[
%type={CC},
%modifier={by-nc-sa},
%version={4.0},
%]{doclicense}
%
%\usepackage[unicode,bookmarks=true]{hyperref}
%\hypersetup{
%	% pdfborder={0 0 0}			% no boxed around links
%	pdfborderstyle={/S/U/W 1},	% underlining insteas of boxes
%	linkbordercolor=cdblue,
%	urlbordercolor=cdblue
%	%	colorlinks,
%	%	citecolor=black,
%	%	filecolor=cddarkblue!80,
%	%	linkcolor=black,
%	%	urlcolor=cddarkblue!80
%}
%
%\usepackage{cleveref}
%\crefname{satz}{Satz}{Sätze}
%\crefname{lemma}{Lemma}{Lemmata}
%\crefname{definition}{Definition}{Definitionen}
%\crefname{bemerkung}{Bemerkung}{Bemerkungen}
%\crefname{beispiel}{Beispiel}{Beispiele}
%\usepackage{bookmark}		% pdf-bookmarks

\usepackage{abstract}
\renewcommand{\descriptionlabel}[1]{\hspace{\labelsep}\textbf{#1}:}

\begin{document}
	
	\hspace{-18.6mm}
	\includegraphics[scale=0.6]{TUD-blue.pdf} \\
		\setlength{\tabcolsep}{0pt}
		\begin{tabular}{p{\textwidth}}
			\hline
			\small \sffamily \textbf{Math--Ba--EBWII --- Oral communication in university and career}  Mr Mueller  \\
			\hline
		\end{tabular} 
		\par 
		\vspace{3\parskip}
	\begin{center}
		\bfseries \large \MakeUppercase{Formal Languages}  \\
		\mdseries \smaller with Respect to the \person{Chomsky}-\person{Schützenberger} Hierarchy
		
		{\scshape \normalsize Eric Kunze} \\		
		{\footnotesize 27 January 2020}
	\end{center}
	
	\vspace{2\parskip}
	
%%%%%%%%%%%%%%%%%%%%%%%%%%%%%%%%%%%%%%%%%%%%%%%%%%%%%%%%%%%%%%%%%%%%%%%%%%%%%%%%

	\section*{Outline}
	\begin{enumerate}
		\item Foundations of Formal Languages
		\item Formal Grammar and the \person{Chomsky}-Hierarchy
		\item Consequences and applications of Formal Languages
	\end{enumerate}
	

	\section*{Keywords}
	Formal Language, Formal Grammar, \person{Chomsky}-\person{Schützenberger} Hierarchy, Logic, Computability Theory
	
	\section*{Abstract}
%	\begin{onehalfspace}
	Every natural language complies with long time developed rules called grammar. A \begriff{formal language} describes the syntax of a language in a more formal way as a set of \begriff{words} over an (arbitrary) \begriff{alphabet}. In this way the structured construction, analysis and classification of languages are possible. In general, there are two methods of describing the syntax of words: generating strings by replacement starting with a fixed start symbol, or accepting given strings via an automaton. As \person{Noam Chomsky} in the 1950’s did we focus on the former and analyse (formal) \begriff{grammar} connected with their generated languages. Afterwards we can classify languages by differentiating the corresponding grammar and especially their replacement behaviour. As result we get the so-called \begriff{Chomsky-Schützenberger Hierarchy} which distributes languages into four types: recursively enumerable, context-sensitive, context-free and regular languages, which are more restrictive in each step. This knowledge can be applied both in mathematics and computer science. Defining a language of logic, we get the fundament of (pure) mathematics and we can formalize terms like e.g. \enquote{proof} or \enquote{theorem}. Furthermore, problems of computability theory can be specified and proven with the result that we are able to express the limits of computation.
%	\end{onehalfspace}
	
	\section*{Vocabulary}
%	\begin{onehalfspace}
	\begin{description}
		\item[Alphabet $\Sigma$] set of symbols
		\item[Language $L$] subset of words over an alphabet $\Sigma$ ($L \subseteq \Sigma^\ast$)
		\item[Grammar $G = (N, \Sigma, P, S)$] Language description with an (finite) alphabet $\Sigma$ (terminal symbols), a (finite) set $N$ of non-terminal symbols, a set $P$ of productions (replacement rules) and a start symbol $S \in N$
		\item[Production $A \to aAb$] replacement of $A$ by $aAb$
		\item[Chomsky-Schützenberger Hierarchy] Classification of formal languages into four types
		\item[(Mathematical) Logic] theory of propositions and their logical connections; fundament of mathematics; propositional logic can be defined as a formal language
		\item[Computability Theory] examines the question if a function is computable and especially with which (mathematical model of a) machine the computation can be done 
	\end{description}
%	\end{onehalfspace}

	\nocite{*}	
	\bibliographystyle{abbrv}
	\bibliography{FL_bib}
	
\end{document}