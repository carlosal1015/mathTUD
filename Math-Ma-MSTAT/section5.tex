% !TEX root = MSTAT19.tex
% This work is licensed under the Creative Commons
% Attribution-NonCommercial-ShareAlike 4.0 International License. To view a copy
% of this license, visit http://creativecommons.org/licenses/by-nc-sa/4.0/ or
% send a letter to Creative Commons, PO Box 1866, Mountain View, CA 94042, USA.

\section{Verteilungskonvergenz in \texorpdfstring{$ℝ^d$}{R\textasciicircum d}} %5
% Korollar \ref{korollar4.5} bzw.\ dessen Erweiterungen \ref{lemma4.6Einhalb} \ref{it:4.6multiDim} liefern eine \define{analytische} Methode zum Nachweis von $X_n\distrto X\text{ in }ℝ^d$.
% Eine weitere Methode fußt auf
\paragraph{Ziel} Beschreibe $X_n \distrto X$ in $ℝ^d$. Dazu
\begin{definition}\label{def5.1}
	Sei $X$ Zufallsvariable in $ℝ^d$ über $(Ω,\A,\P)$ und
	\begin{align*}
		\scaProd xy :=:x'y:=\sum_{i=1}^d x_i· y_i\qquad x=(x_1,…,x_d),y=(y_1,…,y_d)∈ℝ^d
	\end{align*}
	das Standard-Skalarprodukt in $ℝ^d$. Dann heißt
	\begin{align*}
		φ_X(t):=\Earg[\big]{\exp\klammern[\big]{\ii \scaProd tX}}\qquad∀ t∈ℝ^d
	\end{align*}
	\define{charakteristische Funktion} von $X$.
\end{definition}

\begin{satz}[Eindeutigkeitssatz]\label{satz5.2Eindeutigkeitssatz}
	\begin{align*}
		X\stackeq{\L} Y⇔φ_X\equivφ_Y
	\end{align*}
\end{satz}

\begin{proof}
	Siehe Buch \cite[Seite 107f.]{jacod2012probability}.% \undefine{Essentials in Probability} von Jacod und Protter (2000), Seite 107-108.
\end{proof}

\begin{satz}[Stetigkeitssatz für charakteristische Funktionen]\label{satz5.3Stetigkeitssatz}
	\begin{align*}
		X_n\distrto  X\text{ in }ℝ^d⇔∀ t∈ℝ^d: φ_{X_n}(t)\ntoinf φ_X(t)
	\end{align*}
\end{satz}

\begin{proof}
	Siehe Vorlesung Wahrscheinlichkeitstheorie (Bachelor) oder \cite[Seite 163ff.]{jacod2012probability}.%Jacod und Protter (2000), Seite 163 ff.
\end{proof}
Der Satz \ref{satz5.3Stetigkeitssatz} wird üblicherweise genutzt um den zentralen
Grenzwertsatz zu beweisen.

Sehr nützlich ist:
\begin{satz}[Cramér-Wold-Device]\label{satz5.4CramerWoldDevice}%
	\footnote{Device bedeutet u.\,A.\ Trick. Das ist kein Name.}
	\footnote{Beliebte Prüfungsfrage laut Prof.\ Ferger.}
	Folgende Aussagen sind äquivalent:
	\begin{enumerate}[label=(\arabic*)]
		\item \label{it:5.4X} $\begin{aligned}
			X_n\distrto  X\text{ in }ℝ^d
		\end{aligned}$
	\item \label{it:5.4tX} $\begin{aligned}
			\scaProd t{X_n} \distrto \scaProd tX \text{ in } ℝ \qquad ∀ t ∈ ℝ^d
		\end{aligned}$
	\end{enumerate}
\end{satz}

\begin{proof}
	Sei
	\begin{align*}
		φ_X(t)
		&\stackeq{\text{Def}}\Earg{\exp\klammern[\big]{\ii·\scaProd tX}}
		\stackeq{d=1}\Earg[\big]{\exp(\ii· X· t)}
		\qquad∀ t∈ℝ^d
	\end{align*}
	\paragraph{Zeige \ref{it:5.4X} $⇒$ \ref{it:5.4tX}:}
	Sei $t∈ℝ^d$. Dann ist $x ↦ \scaProd tx$ stetig auf $ℝ^d$.
	Aus Satz \ref{satz4.10ContinuousMappingTheorem} (CMT) folgt \ref{it:5.4tX}.
	\paragraph{Zeige \ref{it:5.4tX} $⇒$ \ref{it:5.4X}:}
	\begin{align*}
		&φ_{X_n}(t)
		\stackeq{\text{Def}}\Earg[\Big]{\exp\klammern[\big]{\ii·\scaProd t{X_n}· 1}}
		\stackeq{\text{Def}}φ_{\scaProd t{X_n}}(1)
		\stackrelnew{\ref{satz5.3Stetigkeitssatz}+ \ref{it:5.4tX}}{n⟶∞}{\longrightarrow}
		φ_{\scaProd tX}(1) = φ_X(t) \\
		& ⇒ φ_{X_n}\ntoinf  φ_X \text{ auf } ℝ^d
		\overset{\ref{satz5.3Stetigkeitssatz}}{⇒} \ref{it:5.4X}
		\qedhere
	\end{align*}
\end{proof}
