% This work is licensed under the Creative Commons
% Attribution-NonCommercial-ShareAlike 4.0 International License. To view a copy
% of this license, visit http://creativecommons.org/licenses/by-nc-sa/4.0/ or
% send a letter to Creative Commons, PO Box 1866, Mountain View, CA 94042, USA.

\subsection{Das Continuous Mapping Theorem (CMT)}
Sei $h \colon (\S,d) \to (\S',d')$ messbar.
Wir wollen nun Bedingung an $h$ finden, sodass die beiden folgenden Implikationen gelten:
\begin{enumerate}[label=(\arabic*)]
	\item \label{it: 4.1conditionhweakto} 
	$\begin{aligned}
		P_n \weakto P \follows P_n \circ h^{-1} \weakto P \circ h^{-1}
	\end{aligned}$
	\item \label{it: 4.1conditionhDistrto} 
	$\begin{aligned}
		X_n \distto X \text{ in } (\S,d) \follows h(X_n) \distto  h(X) \text{ in }(\S',d')
	\end{aligned}$
\end{enumerate}
Man beachte dabei, dass
\begin{align*}
	P \circ \brackets{h(X_n)}^{-1}
	&= P \circ \brackets{h \circ X_n}^{-1}
	= \brackets{P \circ X_n^{-1}} \circ h^{-1} \\
	P \circ \brackets{h(X)}^{-1}
	&= \brackets{P \circ X^{-1}} \circ h^{-1}
\end{align*}
gilt. Also folgt \ref{it: 4.1conditionhDistrto} aus \ref{it: 4.1conditionhweakto}. 
Ist $h$ stetig auf $\S$, dann gilt \ref{it: 4.1conditionhweakto} schon. Um das zu zeigen, sei $f\in C^b(\S')$ beliebig --- es gilt:
\begin{align*}
	\int_{\S'} f \diffskip \brackets{P_n \circ h^{-1}}
	\overset{\eqref{eqTrafo}}{=}
	\int_{\S} \underbrace{f \circ h}_{\in C^b(\S)} \diffskip{P_n}
	\overset{\text{Def.}}&{\longrightarrow} \int f \circ h \diffskip P
	\overset{\eqref{eqTrafo}}{=}
	\int f \diffskip{(P \circ h^{-1})} \\
	\overset{\text{Def.}}{\follows}
	P_n \circ h^{-1} &\weakto P \circ h^{-1} \satzende
\end{align*}
Auf Stetigkeit von $h$ kann im Allgemeinen nicht verzichtet werden, denn es gilt das nachfolgende Gegenbeispiel.
%
% Nebenbeibemerkung: Prof. Ferger ist bei research gate angemeldet, das Facebook der Wissenschaftler: dort bekommt er etwas Rückmeldung zu den Papern. Bei üblichen Veröffentlichungen gibt es das nicht. Schade.
%
\begin{beispiel}\label{beispiel: 4.8}
	% Beweis unvollständig, als Übung belassen, hier vollständig
	Sei $\S = \S' = [0,1]$ und $h \colon [0,1] \to [0,1]$ mit
	\begin{align*}
		h(x)\defeq
		\begin{cases}
			1, &\falls x \in \menge{0} \cup \menge{\frac{1}{2n} : n \in \N} \\
			0, & \text{sonst} \satzende
		\end{cases}
	\end{align*}
	Sei $P_n \defeq \delta_{\frac{1}{n}}$ ($n\in\N$) und $P \defeq \delta_{0}$, wobei $\delta_x$ das Dirac-Maß bezeichnet.
	Dann gilt $P_n \weakto P$, denn
	\begin{equation*}
		\int f \diffskip{P_n} = f \brackets{\frac{1}{n}} \ntoinf f(0) = \int f \diffskip P \qquad \forall f \in C^b \brackets{[0,1]} \satzende
	\end{equation*}
	Aber wegen $\delta_x \circ h^{-1} = \delta_{h(x)}$ gilt für
	\begin{align*}
		Q_n \defeq P_n \circ h^{-1} 
		= \delta_{\frac{1}{n}} \circ h^{-1}
		=\delta_{h\brackets{\frac{1}{n}}} 
		\komma \qquad 
		Q \defeq P \circ h^{-1} = \delta_1
	\end{align*}
	das Folgende:
	\begin{equation*}
		\int f \diffskip{Q_n} = f \brackets{h \brackets{\frac{1}{n}}}
		=
		\begin{cases}
			f(1), & \falls n \text{ gerade }\\
			0, & \falls n \text{ ungerade } \satzende
		\end{cases}
	\end{equation*}
	Sei $f \in C^b([0,1])$ mit $f(0) \neq f(1)$ (z.\,B.\ $f = \id$).
	Folglich ist die Folge $\brackets{\int f \diffskip{Q_n}}_{n \in \N}$ divergent.
	Somit ist $\int f \diffskip{Q_n} \to \int f \diffskip{Q}$, was $Q_n \weakto Q$ zeigt.
\end{beispiel}

Die Forderung der Stetigkeit lässt sich aber abschwächen so, dass \cref{it: 4.1conditionhweakto} noch gilt.
Dazu definieren wir die \begriff{Menge der Unstetigkeitsstellen}
\begin{equation*}
	D_h \defeq \menge{ x \in \S : h \text{ nicht stetig in } x } \satzende
\end{equation*}

\begin{lemma} \label{lemma: 4.9}
	Für alle $h \colon \S \to \S'$ gilt $D_h \in \B(\S)$. $h$ muss insbesondere nicht einmal messbar sein.
\end{lemma}

\begin{proof}[nach {\cite[Appendix]{billingsley2013convergence}}]
	Mit der Dreiecksungleichung überlegt man sich leicht:
	\begin{align*}
		h \text{ stetig in } x
		\equivalent
		&\forall 0 < \epsilon \in \Q : \exists 0 < \delta \in \Q : \forall y,z \in \S : \\
		&d(x,y) < \delta \und d(x,z) < \delta
		\follows d'\brackets{h(y),h(z)} < \epsilon
		\satzende
	\end{align*}
	Damit folgt
	\begin{equation} \label{eqProof4.9Plus}
		D_h = \bigcup_{ 0 < \epsilon \in \Q} \bigcap_{0 < \delta \in \Q} \underbrace{\menge{x \in \S :
		\exists y,z \in \S \mit d(x,y) < \delta \und d(x,y) < \delta \und d'\brackets{h(y),h(z)} \ge \epsilon}}_{\defqe A_{\epsilon,\delta}} \satzende
	\end{equation}
	Zeige nun
	\begin{equation}\label{eqProof4.9Stern}
		A_{\epsilon,\delta} \in \G(\S) \qquad \forall \epsilon,\delta > 0 \satzende
	\end{equation}
	Dazu sei $x_0 \in A_{\epsilon,\delta}$.
	Dann existieren $y,z \in \S$ mit $d(x_0,y) < \delta$ und $d(z_0,z) < \delta$, aber $d'\brackets{h(y),h(z)} \ge \epsilon$.
	Wähle
	\begin{equation*}
		\delta_0 \defeq \min\menge{\delta -  d(x_0,y), \delta - d(x_0,z)} > 0 \satzende
	\end{equation*}
	Ferner gilt
	\begin{equation}\label{eqProof4.9i}
		B(x_0,\delta_0) \subseteq A_{\epsilon,\delta},
	\end{equation}
	denn: Sei $x \in B(x_0,\delta_0)$.
	Dann gilt $d(x,y) \le d(x,x_0) + d(x_0,y) < \delta_0 + d(x_0,y) < \delta$
	und analog $d(x,z) < \delta$
	Also folgt $x \in A_{\epsilon,\delta}$, denn $d'\brackets{(h(y),h(z)} \ge \epsilon$.
	Wegen \eqref{eqProof4.9i} ist $x_0$ innerer Punkt von $A_{\epsilon,\delta}$.
	Also folgt \eqref{eqProof4.9Stern} und mit \eqref{eqProof4.9Plus} dann die Behauptung.
\end{proof}

\begin{satz}[Continuous Mapping Theorem -- CMT] \label{satz: 4.10ContinuousMappingTheorem}
	Sei $h \colon (\S,d) \to (\S',d')$ $\B_d(\S)$-$\B_{d'}(\S')$-messbar.
	% hier ist D_h BDSM (B_d(\S) messbar)
	Dann gilt:
	\begin{enumerate}[label=(\arabic*)]
		\item \label{it:4.10CMTweakto} $\begin{aligned}
			P_n \weakto  P \und P(D_h) = 0
			\follows P_n \circ h^{-1} \weakto P \circ h^{-1}
		\end{aligned}$
	\item \label{it:4.10CMTdistrto} $\begin{aligned}
			X_n \distto X \text{ in }(\S,d) \und \P(X \in D_h) = 0
			\follows h(X_n) \distto h(X)\text{ in } (\S',d')
		\end{aligned}$
	\item \label{it:4.10CMThcontinous} Sei $h$ stetig. Dann gilt: $X_n \distto X \text{ in } (\S,d) \follows h(X_n) \distto h(X) \text{ in } (\S',d')$.
	\end{enumerate}
\end{satz}

\begin{proof}
	\textit{Der Beweis ist eine beliebte Prüfungsfrage laut Prof.\ Ferger.} \\
	Wegen des Portmanteau-Theorems (\cref{satz: 4.2}) zeigen wir die äquivalente Aussage über den Limes superior.
	\begin{enumerate}[label=(zu \arabic*), leftmargin=*]
		\item 
		Sei $F\in\F(\S')$ (d.\,h.\ $F\subseteq\S'$ abgeschlossen) beliebig.
		Dann gilt:
		\begin{align}\label{eqProof1.4.10(i)}
			\limsup_{n \to \infty} P_n \circ h^{-1}(F)
			&= \limsup_{n \to \infty} P_n \brackets{\underbrace{h^{-1}(F)}_{\subseteq \abschluss{h^{-1}(F)}}} \\\nonumber
			&\le \limsup_{n \to \infty} P_n \brackets{\underbrace{\abschluss{h^{-1}(F)}}_{\in\F(\S)}} \\
			\overset{\ref{satz: 4.2}}&{\le}
			P\brackets{\abschluss{h^{-1}(F)}}
		\end{align}
		Wir zeigen nun, dass
		\begin{equation} \label{eqProof1.4.10(ii)}
			\abschluss{h^{-1}(F)} \subseteq h^{-1}(F) \cup D_h
		\end{equation}
		gilt. Sei dazu $x \in \quer{h^{-1}(F)}$.
		\begin{enumerate}[label=\textbf{Fall \arabic*:}, leftmargin=*]
			\item Sei $x \in D_h$. Dann folgt daraus schon, dass $x \in h^{-1}(F) \cup D_h$.
			\item Sein nun $x \notin D_h$, d.h. $h$ ist stetig in $x$. Per Definition des Abschlusses existiert eine Folge $(x_n)_{n \in \N} \subseteq h^{-1}(F) \subseteq \S$ mit $x_n \ntoinf x$. Für jede solche Folge gilt
			\begin{equation*}
				\underbrace{h(x_n)}_{\in F} \ntoinf h(x)
				\follows h(x) \in \abschluss{F} = F 
				\follows x \in h^{-1}(F)
				\follows x \in h^{-1}(F) \cup D_h \satzende
			\end{equation*} 
		\end{enumerate}
		Mit \eqref{eqProof1.4.10(ii)} setzt sich die Ungleichungskette \eqref{eqProof1.4.10(i)} fort und wir erhalten
		\begin{equation*}
			P\brackets{\quer{h^{-1}(F)}}
			\overset{\eqref{eqProof1.4.10(ii)}}{\le}
			P\brackets{h^{-1}(F) \cup D_h}
			\le P\brackets{h^{-1}(F)} + \underbrace{P(D_h)}_{= 0}
			= P \circ h^{-1}(F) \satzende
		\end{equation*}
		Schließlich liefern  \eqref{eqProof1.4.10(i)} und \cref{satz: 4.2}\ref{it:4.2ClosedSets}  das gewünschte Resultat $P_n \circ h^{-1} \weakto  P \circ h^{-1}$.
		%
		\item Dies folgt direkt aus \ref{it:4.10CMTweakto} nach Definition \cref{definition: 4.1}, denn es ist
		\begin{align*}
			P \circ (h(X_n))^{-1} = P \circ (h \circ X_n)^{-1} = (P \circ X_n^{-1}) \circ h^{-1}
			&\weakto P \circ X^{-1} \\
			&\weakto (P  \circ X^{-1}) \circ h^{-1} = P \circ (h(X))^{-1} \\
			\overset{\text{Def}}{\follows} h(X_n) &\distto h(X) \satzende
		\end{align*}
		%
		\item Dies folgt wiederum direkt aus \ref{it:4.10CMTdistrto}, da für eine stetige Funktion $h$ die Menge $D_h$ stets leer ist.
	\end{enumerate}
\end{proof}

\begin{bemerkung}[Choque-Kapazität]  
	\label{note: choque_kapazitat}
	Es gibt eine Verallgemeinerung des Wahrscheinlichkeitsbegriffs: die so genannte \begriff{Choque-Kapazität}. Auf der Menge aller Choque-Kapazitätäten lässt sich eine Topologie $\tau_{\text{weak}}$ einführen (die so genannte \emph{schwache Topologie}) und ein entsprechendes Portmanteau-Theorems beweisen, vgl. \cite{Ferger2018WeakCO} und \enquote{Dietmar Ferger (2019): \textsl{unveröffentlicht}}.
\end{bemerkung}