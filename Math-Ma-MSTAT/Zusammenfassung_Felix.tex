% This work is licensed under the Creative Commons
% Attribution-NonCommercial-ShareAlike 4.0 International License. To view a copy
% of this license, visit http://creativecommons.org/licenses/by-nc-sa/4.0/ or
% send a letter to Creative Commons, PO Box 1866, Mountain View, CA 94042, USA.

\newcommand{\directoryPrefix}{../latex/} % Je nach Ordnertiefe muss dieser Command angepasst werden. Bei Fragen mich anschreiben.
\input{\directoryPrefix templates}
\TemplateSummary{Felix Hilsky, Basis: Willi Sontopski}{MSTAT}
\input{\directoryPrefix commands_Felix}
\setlist{nosep}
\hfuzz=1.5cm  % silence overfull hboxes less than 2cm
\begin{document}
Alle Beweise und Sätze, die Prof.\ Ferger als typische Prüfungsfragen bezeichnet hat, finden sich durch eine Suche nach \enquote{Prüfung} im Skript.
	\section{Der Median}
	\begin{itemize}
		\item $m$ Median $\gdw m=\argmin_{t}(\Earg{\abs{X-t}})\gdw F(m-)≤\frac{1}{2}≤ F(m)\gdw \Earg{\abs{X-t}}≥\Earg{\abs{X-m}}~∀ t∈ℝ$
		% \item Trafo: $∫_Ω g(X(ω))~\P(\dω)  % keine Ahnung, was das hier sucht
		% =∫_ℝ g(x)~\P_X(\d x)
		% =∫_ℝ g(x)· f_X(x)\ds x$ ($f_X$ Dichte von $P_{X} = P ∘ X^{-1}$)
	\item Median i.\,A.\ \betone{nicht} eindeutig, wähle $m:=F^{-1}(\frac12)\mit F^{-1}(u):=\inf\set{ x∈ℝ:F(x)≥ u}$ \define{verallg.\ Inverse / Quantilfunktion}; $Y$ stetig, aber nicht diffbar; Minimierung über Ableiten unmöglich
		\item empir.\ Median $\hat{m_n}:=\argmin_{t∈ℝ}Y_n(t)\mit Y_n(t) := \frac{1}{n} \sum_{i=1}^n \abs{x_i-t}$
		\item SGGZ sagt: $Y_n(t) \ntoinf Y(t) := ∫_ℝ \abs{x-t} ~ F(\d x)$ f.\,s.\ $∀ t ∈ ℝ$.
	\end{itemize}

	\section{Konzepte aus metrischen Räumen}
	\begin{itemize}
		\item $\S$ metrischer Raum; $\F$ Menge der offenen Teilmengen und $\G$ Menge der geschlossenen Teilmengen
		\item \enquote{Bump-Function}: $∀ A⊆\S,∀ε:∃ f_A:\S⟶[0,1]$ glm.\ stetig s.\,d.\ $f_A \approx \indi_A$ ($f_A(x) = 0$ für $d(x, A) ≥ ε$)
		\item MR \define{separabel} $:\gdw ∃ S_0 ⊆ \S$ abzählbar s.\,d.\ $\abschluss{S_0} = \S \gdw ∃ S_0 ⊆ \S$ abzählbar mit $S_0$ liegt dicht in $\S$
		% CHECKED: '\abschluss' used.
		\item $\G_0⊆\G$ \define{Basis} $:\gdw∀ G∈\G:G$ ist $\Union$ von Mengen aus $\G_0$; $\S$ separabel $\gdw\G(\S)$ hat abzählbare Basis
		\item \define{Produktmetriken}: $d_1× d_2:∈\set[\big]{ √{d_1^2+d_2^2}, d_1+d_2,\max(d_1,d_2)}$; sind äquivalent, erzeugen selbe Topo
	\end{itemize}

	\section{Zufallsvariablen in metrischen Räumen}
	\begin{itemize}
		\item Borel-$σ$-Algebra auf $\S$ ist $\B(\S):=σ(\G(\S))=σ(\F(\S))\overset{\G_0\text{ abz- Basis}}{=}σ(\G_0)$, hängt. i.\,A.\ von $d$ ab.
		\item $\B_{d_1 × d_2}(\S_1 × \S_2) = \B_{d_1}(\S_1) \tens \B_{d_2}(\S_2)$ für \emph{separable} MR
		\item \define{Zufallsvariable} ist $X\colonΩ⟶\S$ die $\A$-$\B(\S)$-messbar ist ($(Ω,\A)$ Messraum, $(\S,d)$ MR).
		\item \define{Verteilung} von $X$ unter $\P$ ($(Ω,\A,\P)$ WR) ist
		$(\P∘ X^{-1})(B):=\P[X∈ B]~∀ B∈\B(\S)$
		\item $(\S,d)$ separabel, $X,Y$ ZV $⇒ d(X,Y)$ ist reelle ZV (jede Metrik stetig)
		\item $X_n\ntoinf  X~\P\text{ f.\,s.}
			:\gdw \P\argu{\set{ω ∈ Ω : d \argu{X_n(ω), X(ω)} \ntoinf 0}} = 1 \gdw d(X_n, X) \ntoinf 0 \text{ f.\,s.}
			\overset{\text{satz3.9}}{\gdw} ∀ ε > 0: \P\argu{\sup_{m ≥ n} d(X_m, X) > ε} \ntoinf 0$
		\item $X_n \ntoinf X$ f.\,s.\ und $f$ messbar und stetig in $X$ $\P$-f.\,s.\ $⇒ f(X_n) \ntoinf f(X)$ f.\,s.
		\item
		$X_n
		\stackrelnew{n⟶∞}{\P}{\longrightarrow}
		X: \gdw ∀ ε > 0:
		\P \argu{\set{d(X_n, X) > ε}}
		\ntoinf
		0$
		\define{stochastische K./ K.\ in Wahrscheinlichkeit}
		\item Konvergenz f.\,s.\ $⇒$ Konvergenz in W.; Umkehrung nicht wegen \enquote{wandernden Hüten}
		\item $X_n \stochto X \gdw$ Zu jeder TF $X_n'$ existiert TTF $(X_n'')$ s.\,d.\ $X_{n''} → X$ f.\,s.
		% CHECKED: '''' used.
		\item $X_n \stochto X$ und $f$ messbar und stetig in $X$ $\P$-f.\,s.\ $⇒ f(X_n) \stochto f(X)$
		\item $(\S_1, d_1)$, $(\S_2, d_2)$ separabel $⇒$ Produktraum $(\S_1 × \S_2, d_1 × d_2)$ separabel
		\item Für beide Konvergenzarten gilt koordinatenweise Konvergenz.
		\item $X,Y$ \define{gleich in Verteilung}:
			$X \overset{\L}{=} Y
			:\gdw \P ∘ X^{-1} = \P∘ Y^{-1}
			\gdw \Earg{f(X)} = \Earg{f(Y)} ~ ∀ f ∈ C^b(\S)$ glm.
		\item $\P_1=\P_2\gdw∫ f\d\P_1=∫ f\d\P_2~∀ f∈ C^b(\S)$ glm. stetig (gilt wegen Bump functions)
	\end{itemize}

	\section{Verteilungskonvergenz von Zufallsvariablen in metrischen Räumen}

	\begin{itemize}
		\item $\P_n\stackrelnew{w}{n⟶∞}{\longrightarrow} \P
			:\gdw
			∫ f\d \P_n\ntoinf ∫ f\d \P~∀ f∈ C^b(\S)$
			\define{schwache Konvergenz von Maßen}
		\item $X_n\distrto  X\text{ in }(\S,d)
			:\gdw
			\P∘ X_n^{-1}\stackrelnew{w}{n⟶∞}{\longrightarrow}\P∘ X^{-1}$
			\define{Konvergenz in Verteilung}

		\item $\P_n\overset{w}{\longrightarrow}\P\gdw F_n\rightharpoonup F:\gdw F_n(x)\ntoinf  F(x)~∀ x∈ C_F$ (Stetigkeitsstelle) \define{schwache Konvergenz von Verteilungsfunktionen} ($\P_n$ zu $F_N$ assoziiert)
		\item $X_n\distrto  X
			\overset{\text{Def}}{\gdw}
			P_n = \P∘ X_n^{-1}
			\weakto  \P∘ X^{-1} =P
			\gdw
			F_n (x) = \P(X_n≤ x)
			\ntoinf
			\P(X≤ x) =F(x)$ $∀ x ∈ C_F$
	\end{itemize}
\begin{minipage}{0.49\textwidth}
	\define{Portmanteau-Theorem 1}: Äquivalent:
		\begin{enumerate}[label=(\arabic*)]
		\item $\begin{aligned}
			\P_n\weakto  \P
		\end{aligned}$
		\item $\begin{aligned}
			∫ f\d \P_n\overset{}{\longrightarrow}∫ f\d \P~∀ f∈ C^b(\S)\text{ glm.\ stetig}
		\end{aligned}$
		\item $\begin{aligned}
			\limsup_{n⟶∞} \P_n(F)≤ \P(F)~∀ F∈\F(\S)
		\end{aligned}$
		\item $\begin{aligned}
			\liminf_{n⟶∞} \P_n(G)≥ \P(G)~∀ G∈\G(\S)
		\end{aligned}$
		\item $\begin{aligned}
			\limn \P_n(B)=\P(B)~∀ B∈\B(\S)
		\end{aligned}$\\ $\mit \P(∂ B)=0$, also \define{$\P$-randlos}.
	\end{enumerate}
\end{minipage}
\begin{minipage}{0.49\textwidth}
	\define{Portmanteau 2}: Äquivalent:
		\begin{enumerate}[label=(\arabic*)]
		\item $\begin{aligned}
			X_n\distrto  X\text{ in }(\S,d)
		\end{aligned}$
		\item $\begin{aligned}
			\Earg[\big]{f(X_n)}\ntoinf \Earg[\big]{f(X)}~∀ f∈ C^b(\S)
		\end{aligned}$ glm.\ stetig
		\item $\begin{aligned}
			\limsup_{n⟶∞}\P(X_n∈ F)≤\P(X∈ F)~∀ F∈\F
		\end{aligned}$
		\item $\begin{aligned}
			\liminf_{n⟶∞}\P(X_n∈ G)≥\P(X∈ G)~∀ G∈\G
		\end{aligned}$
		\item $\begin{aligned}
			\P(X_n∈ B)\ntoinf \P(X∈ B)~∀ B∈\B(\S)
		\end{aligned}$ $\mit\P(X∈∂ B)=0$
	\end{enumerate}
\end{minipage}

\begin{itemize}
	\item Portmanteau 1: (1) $⇒$ (2): Def, (2) $⇒$ (3): Bump fs;
	Portmanteau 2 folgt aus 1 mit $P_n:=\P∘ X_n^{-1}$ und $P:=\P∘ X^{-1}$.
	\item \define{CMT}: Für $h\colon(\S,d)⟶(\S',d')$ $\B(\S)$-$\B(\S')$-messbar gilt ($D_h$ Menge der Unstetigkeitsstellen von $h$):
	\begin{enumerate}[label=(\arabic*)]
		\item $\begin{aligned}
			P_n\weakto  P∧ P(D_h)=0
			⇒ P_n∘ h^{-1}\weakto  P∘ h^{-1}
		\end{aligned}$
		\item $\begin{aligned}
			X_n\distrto X\text{ in }(\S,d)∧\P(X∈ D_h)=0
			⇒ h(X_n)\distrto  h(X)\text{ in }(\S',d')
		\end{aligned}$ (\enquote{$h$ stetig in $X$})
	\end{enumerate}
	\item \undefine{Beweis.} (2) folgt direkt aus (1) nach Definition.
	Nutze Portmanteau 1, ((1) $\gdw$ (3)):\\
	$\limsup_{n⟶∞} \P_n∘ h^{-1}(F)
	=\limsup_{n⟶∞} \P_n \argu[\big]{\underbrace{h^{-1}(F)}_{⊆ \overline{h^{-1}(F)}}}
	≤ \limsup_{n⟶∞} \P_n \argu[\big]{\underbrace{\overline{h^{-1}(F)}}_{∈\F(\S)}}
		\overset{\text{Portm}}{≤}
		\P\klammern[\big]{\overline{h^{-1}(F)}}\\
		\overset{(*)}{≤}
		\P\klammern[\big]{h^{-1}(F)∪ D_h}
		≤ \P\klammern[\big]{h^{-1}(F)}+\underbrace{\P(D_h)}_{\overset{\Vor}{=}0}
		=\P∘ h^{-1}(F)
		$
		Bei $(*)$: Sei $x∈\overline{h^{-1}(F)}$. Falls $x∈ D_h$ trivial, $x\not∈ D_h\leadsto h$ stetig in $x$.
		Sei $(x_n)_{n∈ℕ}⊆\S\mit x_n\longrightarrow x$ (existiert nach Def des Abschlusses) $\leadsto h(x_n)=h(x)\overset{!}{∈} \overline{F}=F\leadsto x∈ h^{-1}(F)$ \hfill $\square$
	\item TF-Prinzip für schwache Konvergenz:
	\subitem $\begin{aligned}
			Q_n\weakto  Q
			⇔
			\text{Jede TF }(Q_{n'})⊆(Q_n)_{n∈ℕ}\text{ enthält TF }(Q_{n''})⊆(Q_{n'}):Q_{n''}\weakto  Q
			% CHECKED: '''' used.
		\end{aligned}$
	\subitem $\begin{aligned}
			X_n\distrto  X⇔\text{Jede TF }(X_{n'})⊆(X_n)_{n∈ℕ}\text{ enthält TF }(X_{n''})⊆(X_{n'}):X_{n''}\distrto  X
			% CHECKED: '''' used.
		\end{aligned}$
	\item $X_n\stochto X⇒ X_n\distrto  X$; Umkehrung i.\,A.\ nicht, nur falls $X$ f.\,s.\ konstant
	\item \define{Cramér}: Seien $(X_n)_{n∈ℕ},(Y_n)_{n∈ℕ}$ Folgen im separablen metrischen Raum $(\S,d)$, die \define{stochastisch äquivalent sind}, d.\,h.\ $d(X_n,Y_n)\stochto0$
	Dann gilt:
	$X_n\distrto  X⇔ Y_n\distrto  X$
	\item \define{Cramér-Slutsky} (Koordinatenweise Konvergenz): $(\S,d),(\S',d')$ separable MR, $X_n\distrto X$ in $(\S,d)$, $Y_n\distrto Y$ in $(\S',d')$ und  $Y$ f.\,s.\ konstant. Dann gilt: $(X_n,Y_n)\distrto (X,Y)$ in $(\S×\S',d× d')$. (folgt aus Cramér)
	\item $\begin{aligned}
			\P_n\stackrelnew{ω}{}{\longrightarrow} \P
		\end{aligned}\gdw\begin{aligned}
			\P_n(A_1× A_2)\ntoinf  \P(A_1× A_2)~∀ A_i∈\B(\S_i)~\P_i\text{-randlos mit } i=1,2
		\end{aligned}$ für $(\S,d)$ separabel
\item $\klammern[\big]{\P_n^{(1)}\otimes\P_n^{(2)}} \weakto \P^{(1)} \otimes \P^{(2)} ⇔
	\begin{aligned}
			\P_n^{(1)}\weakto \P^{(1)}∧
			\P_n^{(2)}\weakto \P^{(2)}
		\end{aligned}$ im separablem Produktraum $\S=\S_1×\S_2$
	\item $X_n\distrto  X\text{ in }\S_1∧ Y_n\distrto  Y\text{ in }\S_2
		⇔ (X_n,Y_n)\distrto (X,Y)\text{ in }\S_1×\S_2$ für $\S=\S_1×\S_2$ separabel, $X_n$ und $Y_n$ sind unabhängig für alle $n∈ℕ$; $X$ und $Y$ sind unabhängig
	\item Sei $(X_n)_{n∈ℕ}$ i.\,i.\,d.\ mit $\Earg{X_i}=:μ$ und $σ^2:=\Var(X_i)∈(0,∞)$.
	Dann (SGGZ + ZGWS + CMT): $√{n}·(\overline{X}-μ)\distrto \Nor(0,σ^2)$ mit $\overline{X}$ emp. Erwartung;
	$\klammern[\big]{\overline{X}_n,S_n^2}\ntoinf \klammern[\big]{μ,σ^2}\text{ in }ℝ^2$ f.\,s.
	\item Also: $(\overline{X}_n)_{n∈ℕ}$ und $(S_n^2)_{n∈ℕ}$ \define{asymptotisch normal}, d.\,h.
	$√{n}·\klammern[\big]{\overline{X}_n-μ}\distrto \mathcal{N}(0,σ^2),\qquad
	√{n}·\klammern[\big]{S_n^2-σ^2}\distrto \mathcal{N}(0,τ^2)$
\end{itemize}

\section{Verteilungskonvergenz in \texorpdfstring{$ℝ^d$}{Rd}}
\define{CF} $φ_X(t):=\E[\exp(\ii·\scaProd{t}{X})]$; Eindeutigkeitssatz: $X\overset{\L}{=}Y\gdwφ_X=φ_Y$;\\
Stetigkeitssatz (SSS): $X_n\distrto X$ in $ℝ^d\gdw∀ t∈ℝ^d:φ_{X_n}(t)\ntoinf φ_X(t)$\\
\define{Cramér-Wold-Device}: $X_n\distrto X$ in $ℝ^d ⇔ \scaProd{t}{X_n} \distrto \scaProd{t}{X}$ in $ℝ$ für alle $t∈ℝ^d$\\
\undefine{Beweis.} \enquote{(1) $⇒$ (2)} folgt aus CMT, da $x ↦ \scaProd{x}{t}$ stetig.\\
\enquote{(2) $⇒$ (1)}:
$	φ_{X_n}(t)
\stackeq{\text{Def}}\Earg[\Big]{\exp\argu[\big]{i · \scaProd{t}{X_n} · 1}}
\stackeq{\text{Def}}φ_{\scaProd{t}{X_n}}(1)
\stackrelnew{\text{SSS+(2)}}{n⟶∞}{\longrightarrow}\underbrace{φ_{\scaProd{t}{X}}(1)}_{=φ_X(t)}
	\leadstoφ_{X_n}\ntoinf φ_X
	\overset{\text{SSS}}{⇒}(1)$

\section{Der multivariate zentrale Grenzwertsatz (ZGWS) für Dreiecksschemata}
\begin{itemize}
	\item \define{$\triangle$-Schema} ist $\set{ X_{n,k}:1≤ k≤ n,n∈ℕ}$ mit $X_{n,k}$ unabhängige reelle ZV.
	\item \define{Lindeberg-ZGWS}: $\triangle$ mit $\Earg{X_{n,k}}=0$, $σ_{n,k}^2:=\Earg{X_{n,k}^2}<∞$ und $s_n^2:=\sum_{k=1}^nσ_{n,k}^2=\Var\klammern[\Big]{\sum_{k=1}^n X_{n,k}}$ und \define{LB}
	$\sum_{k=1}^n\Earg[\Big]{X_{n,k}^2·\indi_{\set{\abs{X_{n,k}}>ε}}} \ntoinf 0~∀ε>0$ und $s_n^2\ntoinf σ^2∈(0,∞)$ Dann: $\sum_{k=1}^n X_{n,k}\distrto \Nor(0,σ^2)$
	\item multivariater Fall: $\set{X_{n,k} = \klammern{X_{n,k}^{(1)},…,X_{n,k}^{(d)}}: k ≤ n, n ∈ ℕ}$ $Δ$-Schema
	% $X_{n,k}=\klammern{X_{n,k}^{(1)},…,X_{n,k}^{(d)}}\text{ in }ℝ^d$
	mit \define{zeilenweiser Unabhängigkeit}:
	$X_{n,1},…,X_{n,n}\text{ sind unabhängig} ~ ∀ n ∈ ℕ$
	% (Also die Vektoren seien unabhängig. Daraus folgt nicht, dass deren Komponenten unabhängig sind.)
	und
	$\Earg[\big]{X_{n,k}} = 0 ∈ ℝ^d$ und $\Var \argu{X_{n, k}^{(j)}} < ∞$
	$∀ k,n, j$.
	%und
	%$\Earg{\klammern{X_{n,k}^{(j)}}} < ∞
	%~∀ 1≤ j≤ d,∀ n,k∈ℕ$
	%(Kovarianzmatrix?)
	Es gelte LB
	$\sum_{k=1}^n \Earg[\Big]{ \norm{X_{n,k}}^2 · \indi_{ \set{ \norm{X_{n,k}} > ε } } } \ntoinf 0 \quad ∀ ε > 0$
	und Norm.beding.
	$\sum_{k=1}^n \Cov \argu{X_{n,k}}
	%\stackrelnew{\substack{\text{kompo-}\\\text{nentenweise}}}{n⟶∞}{\longrightarrow}
	→
	Γ$ mit $Γ ∈ ℝ^{d × d} \text{ positiv definit}
	$
	Dann MZGWS:
	$\sum_{k=1}^n X_{n,k}\distrto \mathcal{N}_d(0,Γ)\text{ in }ℝ^d$
	\item Koro ($d=1$ ist klassischer ZGWS):
	Sei $(X_i)_{i∈ℕ}$ i.\,i.\,d.\ in $ℝ^d$ mit
	$\Earg[\Big]{(X_1^{(j)})^2}<∞~∀ 1≤ j≤ d
	;\\
	μ:=\Earg[\big]{X_1}=\klammern{\Earg{X_1^{(1)}}, …, \Earg{X_1^{(d)}} } ∈ ℝ^d, ~
	Γ:=\Cov(X_1)
	= \klammern{\Cov \argu{X_1^{(i)},X_1^{(j)} }}_{i,j=1}^d\\
	= \klammern{ \Earg{ \klammern{X_1^{(i)} - μ_i } · \klammern{X_1^{(j)} - μ_j } } }_{i,j=1}^d \text{ positiv definit}
	$
	Dann:
	$
	\frac{1}{√{n}}·\sum_{i=1}^n(X_i-μ)\distrto \mathcal{N}_d(0,Γ)
	$
\end{itemize}

\section{Verteilungskonvergenz im Raum stetiger Funktionen}
\begin{itemize}
	\item $\B(C)
		=σ\argu{π_t:t∈ I}
		=σ \argu{π_T:T⊆ I,T\text{ endlich}}
		=σ \argu{\set[\big]{π_t^{-1}(B) : t ∈ I, B∈\B(ℝ)}}$ \\
	\enquote{kleinste $σ$-Algebra, sodass alle $π_t$ messbar sind}
	\item $X\colon I⟶ℝ,\qquad t↦ X(t,ω)$ heißt \define{Pfad} für jedes $ω∈Ω$
	\item $X\colonΩ⟶ℝ,\qquadω↦ X(t,ω)$ ist eine einzelne ZV für jedes $t∈ I$
	\item $X\colonΩ⟶ (I⟶ℝ),\qquad ω↦(t↦ X(t,ω))$ heißt \define{Pfadabbildung}
	\item Falls alle Pfade stetig (also $X$ ein \define{stetiger stochastischer Prozess} ist) sind, gilt $I⟶ℝ=C(I)$.
	\item Man kann einen (stetigen) stochastischen Prozess mit seiner Pfadabbildung identifizieren:
	$\set[\big]{ X(t)\mid t∈ I, X(t)\colonΩ⟶ℝ}
		\cong X\colonΩ⟶ C(I)
	$
	\item $X\text{ ist }\A\text{-}\B(C)\text{-messbar}⇔∀ t∈ I:π_t∘ X\text { ist }\A\text{-}\B(ℝ)\text{-messbar}$
	% das ist irgendwie Quark:
	% \item Alle Pfadabbildungen $X_t : Ω ⟶ C(I)$ eines stetigen SP sind $\A$-$\B(S)$-messbar.
	% Es gibt nicht „alle“, sondern nur eine Pfadabbildung X_{·} und X_t is das der Zeile davor
	\item Maße $P=Q\gdw∀ T⊆ I$ endlich$:P∘π_T^{-1}=Q∘π_T^{-1}$;
	$X\overset{\L}{=}Y\gdwπ_T(X)\overset{\L}{=}π_T(Y)$
	\item \define{endlich dim. Randverteilungen (fidis)}: $π_T∘ X:Ω⟶ℝ^d$ ist Zufallsvektor, genauer:\\
		$π_T∘ X\colonΩ⟶ C⟶ℝ_k,~
		π_T\klammern[\big]{X(ω)}= \klammern[\big]{X(ω)(t_1),…,X(ω)(t_k)} ∀ ω ∈ Ω, ∀ T= \set{t_1, …, t_k} ⊆ I$
	\item \define{Stetigkeits- / Oszillationsmodul}
		$ω(f,δ):=\sup\set{ \abs{f(s)-f(t)}:s,t∈ I\mit \abs{s-t}≤δ}$;
		$f∈ C(I)\gdwω(f,δ)\overset{δ⟶0}{\longrightarrow}0$
	\item $X_n\overset{\fd}{\longrightarrow}X:\gdwπ_T∘ X_n\distrto π_T∘ X~∀ T⊆ I$ endlich \define{Konvergenz der fidis}
	\item $X_n\overset{\fd}{\longrightarrow}X$ und $∃$ Folge
		$(δ_k)_{k∈ℕ}⊆(0,∞)\mitδ_k\downarrow0$, s.\,d.\ $∀ε>0:
		\lim\limits_{k⟶∞} \limsup\limits_{n⟶∞} \P \argu[\Big]{ω \argu[\big]{X_n,δ_k} > ε} = 0$.
		Dann gilt:
		$X_n\distrto X\text{ in }(C,d)$
	\item \define{Momentenkriterium von Kolmo}: $X_n \fdto X$ und $∃γ>0,∃α>1$ und $F\colon I⟶ℝ$ stetig + monoton wachsend mit
		$\Earg[\Big]{\abs{X_n(s) - X_n(t)}^γ } ≤ \klammern[\big]{F(s) - F(t)}^α ~ ∀ s > t, s, t ∈ I$ Dann:
		$X_n\distrto  X\text{ in }\klammern[\big]{C(I),d}$
	\item Konvergenz von SP $⇐$ Konvergenz der fidis + \enquote{Straffheit} (Momentenkriterium); bei konvexen SP reicht Konvergenz der fidis (Straffheit automatisch erfüllt)
	\item \define{BB}: $I=[0,b]$ und $B:=\set[\big]{ B(t):=B(t,ω)}$ stetiger SP mit $B(0)=B(0,ω)=0$, unabhängige Zuwächse:
		$B(t_i)-B(t_{i-1}),~1≤ i≤ r~∀~ 0=:t_0≤ t_1<…<t_r≤ b$;
		Normalverteile Zuwächse $0≤ s<t≤ b⇒ B(t)-B(s)\sim\mathcal{N}(0,t-s)$; BB existiert nach Lévy; Verteilung einer BB eindeutig bestimmt
	\item \define{Donsker}: Sei $(ζ_i)_{i∈ℕ}$ i.\,i.\,d\ mit $\Earg{ζ_1}=0$ und $\Var(ζ_1)=1$, $S_k:=\sum_{i=1}^kζ_i$ (RW!) und $X_n(t)$ der Polygonzug durch die Punkte $\klammern{\frac{k}{n}, \frac{S_k}{√{n}}}_{0≤ k≤ b· n}$ Dann: $X_n \distrto$ BB in $(C([0,b]), d)$. (folgt aus Mom.\ von Kolmo);\\
		Standardisierung $\hat{ζ}:=\frac{ζ_i-μ}{σ}$ kann helfen; BB hängt \betone{nicht} von Verteilung ab
	\item \define{Change-Point-Problem:} $X_{1,n},…,X_{n,n},n∈ℕ$ unabhängig mit
		$\begin{cases}
			X_{i,n}\text{ i.\,i.\,d.}\sim(μ,σ^2), &\falls 1≤ i≤τ_n\\
			X_{i,n}\text{ i.\,i.\,d.}\sim(ν,τ^2), &\falls τ_n< i≤ n
		\end{cases}$
		wobei $τ_n∈\set{1,…,n}$ der \emph{unbekannte} \define{Change-point} \enquote{$\triangle$-Schema, damit change-point mitwandern kann}
	\item $H_0: τ_n = n$, $H_1: τ_n < n$. \define{Asymptotischer Niveau-$α$-Test für $H_0$} gesucht mit Signifikanzniveau $\intervallO01 \ni α \ll 1$ heißt:
		Fehler 1.\ Art $ := \lim_{n → ∞} \P_{H_0}(H_0 \text{ wird verworfen}) = α$.
		Tests sind von Bauart $H_0 \text{ verwerfen } \gdw T_n ≥ k_{α}$ mit versch.\ $T_n$, $k_{α}$.
% \end{itemize}

\begin{minipage}{0.49\textwidth}
	$μ, σ^2$ bekannt, $ν, τ^2$ unb., $ν > μ$ \\ $\rightsquigarrow$
	$T_n := \max_{0 ≤ k ≤ n} S_k$\\
	F.\ 1.\ Art $\ntoinf \P(\abs{\mathcal{N}(0,1)} ≤ k_{α})$\\
	$⇒ k_{α} = Φ^{-1}(1- \frac{α}{2}) =: u_{1- \frac{α}{2}}$ \\
	($Φ$ Vtl.fkt.\ von $\Nor(0, 1)$)
\end{minipage}
\begin{minipage}{0.49\textwidth}
	$μ, σ^2$ bekannt, $ν, τ^2$ unb., $ν \neq μ$ \\ $\rightsquigarrow$
	$T_n := \max_{0 ≤ k ≤ n} \abs{S_k}$\\
	F.\ 1.\ Art $\ntoinf \P(\norm{B}_{∞} ≤ k_{α})$\\
	$H(x) = \frac{4}{π} \sum\limits_{k ∈ ℕ_0} \frac{(-1)^k}{2k + 1} \exp\argu{\frac{-(2k + 1)^2 π^2}{8 x^2}}$ \\
	$⇒ k_{α} = H^{-1}(1- α)$
\end{minipage}

\begin{minipage}{0.49\textwidth}
	$μ, σ^2, ν, τ^2$ unb., $ν \neq μ$ \\ $\rightsquigarrow$
	Nutze Schätzer. \\
	$T_n^* \! := \! (\hat {σ}_n^2 n)^{-\frac12} \max\limits_{0 ≤ k ≤ n} \abs{\sum\limits_{i = 1}^k(X_{i, n} \!-\! \overline{X_n})}$\\
	% CHECKED \overline
	F.\ 1.\ Art $\ntoinf \P(\norm{B_0}_{∞} ≤ k_{α})$\\
	$H_0(x) = 1 - \sum\limits_{k≥ 1}(-1)^{k+1} \exp\argu{-2 k^2 x^2}$ \\
	$⇒ k_{α} = H_0^{-1}(1- α)$
\end{minipage}
\begin{minipage}{0.49\textwidth}
	$μ, σ^2, ν, τ^2$ unb., $ν > μ$ \\ $\rightsquigarrow$
	Nutze Schätzer. \\
	$\tilde{T}_n^* \! := \! (\hat {σ}_n^2 n)^{-\frac12} \max\limits_{0 ≤ k ≤ n} \sum\limits_{i = 1}^k(X_{i, n} \!-\! \overline{X_n})$\\
	% CHECKED \overline
	F.\ 1.\ Art $\ntoinf \P(\sup_t B_0(t)) ≤ k_{α})$\\
	$H_0^+(x) = 1 - \exp(-2 x^2)$ \\
	$⇒ k_{α} = (H_0^+)^{-1}(1 - α) = √{-\frac 12 \log(α)}$
\end{minipage}

\begin{minipage}{0.49\textwidth}
	$μ, σ^2, ν, τ^2$ unb., $ν \neq μ$ (mit Gumbel) \\
	$\rightsquigarrow$
	$U_n^* := \hat{σ}_n^{-1}√n \max_{1 ≤ k ≤ n-1} \frac{\abs{\sum_{i = 1}^k (X_i - \overline{X}_n)}}{√{k (n - k)}}$ \\
	F.\ 1.\ Art $= \P_{H_0}(U_n^* > \frac{t_{α} + D_n}{A_n}) \ntoinf 1 - G(t_{α})$ \\
	$G(x) = e^{-2 e^{-t}} ⇒ t_{α} = -\log(-\frac12 \log(1 - α))$
\end{minipage}
\begin{minipage}{0.49\textwidth}
	$μ, σ^2, ν, τ^2$ unb., $ν \neq μ$ (mit Ferger) \\
	$\rightsquigarrow$
	$R_n := \hat{σ}_n^{-1}√n \frac{\max_{1 ≤ k ≤ n-1} \abs{\sum_{i = 1}^k (X_i - \overline{X}_n)}}{√{k (n - k)}}$ \\
	mit $\hat k_n = \min \set{1 ≤ k ≤ n : \abs{C_k} = \max\limits_{1 ≤ j ≤ n - 1} \abs{C_j}}$,\\
	\text{wobei } $C_j = \sum_{i = 1}^j (X_i - \overline{X}_n)$ \\
	F.\ 1.\ Art $\ntoinf \P_{H_0} \argu{\frac{\norm{B_0}_{∞}}{√{\argmax \abs{B_0}(1 - \argmax \abs{B_0})}} > k_{α}}$ \\
	$k_{α} = Φ^{-1}(1- α)$ \\
	für $ν > μ$: ohne $\abs{·}$
\end{minipage}

% \begin{itemize}
	\item \define{Brownsche Brücke} ist SP $B_0(t) := B(t) - t · B(1) \qquad ∀ t ∈ \intervall01$ für BB auf $\intervall01$ ($B_0(0) = 1$ und $B_0(1) = 0$)
	\item Der bisherige Fall $C(I)$ mit $I = \intervall ab$ deckt $C(ℝ)$ nicht ab.
		$C(ℝ)$ ist vollständiger separabler MR; $\B_d(C(ℝ))=σ(π_t:t∈ℝ)=σ(π_T:T⊆ℝ$ endlich) (analog);
		$X$ $\A$-$\B_d(C)$-messbar $\gdw∀ t∈ℝ:π_t∘ X$ $\A$-$\B(ℝ)$-messbar
	\item Für $X,Y$ ZV in $(C(ℝ),d)$ gilt: $X\overset{\L}{=}Y\gdwπ_T∘ X\overset{\L}{=}π_T∘ Y\gdw\klammern[\big]{X(t_1),…,X(t_k)}\stackeq{\L}\klammern[\big]{Y(t_1),…,Y(t_k)}~∀ t_j$
	\item $X_n\distrto X\text{ in }\klammern[\big]{C(ℝ),d}
		⇔∀ j∈ℕ:
		\restr{X_n}{I_j}\distrto  \restr{X}{I_j} \text{ in } \klammern[\big]{C(I_j),d_j} \mit I_j:=\intervall{-j}j$
\end{itemize}

\section{Argmin-Theoreme in \texorpdfstring{$C(ℝ)$}{C(R)}} %8
\begin{itemize}
	\item Wann überträgt sich die Konvergenz (f.\,s.\ oder in Verteilung) von stetigen stochastischen Prozessen auf deren Minimalstellen?
	\item Für $f∈ C(ℝ)$: $A(f):=\argmin(f):=\set[\big]{ t∈ℝ:f(t)=\inf_{s∈ℝ}f(s)}$ Menge aller Min-Stellen;\\
	$τ∈ A(f)$ \define{wohl-separiert} $:\gdw\inf\set{ f(t):\abs{t-τ}≥ε}>f(τ)~∀ 0<ε∈ℚ⇒τ$ eindeutig (siehe Bild)
	\item 8.3: $f,f_n,n∈ℕ$ aus $C(ℝ)$, $τ_n∈ A(f_n)\neq∅~∀ n$ und $τ∈ A(f)$ wohlsepariert und $\norm{ f_n-f}_∞\ntoinf 0$ Dann: $τ_n\ntoinf τ$; lässt sich übertragen auf offene und kompakte Intervalle (da muss $τ$ nur eindeutig sein)
	\item 8.4: $f,f_n,n∈ℕ$ aus $C([a,b])$. Dann: $A(f_n)\neq∅$ (da kompakt) und:
	Falls $\set{τ}=A(f)$ und $\norm{ f_n-f}_∞\ntoinf 0$ so gilt für \emph{jede} Auswahl $τ_n∈ A(f_n):τ_n\ntoinf τ$
	\item 8.5: $M,M_n,n∈ℕ$ SP mit Pfaden in $C(ℝ)$ (d.\,h.\ $M(ω)\colon ℝ⟶ℝ$ stetig), $τ(ω)∈ A\klammern[\big]{M(·,ω)}$ %\overset{\Def}{=}			\set{t∈ R:\inf_{s∈ℝ}M(s,ω)=M(t,ω)}$
	f.\,s.\ für ZV $τ\colonΩ⟶ℝ$; $\inf\set[\big]{ M(t):\abs{t-τ}≥ε}>M(τ)\text{ f.\,s.\ } ∀ 0 < ε ∈ ℚ$;
	$\norm[\big]{M_n-M}_∞\ntoinf 0$ f.\,s.;
$∀(τ_n)_{n∈ℕ}$ von ZV mit $τ_n∈ A(M_n)$ f.\,s.:
Dann: $τ_n\ntoinf τ$ f.\,s.
\undefine{Beweis.} Abzählbare Schnitte von 4 Einsmengen + 8.3 $\square$
	\item 8.6: $M$,$M_n$, $n∈ℕ$ SP mit Pfaden in $C(I)$, $I$ kompakt, $∃τ$ ZV eindeutige Minstelle f.\,s.; $\norm{M_n-M}_∞$ f.\,s.; $(τ_n)_{n∈ℕ}$ mit $τ_n∈ A(M_n)$ f.\,s.\ beliebig.
	Dann: $τ_n\ntoinf τ$ (folgt aus 8.4)
	% \item 8.7: Exponential-Familie, $Θ⊆ℝ$ kompakt. Dann MLE stark konsistent (nach 8.6), also $\hat{θ}_n\ntoinf θ_0$ f.\,s. → nicht in WiSe 2019/20 behandelt
	\item CPP: Schätzer $\hat{τ}_n:=\argmax_{0≤ k≤ n}\abs{S_k}$ mit $S_k:=\sum_{i=1}^k(X_i-\overline{X}_n)$ für Wechselzeitpunkt $τ$. Bedingung: $p$te Momente von $X_k$ müssen endlich sein für ein $p > 2$.
	\item $(σ_n)_{n∈ℕ}$ \define{stochastisch beschränkt} $:\gdw\lim_{t⟶∞}\limsup_{n⟶∞}\P(\abs{σ_n}>j)=0$
	\item Argmin für $\stochto$: $Z_n \stochto Z$ in $C(ℝ)$. Dann $\limsup_{n → ∞} \P(σ_n ∈ K) ≤ \P( A(Z) ∩ K \neq ∅ ) =: μ^*(K)  = \text{Hitting-probability}$ für alle $K ⊂ ℝ$ kompakt.
		\subitem Wenn $(σ_n)_n$ stochastisch beschränkt sind, so folgt $\limsup_{n → ∞} \P( σ_n ∈ F ) ≤ μ^*(F)$ für $F ∈ ℝ$ abgeschlossen.
		\subitem Falls zusätzlich $A(Z) = \set{ σ }$ ($σ$ ZV), also eindeutige Minimalstelle, so gilt $σ_n \stochto σ$.
\end{itemize}

\section{Verteilungskonvergenz im Raum konvexer Funktionen}
\begin{itemize}
	\item $O⊆ℝ$ \define{konvex} $:\gdw x,y∈ O⇒∀λ∈(0,1):λ x+(1-λ)y∈ O$ (Intervall);
	$f\colon O⟶ℝ$ \define{konvex} $:\gdw f(λ x+(1-λ)y)≤λ f(x)+(1-λ) f(y)~∀ x,y∈ O,λ∈(0,1)$;
	$C_c(O):=\set{f\colon O⟶ℝ:f\text{ konvex}}$ (konvex $⇒$ stetig);
	SP $X = \set{X(t) : t ∈ O}$ mit Pfaden in $C_c(O)$ heißt \define{konvex}.
	\item $C_c(O)$ separabel;
		$\B_d\argu[\big]{C_c(O)} = σ\argu[\big]{π_t:t∈ O}=σ \argu[\big]{π_T:T⊆ O\text{ endlich}}$
	\item $X\overset{\L}{=}Y\gdwπ_T∘ X=π_T∘ Y$ (fidis gleich)
	\item $f_n$ konvex, $f\colon O⟶ℝ$, $D⊆ O$ dicht, $f_n(x)\ntoinf f(x)~∀ x∈ D$ Dann: Konvergenz $∀ x∈ O$, $f$ konvex und $\norm{f_n-f}_∞\ntoinf 0~∀ K∈ O$ kompakt
	\item  $X_n,n∈ℕ$ konvexe SP und sei $X$ stetiger SP auf $ℝ$.
	Dann: $X_n\fdto X\gdw X_n\distrto  X\text{ in }\klammern[\big]{C(ℝ),d}⇒ X$ konvexer SP
	\undefine{Beweis.}
	(1) $⇒$ (2): Portmanteau; (2) $⇒$ (1): CMT $\square$
	\item \define{Subspace}: $Z_n\distrto  Z\text{ in }(\S,d)
	⇔
	Z_n\distrto  Z\text{ in }(U,d)$
	\item Koro: $X_n$ konvexer SP: $X_n\fdto X⇒
	X_n\distrto  X\text{ in }\klammern[\big]{C_c(ℝ),d}$ (folgt aus Subspace-Lemma + vorherigem)
\end{itemize}

\section{Argmin-Theoreme in \texorpdfstring{$C_c(ℝ)$}{CcR}}
\begin{itemize}
	\item 10.2: Sei $D⊆ℝ^d$ dicht in $ℝ^d$ (z.\,B.\ $D=ℚ^d$) und seien $f,f_n,n∈ℕ$ konvexe Funktionen auf $ℝ^d$, wobei $f$ eine eindeutige Minimalstelle $τ$ besitze ($A(f)=\set{τ}$).
	Seien $τ_n∈ A(f_n)\neq∅$ $∀ n≥ N_0∈ℕ$.
	Dann: $\klammern{f_n(t) \ntoinf f(t) \qquad ∀ t ∈ D} ⇒ τ_n \ntoinf τ$
	\item 10.3: Seien $M,M_n,n∈ℕ$ konvexe SP auf $ℝ^d$ mit
	$A(M)=\set{τ}$ f.\,s.\ für eine Zufallsvariable $τ$;
	$M_n(t)\ntoinf  M(t)~\P\text{-f.\,s.\ } ∀ t ∈ ℝ^d$;
	Seien $τ_n$ ZV mit $τ_n∈ A(M_n)$ f.\,s.\ $∀ n∈ℕ$.
Dann: $τ_n\ntoinf τ~\P\text{-f.\,s.\ in } ℝ^d$ (folgt aus 10.2 über Einsmengen)
	\item 10.4: $Z,Z_n,n∈ℕ$ konvexe SP auf $ℝ$ mit $A(Z)=\set{σ}$ f.\,s.\ für eine ZV $σ$ und $Z_n\overset{\fd}{\longrightarrow}Z$.
	Dann gilt für jede Auswahl $σ_n∈ A(Z_n)$: $\argmin(Z_n)=σ_n\distrto σ$
	\item Sei $U_n,n∈ℕ$ reelle ZV mit $\Earg{U_n}\ntoinf μ$ und $\Var(U_n)\ntoinf 0$. Dann: $U_n\stochtoμ$.
\end{itemize}

\section{Rekapitulation des Einführungskapitels}
Der Median ist $m:=\argmin_{t∈ℝ}Y(t)$, $Y∈ C(ℝ)$, $Y(t):=\Earg{\abs{X-t}}\overset{\text{Trafo}}{=}∫_ℝ\abs{x-t}~(F\d x)$, Schätzer\\ $Y_n(t):=∫_ℝ\abs{x-t}~(F_n\d x)\overset{\text{emp}}=\sum_{i=1}^n\abs{X_i-t}\underset{\text{SGGZ}}{\ntoinf }\Earg{\abs{X_1-t}}~∀ t∈ℝ$\\
$M:=Y$ und $M_n:=Y_n$ lassen sich als SP auffassen, sind sogar konvex. Aus Argmin-Theorem 10.3 folgt \emph{sofort} die Konvergenz des $\argmin$'s, ohne weitere Voraussetzungen an die starke Konsistenz des Medians (abgesehen Eindeutigkeit der Minimalstelle): $\hat{m}_n\ntoinf  m$\\
Um die Verteilungskonvergenz des Medians zu zeigen, betrachte den \define{reskalierten Prozess}\\ $Z_n(t):=a_n·√{n}·\klammern[\big]{M_n\klammern[\big]{m+\frac{t}{a_n}}-M_n(m)}$.
Dann gilt: $a_n(\hat{m}_n-m)=\argmin_{t∈ℝ}Z_n(t)$ und Konvergenz der fidis $Z_n\overset{\fd}{\longrightarrow}Z$

% !TEX root = MSTAT19.tex
% This work is licensed under the Creative Commons
% Attribution-NonCommercial-ShareAlike 4.0 International License. To view a copy
% of this license, visit http://creativecommons.org/licenses/by-nc-sa/4.0/ or
% send a letter to Creative Commons, PO Box 1866, Mountain View, CA 94042, USA.

Folgende Tabelle zeigt die Übersicht über die Voraussetzungen der verschiedenen\\ Argmin-Theoreme:\nl
\begin{tabular}{c|c|c|c}
	& $C(\R)$ & $C(I)$ mit $I$ kompakt & $C_c(\R)$ konvex \\
	\hline
		\makecell{$f,f_n$\\ Fkt.} &
		\makecell{$\tau\in A(f)$ wohlsepariert\\ $A(f_n)\neq\emptyset\forall n\geq N_0\in\N$\\ $f_n\to f$ glm.} &
		\makecell{$\tau\in A(f)$ eindeutig\\ $f_n\to f$ glm.} &
		\makecell{$\tau\in A(f)$ eindeutig\\ $A(f_n)\neq\emptyset~\forall n\geq N_0\in\N$\\ $f_n\to f$ pktw. $D\overset{\text{dicht}}{\subseteq}\R^d$}\\
	\hline
		\makecell{$M,M_n$\\ SP} &
		\makecell{$\tau\in A(M)$ wohlsep. f.s.\\ $A(M_n)\neq\emptyset$ f.s. \\ $M_n\to M$ glm. f.s.} &
		\makecell{$\tau\in A(f)$ eindeutig f.s.\\ $M_n\to M$ glm. f.s.} &
		\makecell{$\tau\in A(M)$ eindeutig f.s.\\ $A(f_n)\neq\emptyset$ f.s.\\ $M_n(t)\to M(t)$ pktw. f.s.}\\
	\hline
		\makecell{$Z_n,Z$\\ SP} &
		\makecell{$\sigma\in A(Z)$ eindeutig f.s.\\ $Z_n\overset{\L}{\to}Z$\\ $(\sigma_n)_{n\in\N}$ stoch. besch.} &
		------ &
		\makecell{$\sigma\in A(Z)$ eindeutig f.s.\\ $Z_n\overset{\fd}{\to}Z$ \\ $A(Z_n)\neq\emptyset$}\\
	\end{tabular}\nl
	Hierbei folgt für die letzte Zeile stets Konvergenz in Verteilung des Argmins, bei den anderen Zellen nur fast sichere Konvergenz.
	Als Faustregel kann man sich auch merken, dass bei stochastischen Prozessen alle Voraussetzungen nur $\P$-fast sicher gelten müssen.


\end{document}
