% !TEX root = MSTAT19.tex
% This work is licensed under the Creative Commons
% Attribution-NonCommercial-ShareAlike 4.0 International License. To view a copy
% of this license, visit http://creativecommons.org/licenses/by-nc-sa/4.0/ or
% send a letter to Creative Commons, PO Box 1866, Mountain View, CA 94042, USA.

\subsection{Anwendung: Verteilungskonvergenz des Medians} %nonumber
Seien $X_1, …, X_n$ \iid\ $\sim F$.
$F$ besitze Median $m∈ℝ$ und erfülle
\begin{align}\label{eqAnwendungVerteilungskonv_B}\tag{B}
	∃ψ\colonℝ⟶ℝ,∃(a_n)_{n∈ℕ}\mit a_n⟶∞:
	\limn√{n} \klammern{ F \argu{m + \frac{t}{a_n} } - F(m)} = ψ(t)
	~~ ∀ t ∈ ℝ
\end{align}

\begin{beisp} %nonumber
	Sei $F$ differenzierbar an der Stelle $m$.
	Dann gilt:
	\begin{align*}
		\underbrace{\frac{F\klammern{m+\frac{t}{√{n}}}-F(m)}{\frac{1}{√{n}}· t}}_{
			\ntoinf F'(m)
		}· t
		\ntoinf F'(m)· t\\
		ψ(t)=
		\begin{cases}
			F_+'(m)· t, & \falls t≤ 0\\
			F_-'(m)· t, & \falls t<0
		\end{cases}
	\end{align*}
\end{beisp}

Erinnerung:
\begin{align*}
	\hat{m}_n&:=\argmin_{t∈ℝ} M_n(t),\qquad
	M_n(t)=\frac{1}{n} \sum_{i=1}^n\abs[\big]{X_i-t}
\end{align*}
Idee: Betrachte den \define{reskalierten Prozess zu $a_n$}:
\begin{align*}
	Z_n(t)&:=a_n·√{n}·\klammern{M_n \argu{m + \frac{t}{a_n} } - M_n(m) } \qquad ∀ t ∈ ℝ
\end{align*}
Dann gilt (bereits gesehen):
\begin{align*}
	a_n\klammern[\big]{\hat{m}_n-m}
	&=\argmin_{t∈ℝ} Z_n(t)
\end{align*}
Die $T_n$ sind konvexe stochastische Prozesse.
\paragraph{Ziel:} Konvergenz der fidis $Z_n\fdto Z$.

Dazu verwende

\begin{lemma}[Keith Knight]\label{lemma10.5KeithKnight}
	\begin{align*}
	\abs{x - y} - \abs{x} &= -y \klammern{1 - 2 \indi_{\set{x ≤ 0}}} + R(x,y)
		\qquad∀ x\neq y∈ℝ\quad\mit\\
		R(x,y)&:=2·∫_0^y\indi_{(∞,s]}(x)-\indi_{(-∞,0]}(x)~λ(\d s)
	\end{align*}
\end{lemma}

Somit folgt für uns:
\begin{align*}
	Z_n(t)
	&=a_n √{n} \klammern{M_n \argu{m + \frac{t}{a_n}} - M_n(m)} \\
	&=\frac{a_n}{√{n}} \sum_{i=1}^n \abs[\bigg]{
		\underbrace{X_i-m}_{\hat{=}x}
		- \underbrace{\frac{t}{a_n}}_{\hat{=}y}} - \abs[\big]{\underbrace{X_i-m}_{\hat{=}x}} \\
	\overset{\ref{lemma10.5KeithKnight}}&=
	\frac{a_n}{√{n}} \sum_{i=1}^n
	- \frac{t}{a_n} \klammern{1 - 2 \indi_{\set{X_i ≤ m}}} + R\argu{X_i - m_i \frac{t}{a_n}} \\
	&= \underbrace{\frac{-t}{√{n}} \sum_{i=1}^n \klammern{1 - 2 \indi_{\set{X_i ≤ m}}}}_{
		=:S_n(t)
	} + \underbrace{\frac{a_n}{√{n}} \sum_{i=1}^m R\argu{X_i - m_i \frac{t}{a_n}}}_{
		:=D_n(t)
	}
\end{align*}
Mit
\begin{align*}
	ζ_i &:= 1 - 2 \indi_{\set{X_i ≤ m}} \qquad ⇒ ζ_i \text{ i.\,i.\,d.}
\end{align*}
gilt
\begin{align*}
	\Earg[\big]{ζ_i}
	&= 1 - 2 \underbrace{F(m)}_{=\frac{1}{2}} = 0
\end{align*}
Hier geht Lemma \ref{lemmaMedian}
ein. Das $F$ stetig ist, folgt aus \eqref{eqAnwendungVerteilungskonv_B}.
\begin{align*}
	\Var(ζ_i)
	&= 4 F(m) \klammern[\big]{1-F(m)}=1 \\
	\overset{\text{ZGWS}}&{⇒}
	\frac{1}{√{n}} \sum_{i=1}^n ζ_i
	\distrto \Nor(0,1)\\
	\overset{\text{CMT}}&{⇒}
	S_n(t)
	\distrto -N· t\qquad∀ t∈ℝ
\end{align*}

Für die Behandlung von $D_n(t)$ verwende:

\begin{lemma}\label{lemma10.6}
	Seien $U_n$, $n ∈ ℕ$ reelle Zufallsvariablen mit
	\begin{align*}
		 \Earg[\big]{U_n}\ntoinf
		 % todo: to was?
		 \qquad\text{und}\qquad
		 \Var(U_n)\ntoinf 0.
	\end{align*}
	Dann gilt
	\begin{align*}
		U_n \stochto μ.
	\end{align*}
\end{lemma}

\begin{proof}
	\begin{align*}
		\P\argu{\abs[\big]{U_n - \Earg{U_n}} > ε }
		\overset{\text{Markov}}&{≤}
		ε^{-2} \Var\klammern[\big]{U_n}\ntoinf 0\\
		U_n - \Earg{U_n} \stochto 0
		⇒ U_n &= \underbrace{(U_n - \Earg{U_n}}_{\stochto0}
		+ \underbrace{\Earg{U_n}}_{→ μ}
		\qedhere
	\end{align*}
\end{proof}

Somit erhalten wir (hier der Fall $t>0$):
\begin{align*}
	\Earg[\big]{D_n(t)}
	&= a_n √{n} \, \Earg{R\argu{X_1 - m, \frac{t}{a_n}}} \\
	&= -2 a_n √{n}\, \Earg[\Bigg]{∫_0^{\frac{t}{a_n}}
		\underbrace{\indi_{\set{X_1 ≤ s + m}} - \indi_{\set{X_1 ≤ m}}}_{
		≥ 0,~\falls t≥0
	}~λ(\d s) } \\
	\overset{\text{Fubini}}&=
	2 a_n √{n} \, ∫_0^{\frac{t}{a_n}} F(m + s) - F(m) ~ \underbrace{λ(\d s}_{=\d s}) \\
	\overset{\text{Subs}}&=
	2 a_n √{n} \frac{1}{a_n} ∫_0^{t}F\argu{m + \frac{u}{a_n}} - F(m) \d u \\
	&= 2 ∫_0^t \underbrace{√{n} \klammern{F\argu{m + \frac{u}{a_n}}-F(m)}}_{
		\ntoinf ψ(u)\text{ gemäß }\eqref{eqAnwendungVerteilungskonv_B}
	} \d u \\
\end{align*}
Bei der Substitution wurde benutzt:
$u= s a_n$, $s=\frac{u}{a_n}$, $\frac{\d s}{\d u} = \frac{1}{a_n}$.

Da $ψ$ integrierbar auf $\intervall{0}{t}$ (vergleiche Bemerkung \ref{bemerkung10.7} unten) folgt mit dem Satz der dominierten Konvergenz:
\begin{align*}
	\limn \Earg[\Big]{D_n(t)} = 2 ∫_0^t ψ(u) \d u
\end{align*}
Betrachte nun die Varianz
\begin{align*}
	&\Var\klammern[\Big]{D_n(t)}\\
	\overset{\text{unab}}&=
	a_n^2·\frac{1}{n} n \Var\argu{R\argu{X_1-m,\frac{t}{a_n}}}\\
	&=a_n^2 \Var\argu{R\argu{X_1 - m, \frac{t}{a_n} }} \\
	&≤ a_n^2 \Earg{ \klammern{ R \argu{X_1 - m, \frac{t}{a_n}} }^2 } \\
	\overset{\ref{lemma10.5KeithKnight}}&=
	a_n^2·4 \Earg{
		∫_0^{\frac{t}{a_n}} ∫_0^{\frac{t}{a_n}} \klammern{
		\indi_{\set{ X_1≤ m+s}}
		-\indi_{\set{ X_1≤ m}}
		}· \klammern{
		\indi_{\set{X_1 ≤ m + u}}
		- \indi_{\set{X_1 ≤ m}}
	} \d s\d u}\\
	\overset{s,u≥0}&{=}
	a_n^2·4 \Earg{
		∫_0^{\frac{t}{a_n}} ∫_0^{\frac{t}{a_n}}
			\indi_{\set{X_1 ≤ m + \min(s, u)}} - \indi_{\set{X_1 ≤ m}}
		\d s \, \d u
	} \\
	\overset{\text{Fubini}}&=
	4· a_n^2 ∫_0^{\frac{t}{a_n}} ∫_0^{\frac{t}{a_n}}
		F\argu[\big]{m + \min(s, u)} - F(m) \d s \, \d u \\
	\overset{\text{2x Subs}}&=
	4 ∫_0^t ∫_0^t \underbrace{F\argu{m + \frac{\min(s, u)}{a_n}} - F(m)}_{
		\ntoinf 0\text{, da $F$ stetig in }m
	}\d s \, \d u \\
	\overset{n⟶∞}&{\longrightarrow}0 \text{ gemäß dominierter Konvergenz} \\
	\overset{\ref{lemma10.6}}&{⇒}
	D_n(t) \stochto 2 ∫_0^t ψ(u) \d u =: D(t) \\
	&⇒
	Z_n(t) = \underbrace{S_n(t)}_{\distrto -N· t} + \underbrace{D_n(t)}_{\stochto D(t)}
	\distrto -N· t + D(t) := Z(t)
\end{align*}
Die letzte Verteilungskonvergenz geht wegen Cramér-Slutsky \ref{satz4.15CramerSlutsky} und CMT \ref{satz4.10ContinuousMappingTheorem}.

Oben geht auch ein allgemeines Prinzip ein:
\begin{align*}
	\klammern{∫_0^af(u)\d u}^2=∫_0^a∫_0^a f(u)· f(y)\d u \, \d y
\end{align*}

%Ferger: "Die jungen Studentinnen hätten mir zum Geburtstag doch mal einen Kuchen backen können. Oder eine Flasche Bier."
%Ferger: "Kennen Sie den Kollegen Matthies? Da will ich modisch auch noch hin! Der Kollege Matthies so einen Wischer. Die wünsche ich mir zu Weihnachten."

Also haben wir insgesamt gezeigt $(t≤ 0$ geht analog):
\begin{align*}
	Z_n(t) \distrto Z(t) \overset{\text{Def}}{=} -N t + 2 ∫_0^t ψ(u)\d u \qquad ∀ t ∈ ℝ
\end{align*}

Dies zeigt die Konvergenz der \emph{eindimensionalen} fidis.
Mit dem Cramér-Wold-Device \ref{satz5.4CramerWoldDevice} folgt tatsächlich
\begin{align*}
	\klammern[\Big]{Z_n(t_1),…,Z_n(t_k)}
	\distrto
	\klammern[\Big]{Z(t_1),…,Z(t_k)}
	\qquad∀ t_1,…,t_2∈ℝ,∀ k∈ℕ
\end{align*}

Es ist $ψ$ insbesondere (zwei Mal) stetig differenzierbar auf $ℝ\setminus\set{0}$ (vergleiche Bemerkung \ref{bemerkung10.7} unten) und streng monoton wachsend.
Dann folgt
\begin{align*}
	Z'(t)&=- N+2·ψ(t),\qquad N\sim\Nor(0,1)\\
	Z'(t)\overset{!}&{=}0⇒ψ(t)=\frac{1}{2} N\\
	&⇒ σ:=ψ^{-1}\argu{\frac{1}{2} N}\text{ ist Nullstelle von }Z'\\
	Z''(t)&=2·ψ'(t)\overset{\text{strenge Monotonie}}{>0}\qquad∀ t\neq 0\\
	% CHECKED: '''' used.
	&⇒ Z''(σ)>0\quad\P\text{-f.\,s.}
	% CHECKED: '''' used.
\end{align*}
Also ist $σ$ f.\,s.\ eindeutige Minimalstellen von $Z$.
Aus Satz \ref{satz10.4} folgt schließlich Satz \ref{satz10.8}.

\begin{bemerkungnr}[Smirnov]\label{bemerkung10.7}\enter
	$F$ besitze eindeutigen Median $m∈ℝ$ und erfülle die Bedingung \eqref{eqAnwendungVerteilungskonv_B}.
	Dann gilt:
	\begin{align*}
		∃ β,c,d>0:ψ(t)=
		\begin{cases}
			c t^β, &\falls t ≥ 0 \\
			-d\abs{t}^β, &\falls t < 0
		\end{cases}
	\end{align*}
\end{bemerkungnr}

\begin{satz}\label{satz10.8}
	\begin{align*}
		a_n \klammern{\hat{m}_n-m} \distrto ψ^{-1}\argu{\frac{1}{2} N(0,1)}
	\end{align*}
\end{satz}

\begin{beispiel}\label{beispiel10.9}
	Es gelte für beliebig kleines $δ>0$:
	\begin{align*}
		F(x):=
		\begin{cases}
			c \abs{x-m}^β + \frac{1}{2}, &\falls m ≤ x ≤ m + δ \\
			-d \abs{x-m}^β + \frac{1}{2}, &\falls m - δ ≤ x < m
		\end{cases}
		\qquad ∀ x ∈ ℝ
	\end{align*}
	Dann erfüllt $F$ die Bedingung \eqref{eqAnwendungVerteilungskonv_B} mit $a_n = n^{\frac{1}{2 β}}$ und $ψ$ wie in Bemerkung \ref{bemerkung10.7}.
	Dann gilt
	\begin{align*}
		n^{\frac{1}{2 β}} \klammern[\Big]{\hat{m}_n - m} \distrto ψ^{-1}\argu{\frac{1}{2} N}
	\end{align*}
\end{beispiel}

