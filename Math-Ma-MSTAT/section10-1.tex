% !TEX root = MSTAT19.tex
% This work is licensed under the Creative Commons
% Attribution-NonCommercial-ShareAlike 4.0 International License. To view a copy
% of this license, visit http://creativecommons.org/licenses/by-nc-sa/4.0/ or
% send a letter to Creative Commons, PO Box 1866, Mountain View, CA 94042, USA.

\subsection{Anwendung: Starke Konsistenz des empirischen Medians} %noNumber

\begin{align*}
	M_n(t) &:= \frac{1}{n} \sum_{i=1}^n \abs[\big]{X_i - t} &∀& t ∈ ℝ \\
	\overset{\text{SGGZ}}{⇒}
	M_n(t) \overset{n⟶∞}&{\longrightarrow} \Earg[\Big]{\abs[\big]{X_1-t}}
	\overset{\text{Def}}= M(t)\text{ f.\,s.} &∀& t∈ℝ^d
\end{align*}
Falls der Median $m$ von $F$ \emph{eindeutig}, d.\,h.\ $A(M)=\set{ m}$
(Beachte: hier ist $M$ deterministisch, also konstanter stochastischer Prozess),
so folgt aus Satz \ref{satz10.3} sofort für jede messbare Auswahl $\hat{m}_n∈ A(M_n)$:
\begin{align*}
	\hat{m}_n\ntoinf  m~~\P\text{-f.\,s.}
\end{align*}
Zum Beispiel $\hat{m}_n:=F^{-1}\klammern{\frac{1}{2}}$.

Das Argmin-Theorem für Verteilungskonvergenz ist:

\begin{satz}\label{satz10.4}
	Seien $Z,Z_n$, $n∈ℕ$ konvexe stochastische Prozesse auf $ℝ$ über $(Ω, \A, \P)$.
	Es gelte:
	\begin{enumerate}[label=(\arabic*)]
		\item \label{it:10.4argminEindeutig} $\begin{aligned}
			A(Z)=\set{σ}
		\end{aligned}$ f.\,s. für eine Zufallsvariable $σ$.
		\item \label{it:10.4fd} $\begin{aligned}
			Z_n\fdto  Z
		\end{aligned}$
	\end{enumerate}
	Dann gilt für jede messbare Auswahl $σ_n∈ A(Z_n)$:
	\begin{align*}
		σ_n \distrto σ \qquad \klammern{\argmin(Z_n) \distrto σ}
	\end{align*}
\end{satz}

\begin{proof}[Beweisskizze]
	\begin{align*}
		\ref{it:10.4fd}
		&⇒ Z_n\ntoinf Z\text{ in }\klammern[\big]{C(ℝ),d}\\
		&⇒ \limsup_{n⟶∞} \P\argu[\big]{σ_n ∈ K}
		≤ \P \argu{\set{σ} ∩ K \neq ∅}
		=\P \argu[\big]{σ ∈ K} \qquad ∀ K ∈ \mathcal{K}
		\qedhere
	\end{align*}
\end{proof}

