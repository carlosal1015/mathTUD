% This work is licensed under the Creative Commons
% Attribution-NonCommercial-ShareAlike 4.0 International License. To view a copy
% of this license, visit http://creativecommons.org/licenses/by-nc-sa/4.0/ or
% send a letter to Creative Commons, PO Box 1866, Mountain View, CA 94042, USA.

\section{Fast sichere Konvergenz}
\begin{definition} \label{definition: 3.6}
	Seien $X, X_n$ ($n \in \N$) Zufallsvariablen in einem separablen, metrischen Raum $(\S,d)$ über $(\Omega,\A,\P)$.
	\begin{equation*}
		X_n \ntoinf X \quad \P\text{-fast sicher}
		\defequiv
		\P\bigg( \menge{\omega \in \Omega : d\brackets{X_n(\omega),X(\omega)} \ntoinf 0} \bigg) = 1
	\end{equation*}
\end{definition}

Beachte: Die Definition von $M \defeq \menge{\omega \in \Omega : d\brackets{X_n(\omega),X(\omega)} \ntoinf 0}$ mengentheoretisch aufgeschrieben (Schnitt $\leadsto$ \enquote{für alle}; Vereinigung $\leadsto$ \enquote{es gibt}) ergibt
\begin{equation*}
	M =
	\bigcap_{0 < \epsilon \in \Q}
	\bigcup_{m \in \N}
	\bigcap_{n \ge m}
	\Big\{ d(X_n,X) < \epsilon \Big\} \overset{\text{\cref{satz: 3.5}}}{\in} \A
	\text{, denn }d(X_n,X)^{-1} \brackets{(-\infty,\epsilon)} \in  \A \satzende
\end{equation*}
Hierbei gilt $M \in \A$ aufgrund der Abzählbarkeit der beteiligten Schnitte und Vereinigungen. 
Man schreibt auch:
\begin{align*}
	\P\brackets{ \lim_{n \to \infty} X_n = X} = 1 \quad \text{ oder } \quad d(X_n, X) \to  0 \enskip (n \to \infty) \enskip \text{ $\P$-f.\,s.}
\end{align*}

Die bekannten Regeln (Ergebnisse) für \textit{reelle} Zufallsvariablen lassen sich mühelos verallgemeinern; so zum Beispiel der folgende Satz.
\begin{satz} \label{satz: 3.7}
	Der fast-sichere Grenzwert ist $\P$-fast sicher eindeutig:
	\begin{equation*}
		X_n \ntoinf X \quad \P\text{-fast sicher } \und 
		X_n \ntoinf X' \quad\P\text{-fast sicher }
		\follows 
		X = X' \quad \P\text{-fast sicher}
	\end{equation*}
\end{satz}
\begin{proof} 
	Es ist $\menge{ X\neq X'} \subseteq \menge{X_n \not\to  X} \cup \menge{X_n \not\to  X'}$ und $\P(X_n \not\to X) + \P(X_n \not\to X') = 0+0$. Also ist auch $\P(X \neq X') = 0$ und mit dem Gegenereignis gilt $\P\brackets{X = X'} = 1-\P\brackets{X\neq X'} = 1$.
\end{proof}

\begin{satz} \label{satz: 3.8}
	Seien $(X_n)_{n \in \N}$ und $X$ Zufallsvariablen im separablen, metrischen Raum $(\S,d)$ und sei $f \colon (\S,d) \to (\S',d')$ $\B_d(S)$ - $\B_{d'}(S')$-messbar und stetig in $X$ $\P$-fast sicher.
	\footnote{\enquote{stetig in $X$ $\P$-fast sicher} bedeutet,
		dass $\P(\menge{\omega  \in  \Omega : f \text{ in } X(\omega) \text{ stetig}}) = 1$. Insbesondere ist $\menge{\omega  \in  \Omega : f \text{ in } X(\omega) \text{ stetig}}$ messbar, was keine Selbstverständlichkeit ist.}
	Dann gilt:
	\begin{equation*}
		X_n \ntoinf	X \quad \P\text{-fast sicher }
		\follows f(X_n) \ntoinf f(X) \quad \P\text{-fast sicher}
	\end{equation*}
\end{satz}

\begin{proof}
	Wegen der Folgenstetigkeit gilt $\menge{X_n \ntoinf  X} \cap \menge{f \text{ stetig in }X}
		\subseteq
		\menge{f(X_n) \ntoinf f(X)}$.
	Da abzählbare Schnitte von Einsmengen stets Einsmengen sind, d.\,h.
	\begin{equation*}
		\forall i \in \N : \, \P(E_i)=1
		\follows \P\brackets{\bigcap_{i \in \N} E_i}=1
		\qquad \forall E_i \subseteq \Omega \mit \P(E_i)=1
	\end{equation*}
	folgt 
	\begin{equation*}
		1 = \P\brackets{\menge{X_n \ntoinf X \text{ und $f$ stetig in } X}} \le\P\brackets{\menge{f(X_n) \ntoinf  f(X)}} \le 1
	\end{equation*}
	und somit schon $\P\brackets{\menge{f(X_n) \ntoinf  f(X)}} = 1$.
\end{proof}

\begin{satz}[Konvergenz-Kriterium]\label{satz: 3.9}
	\begin{equation*}
		X_n \ntoinf X \quad \P\text{-fast sicher}
		\equivalent
		\forall\epsilon > 0: \lim_{n \to \infty} \P\brackets{\sup_{m\ge n} d(X_m,X) > \epsilon} = 0
	\end{equation*}
\end{satz}
\begin{proof}
	Man ersetze im Beweis für den Fall reeller Zufallsvariablen $\abs{X_n-X}$ durch $d(X_n,X)$.
	Und beachte, dass alle Schlussfolgerungen bestehen bleiben. Hier ausführlich:
	\begin{align*}
		X_n \overset{n\to \infty}{\longrightarrow} X~\P \text{-fast sicher}
		&\equivalent 1 = \P\brackets{\bigcup_{0 < \epsilon} \bigcap_{m  \in  \N} \bigcup_{n \ge m} d(X_n, X) < \epsilon} \\
		&\equivalent 0 = \P\brackets{\bigcap_{0 < \epsilon} \bigcup_{m  \in  \N} \bigcap_{n \ge m} d(X_n, X) \ge \epsilon} \\
		&\equivalent 0 = \P\brackets{\bigcup_{m  \in  \N} \bigcap_{n \ge m} d(X_n, X) \ge \epsilon} \qquad \forall \epsilon > 0
	\end{align*}
	Da für $m \to  \infty$ die Mengen $\menge{\bigcap_{n \ge m} d(X_n, X) \ge \epsilon}$ kleiner werden,
	ist dies äquivalent zu
	\begin{align*}
		\forall \epsilon > 0 \enskip \forall \delta > 0 \enskip \exists\, m  \in \N: \quad \P\brackets{\bigcap_{n \ge m} d(X_n, X) \ge \epsilon} &< \delta \\
		\equivalent \forall \epsilon > 0: \quad \lim_{m \to  \infty} \P\brackets{\bigcap_{n \ge m} d(X_n, X) \ge \epsilon} &= 0 \\
		\equivalent \forall \epsilon > 0: \quad \lim_{m \to  \infty} \P\brackets{\sup_{n \ge m} d(X_n, X) \ge \epsilon} &= 0
	\end{align*}
\end{proof}

Ein sehr nützliches Kriterium ist Folgendes:
\begin{satz}\label{satz: 3.10}
	\begin{align*}
		\forall \epsilon > 0: \quad 
		\sum_{n \in \N_{\ge1}}\P\brackets{d(X_n,X)>\epsilon} < \infty
		\quad \implies \quad  X_n \ntoinf X \quad \P\text{-fast sicher}
	\end{align*}
\end{satz}

\begin{proof}
	Setze $A_n(\epsilon) \defeq \menge{d(X_n,X)>\epsilon} \in \A$ wegen \cref{satz: 3.5}.
	Dann folgt aus dem \textit{ersten Borel-Cantelli-Lemma} $\displaystyle \P\brackets{\limsup_{n\to\infty} A_n(\epsilon)} = 0$ für alle $\epsilon > 0$. Mit
	\begin{align*}
		\liminf_{n \to \infty}\brackets{A_n(\epsilon)^\complement}
		\overset{\text{Def}}{=}
		\bigcup_{m \in \N}\bigcap_{n \ge  m} \brackets{A_n(\epsilon)}^\complement
		&= \brackets{\bigcap_{m \in \N} \bigcup_{n \ge m} \menge{ d(X_n,X) > \epsilon}}^\complement \\
		\overset{\text{Def}}&{=} \brackets{\limsup_{n \to \infty} A_n(\epsilon)}^\complement
	\end{align*}
	folgt dann
	\begin{equation*}
		1 = \P\brackets{\brackets{\limsup_{n\to\infty} A_n(\epsilon)}^\complement}
		= \P\brackets{\liminf_{n\to\infty} \brackets{A_n(\epsilon)^\complement}}
		\qquad \forall \epsilon > 0
	\end{equation*}
	Da abzählbare Durchschnitte von Eins-Mengen (also Mengen mit $\P$-Maß $1$) wieder Eins-Mengen sind, folgt schließlich:
	\begin{equation*}
		\P\brackets{\underbrace{
			\bigcap_{0 < \epsilon \in \Q}\bigcup_{m \in \N}\bigcap_{n\ge m}\menge{d(X_n,X)\le\epsilon}
		}_{\menge{ X_n\to X} = \menge{ d(X_n,X) \to 0}}}=1
	\end{equation*}
\end{proof}

\begin{*bemerkung}
	Die $\P(d(X_n, X) > \epsilon)$ werden in der Statistik \begriff{Fehlerwahrscheinlichkeit} oder \begriff{tail probability} genannt.
	Um Fehlerwahrscheinlichkeiten abzuschätzen, also die Voraussetzung für diesen Satz für einen speziellen Fall zu zeigen, nutzt man häufig sogenannte Maximalungleichungen wie die Markov-Ungleichung und die Tschebychew-Ungleichung.
\end{*bemerkung}

Weitere Eigenschaften der fast sicheren Konvergenz von Zufallsvariablen in metrischen Räumen finden sich z.\,B.\ in \cite[Kapitel 8.2]{gaensslerstute1977Wahrscheinlichkeitstheorie}.
%
