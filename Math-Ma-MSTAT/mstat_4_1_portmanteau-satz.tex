% This work is licensed under the Creative Commons
% Attribution-NonCommercial-ShareAlike 4.0 International License. To view a copy
% of this license, visit http://creativecommons.org/licenses/by-nc-sa/4.0/ or
% send a letter to Creative Commons, PO Box 1866, Mountain View, CA 94042, USA.

\begin{theorem}[Portmanteau-Theorem]\label{satz: 4.2}
	Folgende Aussagen sind äquivalent:
	\begin{enumerate}[label=(\arabic*)]
		\item \label{it:4.2weakConv} $\begin{aligned}
			P_n \weakto P
		\end{aligned}$
	\item \label{it:4.2f} $\begin{aligned}
			\int f \diffskip P_n \longrightarrow \int f \diffskip P \qquad \forall f \in C^b(\S) \text{ glm. stetig}
			% Standardfrage für die Prüfung: was sagt das Portmanteau-Theorem und was ist schwache Konvergenz?
		\end{aligned}$
	\item \label{it:4.2ClosedSets} $\begin{aligned}
			\limsup_{n \to \infty} P_n(F) \le P(F) \qquad \forall F \in \F(\S)
		\end{aligned}$
	\item \label{it:4.2OpenSets} $\begin{aligned}
			\liminf_{n \to \infty} P_n(G)\ge P(G) \qquad \forall G \in \G(\S)
		\end{aligned}$
	\item \label{it:4.2BorelSets} $\lim\limits_{n \to \infty} P_n(B) = P(B) \qquad \forall B \in \B(\S) \mit P(\underbrace{\partial B}_{ \in\F(\S)}) = 0$ \\
		Die Mengen $B \in \B(\S)$ mit $P(\partial B)=0$ heißen \begriff{$P$-randlos}.
	\end{enumerate}
\end{theorem}

\begin{proof}
	\begin{description}
		\item[\ref{it:4.2weakConv} $\Rightarrow$ \ref{it:4.2f}:] 
		Folgt aus \cref{definition: 4.1}.
		%
		\item[\ref{it:4.2f} $\Rightarrow$ \ref{it:4.2ClosedSets}:] 
		Sei $F \in\F(\S)$. Der Beweis von \cref{satz: 3.17} zeigt, dass es eine Folge $(f_k)_{k \in \N}$ von gleichmäßig stetigen und beschränkten Funktionen auf $\S$ gibt mit $f_k \downarrow \one_F$. Dann gilt:
		\begin{equation*}
			\limsup_{n \to \infty} P_n(F)
			= \limsup_{n \to \infty} \int \underbrace{\one_F}_{\le f_k} \diffskip P_n
			\overset{\text{Mon.}}{\le}
			\limsup_{n \to \infty} \int f_k \diffskip P_n
			\overset{\text{Vor.}}{=}
			\int f_k \diffskip P \quad \forall k \in \N 
			\satzende
		\end{equation*}
		Mit monotoner Konvergenz folgt daraus $\int f_k \diffskip P \longrightarrow \int \one_F \diffskip  P = P(F)$ und somit für $k \to \infty$ auch \ref{it:4.2ClosedSets}.
		%
		\item[\ref{it:4.2ClosedSets} $\Leftrightarrow$ \ref{it:4.2OpenSets}:]
		Nutze den Übergang zum Komplement sowie die Rechenregeln für $\liminf$ und $\limsup$. Sei $G$ offen, das heißt $G \in \G(\S)$ und damit $G^\complement \in \F(\S)$. Dann gilt
		\begin{equation*}
			\liminf_{n \to \infty} P_n(G)
			= \liminf_{n \to \infty} \brackets{1-\P_n(G^\complement)}
			= 1 - \underbrace{\limsup_{n \to \infty} P_n(G^\complement)}_{\le P(G^\complement)}
			\ge 1 - P(G^\complement)
			= P(G) 
			\satzende
		\end{equation*}
	\end{description}

%	\paragraph{Zeige \ref{it:4.2ClosedSets} $ \Rightarrow $ \ref{it:4.2weakConv}:}
%	Sei $f \in C^b(\S)$ beliebig. Zeige zunächst:
%	\begin{align}\label{eqProof1.4.2Sternchen}\tag{$\ast$}
%		\limsup_{n \to \infty}\int f\diffskip \P_n\le\int f\diffskip \P
%	\end{align}
%	\subparagraph{1. Schritt:} Sei $0\le f<1$. Setze
%	\begin{align*}
%		F_i:=\set{ f\ge\frac{i}{k}}=\set{ x \in\S:f(x)\ge\frac{i}{k}}
%		= f^{-1}\klammern{\intervallHO{\frac1k}{∞}},\qquad  \forall\; 0\le i\le k,\;k \in\N
%	\end{align*}
%	($F_i$ hängt auch von $k$ ab, aber diese Abhängigkeit wird in der Notation nicht
%	demonstriert.)
%	Dann gilt $F_i \in\F~ \forall i$, da $f$ stetig. Da
%	\begin{align*}
%		\int_{\S}f\diffskip \P
%		\stackeq{\text{Lin}}
%		\sum_{i=1}^k\int\indi_{\set{\frac{i-1}{k}\le f<\frac{i}{k}}}· f\diffskip \P
%	\end{align*}
%	folgt wegen Monotonie des Integrals
%	\begin{align}\label{eqProof1.4.2Plus}\tag{+}
%		\sum_{i=1}^k\underbrace{\frac{i-1}{k}}_{=\frac{i}{k}-\frac{1}{k}}·
%	\P\klammern{\frac{i-1}{k}\le f<\frac{i}{k}}
%		\le
%		\int f\diffskip \P
%		\le
%		\sum_{i=1}^k \frac{1}{k}· \P\klammern[\Bigg]{\underbrace{\frac{i-1}{k}\le f<\frac{i}{k}}_{F_{i-1}\setminus F_i}}
%	\end{align}
%	Die rechte Summe in \eqref{eqProof1.4.2Plus} ist gleich
%	\begin{align*}
%		&\frac{1}{k}·\sum_{i=1}^k i·\klammern[\big]{ \P(F_{i-1})-\P(F_i)}
%		\quad \text{da $F_i ⊂ F_{i-1}$}\\
%		&=\frac{1}{k}·\Big(\P(F_0)-\P(F_1)+2· \P(F_1)-2· \P(F_2)+3· \P(F_2)-3· \P(F_3)+\\
%		% CHECKED: '\Big' used.
%		&\qquad+…+(k-1)· \P(F_{k-2})-(k-1)· \P(F_{k-1})+k· \P(F_{k-1})-k· \P(F_k)\Big)\\
%		% CHECKED: '\Big' used.
%		&=\frac{1}{k}·\klammern[\Big]{\underbrace{\P(F_0)}_{=1}+\P(F_1)+\P(F_2)+…+\P(F_{k-1})- k· \underbrace{\P(F_k)}_{=0}} \quad (0 \le f < 1)\\
%		&=\frac{1}{k}+\frac{1}{k}·\sum_{i=1}^{k-1} \P(F_i)
%	\end{align*}
%	Da die linke Summe in \eqref{eqProof1.4.2Plus} gleich der rechten Summe in \eqref{eqProof1.4.2Plus} minus $\frac{1}{k}$ ist, folgt
%	\begin{align}\label{eqProof1.4.2DoppelSternchen}\tag{$\ast\ast$}
%		\sum_{i=1}^{k-1} \P(F_i)
%		\le\int f\diffskip P
%		\le\frac{1}{k}+\frac 1k \sum_{i=1}^{k-1} \P(F_i)
%	\end{align}
%	Beachte, \eqref{eqProof1.4.2DoppelSternchen} gilt für \emph{jedes} Wahrscheinlichkeitsmaß $\P$, also auch für $\P_n$.
%	Damit folgt:
%	\begin{align*}
%		\limsup_{n \to \infty}\int f\diffskip \P_n
%		\overset{\eqref{eqProof1.4.2DoppelSternchen}}&{\le}
%		\frac{1}{k}+\frac 1k \sum_{i=1}^{k-1}\underbrace{
%			\limsup_{n \to \infty} \P_n(F_i)
%		}_{\overset{\ref{it:4.2ClosedSets}}{\le}\P(F_i)~ \forall i}\\
%		\overset{\ref{it:4.2ClosedSets}}&{\le}
%		\frac{1}{k}+\underbrace{\frac 1k \sum_{i=1}^{k-1} \P(F_i)}_{
%			\overset{\eqref{eqProof1.4.2DoppelSternchen}}{\le}\int f\diffskip \P
%		}\\
%		\overset{\eqref{eqProof1.4.2DoppelSternchen}}&{\le}
%		\frac{1}{k}+\int f\diffskip \P\qquad \forall k \in\N
%	\end{align*}
%	Grenzwertbildung $k \to \infty$ liefert \eqref{eqProof1.4.2Sternchen}.
%	\paragraph{2. Schritt:} Da $f \in C^b(\S)$ beliebig, gilt wegen Beschränktheit von $f$:
%	\begin{align*}
%		∃ a<b:a\le f<b
%		 \Rightarrow  g(x):=\frac{f(x)-a}{b-a}\text{ ist stetig und } 0\le g<1
%	\end{align*}
%	Daraus folgt
%	\begin{align*}
%		\limsup_{n \to \infty}\int f\diffskip \P_n
%		&=\limsup_{n \to \infty}\int (b-a)· g+a\,\diffskip \P_n\\
%		&=\limsup_{n \to \infty}\klammern{(b-a)·\int g\,\diffskip \P_n+a}\\
%		&\le(b-a)·\underbrace{\limsup_{n \to \infty}\int g\diffskip \P_n}_{\le\int g\diffskip P\text{, wg.\ 1.\ Schritt}}+a\\
%		&\le(b-a)·\int g\diffskip \P + a\\
%		\overset{\text{Lin}}&=
%		\int f\diffskip \P
%	\end{align*}
%	Damit ist \eqref{eqProof1.4.2Sternchen} gezeigt. Übergang zu $-f$ in \eqref{eqProof1.4.2Sternchen} liefert
%	\begin{align*}
%		\liminf_{n \to \infty}\int f\diffskip \P_n
%		&=\liminf_{n \to \infty}-\int -f\diffskip \P_n\\
%		&=-\limsup_{n \to \infty}\int \underbrace{-f}_{ \in C^b(\S)}\diffskip \P\\
%		\overset{\eqref{eqProof1.4.2Sternchen}}&{\ge}
%		-\int -f\diffskip \P\\
%		\overset{\text{Lin}}&=
%		\int f\diffskip \P
%		\qquad \forall f \in C^b(\S)\\
%		 \Rightarrow  \intf \diffskip\P &\le \liminf_{n → ∞} \intf\diffskip\P_n \le \limsup_{n → ∞} \intf\diffskip\P_n \overset{\eqref{eqProof1.4.2Sternchen}}{\le} \intf \diffskip\P \\
%		 \Rightarrow  \lim_{n → ∞} \int f \diffskip\P_n &= \intf \diffskip\P \quad  \forall f  \in C^b(S) \\
%		& \Rightarrow \ref{it:4.2weakConv}
%	\end{align*}
%
%	\subparagraph{Zeige \ref{it:4.2ClosedSets} $ \Rightarrow $ \ref{it:4.2BorelSets}:}
%	Sei $B \in\B(\S)\mit \P(\partial B)=0$. Dann gilt:
%	\begin{align*}
%		\P(\overline{B})
%		\overset{\ref{it:4.2ClosedSets}}&{\ge}
%		\limsup_{n \to \infty} \underbrace{\P_n(\overbrace{\overline{B}}^{⊇ B})}_{\ge \P_n(B)}
%		\ge\limsup_{n \to \infty} \P_n(B)
%		\overset{\text{stets}}{\ge}
%		\liminf_{n \to \infty} \underbrace{\P_n(\overbrace{B}^{⊇\overset{∘}{B}})}_{
%			\ge \P_n(\inner B)
%		}\\
%		\overset{\ref{it:4.2ClosedSets}⇔ \ref{it:4.2OpenSets}}&{\ge}
%		\P\klammern[\big]{\inner B}
%		=\P(\overline{B})
%		=\P(B),
%	\end{align*}
%	denn:
%	\begin{align*}
%		0
%		=\P(\overbrace{\partial B}^{\overline{B}\setminus \inner B})
%		=\P(\overline{B})-\P(\inner B)
%		 \Rightarrow 
%		\P(\inner B)\le \P(B)\le \P(\overline{B})=\P(\inner B)
%	\end{align*}
%	Damit folgt $\liminf=\limsup$ und folglich $\lim_{n \to \infty} \P_n(B)=\P(B)$.
%	\paragraph{Zeige \ref{it:4.2BorelSets} $ \Rightarrow $ \ref{it:4.2ClosedSets}:}
%	Sei $F \in\F$ (abgeschlossen) beliebig. Dann gilt $ \forallε>0:$
%	\begin{align}\label{eqProof1.4.2SternchenUnten}\tag{$\ast$}
%		\partial\,\set{ x \in\S:d(x,F)\leε}
%		⊆\set{ x \in\S:d(x,F)=ε}
%	\end{align}
%	denn: Sei $x \in\partial\set{ x \in\S:d(x,F)\leε}$. Dann gilt:
%	\begin{align*}
%		&∃ (x_n)_{n \in\N}: \forall n \in\N:d(x_n,F)\leε∧ \lim_{n \to \infty} x_n=x\\
%		&∃ (ζ_n)_{n \in\N}: \forall n \in\N:d(ζ_n,F)>ε∧ \lim_{n \to \infty} ζ_n=x
%	\end{align*}
%	Da $d(·,F)$ stetig ist gemäß \ref{lemma2.3} \ref{it:distStetig}, folgt
%	\begin{align*}
%		ε \le \lim_{n → ∞} d(ζ_n, F) = d(x,F) = \lim_{n → ∞} d(x_n, F) \le ε.
%	\end{align*}
%	Wegen \eqref{eqProof1.4.2SternchenUnten} sind
%	\begin{align*}
%		A_ε:=\partial\set{ x \in\S:d(x,F)\leε}\qquad \forallε>0
%	\end{align*}
%	paarweise disjunkt, da bereits die Obermengen paarweise disjunkt sind. Dann folgt
%	\begin{align}\label{eqProof1.4.2DoppelSternchenUnten}\tag{$\ast\ast$}
%		E:=\set{ε>0:\P(A_ε)>0}\text{ ist höchstens abzählbar},
%	\end{align}
%	denn:
%	\begin{align*}
%		E=\Union_{m  \in \N} \underbrace{\set{ε>0: \P(A_ε) \ge \frac1m}}_{=:E_m}
%	\end{align*}
%	Wir möchten eine Obergrenze für die Größe der $E_m$ finden. Dafür beachte
%	\begin{align*}
%		1\ge \P\klammern{\Union_{ε  \in E_m} A_{ε}}
%		\stackeq{\text{pw.\,disj.}}
%		\sum_{ε  \in E_m}\underbrace{\P(A_{ε})}_{\ge\frac{1}{m}} \ge \abs{E_m}·\frac{1}{m}  \Rightarrow  \abs{E_m} \le m.
%	\end{align*}
%	% Fergers Variante:
%	% Es gilt $\abs{E_m}\le m$, weil: Angenommen es existieren $0<ε_1<…<ε_{m+1}$ mit
%	% % CHECKED: '|'
%	% \begin{align*}
%	% 	&\P(A_{ε_i})\ge\frac{1}{m}\qquad \forall 1\le i\le m+1\\
%	% 	& \Rightarrow 
%	% 	1\ge \P\left(\Union_{i=1}^{m+1} A_{ε_i}\right)
%	% 	\stackeq{\text{pw. disj.}}
%	% 	\sum_{i=1}^{m+1}\underbrace{\P\klammern[\big]{A_{ε_i}}}_{\ge\frac{1}{m}}\ge(m+1)·\frac{1}{m}>1
%	% \end{align*}
%	% Das ist ein Widerspruch.
%	Damit ist $E$ abzählbare Vereinigung endlicher Mengen, also höchstens abzählbar unendlich.
%	Damit liegt das Komplement
%	\begin{align*}
%		E^C=\set[\big]{ε>0: \P(A_ε)=0}
%	\end{align*}
%	dicht in $[0,∞)$.
%	(Dies kann man durch Widerspruch zeigen: in $E$ kann kein Intervall positiver Länge enthalten sein, denn diese enthalten überabzählbar viele Elemente)
%	Daraus folgt insbesondere:
%	\begin{align*}
%		∃(ε_k)_{k \in\N}⊆ℝ\mitε_k\diffskipownarrow0: \forall k \in\N:
%		F_k:=\set[\big]{ x \in\S:d(x,F)\leε_k}\text{ ist $\P$-randlos}
%	\end{align*}
%	Beachte $A_{ε_k}=\partial F_k$. Wähle also $ε_k  \in E^C$ für alle $k  \in \N$. Da $F⊆ F_k~ \forall k \in\N$, gilt:
%	\begin{align*}
%		\limsup_{n \to \infty} \P_n(F)
%		\le\limsup_{n \to \infty} \underbrace{\P_n(F_k)}_{\text{konv.}}
%		\overset{\ref{it:4.2BorelSets}}&{=}
%		\P(F_k)\qquad \forall k \in\N\\
%		\overset{k \to \infty}{ \Rightarrow }
%		\limsup_{n ⟶ ∞} \P_n(F)
%		\le\lim_{k \to \infty} \P(F_k)
%		&=\P(F)
%	\end{align*}
%	Die letzte Gleichheit gilt, weil $\P$ $σ$-stetig von oben ist und $F_k\diffskipownarrow F$.
%	$F_k\diffskipownarrow F$, denn $F_1⊇ F_2⊇…$, da $ε_k$ monoton fallende Folge ist und
%	\begin{align*}
%		\bigcap_{k \in\N}F_k=F,
%	\end{align*}
%	denn:
%	\begin{align*}
%		x \in\bigcap_{k \in\N}F_k
%		&⇔
%		x \in F_k &  \forall k \in\N\\
%		&⇔
%		d(x,F)\leε_k &  \forall k \in\N\\
%		& \Rightarrow 
%		d(x,F)=0\\
%		\overset{\ref{lemma2.3}~\ref{it:distCharakterisierung}}&{⇔}
%		x \in \overline{F}\stackeq{F \in\F}F
%		&& \qedhere
%	\end{align*}
\end{proof}