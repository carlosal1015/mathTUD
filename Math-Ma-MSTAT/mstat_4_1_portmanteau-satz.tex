% This work is licensed under the Creative Commons
% Attribution-NonCommercial-ShareAlike 4.0 International License. To view a copy
% of this license, visit http://creativecommons.org/licenses/by-nc-sa/4.0/ or
% send a letter to Creative Commons, PO Box 1866, Mountain View, CA 94042, USA.

\begin{theorem}[Portmanteau-Theorem]\label{satz: 4.2}
	Folgende Aussagen sind äquivalent:
	\begin{enumerate}[label=(\arabic*)]
		\item \label{it:4.2weakConv} $\begin{aligned}
			P_n \weakto P
		\end{aligned}$
	\item \label{it:4.2f} $\begin{aligned}
			\int f \diffskip P_n \longrightarrow \int f \diffskip P \qquad \forall f \in C^b(\S) \text{ glm. stetig}
			% Standardfrage für die Prüfung: was sagt das Portmanteau-Theorem und was ist schwache Konvergenz?
		\end{aligned}$
	\item \label{it:4.2ClosedSets} $\begin{aligned}
			\limsup_{n \to \infty} P_n(F) \le P(F) \qquad \forall F \in \F(\S)
		\end{aligned}$
	\item \label{it:4.2OpenSets} $\begin{aligned}
			\liminf_{n \to \infty} P_n(G)\ge P(G) \qquad \forall G \in \G(\S)
		\end{aligned}$
	\item \label{it:4.2BorelSets} $\lim\limits_{n \to \infty} P_n(B) = P(B) \qquad \forall B \in \B(\S) \mit P(\underbrace{\partial B}_{ \in\F(\S)}) = 0$ \\
		Die Mengen $B \in \B(\S)$ mit $P(\partial B)=0$ heißen \begriff{$P$-randlos}.
	\end{enumerate}
\end{theorem}

\begin{proof}
	\begin{description}
		\item[\ref{it:4.2weakConv} $\Rightarrow$ \ref{it:4.2f}:] 
		Folgt aus \cref{definition: 4.1}.
		%
		\item[\ref{it:4.2f} $\Rightarrow$ \ref{it:4.2ClosedSets}:] 
		Sei $F \in\F(\S)$. Der Beweis von \cref{satz: 3.17} zeigt, dass es eine Folge $(f_k)_{k \in \N}$ von gleichmäßig stetigen und beschränkten Funktionen auf $\S$ gibt mit $f_k \downarrow \one_F$. Dann gilt:
		\begin{equation*}
			\limsup_{n \to \infty} P_n(F)
			= \limsup_{n \to \infty} \int \underbrace{\one_F}_{\le f_k} \diffskip P_n
			\overset{\text{Mon.}}{\le}
			\limsup_{n \to \infty} \int f_k \diffskip P_n
			\overset{\text{Vor.}}{=}
			\int f_k \diffskip P \quad \forall k \in \N 
			\satzende
		\end{equation*}
		Mit monotoner Konvergenz folgt daraus $\int f_k \diffskip P \longrightarrow \int \one_F \diffskip  P = P(F)$ und somit für $k \to \infty$ auch \ref{it:4.2ClosedSets}.
		%
		\item[\ref{it:4.2ClosedSets} $\Leftrightarrow$ \ref{it:4.2OpenSets}:]
		Nutze den Übergang zum Komplement sowie die Rechenregeln für $\liminf$ und $\limsup$. Sei $G$ offen, das heißt $G \in \G(\S)$ und damit $G^\complement \in \F(\S)$. Dann gilt
		\begin{equation*}
			\liminf_{n \to \infty} P_n(G)
			= \liminf_{n \to \infty} \brackets{1-\P_n(G^\complement)}
			= 1 - \underbrace{\limsup_{n \to \infty} P_n(G^\complement)}_{\le P(G^\complement)}
			\ge 1 - P(G^\complement)
			= P(G) 
			\satzende
		\end{equation*}
		%
		\item[\ref{it:4.2ClosedSets} $\Rightarrow$ \ref{it:4.2weakConv}:] Sei $f \in C^b(\S)$ beliebig. Wir zeigen zunächst, dass
		\begin{equation}\label{eqProof1.4.2Sternchen}
			\limsup_{n \to \infty} \int f \diffskip P_n \le \int f \diffskip P
		\end{equation}
		\begin{description}
			\item[Schritt 1:] Sei $0 \le f < 1$. Setze $F_i \defeq \menge{ f \ge \frac{i}{k} } = \menge{ x \in \S : f(x) \ge \frac{i}{k} } = f^{-1}([\frac{i}{k}, \infty))$ für alle $0 \le i \le k$ und $k \in \N$.
			Dann gilt $F_i \in \F$ für alle zulässigen $i$, da die Urbilder abgeschlossener Mengen und der stetigen Funktion $f$ wieder abgeschlossen sind. Wegen
			\begin{equation*}
				\int_{\S} f \diffskip P
				\overset{\text{Lin.}}{=}
				\sum_{i=1}^k \int \one_{\menge{\frac{i-1}{k} \le f < \frac{i}{k}}} * f \diffskip P
			\end{equation*}
			folgt mit der Monotonie des Integrals und der Abschätzung des Integranten nach oben und unten mittels der Indikatorfunktion
			\begin{equation} \label{eqProof1.4.2Plus}
				\sum_{i=1}^k\underbrace{\frac{i-1}{k}}_{=\frac{i}{k}-\frac{1}{k}} *
				P\brackets{\frac{i-1}{k} \le f < \frac{i}{k}}
				\le \int f \diffskip P
				\le \sum_{i=1}^k \frac{1}{k} * P \bigg( \underbrace{\frac{i-1}{k} \le f < \frac{i}{k}}_{F_{i-1} \setminus F_i} \bigg)
				\satzende
			\end{equation}
			Die rechte Summe in \eqref{eqProof1.4.2Plus} ist gleich
			\begin{align*}
				&\frac{1}{k} * \sum_{i=1}^k i * \brackets{P(F_{i-1}) - P(F_i)}
				\quad \text{da $F_i \subseteq F_{i-1}$}\\
				&= \frac{1}{k} * \Big( P(F_0) - P(F_1) + P(F_1) - 2 * P(F_2) + 3 * P(F_2) - 3 * P(F_3) + \\
				&\qquad + \dots + (k-1) * P(F_{k-2}) - (k-1) * P(F_{k-1}) + k * P(F_{k-1}) - k * P(F_k) \Big) \\
				&= \frac{1}{k} * \Big( \underbrace{P(F_0)}_{=1} + P(F_1) + P(F_2) + \dots + P(F_{k-1}) - k * \underbrace{P(F_k)}_{=0} \Big) \quad (0 \le f < 1) \\
				&= \frac{1}{k} + \frac{1}{k} * \sum_{i=1}^{k-1} P(F_i) 
				\satzende
			\end{align*}
			Da die linke Summe in \eqref{eqProof1.4.2Plus} gleich der rechten Summe in \eqref{eqProof1.4.2Plus} minus $\frac{1}{k}$ ist, folgt
			\begin{equation} \label{eqProof1.4.2DoppelSternchen}
				\frac{1}{k} \sum_{i=1}^{k-1} P(F_i) 
				\le \int f \diffskip P
				\le \frac{1}{k} + \frac{1}{k} \sum_{i=1}^{k-1} P(F_i) 
				\satzende
			\end{equation}
			Man beachte, dass \eqref{eqProof1.4.2DoppelSternchen} für \emph{jedes} Wahrscheinlichkeitsmaß $P$ gilt, also insbesondere auch für $P_n$.
			Damit folgt
			\begin{align*}
				\limsup_{n \to \infty} \int f \diffskip P_n
				\overset{\eqref{eqProof1.4.2DoppelSternchen}}&{\le}
				\frac{1}{k} + \frac{1}{k} \sum_{i=1}^{k-1} \underbrace{ \limsup_{n \to \infty} P_n(F_i) }_{\overset{\ref{it:4.2ClosedSets}}{\le}P(F_i) ~ \forall i} \\
				\overset{\ref{it:4.2ClosedSets}}&{\le} \frac{1}{k} + \underbrace{\frac{1}{k} \sum_{i=1}^{k-1} P(F_i)}_{\overset{\eqref{eqProof1.4.2DoppelSternchen}}{\le}\int f \diffskip P} \\
				\overset{\eqref{eqProof1.4.2DoppelSternchen}}&{\le} \frac{1}{k} + \int f \diffskip P
			\end{align*}
			für alle $k \in \N$. Der Grenzübergang $k \to \infty$ liefert schließlich \eqref{eqProof1.4.2Sternchen}.
			\item[Schritt 2:] Da $f \in C^b(\S)$ beliebig, gilt wegen Beschränktheit von $f$, dass $a,b \in \S$ existieren, sodass $a \le f < b$ gilt. Damit ist die Funktion 
			\begin{equation*}
				g(x) \defeq \frac{f(x) - a}{b - a}
			\end{equation*}
			stetig und es gilt $0 \le g < 1$. Mit dem Ergebnis von Schritt 1 erhalten wir
			\begin{align*}
				\limsup_{n \to \infty} \int f \diffskip P_n
				&=\limsup_{n \to \infty} \int (b-a) * g + a \diffskip P_n\\
				&=\limsup_{n \to \infty} \brackets{(b-a) * \int g \diffskip P_n + a} \\
				&\le (b-a) * \underbrace{\limsup_{n \to \infty} \int g \diffskip P_n}_{\le \int g \diffskip P} + a \\
				&\le (b-a) * \int g \diffskip P + a \\
				\overset{\text{Lin.}}&=
				\int f \diffskip P
			\end{align*}
		\end{description}
		Damit ist \eqref{eqProof1.4.2Sternchen} gezeigt. Der Übergang zu $-f$ in \eqref{eqProof1.4.2Sternchen} liefert
		\begin{align*}
			\liminf_{n \to \infty} \int f \diffskip P_n
			=\liminf_{n \to \infty} - \int -f\diffskip P_n 
			&= - \limsup_{n \to \infty} \int \underbrace{-f}_{\in C^b(\S)} \diffskip P \\
			\overset{\eqref{eqProof1.4.2Sternchen}}&{\ge}
			-\int -f \diffskip P \\
			\overset{\text{Lin.}}&=
			\int f \diffskip P
			\qquad \forall f \in C^b(\S) \satzende
		\end{align*}
		Daraus folgt nun 
		\begin{equation*}
			\int f \diffskip P \le \liminf_{n \to \infty} \int f \diffskip P_n \le \limsup_{n \to \infty} \int f \diffskip P_n \overset{\eqref{eqProof1.4.2Sternchen}}{\le} \int f \diffskip P \komma
		\end{equation*}
		was via $\lim_{n \to \infty} \int f \diffskip P_n = \int f \diffskip P$ für alle $f  \in C^b(S)$ nun \ref{it:4.2weakConv} impliziert.
		%
		\item[\ref{it:4.2ClosedSets} $ \Rightarrow $ \ref{it:4.2BorelSets}:]
		Sei $B \in\B(\S)$ mit $P(\partial B) = 0$. Zunächst gilt
		\begin{equation*}
			0
			= P(\overbrace{\partial B}^{\overline{B} \setminus \inner B})
			= P(\overline{B}) - P(\inner B)
			\Rightarrow 
			P(\inner B)
			\le P(B)
			\le P(\overline{B}) 
			= P(\inner B)
			\satzende
		\end{equation*}
		Damit folgt dann durch
		\begin{align*}
			P(\overline{B})
			\overset{\ref{it:4.2ClosedSets}}&{\ge}
			\limsup_{n \to \infty} \underbrace{P_n(\overline{B})}_{\ge P_n(B)}
			\ge \limsup_{n \to \infty} P_n(B)
			\ge
			\liminf_{n \to \infty} \underbrace{P_n(B)}_{ \ge P_n(\inner B) } \\
			\overset{\ref{it:4.2ClosedSets} \Leftrightarrow \ref{it:4.2OpenSets}}&{\ge}
			P\brackets{\inner B}
			= P(\quer{B})
			= P(B)
		\end{align*}
		die Gleichheit von Limes inferior und Limes superior und somit existiert der Grenzwert $\lim_{n \to \infty} P_n(B) = P(B)$.
		%
		\item[\ref{it:4.2BorelSets} $\Rightarrow$ \ref{it:4.2ClosedSets}:] 
		Sei $F \in \F$ beliebig. Dann gilt für alle $\epsilon > 0$
		\begin{equation}
			\label{eqProof1.4.2SternchenUnten}
			\partial \menge{ x \in \S : d(x,F) \le \epsilon} \subseteq \menge{ x \in \S : d(x,F)= \epsilon}
		\end{equation}
		denn: Sei $x \in \partial\menge{ x \in \S : d(x,F) \le \epsilon}$. Dann gilt mit der Charakterisierung des Abschlusses als Grenzwerte von Folgen einer Menge:
		\begin{align*}
			&\exists  (x_n)_{n \in \N} : \forall n \in \N : d(x_n,F) \le \epsilon \und \lim_{n \to \infty} x_n = x \komma \\
			&\exists  (\zeta_n)_{n \in \N} :  \forall n \in \N : d(\zeta_n,F) > \epsilon \und  \lim_{n \to \infty} \zeta_n = x \satzende
		\end{align*}
		Da $d( * ,F)$ stetig ist gemäß \cref{lemma: 2.3} \ref{it: distStetig}, folgt
		\begin{equation*}
			 \epsilon \le \lim_{n \to \infty} d(\zeta_n, F) = d(x,F) = \lim_{n \to \infty} d(x_n, F) \le \epsilon \satzende
		\end{equation*}
		Wegen \eqref{eqProof1.4.2SternchenUnten} sind
		\begin{align*}
			A_\epsilon \defeq \partial \menge{ x \in \S : d(x,F) \le \epsilon} \qquad \forall \epsilon > 0
		\end{align*}
		paarweise disjunkt, da bereits die Obermengen paarweise disjunkt sind. Dann folgt
		\begin{equation} \label{eqProof1.4.2DoppelSternchenUnten}
			E \defeq \menge{ \epsilon>0 : P(A_\epsilon) > 0} \text{ ist höchstens abzählbar,}
		\end{equation}
		denn:
		\begin{align*}
			E = \bigcup_{m \in \N} \underbrace{\menge{ \epsilon > 0 : P(A_\epsilon) \ge \frac{1}{m}}}_{ \defqe E_m} \satzende
		\end{align*}
		Es gilt $\card{E_m} \le m$ --- Angenommen es existieren $0 < \epsilon_1 < \dots < \epsilon_{m+1}$ mit
		\begin{equation*}
		 	P(A_{\epsilon_i}) \ge \frac{1}{m} \quad (1 \le i \le m+1) 
			\follows 
		 	1 \ge P\left(\bigcupdot_{i=1}^{m+1} A_{\epsilon_i} \right)
		 	= \sum_{i=1}^{m+1} \underbrace{ P\brackets{A_{\epsilon_i}}}_{\ge \frac{1}{m}} \ge (m+1) * \frac{1}{m} > 1
		\end{equation*}
		Das ist ein Widerspruch.
		Damit ist $E$ abzählbare Vereinigung endlicher Mengen, also höchstens abzählbar unendlich.
		Damit liegt das Komplement
		\begin{align*}
			E^\complement = \menge{\epsilon > 0 : P(A_\epsilon) = 0}
		\end{align*}
		dicht in $[0,\infty)$. (Dies kann man durch Widerspruch zeigen: angenommen es existiert ein $x \ge 0$ mit $x \notin \quer{E^\complement}$. Gemäß einer Übungsaufgabe existiert dann ein $r > 0$ mit $B(x,r) \cap E^\complement = \emptyset$, also mit $B(x,r) \subseteq \brackets{E^\complement}^\complement = E$. Da $B(x,r)$ überabzählbar ist, müsste auch $E$ überabzählbar sein, was eben nicht der Fall ist).
		Daraus folgt insbesondere
		\begin{align*}
			\exists (\epsilon_k)_{k \in \N} \subseteq \R \mit \epsilon_k \downarrow 0 : \forall k \in \N : 
			F_k \defeq \menge{ x \in \S : d(x,F) \le \epsilon_k} \text{ ist $P$-randlos.}
		\end{align*}
		Beachte $A_{\epsilon_k} = \partial F_k$. Wähle also $\epsilon_k  \in E^\complement$ für alle $k \in \N$. Da $F \subseteq F_k$ für alle $k \in \N$ ist, gilt:
		\begin{equation*}
			\limsup_{n \to \infty} P_n(F)
			\le \limsup_{n \to \infty} \underbrace{P_n(F_k)}_{\text{konv.}}
			\overset{\ref{it:4.2BorelSets}}{=}
			P(F_k) \qquad \forall k \in \N
		\end{equation*}
		Für $k \to \infty$ folgt daraus $\limsup\limits_{n \to \infty} P_n(F) \le \lim_{k \to \infty} P(F_k) = P(F)$.
		Die letzte Gleichheit gilt, weil $P$ $\sigma$-stetig von oben ist und $F_k \downarrow F$. Dies gilt, da $(\epsilon_k)$ monton fallend ist und somit $F_1 \supseteq F_2 \supseteq \dots$ gilt; sowie $\bigcap\limits_{k \in\N} F_k = F$, denn:
		\begin{equation*}
			x \in\bigcap_{k \in\N} F_k
			\qquad \Leftrightarrow \qquad
			x \in F_k \quad \forall k \in \N 
			\qquad \Leftrightarrow \qquad 
			d(x,F) \le \epsilon_k \quad \forall k \in \N 
		\end{equation*}
		Dies impliziert dann $d(x,F) = 0$, was wegen \cref{lemma: 2.3} äquivalent zu $x \in \overline{F} = F$ ist.
	\end{description}
\end{proof}