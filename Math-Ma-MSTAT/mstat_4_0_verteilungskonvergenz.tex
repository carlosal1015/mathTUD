% This work is licensed under the Creative Commons
% Attribution-NonCommercial-ShareAlike 4.0 International License. To view a copy
% of this license, visit http://creativecommons.org/licenses/by-nc-sa/4.0/ or
% send a letter to Creative Commons, PO Box 1866, Mountain View, CA 94042, USA.

\section{Verteilungskonvergenz von Zufallsvariablen in metrischen Räumen} %4

Seien $X$ und $X_n$ ($n\in\N$) Zufallsvariablen in $(\S,d)$ über $(\Omega,\A,\P)$.
Dann sind
\begin{equation*}
	P \defeq \P \circ X^{-1}, \qquad P_n \defeq \P \circ X_n^{-1} \qquad (n \in \N)
\end{equation*}
Wahrscheinlichkeitsmaße auf $\B(\S)$.

\begin{definition}[Verteilungskonvergenz] \label{definition: 4.1}\
	\begin{enumerate}[label=(\arabic*)]
		\item \label{it: 4.1Pweak} Seien $\P$ und $\P_n$ ($n \in \N$) Wahrscheinlichkeitsmaße auf $\B(S)$.
		Dann \begriff{konvergiert $\P_n$ schwach} gegen $\P$, in Zeichen
		\begin{equation*}
			\P_n \weakto \P \defequiv \int f \diffskip{\P_n} \ntoinf \int f \diffskip\P \qquad \forall f \in C^b(\S)
		\end{align*}
		\item \label{it: 4.1Xweak}$X_n$ \begriff{konvergiert in Verteilung} gegen $X$ in Raum $(\S,d)$, in Zeichen
			\begin{align*}
				X_n \distto  X \text{ in }(\S,d) \defequiv \P \circ X_n^{-1} \weakto \P \circ X^{-1}
			\end{align*}
			Das $\L$ steht für \enquote{law}. Alternative Schreibweise: $X_n \overset{\d}{\longrightarrow} X$.
	\end{enumerate}
\end{definition}

Der folgende Satz gibt eine äquivalente Charakterisierung von $\weakto $ bzw.\ $\distto$.


% ist sehr lang und daher ausgelagert:
\input{section4-Portmanteau}

Falls $\mathcal{U} = \menge{B \in \B(S) : \P(\rand{B}) = 0}$, so besagt
\cref{satz: 4.2}, \cref{it: 4.2BorelSets}, dass $P_n(B) \to P(B)$ für alle $B \in \mathcal{U}$, also punktweise Konvergenz auf der Klasse $\mathcal{U}$, die schwache Konvergenz $P_n \weakto P$ nach sich zieht.
Es gibt weitere solcher Klassen.
% Der letzte Absatz ersetzt folgenden alten Satz:
% Mitunter folgt schwache Konvergenz aus $\P_n(A)\ntoinf  \P(A)$ für eine spezielle Klasse von Mengen $A$.

\begin{theorem} \label{theorem: 4.3}
	Sei $\U \subseteq \B(\S)$ mit
	\begin{enumerate}[label=(\roman*)]
		\item \label{it: 4.3Schnittstabil} $\U$ ist endlich schnittstabil, d.h. $A,B \in \U \follows A \cap B \in \U$
        \item \label{it: 4.3Basis} Jedes offene $G \subseteq \S$ ist abzählbare Vereinigung von Mengen aus $\U$.
	\end{enumerate}
	Dann gilt: $\forall A \in \U: \P_n(A)\ntoinf \P(A) \follows \P_n \weakto \P$
\end{theorem}

\begin{proof}
	Seien $A_1,\dots,A_m \in \U$. Dann gilt:
	\begin{align*}
		\P_n \brackets{\bigcup_{i=1}^m A_i}
		&= \sum_{k=1}^m (-1)^{k-1}·\sum_{1 \le i_1 < \dots < i_k \le m} \underbrace{\P_n(\overbrace{A_{i_1} \cap \dots \cap A_{i_k}}^{\in\U})}_{\ntoinf  \P(A_{i_1}\cap\dots\cap A_{i_k})}\\
		\overset{n\to\infty}&{\longrightarrow}
		\sum_{k=1}^m\sum_{1\le i_1<\dots<i_k\le m} \P\brackets{A_{i_1}\cap\dots\cap A_{i_k}}\\
		\overset{\text{allg.\ Add.}}&=
		\P\brackets{\bigcup_{i=1}^m A_i}
	\end{align*}
	Sei $G\in\G$, also offen. Dann gilt wegen Voraussetzung \ref{it: 4.3Basis}:
	\begin{align*}
		&\exists(A_i)_{i\in\N}\subseteq\U:G=\bigcup_{i\in\N} A_i\\
		&\follows
		G_m\defeq\bigcup_{i=1}^m A_i\uparrow G,~m\to\infty\\
		\overset{\P~\sigma\text{-stetig}}&{\follows}
		\forall \epsilon>0:\exists m_0\in\N:\forallm > m_0:\P(G)-\P(G_{m_0})\le\epsilon\\
		&\follows \forall\epsilon>0:\exists m_0\in\N:
		\P(G)-\epsilon\le \P(G_{m_0})
		\stackeq{\text{s.\,o.}}
		\limn \P_n(\underbrace{G_{m_0}}_{\subseteq G})\le\liminf_{n\to\infty} \P_n(G)\\
		\overset{\epsilon\to0}&{\follows}
		\liminf_{n\to\infty} \P_n(G)\ge \P(G)
	\end{align*}
	Da $G \in \G$ beliebig war, folgt nun die Behauptung aus dem Theorem \ref{satz4.2} \ref{it:4.2OpenSets}.
\end{proof}

\begin{korollar}\label{korollar4.4}
	Sei $\U\subseteq \B(S)$ endlich durchschnittsstabil mit
	\begin{align}\label{eqKorollar4.4_i}\tag{i}
		\forall x\in S : \forall\epsilon>0:\exists A\in\U:x\in \inner A\subseteq A\subseteq B(x,\epsilon)
	\end{align}
	Ist $(\S,d)$ separabel, so gilt
	\begin{align*}
		\brackets{\forall A\in\U:\P_n(A)\ntoinf  \P(A)}
		\follows \P_n\weakto  \P.
	\end{align*}
\end{korollar}

\begin{proof}
	Gemäß Satz \ref{satz2.9} hat $\G$ eine abzählbare Basis.
	Nach dem \textit{Satz von Lindelőf} (vergleiche \cite[]{schubert423277teubner}):
	% wirklich ein Lindelöf mit witzigen ő-Strichen/Punkten
	\begin{equation}\label{eqSatzVonLindelöf}\tag{L}
		\begin{split}
			&\text{Für jede offene Überdeckung einer beliebigen Teilmenge von $\S$ existiert}\\
			&\text{eine \emph{abzählbare} Teilüberdeckung.}
		\end{split}
	\end{equation}
	Sei nun $G\in\G$ beliebig.
	Für alle $x\in G$ existiert ein $\epsilon_x>0$ mit $B(x,\epsilon_x)\subseteq G$.
	Gemäß \eqref{eqKorollar4.4_i} findet man ein $A_x\in\U$ mit $x\in \inner{A_x}\subseteq A_x\subseteq B(x,\epsilon_x)\subseteq G$.
	Also folgt
	\begin{align*}
		G=\bigcup_{x\in G}\menge{ x}\subseteq\bigcup_{x\in G}\inner{A_x}\subseteq G
	\end{align*}
	Somit ist $\menge{ \inner{A_x}:x\in G}$ eine offene Teilüberdeckung von $G$.
\taus \eqref{eqSatzVonLindelöf} folgt nun:
	Es existieren $A_{x_i}\in\U,~i\in\N$ mit
	\begin{align*}
		G\subseteq\bigcup_{i\in\N} \inner{A_{x_i}}\subseteq
		\bigcup_{i\in\N}A_{x_i}\subseteq
		G
		\follows
		G=\bigcup_{i\in\N}A_{x_i}
	\end{align*}
	Also erfüllt $\U$ die Voraussetzung \ref{it: 4.3Schnittstabil} und \ref{it: 4.3Basis} in Theorem \ref{theorem: 4.3} und es folgt die Behauptung.
\end{proof}

Als Anwendung/ Beschreibung der schwachen Konvergenz im $\S=\R$.
Erinnere an \begriff{schwache Konvergenz von Verteilungsfunktionen (VF)} $(F_n)_{n\in\N}$ gegen $F$, in Zeichen
\begin{align*}
	F_n\weakVtlto F
	:⇔
	F_n(x)\ntoinf  F(x)\qquad\forall x\in C_F
	\mit C_F\defeq\menge{ x\in\R:F\text{ ist stetig in }x}
\end{align*}

\begin{korollar}\label{korollar4.5}
	Seien $\P,\P_n,n\in\N$ Wahrscheinlichkeitsmaße auf $(\R, \B(\R))$ mit zugehörigen Verteilungsfunktionen $F$ und $F_n,n\in\N$ (d.\,h.\ $F(x) = \P(\intervallOH{-\infty}x)$, $F_n(x) = P_n(\intervallOH{-\infty}x)$, für alle $x \in \R$, $n \in \N$).
	Dann gilt:
	\begin{align*}
		\P_n\weakto\P
		⇔
		F_n\weakVtlto F
	\end{align*}
\end{korollar}

\begin{proof}~
	\paragraph{Zeige \enquote{$\follows$}:}
	Sei $x\in\R$, $B\defeq(-\infty,x],$.
	Dann gilt:
	\begin{align}\label{eq:4.5randlos}\tag{$*$}
		\P(\underbrace{\partial B}_{=\menge{ x}})=0
		&⇔ \P(\menge{ x})=F(x)-\underbrace{F(x-0)}_{\text{Grenzwert}}=0
		\qquad\forall x\in C_F
	\end{align}
	Somit folgt für $x\in C_F$:
	\begin{align*}
		F_n(x)&\stackeq{\text{Def}}
	\P_n\brackets{\underbrace{\intervallOH{-\infty}{x}}_{=B}}
		=\P_n(B)\ntoinf  \P(B)\stackeq{\text{Def}} F(x)\qquad\forall x\in C_F
	\end{align*}
	gemäß Satz \ref{satz4.2} \ref{it:4.2BorelSets}, da $B$ $\P$-randlos ist für alle $x \in C_F$ gemäß \eqref{eq:4.5randlos}.
	\paragraph{Zeige \enquote{$⇐$}:} Sei
	\begin{align*}
		\U\defeq\menge{ \intervallOH{a}{b} : a \le b \in C_F}.
	\end{align*}
	Dann ist $\U$ endlich durchschnittsstabil. Ferner: Die Menge
	\begin{align*}
		D_F\defeqC_F^C = \menge{ x\in\R: F\text{ \emph{nicht} stetig in }x}
		=\menge{ x\in\R:\P(\menge{ x})>0}
	\end{align*}
	ist höchstens abzählbar (vgl.\ \eqref{eqProof1.4.2DoppelSternchen}
	im Beweis von Satz \ref{satz4.2}). Damit ist $D_F^C = C_F$ dicht in $\R$.
	Also folgt
	\begin{align*}
		\forall x\in\R:\forall\epsilon>0:\exists A&=(a,b]\in\U:\\
		x\in (a,b)&=\inner A\subseteq A=(a,b]\subseteq B(x,\epsilon)=(x-\epsilon,x+\epsilon)
	\end{align*}
	denn mit der Dichtheit von $C_F$ gilt:
	In $(x-\epsilon,x)$ muss ein $a\in C_F$ existieren und analog findet man ein $b\in(x,x+\epsilon)$.
	Somit erfüllt $\U$ die Voraussetzungen von Korollar \ref{korollar4.4}.
	Klar: $\S=\R$ ist separabel, da $ℚ$ abzählbar und dicht in $\R$.
	Schließlich gilt:
	\begin{align*}
		\P_n\brackets{\intervallOH{a}{b}}&=
		F_n(b)-F_n(a)\stackrelnew{a,\, b \in C_F}{n\to\infty}{\longrightarrow} F(b)-F(a)
		=\P\brackets{\intervallOH{a}{b}}\qquad\forall a,b\in C_F\\
		&\overset{\ref{korollar4.4}}{\follows}
		\P_n\weakto \P
		\qedhere
	\end{align*}
\end{proof}

\begin{bemerkungnr}\label{bemerkung4.6}\ %4.6
	\begin{enumerate}[label=(\arabic*)]
		\item Seien $X,X_n,n\in\N$ reelle Zufallsvariablen über $(\Omega,\A,\P)$ mit Verteilungsfunktionen $F_n$, bzw.\ $F$. Dann:
		% CHECKED: 'bzw.' used.
			\begin{equation}
				\label{eq:Bemerkung4.6}
				\tag{$\ast$}
				\begin{aligned}
					X_n\distto  X
					\overset{\text{Def}}&{⇔}
					P_n = \P\circ X_n^{-1}
					\weakto \P\circ X^{-1} = P \\
					\overset{\ref{korollar4.5}}&{⇔}
					\P(X_n\le x)
					\ntoinf
					\P(X\le x) \forall x \in C_F
					⇔ F_n \weakVtlto F
				\end{aligned}
			\end{equation}
			% für alle $x$, die Stetigkeitsstellen der Verteilungsfunktion von $X$ sind.
			Damit ist das klassische Konzept der Verteilungskonvergenz reeller
			Zufallsvariablen nur ein Spezialfall für $S = \R$.
		\item \label{it:4.6multiDim} Es gibt Verallgemeinerung von \ref{korollar4.5} bzw \eqref{eq:Bemerkung4.6} auf $\S=\R^k$:
			Seien
			\begin{align*}
				X=\brackets{X^{(1)},\dots, X^{(k)}},
				X_n=\brackets{X^{(1)}_n,\dots,X^{(k)}_n},\qquad(\Omega,\A)\to\brackets{\R^k,\B(\R^k)}
			\end{align*}
			Zufallsvariablen in $\R^k$. Dann gilt:
			\begin{align*}
				X_n\distto  X\text{ in }\R^k
				⇔
				\P(X_n\le x)\ntoinf \P(X\le x)\qquad\forall x=\brackets{x_1,\dots,x_k}\in\R^k
			\end{align*}
		wobei $x_i$ Stetigkeitsstelle der Verteilungsfunktion von $X^{(i)}$ ist für alle $i\in\menge{1,\dots,k}$.
		Beweis ist analog zu \ref{korollar4.5}, vergleiche \cite[Satz 5.58]{witting1995mathematische}.
	\end{enumerate}
\end{bemerkungnr}

Der schwache Limes einer Folge $(P_n)_{n\in\N}$ ist eindeutig, denn es gilt:

\mengecounter{satz}{5} % dazwischen geschoben
\begin{lemma}\label{lemma4.6Einhalb}
	\begin{align*}
		P_n\weakto  P,~P_n\weakto  Q\follows P=Q
	\end{align*}
\end{lemma}

\begin{proof}
	Gemäß Definition gilt:
	\begin{align*}
		\int f\d P_n&\overset{}{\longrightarrow}\int f\d P\qquad\forall f\in C^b(\S)\\
		\int f\d P_n&\overset{}{\longrightarrow}\int f\d Q\qquad\forall f\in C^b(\S)
	\end{align*}
	Der Grenzwert von reellen Zahlenfolgen eindeutig ist, folgt
	\begin{align*}
		\int f\d P &= \int f\d Q\qquad\forall f\in C^b(\S)\\
		\overset{\ref{satz3.17}}{\follows}
		P &= Q
		\qedhere
	\end{align*}
\end{proof}

In Bemerkung \ref{bemerkung4.6} wurde Korollar \ref{korollar4.5} über Definition \ref{definition: 4.1} \ref{it: 4.1Xweak}
übertragen in eine Version für Verteilungskonvergenz von reellen Zufallsvariablen.
Analog lassen sich auch \ref{satz4.2} -- \ref{korollar4.4} und \ref{lemma4.6Einhalb} auf
Zufallsvariablen übertragen. Konkret
% Im Folgenden ist das Ziel die Übertragung unserer Resultate auf Verteilungskonvergenz.

\begin{satz}[Portmanteau-Theorem]\label{satz4.7}
	Folgende Aussagen sind äquivalent:
	\begin{enumerate}[label=(\arabic*)]
		\item $\begin{aligned}
			X_n\distto  X\text{ in }(\S,d)
		\end{aligned}$
		\item $\begin{aligned}
			\E{f(X_n)}\ntoinf \E{f(X)}\qquad\forall f\in C^b(\S)
		\end{aligned}$ gleichmäßig stetig
		\item $\begin{aligned}
			\limsup_{n\to\infty}\P(X_n\in F)\le\P(X\in F)\qquad\forall F\in\F
		\end{aligned}$
		\item $\begin{aligned}
			\liminf_{n\to\infty}\P(X_n\in G)\ge\P(X\in G)\qquad\forall G\in\G
		\end{aligned}$
		\item $\begin{aligned}
			\P(X_n\in B)\ntoinf \P(X\in B)\qquad\forall B\in\B(\S)\mit\P(X\in\partial B)=0
		\end{aligned}$
	\end{enumerate}
\end{satz}

\begin{proof}
	Wende Satz \ref{satz4.2} an auf $P_n\defeq\P\circ X^{-1}_n,~P\defeq\P\circ X^{-1}$ (wegen Def $\distto $).
	Beachte z.\,B.
	\begin{align*}
		P_n(F)&=\P\circ X^{-1}_n(F)
		\overset{\text{Def}}=
		\P\brackets{X_n^{-1}(F)}
		=\P\brackets{(\menge{\Omega\in\Omega:X_n(\Omega)\in F}}
		=\P(X_n\in F)
	\end{align*}
	und
	\begin{align*}
		\int f\d P_n
		=
		\int_{\S} f\d(\P\circ X_n^{-1})
		\stackeq{\eqref{eqTrafo}}
		\int_\Omega f(X_n)\d\P
		=\E{f(X_n)}
		&\qedhere
	\end{align*}
\end{proof}

Ferner erhält man unter den jeweiligen Voraussetzungen in Theorem \ref{theorem: 4.3} bzw.\ Korollar \ref{korollar4.4}:
% CHECKED: 'bzw.' used.
\begin{align*}
	\P(X_n\in A) &\ntoinf \P(X\in A)\qquad\forall A\in\U\\
	\follows X_n &\distto  X\text{ in }(\S,d)
\end{align*}
Und aus Lemma \ref{lemma4.6Einhalb}:
\begin{align*}
	X_n\distto  X,X_n\distto  X'\follows X\overset{\L}= X'
\end{align*}

