% This work is licensed under the Creative Commons
% Attribution-NonCommercial-ShareAlike 4.0 International License. To view a copy
% of this license, visit http://creativecommons.org/licenses/by-nc-sa/4.0/ or
% send a letter to Creative Commons, PO Box 1866, Mountain View, CA 94042, USA.

\section{Satz von Portmanteau}

Seien $X$ und $(X_n)_{n \in \N}$ Zufallsvariablen in $(\S,d)$ über $(\Omega,\A,\P)$.
Dann sind
\begin{equation*}
	P \defeq \P \circ X^{-1} \qquad \und \qquad P_n \defeq \P \circ X_n^{-1} \qquad (n \in \N)
\end{equation*}
Wahrscheinlichkeitsmaße auf $\B(\S)$.

\begin{definition} \label{definition: 4.1}
	\begin{enumerate}[label=(\arabic*)]
		\item \label{it: 4.1Pweak} Seien $P$ und $(P_n)_{n \in \N}$ Wahrscheinlichkeitsmaße auf $\B(S)$.
		Dann \begriff{konvergiert $P_n$ schwach} gegen $P$, in Zeichen
		\begin{equation*}
			P_n \weakto P \defequiv \int f \diffskip{P_n} \ntoinf \int f \diffskip P \qquad \forall f \in C^b(\S)
		\end{equation*}
		\item \label{it: 4.1Xweak} $X_n$ \begriff{konvergiert in Verteilung} gegen $X$ in Raum $(\S,d)$, in Zeichen
			\begin{align*}
				X_n \distto X \text{ in } (\S,d) \defequiv \P \circ X_n^{-1} \weakto \P \circ X^{-1}
			\end{align*}
			Das $\text{d}$ steht für \enquote{distribution}. Alternative Schreibweise: $X_n \overset{\mathcal{L}}{\longrightarrow} X$.
	\end{enumerate}
\end{definition}

Der folgende Satz gibt eine äquivalente Charakterisierung von $\weakto$ bzw.\ $\distto$.

% This work is licensed under the Creative Commons
% Attribution-NonCommercial-ShareAlike 4.0 International License. To view a copy
% of this license, visit http://creativecommons.org/licenses/by-nc-sa/4.0/ or
% send a letter to Creative Commons, PO Box 1866, Mountain View, CA 94042, USA.

\begin{theorem}[Portmanteau-Theorem]\label{satz: 4.2}
	Folgende Aussagen sind äquivalent:
	\begin{enumerate}[label=(\arabic*)]
		\item \label{it:4.2weakConv} $\begin{aligned}
			P_n \weakto P
		\end{aligned}$
	\item \label{it:4.2f} $\begin{aligned}
			\int f \diffskip P_n \longrightarrow \int f \diffskip P \qquad \forall f \in C^b(\S) \text{ glm. stetig}
			% Standardfrage für die Prüfung: was sagt das Portmanteau-Theorem und was ist schwache Konvergenz?
		\end{aligned}$
	\item \label{it:4.2ClosedSets} $\begin{aligned}
			\limsup_{n \to \infty} P_n(F) \le P(F) \qquad \forall F \in \F(\S)
		\end{aligned}$
	\item \label{it:4.2OpenSets} $\begin{aligned}
			\liminf_{n \to \infty} P_n(G)\ge P(G) \qquad \forall G \in \G(\S)
		\end{aligned}$
	\item \label{it:4.2BorelSets} $\lim\limits_{n \to \infty} P_n(B) = P(B) \qquad \forall B \in \B(\S) \mit P(\underbrace{\partial B}_{ \in\F(\S)}) = 0$ \\
		Die Mengen $B \in \B(\S)$ mit $P(\partial B)=0$ heißen \begriff{$P$-randlos}.
	\end{enumerate}
\end{theorem}

\begin{proof}
	\begin{description}
		\item[\ref{it:4.2weakConv} $\Rightarrow$ \ref{it:4.2f}:] 
		Folgt aus \cref{definition: 4.1}.
		%
		\item[\ref{it:4.2f} $\Rightarrow$ \ref{it:4.2ClosedSets}:] 
		Sei $F \in\F(\S)$. Der Beweis von \cref{satz: 3.17} zeigt, dass es eine Folge $(f_k)_{k \in \N}$ von gleichmäßig stetigen und beschränkten Funktionen auf $\S$ gibt mit $f_k \downarrow \one_F$. Dann gilt:
		\begin{equation*}
			\limsup_{n \to \infty} P_n(F)
			= \limsup_{n \to \infty} \int \underbrace{\one_F}_{\le f_k} \diffskip P_n
			\overset{\text{Mon.}}{\le}
			\limsup_{n \to \infty} \int f_k \diffskip P_n
			\overset{\text{Vor.}}{=}
			\int f_k \diffskip P \quad \forall k \in \N 
			\satzende
		\end{equation*}
		Mit monotoner Konvergenz folgt daraus $\int f_k \diffskip P \longrightarrow \int \one_F \diffskip  P = P(F)$ und somit für $k \to \infty$ auch \ref{it:4.2ClosedSets}.
		%
		\item[\ref{it:4.2ClosedSets} $\Leftrightarrow$ \ref{it:4.2OpenSets}:]
		Nutze den Übergang zum Komplement sowie die Rechenregeln für $\liminf$ und $\limsup$. Sei $G$ offen, das heißt $G \in \G(\S)$ und damit $G^\complement \in \F(\S)$. Dann gilt
		\begin{equation*}
			\liminf_{n \to \infty} P_n(G)
			= \liminf_{n \to \infty} \brackets{1-\P_n(G^\complement)}
			= 1 - \underbrace{\limsup_{n \to \infty} P_n(G^\complement)}_{\le P(G^\complement)}
			\ge 1 - P(G^\complement)
			= P(G) 
			\satzende
		\end{equation*}
	\end{description}

%	\paragraph{Zeige \ref{it:4.2ClosedSets} $ \Rightarrow $ \ref{it:4.2weakConv}:}
%	Sei $f \in C^b(\S)$ beliebig. Zeige zunächst:
%	\begin{align}\label{eqProof1.4.2Sternchen}\tag{$\ast$}
%		\limsup_{n \to \infty}\int f\diffskip \P_n\le\int f\diffskip \P
%	\end{align}
%	\subparagraph{1. Schritt:} Sei $0\le f<1$. Setze
%	\begin{align*}
%		F_i:=\set{ f\ge\frac{i}{k}}=\set{ x \in\S:f(x)\ge\frac{i}{k}}
%		= f^{-1}\klammern{\intervallHO{\frac1k}{∞}},\qquad  \forall\; 0\le i\le k,\;k \in\N
%	\end{align*}
%	($F_i$ hängt auch von $k$ ab, aber diese Abhängigkeit wird in der Notation nicht
%	demonstriert.)
%	Dann gilt $F_i \in\F~ \forall i$, da $f$ stetig. Da
%	\begin{align*}
%		\int_{\S}f\diffskip \P
%		\stackeq{\text{Lin}}
%		\sum_{i=1}^k\int\indi_{\set{\frac{i-1}{k}\le f<\frac{i}{k}}}· f\diffskip \P
%	\end{align*}
%	folgt wegen Monotonie des Integrals
%	\begin{align}\label{eqProof1.4.2Plus}\tag{+}
%		\sum_{i=1}^k\underbrace{\frac{i-1}{k}}_{=\frac{i}{k}-\frac{1}{k}}·
%	\P\klammern{\frac{i-1}{k}\le f<\frac{i}{k}}
%		\le
%		\int f\diffskip \P
%		\le
%		\sum_{i=1}^k \frac{1}{k}· \P\klammern[\Bigg]{\underbrace{\frac{i-1}{k}\le f<\frac{i}{k}}_{F_{i-1}\setminus F_i}}
%	\end{align}
%	Die rechte Summe in \eqref{eqProof1.4.2Plus} ist gleich
%	\begin{align*}
%		&\frac{1}{k}·\sum_{i=1}^k i·\klammern[\big]{ \P(F_{i-1})-\P(F_i)}
%		\quad \text{da $F_i ⊂ F_{i-1}$}\\
%		&=\frac{1}{k}·\Big(\P(F_0)-\P(F_1)+2· \P(F_1)-2· \P(F_2)+3· \P(F_2)-3· \P(F_3)+\\
%		% CHECKED: '\Big' used.
%		&\qquad+…+(k-1)· \P(F_{k-2})-(k-1)· \P(F_{k-1})+k· \P(F_{k-1})-k· \P(F_k)\Big)\\
%		% CHECKED: '\Big' used.
%		&=\frac{1}{k}·\klammern[\Big]{\underbrace{\P(F_0)}_{=1}+\P(F_1)+\P(F_2)+…+\P(F_{k-1})- k· \underbrace{\P(F_k)}_{=0}} \quad (0 \le f < 1)\\
%		&=\frac{1}{k}+\frac{1}{k}·\sum_{i=1}^{k-1} \P(F_i)
%	\end{align*}
%	Da die linke Summe in \eqref{eqProof1.4.2Plus} gleich der rechten Summe in \eqref{eqProof1.4.2Plus} minus $\frac{1}{k}$ ist, folgt
%	\begin{align}\label{eqProof1.4.2DoppelSternchen}\tag{$\ast\ast$}
%		\sum_{i=1}^{k-1} \P(F_i)
%		\le\int f\diffskip P
%		\le\frac{1}{k}+\frac 1k \sum_{i=1}^{k-1} \P(F_i)
%	\end{align}
%	Beachte, \eqref{eqProof1.4.2DoppelSternchen} gilt für \emph{jedes} Wahrscheinlichkeitsmaß $\P$, also auch für $\P_n$.
%	Damit folgt:
%	\begin{align*}
%		\limsup_{n \to \infty}\int f\diffskip \P_n
%		\overset{\eqref{eqProof1.4.2DoppelSternchen}}&{\le}
%		\frac{1}{k}+\frac 1k \sum_{i=1}^{k-1}\underbrace{
%			\limsup_{n \to \infty} \P_n(F_i)
%		}_{\overset{\ref{it:4.2ClosedSets}}{\le}\P(F_i)~ \forall i}\\
%		\overset{\ref{it:4.2ClosedSets}}&{\le}
%		\frac{1}{k}+\underbrace{\frac 1k \sum_{i=1}^{k-1} \P(F_i)}_{
%			\overset{\eqref{eqProof1.4.2DoppelSternchen}}{\le}\int f\diffskip \P
%		}\\
%		\overset{\eqref{eqProof1.4.2DoppelSternchen}}&{\le}
%		\frac{1}{k}+\int f\diffskip \P\qquad \forall k \in\N
%	\end{align*}
%	Grenzwertbildung $k \to \infty$ liefert \eqref{eqProof1.4.2Sternchen}.
%	\paragraph{2. Schritt:} Da $f \in C^b(\S)$ beliebig, gilt wegen Beschränktheit von $f$:
%	\begin{align*}
%		∃ a<b:a\le f<b
%		 \Rightarrow  g(x):=\frac{f(x)-a}{b-a}\text{ ist stetig und } 0\le g<1
%	\end{align*}
%	Daraus folgt
%	\begin{align*}
%		\limsup_{n \to \infty}\int f\diffskip \P_n
%		&=\limsup_{n \to \infty}\int (b-a)· g+a\,\diffskip \P_n\\
%		&=\limsup_{n \to \infty}\klammern{(b-a)·\int g\,\diffskip \P_n+a}\\
%		&\le(b-a)·\underbrace{\limsup_{n \to \infty}\int g\diffskip \P_n}_{\le\int g\diffskip P\text{, wg.\ 1.\ Schritt}}+a\\
%		&\le(b-a)·\int g\diffskip \P + a\\
%		\overset{\text{Lin}}&=
%		\int f\diffskip \P
%	\end{align*}
%	Damit ist \eqref{eqProof1.4.2Sternchen} gezeigt. Übergang zu $-f$ in \eqref{eqProof1.4.2Sternchen} liefert
%	\begin{align*}
%		\liminf_{n \to \infty}\int f\diffskip \P_n
%		&=\liminf_{n \to \infty}-\int -f\diffskip \P_n\\
%		&=-\limsup_{n \to \infty}\int \underbrace{-f}_{ \in C^b(\S)}\diffskip \P\\
%		\overset{\eqref{eqProof1.4.2Sternchen}}&{\ge}
%		-\int -f\diffskip \P\\
%		\overset{\text{Lin}}&=
%		\int f\diffskip \P
%		\qquad \forall f \in C^b(\S)\\
%		 \Rightarrow  \intf \diffskip\P &\le \liminf_{n → ∞} \intf\diffskip\P_n \le \limsup_{n → ∞} \intf\diffskip\P_n \overset{\eqref{eqProof1.4.2Sternchen}}{\le} \intf \diffskip\P \\
%		 \Rightarrow  \lim_{n → ∞} \int f \diffskip\P_n &= \intf \diffskip\P \quad  \forall f  \in C^b(S) \\
%		& \Rightarrow \ref{it:4.2weakConv}
%	\end{align*}
%
%	\subparagraph{Zeige \ref{it:4.2ClosedSets} $ \Rightarrow $ \ref{it:4.2BorelSets}:}
%	Sei $B \in\B(\S)\mit \P(\partial B)=0$. Dann gilt:
%	\begin{align*}
%		\P(\overline{B})
%		\overset{\ref{it:4.2ClosedSets}}&{\ge}
%		\limsup_{n \to \infty} \underbrace{\P_n(\overbrace{\overline{B}}^{⊇ B})}_{\ge \P_n(B)}
%		\ge\limsup_{n \to \infty} \P_n(B)
%		\overset{\text{stets}}{\ge}
%		\liminf_{n \to \infty} \underbrace{\P_n(\overbrace{B}^{⊇\overset{∘}{B}})}_{
%			\ge \P_n(\inner B)
%		}\\
%		\overset{\ref{it:4.2ClosedSets}⇔ \ref{it:4.2OpenSets}}&{\ge}
%		\P\klammern[\big]{\inner B}
%		=\P(\overline{B})
%		=\P(B),
%	\end{align*}
%	denn:
%	\begin{align*}
%		0
%		=\P(\overbrace{\partial B}^{\overline{B}\setminus \inner B})
%		=\P(\overline{B})-\P(\inner B)
%		 \Rightarrow 
%		\P(\inner B)\le \P(B)\le \P(\overline{B})=\P(\inner B)
%	\end{align*}
%	Damit folgt $\liminf=\limsup$ und folglich $\lim_{n \to \infty} \P_n(B)=\P(B)$.
%	\paragraph{Zeige \ref{it:4.2BorelSets} $ \Rightarrow $ \ref{it:4.2ClosedSets}:}
%	Sei $F \in\F$ (abgeschlossen) beliebig. Dann gilt $ \forallε>0:$
%	\begin{align}\label{eqProof1.4.2SternchenUnten}\tag{$\ast$}
%		\partial\,\set{ x \in\S:d(x,F)\leε}
%		⊆\set{ x \in\S:d(x,F)=ε}
%	\end{align}
%	denn: Sei $x \in\partial\set{ x \in\S:d(x,F)\leε}$. Dann gilt:
%	\begin{align*}
%		&∃ (x_n)_{n \in\N}: \forall n \in\N:d(x_n,F)\leε∧ \lim_{n \to \infty} x_n=x\\
%		&∃ (ζ_n)_{n \in\N}: \forall n \in\N:d(ζ_n,F)>ε∧ \lim_{n \to \infty} ζ_n=x
%	\end{align*}
%	Da $d(·,F)$ stetig ist gemäß \ref{lemma2.3} \ref{it:distStetig}, folgt
%	\begin{align*}
%		ε \le \lim_{n → ∞} d(ζ_n, F) = d(x,F) = \lim_{n → ∞} d(x_n, F) \le ε.
%	\end{align*}
%	Wegen \eqref{eqProof1.4.2SternchenUnten} sind
%	\begin{align*}
%		A_ε:=\partial\set{ x \in\S:d(x,F)\leε}\qquad \forallε>0
%	\end{align*}
%	paarweise disjunkt, da bereits die Obermengen paarweise disjunkt sind. Dann folgt
%	\begin{align}\label{eqProof1.4.2DoppelSternchenUnten}\tag{$\ast\ast$}
%		E:=\set{ε>0:\P(A_ε)>0}\text{ ist höchstens abzählbar},
%	\end{align}
%	denn:
%	\begin{align*}
%		E=\Union_{m  \in \N} \underbrace{\set{ε>0: \P(A_ε) \ge \frac1m}}_{=:E_m}
%	\end{align*}
%	Wir möchten eine Obergrenze für die Größe der $E_m$ finden. Dafür beachte
%	\begin{align*}
%		1\ge \P\klammern{\Union_{ε  \in E_m} A_{ε}}
%		\stackeq{\text{pw.\,disj.}}
%		\sum_{ε  \in E_m}\underbrace{\P(A_{ε})}_{\ge\frac{1}{m}} \ge \abs{E_m}·\frac{1}{m}  \Rightarrow  \abs{E_m} \le m.
%	\end{align*}
%	% Fergers Variante:
%	% Es gilt $\abs{E_m}\le m$, weil: Angenommen es existieren $0<ε_1<…<ε_{m+1}$ mit
%	% % CHECKED: '|'
%	% \begin{align*}
%	% 	&\P(A_{ε_i})\ge\frac{1}{m}\qquad \forall 1\le i\le m+1\\
%	% 	& \Rightarrow 
%	% 	1\ge \P\left(\Union_{i=1}^{m+1} A_{ε_i}\right)
%	% 	\stackeq{\text{pw. disj.}}
%	% 	\sum_{i=1}^{m+1}\underbrace{\P\klammern[\big]{A_{ε_i}}}_{\ge\frac{1}{m}}\ge(m+1)·\frac{1}{m}>1
%	% \end{align*}
%	% Das ist ein Widerspruch.
%	Damit ist $E$ abzählbare Vereinigung endlicher Mengen, also höchstens abzählbar unendlich.
%	Damit liegt das Komplement
%	\begin{align*}
%		E^C=\set[\big]{ε>0: \P(A_ε)=0}
%	\end{align*}
%	dicht in $[0,∞)$.
%	(Dies kann man durch Widerspruch zeigen: in $E$ kann kein Intervall positiver Länge enthalten sein, denn diese enthalten überabzählbar viele Elemente)
%	Daraus folgt insbesondere:
%	\begin{align*}
%		∃(ε_k)_{k \in\N}⊆ℝ\mitε_k\diffskipownarrow0: \forall k \in\N:
%		F_k:=\set[\big]{ x \in\S:d(x,F)\leε_k}\text{ ist $\P$-randlos}
%	\end{align*}
%	Beachte $A_{ε_k}=\partial F_k$. Wähle also $ε_k  \in E^C$ für alle $k  \in \N$. Da $F⊆ F_k~ \forall k \in\N$, gilt:
%	\begin{align*}
%		\limsup_{n \to \infty} \P_n(F)
%		\le\limsup_{n \to \infty} \underbrace{\P_n(F_k)}_{\text{konv.}}
%		\overset{\ref{it:4.2BorelSets}}&{=}
%		\P(F_k)\qquad \forall k \in\N\\
%		\overset{k \to \infty}{ \Rightarrow }
%		\limsup_{n ⟶ ∞} \P_n(F)
%		\le\lim_{k \to \infty} \P(F_k)
%		&=\P(F)
%	\end{align*}
%	Die letzte Gleichheit gilt, weil $\P$ $σ$-stetig von oben ist und $F_k\diffskipownarrow F$.
%	$F_k\diffskipownarrow F$, denn $F_1⊇ F_2⊇…$, da $ε_k$ monoton fallende Folge ist und
%	\begin{align*}
%		\bigcap_{k \in\N}F_k=F,
%	\end{align*}
%	denn:
%	\begin{align*}
%		x \in\bigcap_{k \in\N}F_k
%		&⇔
%		x \in F_k &  \forall k \in\N\\
%		&⇔
%		d(x,F)\leε_k &  \forall k \in\N\\
%		& \Rightarrow 
%		d(x,F)=0\\
%		\overset{\ref{lemma2.3}~\ref{it:distCharakterisierung}}&{⇔}
%		x \in \overline{F}\stackeq{F \in\F}F
%		&& \qedhere
%	\end{align*}
\end{proof}

Falls $\mathcal{U} = \menge{B \in \B(S) : P(\rand{B}) = 0}$, so besagt \cref{satz: 4.2} in \cref{it: 4.2BorelSets}, dass $P_n(B) \to P(B)$ für alle $B \in \mathcal{U}$, also punktweise Konvergenz auf der Klasse $\mathcal{U}$, die schwache Konvergenz $P_n \weakto P$ nach sich zieht.
Es gibt weitere solcher Klassen.
% Der letzte Absatz ersetzt folgenden alten Satz:
% Mitunter folgt schwache Konvergenz aus $\P_n(A)\ntoinf  \P(A)$ für eine spezielle Klasse von Mengen $A$.

\begin{theorem} \label{theorem: 4.3}
	Sei $\mathcal{U} \subseteq \B(\S)$ mit
	\begin{enumerate}[label=(\roman*)]
		\item \label{it: 4.3Schnittstabil} $\mathcal{U}$ ist endlich schnittstabil, d.h. mit $A,B \in \mathcal{U}$ ist auch $A \cap B \in \mathcal{U}$
        \item \label{it: 4.3Basis} Jedes offene $G \subseteq \S$ ist abzählbare Vereinigung von Mengen aus $\mathcal{U}$.
	\end{enumerate}
	Dann gilt: $P_n(A) \ntoinf P(A) \enskip \forall A \in \mathcal{U} \follows P_n \weakto P$.
\end{theorem}

\begin{proof}
	Seien $A_1,\dots,A_m \in \mathcal{U}$. Dann gilt:
	\begin{align*}
		P_n \brackets{\bigcup_{i=1}^m A_i}
		&= \sum_{k=1}^m (-1)^{k-1} * \sum_{1 \le i_1 < \dots < i_k \le m} \underbrace{P_n(\overbrace{A_{i_1} \cap \dots \cap A_{i_k}}^{\in\mathcal{U}})}_{\ntoinf  P(A_{i_1}\cap\dots\cap A_{i_k})}\\
		\overset{n\to\infty}&{\longrightarrow}
		\sum_{k=1}^m\sum_{1\le i_1<\dots<i_k\le m} P\brackets{A_{i_1}\cap\dots\cap A_{i_k}}\\
		\overset{\text{allg.\ Add.}}&=
		P\brackets{\bigcup_{i=1}^m A_i}
	\end{align*}
	Sei $G\in\G$, also offen. Wegen Voraussetzung \ref{it: 4.3Basis} ist $G$ darstellbar als $G=\bigcup_{i\in\N} A_i$ mit $(A_i){i \in \N} \subseteq \mathcal{U}$. Es gilt $G_m \defeq \bigcup_{i=1}^m A_i \uparrow G$ für $m\to\infty$. Mit der $\sigma$-Stetigkeit von $P$ gilt
	\begin{equation*}
		\begin{aligned}
			&\forall \epsilon>0 \enskip \exists m_0\in\N \enskip \forall m > m_0: \enskip P(G)-P(G_{m_0})\le\epsilon \\
			\follows &\forall\epsilon > 0 \enskip \exists m_0 \in \N: \enskip
			P(G) - \epsilon \le P(G_{m_0})
			\overset{\text{s.\,o.}}{=}
			\lim_{n \to \infty} P_n(\underbrace{G_{m_0}}_{\subseteq G}) \le \liminf_{n \to \infty} P_n(G) \satzende
		\end{aligned}
	\end{equation*}
	Mit $\epsilon \to 0$ folgt daraus $\liminf_{n \to \infty} P_n(G)\ge P(G)$.
	Da $G \in \G$ beliebig war, folgt nun die Behauptung aus dem Theorem \cref{satz: 4.2} \ref{it:4.2OpenSets}.
\end{proof}

\begin{korollar} \label{korollar4.4}
	Sei $\mathcal{U} \subseteq \B(S)$ endlich durchschnittsstabil mit
	\begin{equation} \label{eqKorollar4.4_i}
		\forall x\in S \enskip \forall \epsilon > 0 \enskip \exists A \in \mathcal{U} : \enskip x \in \inner A \subseteq A \subseteq B(x,\epsilon)
		\satzende
	\end{equation}
	Ist $(\S,d)$ separabel, so gilt
	\begin{align*}
		\brackets{\forall A \in \mathcal{U}: P_n(A) \ntoinf P(A)}
		\follows P_n \weakto P
		\satzende
	\end{align*}
\end{korollar}

\begin{proof}
	Gemäß \cref{satz: 2.9} hat $\G$ eine abzählbare Basis. Nach dem \textit{Satz von Lindelőf} (vergleiche \cite{schubert423277teubner}):
	% wirklich ein Lindelöf mit witzigen ő-Strichen/Punkten
	\begin{center}
		Für jede offene Überdeckung einer beliebigen Teilmenge von $\S$ existiert eine \emph{abzählbare} Teilüberdeckung.
	\end{center}
	Sei nun $G\in\G$ beliebig.
	Für alle $x\in G$ existiert ein $\epsilon_x>0$ mit $B(x,\epsilon_x)\subseteq G$.
	Gemäß \eqref{eqKorollar4.4_i} findet man ein $A_x\in\mathcal{U}$ mit $x\in \inner{A_x}\subseteq A_x\subseteq B(x,\epsilon_x)\subseteq G$.
	Also folgt
	\begin{align*}
		G=\bigcup_{x\in G}\menge{ x}\subseteq\bigcup_{x\in G}\inner{A_x}\subseteq G
	\end{align*}
	Somit ist $\menge{ \inner{A_x}:x\in G}$ eine offene Teilüberdeckung von $G$.
	Aus dem Satz von Lindelőf existieren nun $A_{x_i}\in\mathcal{U}$ ($i\in\N$) mit
	\begin{align*}
		G\subseteq\bigcup_{i\in\N} \inner{A_{x_i}}\subseteq
		\bigcup_{i\in\N}A_{x_i}\subseteq
		G
		\follows
		G=\bigcup_{i\in\N}A_{x_i}
	\end{align*}
	Also erfüllt $\mathcal{U}$ die Voraussetzung \ref{it: 4.3Schnittstabil} und \ref{it: 4.3Basis} in \cref{theorem: 4.3} und es folgt die Behauptung.
\end{proof}

Als Anwendung/ Beschreibung der schwachen Konvergenz im $\S=\R$.
Wir erinnern uns an die \begriff{schwache Konvergenz von Verteilungsfunktionen} $(F_n)_{n\in\N}$ gegen $F$, in Zeichen
\begin{equation*}
	F_n \rightharpoonup F
	\defequiv
	F_n(x) \ntoinf  F(x) \qquad\forall x \in C_F \defeq \menge{ x \in \R: F \text{ ist stetig in }x}
\end{equation*}

\begin{korollar}\label{korollar4.5}
	Seien $P$ und $(P_n)_{n\in\N}$ Wahrscheinlichkeitsmaße auf $(\R, \B(\R))$ mit zugehörigen Verteilungsfunktionen $F$ und $(F_n)_{n\in\N}$ (d.\,h.\ $F(x) = P((-\infty, x])$) für alle $x \in \R$, $n \in \N$).
	Dann gilt:
	\begin{equation*}
		P_n \weakto P
		\equivalent
		F_n \rightharpoonup F
	\end{equation*}
\end{korollar}

\begin{proof}
	\begin{description}
		\hinrichtung
		Sei $x\in\R$ und $B\defeq(-\infty,x]$. Dann ist $\partial B = \menge{x}$ und somit
		\begin{equation}\label{eq:4.5randlos}
			P(\partial B) = 0
			\equivalent P(\menge{ x})=F(x)-\underbrace{F(x-0)}_{\text{Grenzwert}}=0
			\qquad\forall x\in C_F \satzende
		\end{equation}
		Damit folgt für $x\in C_F$
		\begin{equation*}
			F_n(x) \overset{\text{Def}}{=}
			P_n\brackets{\underbrace{(-\infty,x]}_{=B}}
			= P_n(B) \ntoinf  P(B) 
			\overset{\text{Def}}{=} F(x) \qquad\forall x\in C_F
		\end{equation*}
		gemäß \cref{satz: 4.2} \ref{it:4.2BorelSets}, da $B$ $P$-randlos ist für alle $x \in C_F$ gemäß \eqref{eq:4.5randlos}.
		%
		\rueckrichtung
		Das Mengensystem $\mathcal{U} \defeq \menge{ (a,b] : a, b \in C_F \mit a \le b}$ ist endlich durchschnittsstabil. Ferner ist die Menge
		\begin{equation*}
			D_F \defeq C_F^\complement = \menge{ x\in\R: F\text{ \textit{nicht} stetig in }x}
			= \menge{ x\in\R:P(\menge{ x})>0}
		\end{equation*}
		höchstens abzählbar (vgl.\ \eqref{eqProof1.4.2DoppelSternchen}
		im Beweis von \cref{satz: 4.2}). Damit ist $D_F^\complement = C_F$ dicht in $\R$.
		Also folgt
		\begin{align*}
			\forall x\in\R:\forall\epsilon>0:\exists A&=(a,b]\in \mathcal{U}:\\
			x\in (a,b)&=\inner A\subseteq A=(a,b]\subseteq B(x,\epsilon)=(x-\epsilon,x+\epsilon)
		\end{align*}
		denn mit der Dichtheit von $C_F$ gilt:
		In $(x-\epsilon,x)$ muss ein $a\in C_F$ existieren und analog findet man ein $b\in(x,x+\epsilon)$.
		Somit erfüllt $\mathcal{U}$ die Voraussetzungen von \cref{korollar4.4}.
		Klar: $\S=\R$ ist separabel, da $\Q$ abzählbar und dicht in $\R$ ist.
		Schließlich gilt:
		\begin{equation*}
			P_n\brackets{(a,b]} =
			F_n(b)-F_n(a) \overset{a,b \in C_F}{\longrightarrow} F(b)-F(a)
			=P\brackets{(a,b]} \qquad\forall a,b\in C_F
		\end{equation*}
		und mit \cref{korollar4.4} folgt daraus $P_n \weakto P$.
	\end{description}
\end{proof}

\begin{bemerkung} \label{bemerkung4.6} %4.6
	\begin{enumerate}[label=(\arabic*)]
		\item Seien $X$ und $(X_n)_{n\in\N}$ reelle Zufallsvariablen über $(\Omega,\A,\P)$ mit Verteilungsfunktionen $F_n$ bzw.\ $F$. Dann gilt:
			\begin{equation}
				\label{eq:Bemerkung4.6}
				\begin{aligned}
					X_n\distto  X
					\overset{\text{Def}}&{\equivalent}
					P_n = \P\circ X_n^{-1}
					\weakto \P\circ X^{-1} = P \\
					\overset{\text{\cref{korollar4.5}}}&{\equivalent}
					\P(X_n\le x)
					\ntoinf
					\P(X\le x) \qquad \forall x \in C_F \\
					&\equivalent F_n \rightharpoonup F
				\end{aligned}
			\end{equation}
			% für alle $x$, die Stetigkeitsstellen der Verteilungsfunktion von $X$ sind.
			Damit ist das klassische Konzept der Verteilungskonvergenz reeller
			Zufallsvariablen nur ein Spezialfall für $\S = \R$.
		\item \label{it:4.6multiDim} Es gibt Verallgemeinerung von \cref{korollar4.5} bzw \eqref{eq:Bemerkung4.6} auf $\S=\R^k$:
			Seien
			\begin{align*}
				X=\brackets{X^{(1)},\dots, X^{(k)}},
				X_n=\brackets{X^{(1)}_n,\dots,X^{(k)}_n},\qquad(\Omega,\A)\to\brackets{\R^k,\B(\R^k)}
			\end{align*}
			Zufallsvariablen in $\R^k$. Dann gilt:
			\begin{align*}
				X_n\distto  X\text{ in }\R^k
				\equivalent
				\P(X_n\le x)\ntoinf \P(X\le x)\qquad\forall x=\brackets{x_1,\dots,x_k}\in\R^k
			\end{align*}
		wobei $x_i$ Stetigkeitsstelle der Verteilungsfunktion von $X^{(i)}$ ist für alle $i\in\menge{1,\dots,k}$.
		Beweis ist analog zu \cref{korollar4.5}, vergleiche \cite[Satz 5.58]{witting1995mathematische}.
	\end{enumerate}
\end{bemerkung}

Der schwache Limes einer Folge $(P_n)_{n\in\N}$ ist eindeutig.

\begin{lemma}\label{lemma4.6Einhalb}
	$\quad P_n \weakto  P \quad \und \quad P_n \weakto  Q \qquad \follows \qquad P=Q$
\end{lemma}

\begin{proof}
	Gemäß Definition gilt:
	\begin{equation*}
		\int f \diffskip P_n \to \int f \diffskip P 
		\quad \und \quad 
		\int f \diffskip P_n \to \int f \diffskip Q \qquad \forall f\in C^b(\S)
	\end{equation*}
	Der Grenzwert von reellen Zahlenfolgen eindeutig ist, folgt $\int f \diffskip P = \int f \diffskip Q$ für alle $f \in C^b(\S)$ und mit \cref{satz: 3.17} folgt daraus schließlich $P = Q$.
\end{proof}

In \cref{bemerkung4.6} wurde \cref{korollar4.5} über \cref{definition: 4.1} übertragen in eine Version für Verteilungskonvergenz von reellen Zufallsvariablen.
Analog lassen sich auch \cref{satz: 4.2}, \cref{theorem: 4.3} und \cref{korollar4.4} sowie \cref{lemma4.6Einhalb} auf Zufallsvariablen übertragen.

\begin{satz}[Portmanteau-Theorem] \label{satz4.7}
	Folgende Aussagen sind äquivalent:
	\begin{enumerate}[label=(\arabic*)]
		\item $\begin{aligned}
			X_n \distto  X\text{ in }(\S,d)
		\end{aligned}$
		\item $\begin{aligned}
			\EW[f(X_n)] \ntoinf \EW[f(X)] \qquad\forall f \in C^b(\S)
		\end{aligned}$ gleichmäßig stetig
		\item $\begin{aligned}
			\limsup_{n\to\infty} \P(X_n\in F)\le\P(X\in F)\qquad\forall F \in \F
		\end{aligned}$
		\item $\begin{aligned}
			\liminf_{n\to\infty}\P(X_n\in G)\ge\P(X\in G)\qquad\forall G\in\G
		\end{aligned}$
		\item $\begin{aligned}
			\P(X_n\in B)\ntoinf \P(X\in B)\qquad\forall B\in\B(\S)\mit\P(X\in\partial B)=0
		\end{aligned}$
	\end{enumerate}
\end{satz}

\begin{proof}
	Wende \cref{satz: 4.2} an auf $P_n \defeq \P \circ X^{-1}_n$ und $P \defeq \P \circ X^{-1}$.
	Beachte z.\,B.
	\begin{equation*}
		P_n(F) = \P\circ X^{-1}_n(F)
		\overset{\text{Def}}{=}
		\P\brackets{X_n^{-1}(F)}
		=\P\brackets{(\menge{\Omega\in\Omega:X_n(\Omega)\in F}}
		= \P(X_n \in F)
	\end{equation*}
	und
	\begin{equation*}
		\int f \diffskip P_n
		=
		\int_{\S} f\diffskip(\P\circ X_n^{-1})
		\overset{\eqref{eqTrafo}}{=}
		\int_\Omega f(X_n) \diffskip\P
		= \E{f(X_n)}  
		\satzende
	\end{equation*}
\end{proof}

Ferner erhält man unter den jeweiligen Voraussetzungen in \cref{theorem: 4.3} bzw.\ \cref{korollar4.4}:
\begin{equation*}
	\P(X_n\in A) \ntoinf \P(X\in A)\qquad\forall A\in \mathcal{U}
	\qquad \implies \qquad
	X_n \distto  X\text{ in }(\S,d)
\end{equation*}
und aus \cref{lemma4.6Einhalb}:
\begin{equation*}
	X_n\distto  X, \quad X_n\distto  X' 
	\quad \implies \quad
	X \disteq X' \satzende
\end{equation*}
