% !TEX root = MSTAT19.tex
% This work is licensed under the Creative Commons
% Attribution-NonCommercial-ShareAlike 4.0 International License. To view a copy
% of this license, visit http://creativecommons.org/licenses/by-nc-sa/4.0/ or
% send a letter to Creative Commons, PO Box 1866, Mountain View, CA 94042, USA.

Nächstes Ziel: Handhabbare Kriterien für den Nachweis der Verteilungskonvergenz in $C$. Dazu:

\begin{definition} %7.7
	Für eine Funktion $f \colon I ⟶ ℝ$ und $δ > 0$ definiere den \define{Stetigkeitsmodul/ Oszillationsmodul}
	\begin{align*}
		ω(f,δ):=\sup \set[\Big]{\abs{f(s)-f(t)}:s, t ∈ I \mit \abs{s-t} ≤ δ}
	\end{align*}
\end{definition}

Aus der Analysis ist bekannt:
\begin{align*}
	f∈ C(I)⇔ω(f,δ)\overset{δ⟶0}{\longrightarrow}0
\end{align*}

\begin{lemma}\label{lemma7.8}
	$ω(·,δ)\colon (C,d)⟶ℝ$ ist stetig für jedes $δ>0$ und damit gemäß Lemma \ref{Lemma3.2} \ref{it:3.2StetigMessbar} auch $\B(C)$-$\B(ℝ)$-messbar.
\end{lemma}

\begin{proof}
	Sei $g∈ C$ und $s,t∈ I$ mit $\abs{s-t}≤δ$.
	Dann gilt:
	\begin{align*}
		\abs{f(s)-f(t)}
		&= \abs[\big]{f(s)-g(s)+g(s)-g(t)+g(t)-f(t)}\\
		\overset{\text{DU}}&{≤}
		\underbrace{\abs{f(s)-g(s)}}_{≤ d(f,g)}
		+ \underbrace{\abs{g(s)-g(t)}}_{ ≤ ω(g,δ)}
		+ \underbrace{\abs{ g(t)-f(t)}}_{≤ d(f,g)}\\
		& ≤
		ω(g,δ)+2· d(f,g)\\
		\overset{\sup}&{⇒}
		ω(f,δ)
		≤ ω(g,δ) + 2 d(f,g)\\
		&⇒ ω(f,δ)-ω(g,δ)
		≤ 2 d(f,g) %& ∀ f,g
		\\
		&⇒ ω(g,δ) - ω(f,δ)
		≤ 2 d(g,f) = 2 d(f,g)
		\\
		&⇒
		\abs{ω(f,δ)-ω(g,δ)}≤ 2 d(f,g)
	\end{align*}
	Das heißt, $ω(·,δ)$ ist sogar Lipschitz-stetig.
\end{proof}

%Ferger: "Ich war übrigens gestern in der Stadt. Da bin ich eigentlich nie. Ich war in mindestens 7 oder Schuhgeschäften. Es war noch jemand dabei. Ich selber habe kein Schuhe gekauft. Und in jedem Geschäft war die Verweildauer sehr lang. Und jedes Schuhgeschäft wurde GESCANNT: Jeder Schuh wurde angefasst und begutachtet. [...] Weihnachten ist was Schönes!"

%Ferger: "Wenn man sich rote Schuhe kauft, muss man dazu auch noch eine passende rote Tasche kaufen. Wussten Sie das?"

% Dieses Jahr regt sich Ferger über die Rente auf. Die Franzosen protestieren, was er
% prinzipiell richtig findet, aber sich wundert, da die Franzosen bisher mit 62 in Rente gehen
% während er als erster Jahrgang mit 67 in Rente gehen wird.
% Das geht noch eine ganze Weile weiter.
% Schließlich regt er sich über Merz auf, der die SPD eine 11%-Partei genannt hat Ferger dachte, Merz
% sei in der FDP. Aber Merz ist CDU.
% Ferger ist Grünen-Wähler.
Erstes Kriterium für Verteilungskonvergenz in:

\begin{satz}\label{satz7.9}
	Seien $X,X_n,n∈ℕ$ Zufallsvariablen in $C$ über $(Ω,\A,\P)$. Falls
	\begin{enumerate}[label=(\arabic*)]
		\item \label{it:7.9fd}
			$X_n\fdto  X$, das heißt
			\begin{align*}
				 \underbrace{\klammern[\big]{X_n(t_1),…,X_n(t_k)}}_{=π_T∘ X_n} \distrto \underbrace{\klammern[\big]{X(t_1),…,X(t_k)}}_{=π_T∘ X}
				\quad∀ t_1,…,t_k∈ I\,∀ k∈ℕ,
			\end{align*}
			die so genannte \define{Konvergenz der fidis} (finite dimensional distributions, gesprochen \enquote{feidies}).
		\item \label{it:7.9limlimsup}
			$\begin{aligned}
				\lim_{k⟶∞}\limsup_{n⟶∞}\P\klammern[\Big]{ω\klammern[\big]{X_n,δ_k}>ε}=0\qquad∀ε>0
			\end{aligned}$\\
			für eine Folge $(δ_k)_{k∈ℕ}$ in $(0,∞)$ mit $δ_k \downarrow 0$
	\end{enumerate}
	so folgt:
	\begin{align*}
		X_n\distrto X\text{ in }(C,d)
	\end{align*}
\end{satz}
% Dieses Kriterium ist häufig nützlich für Martingale mit Maximalungleichungen.
% In seiner Dissertation hat er sich mit diesem Kriterium und den Stetigkeitsmoduln herumgeplagt.
% Das war schwierig.
\begin{proof}
	Siehe \cite[Seite 348]{gaensslerstute1977Wahrscheinlichkeitstheorie}.
\end{proof}
\begin{proof}[Felix' Beweis]\footnote{Nicht in der Vorlesung behandelt.}
	Laut Portmanteau-Theorem \ref{satz4.2} genügt es für $f \colon C(I) → ℝ$ gleichmäßig stetig
	und beschränkt zu zeigen,
	dass
	\begin{equation} \label{eq:7.9Ziel}
		∫_{C(I)} f \d (\P ∘ X_n^{-1}) → ∫_{C(I)} f \d (\P ∘ X^{-1}).
	\end{equation}
	Mit \ref{it:7.9fd} haben wir diese Konvergenz erstmal nur für spezielle $f$,
	nämlich solche der Form $f = \tilde{f} ∘ π_T$ mit $T ⊂ I$, $T$ endlich,
	mit $\tilde f \colon ℝ^{\measure{T}} → ℝ$ gleichmäßig stetig und beschränkt.

	Um \eqref{eq:7.9Ziel} auf die fidis zurückzuführen, betrachten wir den Fehler,
	den wir machen, wenn wir statt $f$ nur endlich viele Punkte betrachten:
	\begin{equation}
		\text{Sei } \tilde{f}_T (x_1, ..., x_{\measure{T}})
		:= f \argu[\Big]{\operatorname{Poly} \argu[\Big]{ (t_1, x_1), ..., (t_{\measure{T}}, x_{\measure{T}}) }},
	\end{equation}
	wobei $T = \set{t_1, ..., t_{\measure{T}}}$, $\inf I = t_1 < ... < t_{\measure{T}} = \sup I$,
	und $\operatorname{Poly}$ der Polygonzug durch die angegebenen Punkte ist,
	analog zur Konstruktion in Donsker \ref{satz7.9}.
	Für $h ∈ C(I)$ sei $\operatorname{Poly}(h, T)$ der Polygonzug, der als Stützpunkte
	die Elemente von $T$ hat und in diesen mit $h$ übereinstimmt.

	Zu $δ_k$ wähle nun $T_k ⊂ I$ so, dass die Abstände aufeinanderfolgender Punkte
	kleiner ist als $δ_k$ (z.\,B.\ gleichverteilt).

	Dann gilt
	\begin{equation} \label{eq:7.9Polyannaehrung}
		\norm{h - \operatorname{Poly}(h, T_k)}_{∞} ≤ 2 ω(h, δ_k) \quad (h ∈ C(I)).
	\end{equation}
	Dies lässt sich sehen, wenn man sich einen Abschnitt zwischen zwei Elementen
	betrachtet:
	\begin{align*}
		\text{Für } s ∈ \intervall{t_i}{t_{i + 1}}:
	  \abs{h(t_i) - X(s)} &< ω(h, δ_k) \\
		\text{und }
		\abs{h(t_i) - \operatorname{Poly}(h, T_k)(s)} &=
		\abs{\operatorname{Poly}(h, T_k)(t_i) - \operatorname{Poly}(h, T_k)(s)} \\
		&≤ \abs{\operatorname{Poly}(h, T_k)(t_i) - \operatorname{Poly}(h, T_k)(t_{i+1})} \\
		&= \abs{h(t_i) - h(t_{i + 1})} < ω(h, δ_k) \\
		⇒ \abs{h(s) - \operatorname{Poly}(h, T_k)(s)} &< 2 ω(h, δ_k)
	\end{align*}
	%
	Betrachte nun den Fehler durch Diskretisierung für einen stochastischen Prozess $Y \colon Ω → C(I)$
	\begin{align*}
		\abs{ ∫_{C(I)} \tilde f ∘ π_{T_k} - f \, \d (\P ∘ Y^{-1})}
		\overset{Δ\neq}&{≤} ∫_{Ω} \abs{ f(\operatorname{Poly}(Y, δ_k)) - f(Y) } \, \d \P \\
		\overset{\eqref{eq:7.9Polyannaehrung}}&{≤} ∫_{Ω} ω(f, 2 ω(Y, δ_k)) \, \d \P
	\end{align*}
	Da $f$ nach Annahme gleichmäßig stetig und $\P$ ein Wahrscheinlichkeitsmaß ist, ist die rechte Seite endlich. (Die Beträge entfallen, da der Stetigkeitsmodul immer nicht-negativ ist.)

	Diese Abschätzung kann nun für $X_n$ und $X$ genutzt werden:
	\begin{align}
		\abs{ ∫_{Ω} f \d (\P ∘ X_n^{-1}) - ∫_{Ω} f \d (\P ∘ X^{-1}) }
		\overset{Δ\neq}&{≤}
		∫_{Ω} ω(f, 2ω(X_n, δ_k)) \d \P \label{eq:7.9Xnpart} \\
		&\, +
		∫_{Ω} ω(f, 2ω(X, δ_k)) \d \P \label{eq:7.9Xpart} \\
		&\, +
		\klammern{ ∫_{Ω} \tilde f ∘ π_{T_k} ∘ X_n \d \P
		- ∫_{Ω} \tilde f ∘ π_{T_k} ∘ X \d \P } \label{eq:7.9discretePart}
	\end{align}
	Die drei Summanden möchten wir nun nach oben abschätzen. Gebe dafür ein $ε > 0$ vor.

	Betrachte \eqref{eq:7.9Xnpart}. $f$ ist beschränkt nach Voraussetzung, das heißt
	$ω(f, δ) < 2 \norm{f}_{∞}$ für alle $δ$. Wähle nun $η > 0$ so klein, dass
	$ω(f, η) < \frac{ε}{6}$. Dies geht, da $f$ gleichmäßig stetig ist.

	Wähle des weiteren $k_0 ∈ ℕ$ so groß, dass \ref{it:7.9limlimsup} für alle $k ≥ k_0$ liefert:
	$\P(2 ω(X_n, δ_k ≥ η) < \frac{ε}{6 · 2\norm{f}_{∞}}$ für alle $n ≥ n_0$, $n_0 ∈ ℕ$.
	Dann kann das Integral \eqref{eq:7.9Xnpart} aufgespalten werden, falls $k > k_0$:
	\begin{align*}
		∫_{Ω} ω(f, 2ω(X_n, δ_k)) \, \d \P
		& ≤ ∫_{\set{2 ω(X_n, δ_k) < η}} ω(f, 2ω(X_n, δ_k)) \, \d \P \\
		& \phantom{≤} + ∫_{\set{2 ω(X_n, δ_k) ≥ η}} ω(f, 2ω(X_n, δ_k)) \, \d \P \\
		& ≤ 1 · ω(f, η) + \frac{ε}{6 \norm{f}_{∞}} ω(f, 2ω(X_n, δ_k)) \\
		& ≤ \frac{ε}{6} + \frac{ε}{6 · 2 \norm{f}_{∞}} · 2 \norm{f}_{∞} = \frac{ε}{3}
	\end{align*}
	%
	$η$ ergibt auch für \eqref{eq:7.9Xpart} Sinn. Da $X$ ein stetiger stochastischer Prozess
	ist, gilt $ω(X, δ_k) → 0$ für $k → ∞$ (und damit $δ_k → 0$) (nicht nur fast) sicher.
	Damit konvergiert dies auch stochastisch, das heißt, es gibt ein $k_1 ∈ ℕ$, sodass
	für alle $k ≥ k_1$ wir $\P(2ω(X, δ_k) > η) < \frac{ε}{6}$ haben.
	Damit ergibt das gleiche Argument wie für \eqref{eq:7.9Xnpart}, dass \eqref{eq:7.9Xpart}
	kleiner als $\frac{ε}{3}$ ist für $k ≥ k_0$.

	Wegen der Konvergenz der fidis \ref{it:7.9fd} geht für jedes feste $δ_k$ \eqref{eq:7.9discretePart} gegen $0$.
	Wähle also $k_2 = \max \set{k_0, k_1}$ und $n_1 ≥ n_0$ so, dass $\eqref{eq:7.9discretePart} < \frac{ε}{3}$ für alle $n ≥ n_1$.
	Damit ist
	\begin{equation*}
		\abs{ ∫_{Ω} f \d (\P ∘ X_n^{-1}) - ∫_{Ω} f \d (\P ∘ X^{-1}) }
		< \frac{ε}{3} + \frac{ε}{3} + \frac{ε}{3} = ε \quad ∀ n ≥ n_1
	\end{equation*}
	Damit gilt $X_n \distrto X$.
\end{proof}

\begin{beispiel}\label{beispiel7.10} Sei
	\begin{align*}
		X_n(t):=A_n+B_n· t+C_n· t^2\quad∀ t∈ I = \intervall {α}{β}
		\quad\mit \klammern[\big]{A_n,B_n,C_n}\distrto (A,B,C)∈ℝ^3
	\end{align*}
	Dann gilt:
	\begin{align*}
		X_n\ntoinf  X\text{ in }(C,d)\qquad\mit\qquad X(t)=A+B· t+C· t^2
	\end{align*}
	Man sagt: $X_n$ und $X$ sind quadratische Funktionen mit zufälligen Koeffizienten.

	\begin{proof}[1. Beweis] Anwendung von \ref{satz7.9}, zur Übung. Hier der Beweis vom Jahr 2018:

		Seien $t_1,…,t_k∈ I$ beliebig.
		Für Voraussetzung \ref{it:7.9fd} in Satz \ref{satz7.9} reicht es gemäß Cramér-Wold-Device \ref{satz5.4CramerWoldDevice} zu zeigen:
		\begin{align*}
			\sum_{j=1}^k λ_j · X_n(t_j)
			\distrto \sum_{j=1}^n λ_j· X(t_j) \text{ in } ℝ
			\qquad ∀ λ = \klammern[\big]{λ_1,…,λ_k} ∈ ℝ^k
		\end{align*}
		%Ich wurde von Prof Ferger bemitleidet, weil ich in Tex nicht alles umsetzen kann, was er an der Tafel durch Wisch-Technik erzeugt.
		Dazu:
		\begin{align*}
			\sum_{j=1}^kλ_j· X_n(t_j)
			&=A_n·\sum_{j=1}^nλ_j+B_n·\sum_{j=1}^kλ_j· t_j+C_n·\sum_{j=1}^kλ_j· t_j^2\\
			\overset{\L,\ref{satz4.10ContinuousMappingTheorem}}&{\longrightarrow}
			A·\sum_{j=1}^kλ_j+B·\sum_{j=1}^kλ_j· t_j+C·\sum_{j=1}^kλ_j· t_j^2
			=\sum_{j=1}^kλ_j· X(t_j)
		\end{align*}
		Zu Voraussetzung \ref{it:7.9limlimsup} in Satz \ref{satz7.9}:
		\begin{align*}
			\abs{X_n(s)-X_n(t)}
			&=\abs[\Big]{B_n·(s-t)+C_n·\underbrace{\klammern[\big]{s^2-t^2}}_{(s-t)·(s+t)}}\\
			\overset{Δ\neq}&{≤}
			\abs{B_n}·\underbrace{\abs{s-t}}_{≤δ}+\abs{C_n}·\underbrace{\abs{s-t}}_{≤δ}·\underbrace{\abs{s+t}}_{≤\abs{s}+\abs{t}≤:K}\\
			&≤
			\abs{B_n}·δ+K·\abs{C_n}·δ\qquad∀ s,t∈ I\mit \abs{s-t}≤δ\\
			\overset{\sup}{⇒}
			ω(X_n,δ)
			&≤\abs{B_n}·δ+K·\abs{C_n}·δ\\
			⇒
			\P\klammern[\Big]{ω\klammern[\big]{X_n,δ}>ε}
			&≤\P\argu[\Big]{\abs{B_n} · δ + K · \abs{C_n} · δ > ε}\\
			\overset{\eqref{eqProofBeispiel7.10}}&{≤}
			\P\argu[\Big]{\abs{B_n}·δ>\frac{ε}{2}}+\P\argu[\Big]{K·\abs{C_n}·δ>\frac{ε}{2}}\\
			&=
			\P\argu[\Big]{\abs{B_n}>\frac{ε}{2 δ}}+\P\argu[\Big]{\abs{C_n}>\frac{ε}{2 K δ}}\\
			&=
			1-F_n\klammern{\frac{ε}{2 δ}}+1-G_n\klammern{\frac{ε}{2 K δ}}
		\end{align*}
		%Ferger: "Warum habe ich das eigentlich so gemacht!?"
		Hierbei ist $F_n$ die Verteilungsfunktion von $\abs{B_n}$ %\distrto \abs{B}$
		und $G_n$ die Verteilungsfunktion von $\abs{C_n}$.
		Erinnerung:
		\begin{align}\label{eqProofBeispiel7.10}
			\set[\big]{ U+V>ε}⊆\set{ U>\frac{ε}{2}}∪\set{ V>\frac{ε}{2}}
		\end{align}
		Mit Korollar \ref{korollar4.5} folgt, dass es eine Folge $(δ_k)_{k∈ℕ}$  mit $δ_k\downarrow0$,
		so dass $\frac{ε}{2 δ_k}$ und $\frac{ε}{2 K δ_k}$ Stetigkeitsstellen der jeweiligen Grenz-Verteilungfunktionen sind,
		%Ferger: "Na gut, einen Nobelpreis für Literatur kriege ich jetzt nicht."
		\begin{align*}
			\limsup_{n⟶∞} \P\argu[\big]{ω \argu{X_n,δ_k} > ε}
			≤ \P\argu[\Big]{\abs{B} > \underbrace{\frac{ε}{2 δ_k}}_{\overset{k⟶∞}{\longrightarrow}∞}}
			+\P\argu[\Big]{\abs{C} > \underbrace{\frac{ε}{2 K δ_k}}_{\overset{k⟶∞}{\longrightarrow}∞}}
			\qquad∀ k∈ℕ
		\end{align*}
		%Ferger: "Hach ich bin so doof. Ich hätte mir das Leben leichter machen können."
		$k⟶∞$ liefert dann \ref{it:7.9limlimsup}.
	\end{proof}
	\begin{proof}[2. Beweis]
		Sei $h \colon ℝ^3 → C$, $(a, b, c) ↦ h(a, b, c) \colon I → ℝ : t ↦ h(a, b, c)(t) := a + bt + ct^2$.
		$h$ ist stetig auf $ℝ^3$, denn: Sei $(a_n, b_n, c_n) → (a, b, c)$.
		Dann
		\begin{align*}
			d(h(a_n, b_n, c_n), h(a, b, c))
			&= \sup_{t ∈ I} \abs{a_n + b_n + c_n t^2 - (a + bt + c^2)} \\
			&≤ \sup_{t ∈ I} \abs[\big]{a_n - a} + \abs[\big]{(b_n - b) t} + \abs{(c_n - n) t^2} \\
			&≤ \abs{a_n - a} + \abs{b_n - b} L + \abs{c_n -c} L^2 \quad L := \max(\abs{α}, \abs{β}) \\
			&→ 0 + 0 · L + 0 · L^2 = 0
		\end{align*}
		Damit ist $h$ stetig.
		Mit dem CMT \ref{satz4.10ContinuousMappingTheorem} folgt
		wegen der Voraussetzung $(A_n, B_n, C_n) \distrto (A, B, C)$
		\begin{align*}
			X_n = h(A_n, B_n, C_n) \distrto h(A, B, C) = X & \qedhere
		\end{align*}
	\end{proof}
\end{beispiel}

Zweite handlicherere Möglichkeit für den Nachweis der Verteilungskonvergenz in $C$ liefert:

\begin{satz}[Momentenkriterium von Kolmogoroff]\label{satz7.11MomentenkriteriumVonKolmogoroff}\enter
	Seien $X,X_n,n∈ℕ$ Zufallsvariablen in $C$ mit
	\begin{align}\label{eqSatz7.11Vor1}\tag{cfd}
		X_n\fdto  X\qquad\text{(vgl.\ \ref{it:7.9fd} in Satz \ref{satz7.9})}
	\end{align}
	Falls es eine Konstante $γ>0$ und $α>1$ sowie eine stetige und monoton wachsende Funktion $F:I⟶ℝ$ gibt, derart, dass
	\begin{align}\label{eqSatz7.11VorM}\tag{M}
		\Earg[\Big]{\abs{X_n(s)-X_n(t)}^γ} ≤ \klammern[\big]{F(s)-F(t)}^α \qquad ∀ s, t ∈ I \mit s>t
	\end{align}
	(\define{Momentenbedingung}).
	Dann gilt:
	\begin{align*}
		X_n\distrto  X\text{ in }\klammern[\big]{C(I),d}
	\end{align*}
\end{satz}

\begin{proof}
	Siehe \cite[Seite 96]{billingsley2013convergence} % B illingsley (1968), \undefine{Convergence of probability measures}, Seite 96.
	%Ferger: "Das Buch hier war früher meine Fibel."
	%Ferger: "Jim Morrison von den Doors. Das war mein Held. Neben Skorokott."
\end{proof}

\paragraph{Ziel:} Verteilungskonvergenz der Partialsummenprozesse $X_n$ aus dem Beispiel \ref{beispiel7.4}.
Dazu:

\begin{definition}[Brownsche Bewegung] \label{def7.12} %7.12
	Sei $I=[0,b]\mit b>0$ und $B:=\set[\big]{ B(t):=B(t,ω), t ∈ I}$ ein stetiger stochastischer Prozess über $(Ω,\A,\P)$ mit
	\begin{enumerate}[label=(\arabic*)]
		\item \label{it:7.12begin0} $\begin{aligned}
			B(0)=B(0,ω)=0\qquad∀ω∈Ω
		\end{aligned}$
	\item \label{it:7.12independantchanges} $∀\, 0=:t_0≤ t_1<…<t_r≤ b$, $r ∈ ℕ$
		sind die Zuwächse $B(t_i)-B(t_{i-1})$, $1≤ i≤ r$ \emph{unabhängig}
	\item \label{it:7.12normal} $\begin{aligned}
			0≤ s<t≤ b⇒ B(t)-B(s)\sim\mathcal{N}(0,t-s)
		\end{aligned}$
	\end{enumerate}
	Dann heißt $B$ \define{Brownsche Bewegung (BB)} auf $I$.

	Vollkommen analog definiert man eine BB auf $I= ℝ_{+} = [0,∞)$.
\end{definition}

%Ferger: "Das ist wie so ein Wunschzettel. Es gelte 1, 2 und 3. Passend zur Weihnachtszeit. Wenn's dumm läuft, kann es noch sein, dass mein Wunsch nicht erfüllt wird, weil er nicht erfüllt werden kann."
%Ferger: "In der Hoffnung das Sie nicht schonmal hier gesessen haben: Weil ich immer dieselben Geschichten erzähle. Ich bin jetzt in so einem Alter wo man alles mehrfach erzählt."
%Ferger: "Da hat mal jemand eine Doktorarbeit geschrieben über eine tolle Funktionenklasse. Und dann kam jemand daher, dass die Funktionenklasse nur aus der Eins-Funktion besteht. Das war echt ein Griff ins Klo. Wäre noch heftiger gewesen, wenn die Funktionenklasse leer gewesen wäre. Aber: So sind wir ja auch alle gestrickt. So werden wir konditionert. Weil, wenn der Mathematiker irgendwas hört, fragt der Mathematiker ständig "Existiert das?""

\begin{satz}[Lévy]\label{satz7.13Levy}
	Eine Brownsche Bewegung existiert.
\end{satz}

\begin{proof}
	Siehe Vorlesung \undefine{Stochastische Prozesse}.
	%Ferger: "Es gibt einen konstruktiven Beweis, der ist auch lehrrreich, aber da haben wir jetzt keine Zeit für.
	%Anmerkung des Autors: Geschichten erzählen ist wohl wichtiger.
\end{proof}

\begin{lemma}\label{lemma7.14}
	Sei $B$ eine BB auf $I$ und $t_1<…<t_r$ aus $I$. Dann gilt:
	\begin{align*}
		\klammern[\Big]{B(t_1),…, B(t_r)}^T\sim\mathcal{N}_r(0,Γ)\text{ wobei }\\
		Γ:= \klammern[\Big]{\Cov\klammern[\big]{B(t_i),B(t_j)}}_{1≤ i,j≤ r}
	=\klammern[\big]{\min(t_i, t_j)}_{1≤ i,j,≤ r}
	\end{align*}
\end{lemma}

\begin{proof}
	Sei $t_0:=0$. Dann gilt:
	\begin{align*}
		B(t_j)
		\overset{\ref{def7.12}\ref{it:7.12begin0}}=
		B(t_j)-\underbrace{B(t_0)}_{=0}
		\overset{\text{Teles}}{=}
		\sum_{i=1}^j\klammern[\big]{B(t_i)-B(t_{i-1})}
		=\sum_{i=1}^j√{t_i-t_{i-1}}·\underbrace{\frac{B(t_i)-B(t_{i-1})}{√{t_i-t_{i-1}}}}_{=:Z_i}
	\end{align*}
	Da $B(t_i)-B(t_{i-1})\sim\mathcal{N}(0,t_i-t_{i-1})$ folgt, dass die $Z_i$ \iid\ $\sim\mathcal{N}(0,1)$ sind.
	Also ist der Vektor
	\begin{align*}
		\begin{pmatrix}
			B(t_1)\\
			\vdots\\
			B(t_r)
		\end{pmatrix}&=\begin{pmatrix}
			√{t_1} & 0 & \hdots & \hdots & 0\\
			√{t_1} & √{t_2-t_1} & 0 & \hdots & 0\\
			\vdots & \vdots & \ddots & \ddots & \vdots\\
			√{t_1} & √{t_2-t_1} & √{t_3-t_2} & \hdots & √{t_r - t_{r - 1}} \\
			% TOCHECK in der Vorlesung: sollte rechts unten nicht √{t_r - t_{r - 1}} statt 0 stehen?
		\end{pmatrix} \begin{pmatrix}
			Z_1\\
			\vdots\\
			Z_r
		\end{pmatrix} \\ ⇒\begin{pmatrix}
			B(t_1)\\
			\vdots\\
			B(t_r)
		\end{pmatrix} & \sim \mathcal{N}_r(μ,Γ) \quad \text{Def.\ normaler Vektor} \\
		μ_i &:= \Earg[\Big]{B(t_i)}
		= \Earg[\Big]{ \underbrace{B(t_i) - B(t_0)}_{\sim \mathcal{N}(0, t_i - t_0)} } \qquad ∀ i \\
		⇒ μ_i &:= \klammern[\big]{μ_1,…,μ_r}=0
	\end{align*}
	Ferner sei $i≤ j$. Dann gilt:
	\begin{align*}
		\Cov\klammern[\big]{B(t_i),B(t_j)}
		&=\Earg[\Big]{B(t_i)· B(t_j)}\\
		&=\Earg[\Big]{B(t_i)·\klammern[\big]{ B(t_j)-B(t_i)+B(t_i)}}\\
		&=\Earg{B(t_i)·\klammern[\big]{B(t_j)-B(t_i)} + \klammern[\big]{B(t_i)}^2} \\
		&=\Earg[\big]{\underbrace{B(t_i)·\klammern[\big]{B(t_j)-B(t_i)}}_{\text{beide Faktoren sind unabh.}}}
		+\Earg{\klammern[\big]{B(t_i)}^2}\\
		\overset{\text{unab}}&=
		\underbrace{\Earg[\big]{B(t_i)}}_{=0}
		· \underbrace{\Earg[\big]{B(t_j)-B(t_i)}}_{=0}
		+ \underbrace{\Earg[\Big]{\klammern[\big]{B(t_i)}^2}}_{= \Var \argu[\big]{B(t_i)}} \\
		\overset{\text{Vert}}&=
		\Var\klammern[\big]{\mathcal{N}(0,t_i)}\\
		&=t_i\\
		\overset{i≤ j}&{=}
		\min(t_i, t_j)
	\end{align*}
	Analog behandle den Fall $i≥ j$.
\end{proof}

\begin{korollar}\label{korollar7.15Folgerung}
	Die Verteilung einer BB ist eindeutig bestimmt.
\end{korollar}

\begin{proof}
	Seien $B$ und $\tilde{B}$ zwei BBs (i.\,d.\,R.\ über unterschiedlichen Wahrscheinlichkeitsräumen definiert). Dann gilt:
	\begin{align*}
		B \stackeq{\L} \tilde{B}
		\overset{\ref{satz7.5}}&{⇔}
		% todo: uncomment. Attention: wrong utf8-sequence
		\klammern[\big]{B(t_1), …, B(t_r)}^T
		\stackeq{\L}\klammern{\tilde{B}(t_1), …, \tilde{B}(t_r)}^T
		& ∀ t_1 < … < t_r ∈ I, ∀ r ∈ ℕ\\
		\overset{\ref{lemma7.14}}&{⇔}
		\mathcal{N}_r(0,Γ)=\mathcal{N}_r(0,Γ)
		&∀ t_1 < … < t_r ∈ I, ∀ r ∈ ℕ
	\end{align*}
	Die letzte Aussage gilt aber, weil jede mehrdimensionale Normalverteilung $\mathcal{N}_r(μ,Γ)$ eindeutig durch $μ$ und $Γ$ festgelegt ist:
	\begin{align*}
		\mathcal{N}(μ,σ^2)=\mathcal{N}(m,s^2)⇔ m=μ\text{ und }s^2=σ^2
		&
		\qedhere
	\end{align*}
\end{proof}

Die Verteilung $\P∘ B^{-1}=:W$ von  einer Brownschen Bewegung $B$ heißt \define{Wiener-Maß} (nach Norbert Wiener).
%Ferger: "Absoluter Schlaukopf. Ein Ammi. Ein US-Amerikaner. Der hat 21 promoviert. Da wo wir noch mit den Klötzchen spielen, hat der schon promoviert."
\begin{align*}
	B:(Ω,\A,\P)⟶\klammern[\Big]{C(I),\B_d\klammern[\big]{C(I)}}
\end{align*}
%Ferger: "Die Brownsche Bewegung wird genannt Brownsche Bewegeung, weil folgendes passiert ist:
% es gab einen schottischen Bonatiker.
% Der hat also eine Flüssigkeit genommen,
% z.\,B.\ Wasser und dann ganz kleine Pollenkörner oder Staubkörnchen,
% also so was ganz klitzekleines oder eie kleine Minischuppe in sein Töpfchen getan,
% in seine Flüssigkeit und hat draufgeguckt und gesagt "Boah, da ist ja was los." [...]
% Ja ich erzähl das so wie bei der Sendung mit der Maus, weil man das so gut versteht. [...]
% Und er stellt fest, dass da viel los ist.
% Die Beobachtung hatten andere Leute wohl auch schon gemacht.
% Die ersten Leute, jeder von denen hat sich natürlich gefragt: Was steckt dahinter
% Die ersten Leute haben als erklärungsversuch gegeben, dass es biologische Energie in sich hat.
% Der Robert Braun hat gesagt: Nein, das ist nicht so. Können Sie ja mal googeln.
% Entscheidend ist Folgendes: Das war ca. 182?.
% Dann kam Albert Einstein und hat gesagt: Nenene bzw.\ jajaja, was hier passiert ist Folgendes:
% So eine Flüssigkeit besteht ja aus ganz vielen Molekülen, die in Bewegung sind.
% Und die schießen wie wild hin und her. Die Moleküle sind aber klitzeklitzeklein.
% Das kleine Teilchen ist aber im mikroskopischen Bereich,
% also im Vergleich zu den Molekülen ein Riesen-Oschi.
% Und die kleinen Teilchen hauen die ganze Zeit dagegen.
% Der Einstein ist hergegangen und hat das für Physiker-Verhältnisse sehr mathematisch beschrieben.
% Der Wiener hat das nochmal auf saubere, mathematische Füße gestellt
% (Funtionenraum, Verteilungen und solche Sachen...)
% Die haben das ein oder andere wohl handwaving gemacht und
% Norbert Wiener hat das dann mal mathematisch sauber gemacht.
% Deshalb heißt die Verteilung einer Brownschen auch Wiener Maß.

% Ferger: Sind ja immernoch 7 Minuten. Mist. Na da kann ich ja noch eine Geschichte erzählen.

% !TEX root = MSTAT19.tex
% This work is licensed under the Creative Commons
% Attribution-NonCommercial-ShareAlike 4.0 International License. To view a copy
% of this license, visit http://creativecommons.org/licenses/by-nc-sa/4.0/ or
% send a letter to Creative Commons, PO Box 1866, Mountain View, CA 94042, USA.

\begin{satz}[Donsker]\label{satz7.16Donsker}%\enter
	Seien $(ξ)_{i∈ℕ}$ \iid\ mit $\Earg{ξ_1}=0$ und $\Var(ξ_1)=1$.
	\begin{align*}
		S_k &:= \sum_{i=1}^k ξ_i & ∀ & k ∈ ℕ_0 \\
		X_n(t) &:= \frac{1}{√{n}} S_{\floor{n t}}
		+ \frac{1}{√{n}} \klammern[\big]{n t - \floor{n t}} ξ_{\floor{n t} + 1} & ∀ & t ∈ I = \intervall{[0}{b}
	\end{align*}
	($b>0$ fest), d.\,h.\ $X_n$ ist der Polygonzug durch die Punkte $\klammern{\frac{k}{n}, \frac{1}{√{n}} S_k}_{0 ≤ k ≤ bn}$. Dann gilt:
	\begin{align*}
		X_n\distrto  B\text{ in }\klammern[\big]{C(I),d}
	\end{align*}
	wobei $B$ eine BB auf $I=[0,b]$ ist.
\end{satz}
% Vielleicht erzähle ich doch die Geschichte.
% Der Donsker hat die Geschichte 1952 gemacht. Motiviert durch eine andere Arbeit von 1949
% von Duqe (oder so ähnlich). Der hat da ausgerechnet, was die Verteilung des Funktionals
% Brownsche Brücke
% \sup_{0 ≤ t ≤ 1} \abs{B(t)} ist. Wird Ferger uns auch noch zeigen.
% Dann haben Leute wie Kolmogoroff oder Smirnoff sich
% \sup_{0 ≤ t ≤ 1} √n \abs{F_n(t) - F(t)}
% angeschaut und für jedes feste n die Verteilungsfunktion H_n ausgerechnet
% und gesehen, dass dieses schwach gegen eine Verteilung H konvergiert.
% Duqe bemerkt dann seine Verteilung genau dieses H ist.
% Die Vermutung war dann, dass die stochastischen Prozesse auch in irgendeiner Form konvergieren.
\begin{proof}
	Anwendung von Satz \ref{satz7.11MomentenkriteriumVonKolmogoroff}.
	Zeige also dessen Voraussetzungen:
	\paragraph{Zeige Konvergenz der fidis \eqref{eqSatz7.11Vor1}:}
	Seien $0 = t_0 ≤ t_1<…<t_r≤ b$.
	Um den Satz \ref{satz4.14Cramer} von Cramér zu nutzen, betrachte:
	\begin{align*}
		% \norm{\klammern[\big]{X_n(t_i)}_{1≤ i≤ r}-\left(\frac{1}{√{n}}·\sum_{j=1}^{\floor{n· t_i}}ξ_j\right)_{1≤ i≤ r}}
		\norm{\klammern[\big]{X_n(t_i)}_{1≤ i≤ r}
			-\klammern{\frac{1}{√{n}}S_{\floor{nt_i}}}_{1≤ i≤ r}
		}
		% &=\Bigg(\sum_{i=1}^r\underbrace{\left|X_n(t_i)-\frac{1}{√{n}}·\sum_{j=1}^{\floor{n t_i}}ξ_j\right|^2}_{\frac{1}{√{n}}·\underbrace{\big|n· t_i-\floor{n t_i}\big|}_{≤1}·ξ_{\floor{n t_i}}}\Bigg)^{\frac{1}{2}}\\
		&=\klammern[\Bigg]{
			\sum_{i=1}^r
				\underbrace {\abs{X_n(t_i)-\frac{1}{√{n}} S_{\floor{nt_i}}}^2}
				_{ \frac1{√n} \abs[\Big]{
					\underbrace {(n t_i - \floor{n t_i})}
						_{≤ 1}
					ξ_{\floor{n t_i} + 1}
				}
				}
		}^{\frac{1}{2}}\\
		&≤
		\frac1{√n} \klammern{\sum_{i=1}^r ξ_{\floor{n t_i} + 1}^2}^{\frac12}
		% \\⇒
		% \P\klammern{\norm{\klammern[\big]{X_n(t_i)}_{1 ≤ i ≤ r} - \klammern{\frac{1}{√{n}} \sum_{j=1}^{\floor{n t_i}} ξ_j}_{1≤ i≤ r}}}
		% &≤ \P \argu{\sum_{i=1}^r ξ_{\floor{n t_i} + 1}^2 > n ε^2}\\
		% \overset{\text{Markov}}&{≤}
		% \frac1n ε^{-2} \sum_{i=1}^r\underbrace{\Earg{ξ^2_{\floor{n t_i} + 1}}}_{=1~∀ i}\\
		% &=
		% \frac1n ε^{-2} r\ntoinf 0 \qquad ∀ ε > 0 \\
		% ⇒ \norm{(...) - (...)} &≤
		% \frac 1{√n} \klammern{\sum_{i = 1}^r ξ_{\floor{n t_i} + 1} ^2}^{\frac 12}
		\\
		⇒ \P( \norm{...} > ε)
		&≤ \P\argu{ (...)^{\frac 12} > ε √n} %\\
		%&
		= \P\argu{\sum_{i = 1}^r ξ_{\floor{nt_i} + 1} ^2 > ε^2 n} \\
		\overset{\text{Markov}\neq}&{≤}
		\frac 1n ε^{-2} \Earg{\sum_{i = 1}^r ξ^2_{\floor{n t_i} + 1}} \\
		&= \frac 1n ε^{-2} r \ntoinf 0 \quad
		∀\, ε > 0
	\end{align*}
	% $⇒ \klammern[\big]{X_n(t_i)}_{1 ≤ i ≤ r}$ und $\klammern{\frac 1{√n} S_{\floor{n t_i}}}_{1 ≤ i ≤ r}$ sind stochastisch äquivalent. Wegen \ref{satz4.14Cramer} reicht es zu zeigen:
	% \begin{align*} \tag{$*$} % do not know if we use this somewhere
	% 	\klammern{\frac 1{√n} S_{\floor{n t_i}}}_{1 ≤ i ≤ r}
	% 	&\distrto \klammern{B(t_1), ..., B(t_r)} \text{ in } ℝ^r \\
	% 	\overset{\text{CMT} \ref{satz4.10ContinuousMappingTheorem}}{⇔}
	% 	\frac 1{√n} \klammern[\big]{S_{\floor{n t_i}} - S_{\floor{n t_{i-1}}}}_{1 ≤ i ≤ r}
	% 	&\distrto \klammern[\big]{B_{t_i} - B_{t_{i-1}}}_{1 ≤ i ≤ r} \text{ in } ℝ^r
	% \end{align*}
	% denn: $h \colon ℝ^r → ℝ^r$ mit $h(x_1, ..., x_r) = (x_1, x_2-x_1, x_3- x_2, ..., x_r - x_{r-1})$
	% ist stetig auf $ℝ^r$ mit Inverse ... s. unten
  %
	Folglich sind die beiden Folgen
	\begin{align*}
		\klammern[\big]{X_n(t_i)}_{1 ≤ i ≤ r} \qquad \text{und} \qquad
		\klammern{\frac1{√n} \sum_{j=1}^{\floor{n t_i}} ξ_j}_{1≤ i≤ r}
	\end{align*}
	stochastisch äquivalent.
	Wegen Cramér (Satz \ref{satz4.14Cramer}) genügt es
	\begin{align}\label{eqProof7.16fd}\tag{fd}
		\klammern{ \frac1{√n} S_{\floor{n t_i}}}_{1≤ i≤ r}
		\distrto \klammern[\big]{B(t_1),…,B(t_r)}
	\end{align}
	zu zeigen.
	Aber \eqref{eqProof7.16fd} ist äquivalent zu
	\begin{align}\label{eqProof7.16Stern}\tag{$\ast$}
		\frac1{√n} \klammern{S_{\floor{n t_i}} - S_{\floor{n t_{i-1}}}}_{1≤ i≤ r}
		\distrto \klammern[\big]{B(t_i)-B(t_{i-1})}_{1 ≤ i ≤ r}
	\end{align}
	denn:
	\begin{align*}
		h \colon ℝ^r ⟶ ℝ^r, \qquad
		h(x_1,…,x_r) := \klammern{x_1, x_2 - x_1, x_3 - x_2, … , x_r - x_{r-1}}
	\end{align*}
	ist stetig auf $ℝ^r$ und hat stetige Inverse $h^{-1}$ mit
	\begin{align*}
		h^{-1}(y_1,…,y_r) = \klammern{y_1, y_1 + y_2, …, y_1 + … + y_r}
	\end{align*}
	Also ist \eqref{eqProof7.16fd} äquivalent zu \eqref{eqProof7.16Stern} gemäß CMT \ref{satz4.10ContinuousMappingTheorem}.

	Beachte wegen \undefine{Blockungslemma}\footnote{%
		Erläuterung von Felix:
		Die Zuwächse $\frac1{√n} \klammern{S_{\floor{n t_i}}-S_{\floor{n t_{i-1}}}}_{1≤ i≤ r}$
		sind Summen von disjunkten Teilmengen von den $(ξ_k)_k$.
		Die $(ξ_k)_k$ sind unabhängig, damit mit dem Blockungslemma auch die
		Tupel aus mehreren $ξ_{\floor{n t_{i-1}}}$ bis $ξ_{\floor{n t_i}}$
		und damit auch die Summen, da Summenbildung eine messbare Abbildung von
		$ℝ^{\floor{n t_{i - 1}} - \floor{n t_i}}$ nach $ℝ$ ist.
	}
		sind die Zuwächse
	$\frac1{√n} \klammern{S_{\floor{n t_i}}-S_{\floor{n t_{i-1}}}}_{1≤ i≤ r}$ \emph{unabhängig}
	und gemäß \ref{def7.12} \ref{it:7.12independantchanges} sind auch
	$\klammern[\big]{B(t_i)-B(t_{i-1})}$, $1 ≤ i ≤ r$ unabhängig.
	Somit ist gemäß Satz \ref{satz4.21} \eqref{eqProof7.16Stern} äquivalent zu
	\begin{align}\label{eqProof7.16SternStern}\tag{$\ast\ast$}
		\frac1{√n} \klammern{S_{\floor{n t_i}}-S_{\floor{n t_{i-1}}}}
		\distrto
		\klammern[\big]{B(t_i) - B(t_{i-1})} \text{ in } ℝ \qquad ∀ 1 ≤ i ≤ r
	\end{align}
	Dazu setze $k_n:=\floor{n t_i} - \floor{n t_{i-1}}$. Dann gilt:
	\begin{align*}
		&\frac1{√n} \klammern{S_{\floor{n t_i}}-S_{\floor{n t_{i-1}}}}\\
		&=\frac1{√n} \sum_{j=\floor{n t_{i-1}} + 1}^{\floor{n t_i}}ξ_j
		=\frac1{√n} \sum_{j=1}^{k_n}ξ_{\floor{n t_{i-1}} + j}
		\overset{\L}{=}
		\frac1{√n} \sum_{j=1}^{k_n} ξ_j\\
		&= %\underbrace{
		√{\frac{k_n}{n}} %}_{=√{\frac{\floor{n t_i} - \floor{n t_{i-1}}}{n}}}
		· \frac{1}{√{k_n}} \sum_{j=1}^{k_n}ξ_j\\
		&= \underbrace{√{\frac{\floor{n t_i} - \floor{n t_{i-1}}}{n}}}_{\ntoinf √{t_i-t_{i-1}}}
		· \underbrace{\frac1{√{k_n}} \sum_{j=1}^{k_n} ξ_j}_{
				\stackrelnew{\text{ZGWS}}{n⟶∞}{\longrightarrow}
				\mathcal{N}(0,1)}
			\stackrelnew{\ref{beisp4.18} \ref{it:4.18einDim}}{\L}{\longrightarrow}
			√{t_i-t_{i-1}} \mathcal{N}(0,1)
			\overset{\L}{=} \underbrace{\mathcal{N}(0,t_i-t_{i-1})}_{
				\overset{\ref{def7.12}}{=}B(t_i)-B(t_{i-1})}
	\end{align*}
	Damit ist Voraussetzung \eqref{eqSatz7.11Vor1} aus \ref{lemma7.14} gezeigt.

	\paragraph{Momentenbedingung} Wir zeigen \eqref{eqSatz7.11VorM} für $γ=4$ und $α=2$, \emph{falls} $μ_4:=\Earg[\big]{ξ_1^4}<∞$ (das ist eine stärkere Voraussetzung!)

	Seien $s>t$ aus $I = \intervall0b$.
	Dann gilt:
	\begin{equation}\label{eqProof7.16Plus}\tag{+}
		X_n(s) - X_n(t)
		= \frac1{√n} \sum_{j = \floor{n t} + 1}^{\floor{n s}} ξ_j
		+ \frac1{√n} \klammern{n s - \floor{n s}} ξ_{\floor{n s}+1}
		- \frac1{√n} \klammern{n t - \floor{n t}} ξ_{\floor{n t}+1}
	\end{equation}
	Da
	\begin{align}\label{eqProof7.16PlusPlus}\tag{++}
		\floor{n t} ≤ n t < \floor{n t} + 1 \qquad ∀ t ≥ 0
	\end{align}
	folgt für
	\begin{align*}
		k := \floor{n t} \und l := \floor{n s}: \qquad
		\frac kn ≤ t < \frac{k+1}n \und
		\frac ln ≤ s < \frac{l+1}n
	\end{align*}
	\paragraph{Fall 1: $s - t ≤ \frac1n$}
	\begin{enumerate}[label=(\roman*)]
		\item \label{it:7.16proof1} $s$ und $t$ liegen im selben Intervall $\intervallHO{\frac kn}{\frac{k+1}n}$
		\item \label{it:7.16proof2} $s$ und $t$ liegen in benachbarten Intervallen
			$t ∈ \intervallHO{\frac kn}{\frac{k + 1}n}$,
			$s ∈ \intervallHO{\frac{k + 1}n}{\frac{k+2}n}$
	\end{enumerate}
	%Ferger: Habe ich schon erwähnt, dass ich in Deutsch richtig schlecht war?
	\subparagraph{Fall \ref{it:7.16proof1}:} $l = \floor{n s} = \floor{n t} = k$ und
	\begin{align*}
		X_n(s)-X_n(t)
		&=\frac1{√n} (n  s - n t) ξ_{\floor{n t}+1}
		=√n (s - t) ξ_{\floor{n t} + 1} \\
		⇒
		\Earg{\abs[\big]{X_n(s)-X_n(t)}^4}
		&=n^2·(s-t)^4·μ_4
		=μ_4·(s-t)^2·\underbrace{(s-t)^2}_{<\frac{1}{n^2}}· n^2\\
		&≤μ_4·(s-t)^2
	\end{align*}
%Ferger: Sie haben vielleicht mitbekommen es wird momentan von der Digitalisierung geredet. Und der Bund will 5 mrd € zur Verfügung stellen. Aber die Lehrer sagen: "Wir müssen die Kinder in das Zeitalter der Digitalisierung bringen". Gestern, ich lag gerade so auf meiner Couch mit einem Glas Wein und ein Journalist sagte dann "Man darf nicht zurück in die Kreidezeit (Lehrer an der Tafel)" Ich bin so kurz zusammengezuckt und habe mir gedacht: "Ach du scheiße, was machst du denn in Dresden..." Also es gibt tatsächlich Leite, die sich auch damit beschäftigen und sagen das ist Mühsam mit der Tafel. Aber indem man schreibt, beschäftigt man sich schon intensiver mit Stoff.
%Ferger: Manchmal ist weniger mehr.
	\subparagraph{Fall \ref{it:7.16proof2}:} $k=\floor{n t}$ und $l=k+1=\floor{n s}$. Dann folgt aus \eqref{eqProof7.16Plus}:
	\begin{align*}
		X_n(s)-X_n(t)
		&=\frac1{√n} ξ_{k+1}
		+ \frac1{√n} \klammern[\big]{n s - (k+1)} ξ_{k+2}
		- \frac1{√n} (n t - k) ξ_{k+1} \\
		&=√n \klammern{s -\frac{k+1}n} ξ_{k+2}
		+ √n \klammern{\frac{k+1}{n} - t} ξ_{k+1}
	\end{align*}

	\begin{lem}[$c_r$-Ungleichung]
		Seien $a_1, …, a_m ∈ ℝ$ (paarweise verschieden) und $r ≥ 1$. Dann gilt:
		\begin{align}\label{eqCrUngleichung}\tag{$c_r$}
			\abs{\sum_{i=1}^m a_i}^r ≤ c_r \sum_{i=1}^m \abs{a_i}^r \mit c_r := m^{r-1}
		\end{align}
	\end{lem}

	\begin{proof}
		Sei $Z$ diskrete Zufallsvariable mit $\P(Z=a_i)=\frac{1}{m}$ für $1≤ i≤ m$. Dann gilt:
		\begin{align*}
			\abs{\frac1m \sum_{i=1}^m a_i}^r = \abs[\Big]{\Earg{Z}}^r
			\overset{\text{Jensen}} &{≤}
			\Earg[\big]{\abs{Z}^r}
			= \frac{1}{m} \sum_{i=1}^m\abs{a_i}^r
			⇒
			\abs{\sum_{i=1}^m a_i}^r
			≤ m^{r - 1} \sum_{i=1}^m \abs{a_i}^r
			\qedhere
		\end{align*}
	\end{proof}

	Mit $m=2$ und $r=4$ folgt:
	\begin{align*}
		\abs{X_n(s)-X_n(t)}^4
		&≤ 8 \klammern[\bigg]{n^2 ξ^4_{k+2} \klammern[\Big]{\underbrace{s - \frac{k+1}n}_{≤ s - t}}^4
		+n^2 ξ^4_{k+1} \klammern[\Big]{\underbrace{\frac{k+1}n - t}_{≤ s - t}}^4}\\
		⇒
		\Earg{\abs[\big]{X_n(s) - X_n(t)}^4}
		&≤ 16 μ_4 \underbrace{n^2·(s-t)^2}_{≤ 1 ~ \text{(Fall 1)}} (s-t)^2
		≤
		16μ_4 (s-t)^2
	\end{align*}

	\paragraph{Fall 2: $s-t≥\frac1n$}
	Aus \eqref{eqProof7.16Plus} und \eqref{eqCrUngleichung} mit $r=4$ und $m=3$ folgt wegen $\abs[\big]{n s - \floor{n s}} < 1$ und $\abs[\big]{n t - \floor{n t}} < 1$:
	\begin{align} \label{eq:7.16proof2} \tag{$*$}
		\Earg{\abs[\big]{X_n(s)-X_n(t)}^4}
		\overset{}&{≤}
		27 \klammern{\frac1{n^2} \Earg{\abs{\sum_{i = \floor{n t} + 1}^{\floor{n s}} ξ_i}^4}
		+ \underbrace{\frac1{n^2}}_{≤ (s - t)^2} μ_4
		+ \underbrace{\frac1{n^2}}_{≤ (s - t)^2} μ_4}
		\\
		&=
		27 \klammern{\frac1{n^2} \Earg{\abs{\sum_{i = 1 }^{\floor{n s} - \floor{n t}} ξ_{\floor{n t} + 1}}^4}
			+ 2 (s - t)^2 μ_4
		}
	\end{align}

	\begin{lem}[Momentenungleichung]
		Seien $ξ_1,…,ξ_n$ \iid, zentriert mit $μ_2 := \Earg{ξ_1^2}$ und $μ_4 := \Earg{ξ_1^4} < ∞$.
		Dann gilt:
		\begin{align*}
			\Earg{\abs{\sum_{i=1}^nξ_i}^4} = nμ_4 + 3n(n-1)μ_2^2
			≤ 4 μ_4 n^2
		\end{align*}
	\end{lem}

	\begin{proof}[Induktionsbeweis vom zweiten Jahr]
		Sei $S_n := \sum_{i = 1}^{n} ξ_i$. Dann $S_{n + 1} = S_n + ξ_{n + 1}$,
		wobei $S_n$ und $ξ_{n + 1}$ unabhängig sind.
		Zeige die Gleichheit per Induktion nach $n$.

		Der Fall $n = 1$ ist trivial.
		Induktionsschritt:
		\begin{align*}
			\Earg{\abs{S_{n + 1}}^4}
			&= \Earg{(S_n + ξ_{n + 1})^4} \\
			&= \Earg{S_n^4} + 3 \Earg{S_n^3 ξ_{n + 1}}
			+ 6 \Earg{S_n^2 ξ_{n + 1}^2} + 3 \Earg{S_n ξ_{n + 1}^3}
			+ \Earg{ξ_{n + 1}^4} \\
			&= \Earg{S_n^4} + 3 \Earg{S_n^3} \underbrace{\Earg{ξ_{n + 1}}}_{= 0}
			+ 6 \underbrace{\Earg{S_n^2}}_{=n μ_2}  \underbrace{\Earg{ξ_{n + 1}^2}}_{= μ_2}
			+ 3 \underbrace{\Earg{S_n}}_{ = 0} \Earg{ξ_{n + 1}^3}
			+ \underbrace{\Earg{ξ_{n + 1}^4}}_{= μ_4} \\
			\overset{\text{IV}}&{=}
			n μ_4 + 3n(n - 1) μ_2^2 + nμ_2 μ_2 + μ_4
			= (n + 1)μ_4 + 3n(n - 1 + 1) μ_2^2 \\
			&= (n + 1)μ_4 + 3n((n + 1) - 1) μ_2^2
		\end{align*}
		Zur Ungleichung:
		\begin{align*}
			% nμ_4 + 6μ_2^2 \binom n2 = nμ_4 + 6μ_2^2 \frac{n(n - 1)}{2} \\
			\underbrace{n}_{≤ n^2} μ_4
				+ 3 n \underbrace{(n-1)}_{≤ n} \underbrace{μ_2^2}_{=\klammern[\big]{\Earg{ξ_1^2}}^2\overset{\text{Jensen}}{≤} μ_4}
				≤ 4 n^2 μ_4 & \qedhere
	% 	μ_2^2 = (\Earg{ξ_1^2})^2
		% 	\overset{\text{Jensen}}{≤}
		% 	\Earg{ξ_1^4} = μ_4
		% 	⇒ n
		\end{align*}
	\end{proof}
	\begin{proof}[Beweis vom ersten Jahr]

		Zeige zuerst das Gleichheitszeichen:
		\begin{align*}
			\Earg{\abs{\sum_{i=1}^n ξ_i}^4}
			&=\Earg{\klammern{\sum_{i=1}^nξ_i}^4}\\
			&=\Earg{\sum_{1 ≤ i,j,k,l ≤ n} ξ_i·ξ_j·ξ_k·ξ_l} \\
			&=\sum_{1 ≤ i,j,k,l ≤ n} \underbrace{\Earg[\Big]{ξ_i·ξ_j·ξ_k·ξ_l}}_{=: μ_{i,j,k,l}}
		\end{align*}
		Die Tupel $(i,j,k,l)∈\set{1,…,n}^4$ mit mindestens drei verschiedenen Komponenten liefern $μ_{i,j,k,l}=0$,
		denn z.\,B.\ (verschiedene Buchstaben repräsentieren verschiedene Zahlen):
		\begin{align*}
			μ_{i,j,k,j}
			&=\Earg[\Big]{ξ_i·ξ_j·ξ_k·ξ_j}
			=\Earg[\Big]{ξ_i·ξ_j^2·ξ_k}
			=\underbrace{\Earg[\big]{ξ_i}}_{=0} · \Earg[\big]{ξ_i^2} \Earg[\big]{ξ_k}
		\end{align*}
		Folglich reduziert sich die obige auf (!)
		\begin{align*}
			&\sum_{i=1}^4 \underbrace{\Earg[\Big]{ξ_i^4}}_{=μ_4}
			+ 6 \sum_{1 ≤ i ≤ j ≤ n}
			\underbrace{\Earg[\Big]{ξ_i^2 ξ_j^2}}_{=μ_2^2}
			+\underbrace{
				16 \sum_{1 ≤ i ≤ j ≤ n}
					\underbrace{\Earg[\Big]{ξ_i}}_{=0}
					\Earg[\Big]{ξ_j^3}
				+ 4 \sum_{1 ≤ i ≤ j ≤ n}
					\Earg[\Big]{ξ_i^3}
					· \underbrace{\Earg[\Big]{ξ_j}}_{=0}
			}_{=0} \\
			&= nμ_4 + 6μ_2^2 \binom n2 = nμ_4 + 6μ_2^2 \frac{n(n - 1)}{2} \\
			&= \underbrace{n}_{≤ n^2} μ_4
				+ 3 n \underbrace{(n-1)}_{≤ n} \underbrace{μ_2^2}_{=\klammern[\big]{\Earg{ξ_1^2}}^2\overset{\text{Jensen}}{≤} μ_4}\\
			&≤ 4 n^2 μ_4
		\end{align*}
		Siehe auch \cite{ferger2014moment} %\undefine{Turkish Journal of Mathematics, Moment equalities via integer partitions} (2014) von Dietmar Ferger
		für mehr Hintergründe.
	\end{proof}

	Mit diesem Lemma folgt:
	\begin{align*}
		% \frac1{n^2} \Earg{\abs{\sum_{i = \floor{n t}+1}^{\floor{n s}} ξ_i}^4}
		% &=
		\frac1{n^2} \Earg{\abs{\sum_{i=1}^{\floor{n s} - \floor{n t}} ξ_{i + \floor{n t}}}^4}
		%\\
		% &≤ \frac4{n^2} μ_4
		% 	\klammern[\Big]{\underbrace{\floor{n s}-\overbrace{\floor{n t}}^{>n· t-1}}_{≤\underbrace{ n· s-n· t}_{=n·(s-t)}+\underbrace{1}_{\overset{\text{2. Fall}}{≤}n·(s-t)}≤2· n·(s-t)}}^2\\
		% CHECKED: '\Big' used.
		&≤ \frac4{n^2} μ_4 \klammern[\big]{\floor{n s} - \floor{n t}}^2 \\
		&= \frac4{n^2} μ_4 \klammern{n s - n t + (nt - \floor{nt}) - (ns - \floor{ns})}^2 \\
		&≤ \frac4{n^2} μ_4 \klammern{n (s-t) + 1 - 0}^2 \\
		\overset{\text{Fall 2}}&{≤} \frac4{n^2} μ_4 \klammern{2 n (s-t)}^2 \\
		% &= \frac1{n^2} 4μ_4 4 n^2·(s-t)^2\\
		&= 16 μ_4 (s-t)^2\\
		\overset{\eqref{eq:7.16proof2}}{⇒}
		\Earg[\Big]{\abs[\big]{X_n(s) - X_n(t)}^4}
		&≤ 27·18μ_4(s-t)^2 \qquad ∀ s e > t ∈ I \\
		&=\klammern[\Big]{F(s)-F(t)}^2, \text{ wobei } F(s):=√{27·18μ_4} s
	\end{align*}
	Offenbar ist $F$ stetig und streng monoton wachsend auf $I = \intervall0b$.
	Folglich ist \eqref{eqSatz7.11VorM} aus Satz \ref{satz7.11MomentenkriteriumVonKolmogoroff} erfüllt mit $μ=4$ und $α=2>1$.

	Wir haben Donsker \ref{satz7.16Donsker} gezeigt, allerdings unter der \emph{stärkeren} Voraussetzung $\Earg[\big]{ξ_1^4} < ∞$.
	Den allgemeinen Fall $\Earg{ξ_1^2}<∞$ zeigt man mit der sogenannten \undefine{Methode des Stutzens (truncation)},
	vergleich \cite{klenke2006wahrscheinlichkeitstheorie}.
	% vergleiche Achim Klenke (2008) \undefine{Wahrscheinlichkeitstheorie}
\end{proof}

%Ferger: Ich bin mit meinem Leben ja sehr zufrieden. [...] Das anstregendste als Professor ist das Tafelwischen. Ansonsten ist der Job leicht verdientes Geld.

\setcounter{satz}{15}
\begin{bemerkungnr}\label{bemerkung7.16Einhalb} %7.16. Einhalb
%\begin{bemerkung}
	Unser Beweis lässt sich sofort übertragen auf Dreiecksschemata:

	$\set[\big]{ξ_{n,i}:1 ≤ i ≤ n, n ∈ ℕ}$ mit $ξ_{n,1}, …, ξ_{n,n}$ \iid\ $\sim H$ (verteilt nach Verteilungsfunktion $H$),
	wobei die Verteilungsfunktion $H$ nicht von $n$ abhängt und
	$\Earg[\big]{ξ_{n, i}} = 0$ und $\sup_{n ∈ ℕ} \Earg[\big]{ξ_{n,i}^4} < ∞$ und $\Var(ξ_{n,i}) = 1$.

	\cite{prokhorov1956convergence}
	% Prokhorov (1956), \undefine{Convergence of random processes and limit theorems in probability theory, Theory of Probability and its applications 1},
	%Seite 157-214,
	zeigt, dass auch hier die Existenz zweiter Momente $\sup_{n ∈ ℕ} \Earg[\big]{ξ_{n, i}^2} < ∞$ ausreicht.
%\end{bemerkung}
\end{bemerkungnr}

% Das folgende in dieser Datei wurde 2019 nicht mehr erzählt.
\footnote{Zusätzlich im Wintersemester 2018/19:

	Seien $(ξ_i)_{i≥1}$ \iid\ $\sim F$ mit $\Earg[\big]{ξ_1} = μ ∈ ℝ$ und $σ^2 := \Var(ξ_1) ∈ (0,∞)$.
	Dann ist Donsker \ref{satz7.16Donsker} anwendbar auf die \define{standardisierten Zufallsvariablen}
	\begin{align*}
		\tilde{ξ}_i:=\frac{ξ_i-μ}{σ},\qquad∀ i≥1
	\end{align*}
	Beachte: Die Grenzverteilung $W$ in Satz \ref{satz7.16Donsker} (also das Wiener-Maß) hängt \emph{nicht} von $F$ ab.
	Die Grenzverteilung ist also invariant unter $F$. Deshalb heißt Satz \ref{satz7.16Donsker} auch \define{Invarianzprinzip}.
	(Andere Formulierung: \define{Funktionaler Grenzwertsatz})

	Sei $h\colon C\argu[\big]{\intervall0b} ⟶ ℝ$ messbar und $W$-fast überall stetig.
	Dann gilt wegen Satz \ref{satz7.16Donsker} und \ref{satz4.10ContinuousMappingTheorem}:
	\begin{align}\label{eqUnder7.16Eins}\tag{1}
		h(X_n)\distrto  h(B)
	\end{align}
	Kennt man die Verteilung von $h(B)$ (= Funktional der Brownschen Bewegung, dazu existiert umfangreiche Literatur, z.\,B.\ \cite{borodin2012handbook}% Borodin und Salminen (2002),
	%\undefine{Handbook of Brownian motion})
	), so auch die Grenzverteilung von $h(X_n)$. Dies macht man sich in der asymptotischen Statistik zunutze
	(siehe Beispiel später).
	Umgekehrt lässt sich oft die Grenzverteilung $h(W)$ für besonders einfache Verteilungfunktionen $F$ bestimmen.
	\begin{align*}
		h\klammern[\big]{X_n}\distrto Z
		\overset{\eqref{eqUnder7.16Eins}~\&~\ref{lemma4.6Einhalb}}{⇒}
		h(B)=Z
	\end{align*}
	Damit ist der Satz \ref{satz7.16Donsker} von Donsker auch nützlich in der Wahrscheinlichkeitstheorie.
}


