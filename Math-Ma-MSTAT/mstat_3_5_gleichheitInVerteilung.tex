% This work is licensed under the Creative Commons
% Attribution-NonCommercial-ShareAlike 4.0 International License. To view a copy
% of this license, visit http://creativecommons.org/licenses/by-nc-sa/4.0/ or
% send a letter to Creative Commons, PO Box 1866, Mountain View, CA 94042, USA.

\subsection{Gleichheit in Verteilung}
\begin{definition} \label{definition: 3.16}
	Seien $X_i$ ($i = 1,2$) Zufallsvariablen in $(\S,d)$ über $(\Omega_i,\A_i,\P_i)$. Sie heißen \begriff{gleich in Verteilung}, in Zeichen
	\begin{align*}
		X_1 \disteq X_2 \defequiv \P_1 \circ X_1^{-1} = \P_2 \circ X_2^{-1}
	\end{align*}
\end{definition}

Wir wollen Verteilungsgleichheit in folgendem Satz charakterisieren.

\begin{satz} \label{satz: 3.17}
	Es gilt:
	\begin{enumerate}[label=(\alph*)]
		\item \label{it: 3.17f} Seien $\P,\Q$ Wahrscheinlichkeitsmaße auf $\B(\S)$. Dann gilt:
		\begin{equation*}
			\P \equiv \Q
			\equivalent
			\int f \diffskip\P = \int f \diffskip\Q \qquad \forall f \in C^b(\S) \text{ glm.\ stetig}
		\end{equation*}
		\item \label{it: 3.17E} Es gilt
		\begin{equation*}
			X \disteq Y \equivalent
			\EW[f(X)] = \int_{\Omega_1} f(X_1)  \diffskip{\P_1} = \int_{\Omega_2} f(X_2)  \diffskip{\P_2} = \EW[f(Y)]
		\end{equation*}
		für all $f \in C^b(\S)$ gleichmäßig stetig.
	\end{enumerate}
\end{satz}

\begin{proof}
	\begin{enumerate}[label=(zu \alph*), leftmargin=*]
		\item \begin{description}
			\hinrichtung Klar.
			\rueckrichtung Aus \cref{lemma: 3.2}, \cref{it: 3.2-OpenClosed} wissen wir, dass $\B(\S) = \sigma\brackets{\mathcal{F}(\S)}$ und $\F$ ist durchschnittsstabil. Wegen des \textit{Maßeindeutigkeitssatz} reicht es zu zeigen, dass $\P(F) = \Q(F)$ für alle $F \in \mathcal{F}(S)$. Sei nun $F \subseteq \S$ abgeschlossen. Setze $f_k (x) \defeq \phi \brackets{k * d(x,F)}$ (vgl.\ \cref{satz: 2.4}, $\phi$ wie dort). Aus \cref{lemma: 2.3} folgt, dass die $f_k$ beschränkt und gleichmäßig stetig sind mit $f_k \searrow \one_F$ für $k \to \infty$. Also gilt:
			\begin{align*}
				\P(F) 
				&= \int \one_F \diffskip{\P}
				= \int \lim_{k\to\infty} f_k  \diffskip\P
				\intertext{und mit monotoner Konvergenz}
				&= \lim_{k \to \infty} \int f_k \diffskip\P
				= \lim_{k \to \infty} \int f_k \diffskip\Q
				\intertext{Wiederum mit monotoner Konvergenz}
				&= \int \lim_{k \to \infty} f_k \diffskip\Q
				= \int \one_F \diffskip\Q
				= \Q(F)
			\end{align*}
			Da $F$ beliebig war, folgt die Behauptung.
		\end{description}
		\item Dies folgt aus \cref{it: 3.17f} mit dem Transformationssatz \eqref{eqTrafo}:
			\begin{align*}
			X \disteq Y
			&\overset{\ref{definition: 3.16}}{\Leftrightarrow}
			\P \circ X^{-1} = \P \circ Y^{-1} \\
			&\overset{\ref{it: 3.17f}}{\Leftrightarrow}
			\int_{\S} f \diffskip{(\P\circ X^{-1})} = \int_{\S} f \diffskip{(\P\circ Y^{-1})}
			&\forall f \in C^b(\S) \text{ glm. stetig} \\
			\overset{\eqref{eq: Proof3.17}}&{\Leftrightarrow}
			\EW[f(X)] = \EW[f(Y)]
			&\forall f \in C^b(\S) \text{ glm. stetig}
		\end{align*}
		Verwende dafür:
		\begin{equation}
			\int_{\S} f \diffskip{(\P\circ X^{-1})}
			\overset{\text{Trafo}}{=}
			\int_{\Omega} \underbrace{f \circ X}_{\defqe f(X)} \diffskip\P
			\overset{\text{Def}}{=}
			\EW[f(X)]
			\tag{$\ast$} \label{eq: Proof3.17} 
		\end{equation}
	\end{enumerate}
\end{proof}
