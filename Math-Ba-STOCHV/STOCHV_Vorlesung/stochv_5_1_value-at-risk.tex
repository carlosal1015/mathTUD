\section{Value-at-Risk}

\begin{*definition}
	Der Value-at-Risk von $L$ mit Verteilungsfunktion $F_L$ zum Level $\alpha \in (0,1)$ ist gegeben durch
	\begin{equation*}
		\VaR_\alpha(L) \defeq \inf \menge{\ell \in \R: F_L(\ell) \ge \alpha} = \inf \menge{\ell \in \R: \P(L > l) \le 1 - \alpha}
	\end{equation*}
\end{*definition}

\begin{itemize}[nolistsep]
	\item In Worten: $\VaR_\alpha(L)$ ist die kleinste Zahl $\ell$ mit der Eigenschaft, dass die Wahrscheinlichkeit einen Verlust $L$ zu erleiden, der $\ell$ übersteigt, kleiner als $1-\alpha$ ist.
	\item Value-at-Risk entspricht genau dem $\alpha$-Quantil $q_\alpha(L)$ von $F_L$.
	\item Typische Werte für $\alpha$ sind beispielsweise $0.95$ oder $0.99$.
\end{itemize}

Die Quantilfunktion
\begin{equation*}
	q_\alpha(L) = \inf \menge{\ell \in \R: F_L(\ell) \ge \alpha}
	\tag{$\star$} \label{eq: quantil-star}
\end{equation*}
wird auch als verallgemeinerte Inverse von $F_L$ bezeichnet.

\textbf{Skizze:} Dichte von $L$, Verteilung von $L$
\begin{center}
	%TODO BILD
	\includegraphics[width=.5\textwidth]{example-image}
	\captionof{figure}{Value-at-Risk}
\end{center}

\begin{lemma}
	\begin{enumerate}[label=(\alph*)]
		\item Wenn $F_L$ stetig und streng monoton ist, dann gilt $q_\alpha(L) = F_L^{-1}(\alpha)$ für alle $\alpha \in (0,1)$.
		\item Wenn $F_L$ eine Dichte $f_L$ besitzt, dann gilt $\int_{-\infty}^{q_\alpha(L)} f_L(u) \diff{u} = \alpha$ für alle $\alpha \in (0,1)$.
		\item Ein Wert $x_0 \in \R$ ist das $\alpha$-Quantil einer Verteilung genau dann, wenn gilt
		\begin{equation*}
			F_L(x_0) \ge \alpha \und F_L(x) < \alpha \text{ für alle } x < x_0
		\end{equation*} 
	\end{enumerate}
\end{lemma}

\begin{*bemerkung}[Problematische Fälle]
	\begin{itemize}
		\item $F_L$ nicht stetig (Sprung über Schwelle $\alpha$). In diesem Fall ist $F_L(q_\alpha(L)) \neq \alpha$.
		\item $F_L$ ist nicht streng monoton (flache Stelle). Dann hat $F_L(x_0) = \alpha$ mehrere Lösungen, d.h. $F_L^{-1}$ ist nicht wohldefiniert.
	\end{itemize}
\end{*bemerkung}

\begin{proof}
	Sei $F_L$ stetig, dann ist $\abb{F_L}{\R}{(0,1)}$ surjektiv und das Infimum in \eqref{eq: quantil-star} wird angenommen, d.h. $F_L(q_\alpha(L)) = \alpha$.
	\begin{enumerate}[label=(zu \alph*), leftmargin=*, nolistsep]
		\item $F_L$ ist streng monoton, d.h. auch injektiv und somit mit der Eingangsfeststellung bijektiv. Somit existiert $F_L^{-1}$ und es gilt $q_\alpha(L) = F_L^{-1}(\alpha)$.
		\item $F_L$ besitze eine Dichte $f_L$. Dann ist $F_L(x) = \int_{-\infty}^x f_L(u) \diff{u}$ stetig. Mit $x = q_\alpha(L)$ gilt $\alpha = \int_{-\infty}^{q_L(\alpha)} f_L(u) \diff{u}$.
		\item entspricht der Definition des Infimums als größte untere Schranke
	\end{enumerate}
\end{proof}

\begin{*beispiel}
	\begin{enumerate}[leftmargin=*]
		\item Sei $L = \Normal(\mu, \sigma^2)$, d.h. $F_L(x) = \Phi\brackets{\frac{x-\mu}{\sigma}}$. 
		\begin{equation*}
			\alpha = F_L(x_0) \follows \alpha = \Phi\brackets{\frac{x_0 - \mu}{\sigma}} \follows\Phi^{-1}(\alpha) = \frac{x_0 - \mu}{\sigma} \follows x_0 = \mu + \sigma \Phi^{-1}(\alpha)
		\end{equation*}
		d.h. $\VaR_\alpha(L) = \mu + \sigma \Phi^{-1}(\alpha)$.
		\item Verallgemeinerung: Sei $L \sim F_{\mu, \sigma}$ mit $F_{\mu, \sigma}(x) = F(\frac{x-\mu}{\sigma})$ und $F$ ist stetig und streng monoton
		(\enquote{Lage-Skalen-Familie} von Verteilungen). Dann gilt
		\begin{equation*}
			\VaR_\alpha(L) = \mu + \sigma F^{-1}(\alpha)
		\end{equation*}
		Aber im Allgemeinen muss $\mu$ nicht dem Erwartungswert von $L$ entsprechen und $\sigma^2$ nicht der Varianz.
		\item Betrachte die Student-t-Verteilung mit $\nu > 0$ Freiheitsgraden und Dichte
		\begin{equation*}
			f(x) = c_\nu * \brackets{1 + \frac{x^2}{\nu}}^{-\brackets{\frac{\nu+1}{2}}}
		\end{equation*}
		Es sei $L \sim F_{\mu, \sigma}$ mit $F_{\mu, \sigma}(x) = F\brackets{\frac{x-\mu}{\sigma}}$ als Verteilungsfunktion von $L$.
		Es gilt: 
		\begin{equation*}
			\EW[L] = \begin{cases}
			\text{n. def.} & \text{für } \nu \in [0,1] \\ \mu &\text{für } \nu > 1
			\end{cases} \quad \und \quad
			\Var[L] = \begin{cases}
			\text{n. def.} & \text{für } \nu \in [0,2] \\ \sigma \sqrt{\frac{\nu}{\nu-1}} &\text{für } \nu > 2
			\end{cases}
		\end{equation*}
		\item Sei $L = \sigma B_t$ mit einer Brownschen Bewegung $B$ (d.h. $B_t \sim \Normal(0,t)$). Es gilt $\VaR_\alpha(L) = \sigma \sqrt{t} * \Phi^{-1}(\alpha)$.
		Intuition: $B_t$ beschreibt die Summe viele kleiner Verluste über ein Zeitintervall $[0,t]$ $\to \VaR$ skaliert mit $\sqrt{t}$ (\enquote{square-root-law})
	\end{enumerate}
\end{*beispiel}

\textbf{Vorteile des $\VaR$:}
\begin{itemize}[nolistsep, topsep=-\parskip]
	\item für jede Verteilung $F_L$ definiert
	\item einfach zu berechnen
	\item einfach zu verstehen / zu erklären
\end{itemize}

\textbf{Nachteile des $\VaR$:}
% TODO Skizze
\begin{itemize}[nolistsep, topsep=-\parskip]
	\item berücksichtigt \textit{Wahrscheinlichkeit} hoher Verluste, aber nicht deren \textit{Ausmaß}
	\item nicht subadditiv, d.h. bestraft unter Umständen Risikostreuung bzw. Diversifikation (mehr dazu im Kapitel 5.3)
\end{itemize}

