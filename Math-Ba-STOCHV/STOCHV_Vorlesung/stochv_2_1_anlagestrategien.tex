\section{Anlagestrategien im CRR-Modell}

Eine Anlagestrategie soll durch ein Anfangskapital $w$ und einen stochastischen Prozess $\folge{\eta_n, \xi_n}{n \in [N]}$ dargestellt werden. 
\begin{itemize}
	\item Dabei steht $\xi_n$ für die ''Anzahl`` der Wertpapiere, die im Zeitintervall $(n-1,n]$ gehalten werden. Negative Werte von $\xi_n$ sind sogenannte Leerverkäufe. Wir erlauben beliebige Anteile $\xi_n \in \R$.
	\item Weiter beschreibt $\eta_n$ den Stand des Verrechnungskontos in Geldeinheiten zum Zeitpunkt Null. Negative Werte von $\eta_n$ entsprechend einer Kreditaufnahme.
	Außerdem sei das Anfangskapital $w \in \R$.
\end{itemize}

\textbf{\underline{Gesamtwert des Portfolios}}

\begin{equation*}
\begin{aligned}
\Pi_n &= \eta_n * S_n^0 + \xi_n * S_n \\
\Pi_0 &= w
\end{aligned} \tag{Port} \label{eq: port}
\end{equation*}
Man nennt \eqref{eq: port} die \begriff{Portfoliogleichung}.

Annahmen an die Strategie:
\begin{itemize}
	\item Die Strategie darf nur von Beobachtungen der Vergangenheit abhängen, d.h. $(\eta_n, \xi_n)$ ist ein \textit{vorhersehbarer} Prozess.
	\item Portfolio wird nur zwischen $S$ und $S_0$ umgeschichtet, d.h. es wird kein Kapital zugeschossen oder abgezogen:
	
	\item $\Pi_n = \eta_n S_n^0 + \xi_n S_n$ Wert zum Zeitpunkt $n$ vor dem Umschichten
	\item $\Pi_n = \eta_{n+1} S_n^0 + \xi_{n+1}S_n$ Wert zum Zeitpunkt $n$ nach dem Umschichten
	\item $\Pi_{n+1} = \eta_{n+1} S_{n+1}^0 + \xi_{n+1} S_{n+1}$ Wert zum Zeitpunkt $n+1$ vor dem Umschichten.
	
	\item Subtrahiere Gleichung 2 und 3: 
	\begin{equation}
		\begin{aligned}
			\follows \Pi_{n+1} - \Pi_n &= \eta_{n+1} \brackets{S_{n+1}^0 - S_n^0} + \xi_{n+1} \brackets{S_{n+1} - S_n} \\
			\text{kurz: } \Delta \Pi_n &= \eta_n \Delta S_n^0 + \xi_m \Delta S_n
		\end{aligned}
		\tag{SF} \label{eq: sf}
	\end{equation}
	Wir bezeichnen \eqref{eq: sf} als \begriff{Selbstfinanzierungseigenschaft}. Beachte: $\Delta \Pi_n \defeq \Pi_n - \Pi_{n-1}$
	
	Diskontieren:
	\begin{equation}
		\schlange{\Pi}_n = \frac{\Pi_n}{S_n^0} 
		\tag{diskontierte Portfoliogleichung}
	\end{equation}
	Wir erhalten aus \eqref{eq: port}
	\begin{equation}
		\begin{aligned}
		\schlange{\Pi}_n &= \eta_n + \xi_n \schlange{S}_n
		\schlange{\Pi}_0 &= w
		\end{aligned}
		\tag{$\schlange{\text{Port}}$}
	\end{equation}
	
	Aus \eqref{eq: sf} erhalten wir
	\begin{equation}
		\Delta \schlange{\Pi}_n = \xi_n \Delta \schlange{S}_n
		\tag{$\schlange{\text{SF}}$}
	\end{equation}
	d.h.
	\begin{align}
			\schlange{\Pi}_n  &= w + \sum_{k=1}^n \brackets{\schlange\Pi_k - \schlange\Pi_{k-1} }
			\overset{\eqref{eq: sf}}{=} w + \sum_{k=1}^n \xi_k \brackets{\schlange S_k - \schlange S_{k-1}} 
			= w + \brackets{\xi \bullet \schlange{S}}_n \nonumber \\
			\schlange\Pi_n &= w + \brackets{\xi \bullet \schlange{S}}_n
			\tag{$\schlange{\text{Int}}$}
	\end{align}
\end{itemize}

\begin{lemma} %2.2
	Eine selbstfinanzierende Anlagestrategie $\folge{\eta_n, \xi_n}{n \in \N}$ mit Anfangskapital $w \in \R$ und ihr Werteprozess $\Pi_n$ sind durch $w$ und $\folge{\xi_n}{n \in \N}$ vollständig definiert. 
	\begin{enumerate}[label=(\alph*), leftmargin=*, nolistsep, topsep=-\parskip]
		\item Der diskontierte Werteprozess lässt sich darstellen als
		\begin{equation*}
		\schlange{\Pi}_n = w + \sum_{k=1}^n \xi_k (\schlange{S}_k - \schlange{S}_{k-1}) = w + (\xi \bullet \schlange{S})_n
		\end{equation*}
		\item Der Anteil $\eta_n$ ist eindeutig gegeben durch
		\begin{equation*}
		\eta_n = \schlange{\Pi}_n - \xi_n \schlange{S}_n
		\end{equation*}
	\end{enumerate}
\end{lemma}
\begin{proof}
	siehe Herleitung oben
\end{proof}
