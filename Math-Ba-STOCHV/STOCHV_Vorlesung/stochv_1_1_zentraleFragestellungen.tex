\section{Zentrale Fragestellungen der Finanzmathematik}

\subsection{Bewertung von Derivaten und Absicherung gegen aus deren Kauf/Verkauf entstehende Risiken}

\begin{definition}[Derivat]
	Ein \begriff{Derivat} ist ein Finanzprodukt, dessen Auszahlung sich vom Preis eines oder mehrerer Basisgüter [underlying] ableitet.
\end{definition}

\begin{beispiel}
	\begin{itemize}[nolistsep, leftmargin=*, topsep=-\parskip]
		\item Recht in drei Monaten 100.000 GBP gegen 125.000 EUR zu erhalten (Call-Option; underlying: Wechselkurs GBP in EUR)
		\item Recht innerhalb des nächsten Jahres 100.000 MWh elektrische Energie zum Preis von 30 EUR/MWh zu konsumieren mit Mindestabnahme 50.000 MWh (Swing-Option; underlying: Strompreis)
		\item Kauf- und Verkaufsoptionen auf Aktien (underlying: Aktienkurs)
	\end{itemize}
\end{beispiel}

\textbf{Fragestellungen:}
\begin{itemize}[nolistsep, label=--, topsep=-\parskip]
	\item Was ist der ''faire`` Preis für solch ein Derivat? (''Pricing`` / Bewertung)
	\item Wie kann sich der Verkäufer gegen die eingegangenen Risiken absichern? (''Hedging`` / Absicherung)
\end{itemize}

\subsection{Optimale Investition: Zusammenstellen von nach Risiko-/ Ertragsgesichtspunkten optimalen Portfolios}

\begin{itemize}[nolistsep, leftmargin=*, topsep=-\parskip]
	\item Wie wäge ich Risiko gegen Ertrag ab?
	\item Was bedeutet optimal?
	\item Lösung des resultierenden Optimierungsproblems
\end{itemize}

\subsection{Risikomanagement und Risikomessung}

gesetzliche Vorschriften (Basel und Solvency) sollen Stabilität des Banken- und Versicherungssystems angesichts verschiedener Risiken sicherstellen \\
$\to$ mathematische Theorie der konvexen und kohärenten Risikomaße

\textbf{Mathematische Werkzeuge:} Wahrscheinlichkeitstheorie und stochastische Prozesse (Dynamik in der Zeit), zusätzlich etwas lineare Algebra, Optimierung, Maßtheorie