\section{Martingalmaß und Arbitrage im CRR-Modell}

\textbf{bisher:} CRR-Modell auf einem Wahrscheinlichkeitsraum $(\Omega, \F, \P)$ \\
\textbf{jetzt:} weiteres Wahrscheinlichkeitsmaß $\Q$ auf $(\Omega, \F)$.

Das bedeutet nun, dass wir die Baumstruktur unverändert lassen, jedoch die Über"-gangs"-wahrscheinlich"-keiten ändern von $p = \P(R_n = b)$ zu $q = \Q(R_n = b)$. Dabei bleibt die Unabhängigkeit der $R_n$ erhalten.

\textbf{Notation:} Wir bezeichnen mit $\E^\Q[\ \cdot \ ]$ den Erwartungswert unter $\Q$.

\begin{*definition}
	Ein Wahrscheinlichkeitsmaß $\Q$ auf $(\Omega, \F)$ heißt äquivalentes Martingalmaß (EMM) für das CRR-Modell, wenn gilt
	\begin{enumerate}
		\item $\Q \sim \P$ ($\Q$ äquivalent zu $\P$)
		\item diskontierter Preisprozess $\folge{\schlange{S}_n}{n \in [N]}$ ist ein $\Q$-Martingal, d.h.
		\begin{equation*}
		\E^\Q \sqbrackets{\schlange{S}_{n+1} \mid \F_n} = \schlange{S}_n \qquad \forall n \in [N-1]_0
		\end{equation*}
	\end{enumerate}
\end{*definition}

\begin{*bemerkung}
	Seien $\P, \Q$ Wahrscheinlichkeitsmaße auf $(\Omega, \F)$.
	\begin{itemize}
		\item Äquivalenz: $\P \sim \Q \defequiv \brackets{\P(A) = 0 \equivalent \Q(A) = 0 \quad \forall A \in \F}$
		\item Absolutstetigkeit: $\Q \ll \P  \defequiv \brackets{\P(A) = 0 \follows \Q(A) = 0 \quad \forall A \in \F}$
		\item Es gilt $\Q \sim \P \equivalent \brackets{\Q \ll \P \land \P \ll \Q}$
	\end{itemize}
\end{*bemerkung}

\begin{theorem}
	\begin{enumerate}[label=(\alph*)]
		\item Im CRR-Modell existiert ein EMM genau dann, wenn $a < r < b$ gilt.
		\item Das EMM $\Q$ ist eindeutig und es gilt
		\begin{align*}
			q = \Q(R_n = b) = \frac{r-a}{b-a} \quad \und \quad 1-q = \Q(R_n = a) = \frac{b-r}{b-a} \qquad \forall n \in [N]
		\end{align*}
	\end{enumerate}
\end{theorem}

\begin{*bemerkung}
	$q$ und $1-q$ sind genau die risikoneutralen Gewichte, die in \eqref{eq: 2_2_rek} auftauchen.
\end{*bemerkung}

\begin{proof}
	Sei $\Q$ ein beliebiges Wahrscheinlichkeitsmaß auf $(\Omega, \F)$. Setze $q_n \defeq \Q(R_n = b \mid \F_{n-1}) \in [0,1]$.
	\begin{equation*}
		\begin{aligned}
			\E^\Q\sqbrackets{\schlange{S}_n \mid \F_{n-1}} &= \E^\Q\sqbrackets{\schlange{S}_{n-1} * \brackets{\frac{1+R_n}{1+r}}  \mid \F_{n-1}} \\
			&= \schlange{S}_{n-1} * \frac{1}{1+r} * \E^\Q\sqbrackets{1+R_n \mid \F_{n-1}} \\
			&= \schlange{S}_{n-1} * \frac{1}{1+r} \brackets{q_n (1+b) + (1-q_n)(1+a)}
		\end{aligned}
	\end{equation*}
	Somit gilt 
	\begin{equation*}
		\begin{alignedat}{2}
			\folge{\schlange{S}_n}{n \in [N]} \text{ ist } \Q \text{-Martingal} &\equivalent &\frac{1}{1+r} \brackets{q_n (1+b) + (1-q_n)(1+a)} &= 1 \\
			&\equivalent &q_n b + (1-q_n) a &= r \\
			&\equivalent &q_n (b-a) &= r-a \\
			&\equivalent &q_n &= \frac{r-a}{b-a}
		\end{alignedat}
	\end{equation*}
	Es gilt $q_n \in [0,1]$ genau dann, wenn $a \le r \le b$. \\ 
	$\Q \sim \P$: $q_n \in (0,1) \equivalent a < r < b$.
	Somit folgt schließlich: $\Q$ ist EMM $\equivalent a < r < b$.
\end{proof}

\begin{theorem}[Risikoneutrale Bewertungsformel]
	\label{theorem: 2.7}
	Sei $C= h(S_1, \dots, S_N)$ ein Derivat im CRR-Modell mit EMM $\Q$. Für den Preisprozess $\folge{\Pi_n}{n \in [N]}$ von $C$ gilt
	\begin{equation*}
		\Pi_n = (1+r)^{-(N-n)} * \E^\Q\sqbrackets{C \mid \F_n}
	\end{equation*}
	Insbesondere gilt
	\begin{equation*}
		w = \Pi_0 = (1+r)^{-N} * \E^\Q[C]
	\end{equation*}
	In Worten: Der faire Preis von $C$ ist eindeutig und gegeben durch den diskontierten Erwartungswert von $C$ unter dem Martingalmaß $\Q$.
\end{theorem}
\begin{proof}
	Der Wahrscheinlichkeitsraum für das CRR-Modell ist endlich, d.h. $\card{\Omega} = 2^N < \infty$. Somit ist jede Zufallsvariable beschränkt, inbesondere $C$ und $\folge{\xi_n}{n \in [N]}$. Sei $\folge{\xi_n}{n \in [N]}$ eine Replikationsstrategie für $C$ mit diskontiertem Werteprozess $\folge{\schlange{\Pi}_n}{}$, d.h.
	\begin{equation*}
		\begin{aligned}
			\schlange{\Pi}_n &= w + \sum_{k=1}^n \xi_k \brackets{\schlange{S}_k - \schlange{S}_{k-1}} = w + (\xi \bullet \schlange{S})_n \\
			\schlange{\Pi}_N &= (1+r)^{-N} C
		\end{aligned}
	\end{equation*}
	$\Q$ ist EMM, also ist $\folge{\schlange{S}_n}{}$ ein $\Q$-Martingal. Mit Theorem 1.6 ist demnach auch $\folge{\xi \bullet \schlange{S}}{n}$ ein $\Q$-Martingal, also ist ebenso $\schlange{\Pi}_n$ ein $\Q$-Martingal.
	\begin{equation*}
		\Pi_n = (1+r)^{n} * \schlange{\Pi}_n = (1+r)^n * \E^\Q \sqbrackets{\schlange{\Pi}_N \mid \F_n} = (1+r)^{-(N-n)} * \E^\Q\sqbrackets{C \mid \F_n}
	\end{equation*}
\end{proof}

\begin{*bemerkung}[Martingalbedingung für $\Q$]
	Wir schreiben (etwas umständlich) $q_b = \Q(R_n = b)$ und $q_a = \Q(R_n = a)$. Da $\Q$ ein Wahrscheinlichkeitsmaß sein soll, muss $q_a + q_b = 1$, oder äquivalent dazu $q_b (1+r) + q_a (1+r) = 1+r$ gelten. Die Martingalbedingung liefert $(1+b) q_b + (1+a) q_a = 1+r$. Beide Gleichungen zusammen sind äquivalent zu $q_b (b-r) + q_a (a-r) = 0$. Als lineares Gleichungssystem geschrieben ergibt sich als
	\begin{equation*}
		\begin{pmatrix}
			1+r & 1+r \\
			b-r & a-r
		\end{pmatrix}
		* 
		\begin{pmatrix}
			q_b \\ q_a
		\end{pmatrix}
		=
		\begin{pmatrix}
		1+r \\ 0
		\end{pmatrix}
		\tag{LGS-2}
		\label{eq: 2_3_lgs-2}
	\end{equation*}
	Vergleicht man dies mit \eqref{eq: 2_2_lgs-1}, dann stellt man fest, dass die Matrizen gleiche Einträge haben, aber transponiert zueinander sind. 
	\eqref{eq: 2_2_lgs-1} ist also eine Bedingung für die Replikationsstrategie, \eqref{eq: 2_3_lgs-2} dagegen eine Bedingung für äquivalente Martingalmaße.
	Multinomial würden Spalten hinzukommen, also wäre das LGS unterbestimmt und es gäbe mehrere EMM.
\end{*bemerkung}

\begin{*bemerkung}
	Das äquivalente Martingalmaß ist genau dann eindeutig, wenn ein vollständiges Modell vorliegt.
\end{*bemerkung}

\subsection{Arbitrage im CRR-Modell}

\begin{*definition}
	Eine Anlagestrategie $\folge{\xi_n}{n \in [N]}$ mit Zeithorizont $N$ und diskretem Werteprozess $\folge{\schlange{\Pi}_n}{n \in [N]}$ heißt Arbitrage, wenn gilt
	\begin{equation*}
		\begin{aligned}
			\schlange{\Pi}_0 &= 0 	&&\text{(kein Anfangskapital)} \\
			\P(\schlange{\Pi}_N \ge 0) &= 1 &&\text{(kein Verlustrisiko)} \\
			\P(\schlange{\Pi}_N > 0) &> 0 &&\text{(positiver Gewinn mit positiver Wahrscheinlichkeit)}
		\end{aligned}
		\tag{Arb} \label{eq: Arb}
	\end{equation*}
\end{*definition}

\begin{theorem}
	Im CRR-Modell sind äquivalent:
	\begin{enumerate}
		\item Es existiert keine Arbeitrage (NA = ''No Arbitrage``).
		\item Es existiert ein EMM $\Q$.
	\end{enumerate}
\end{theorem}

\begin{*bemerkung}
	Dieser Satz gilt im Wesentlichen in allen Finanzmarktmodellen und heißt dann auch ''Erster Hauptsatz der Preistheorie``.
\end{*bemerkung}

\begin{proof}
	\begin{equivalence}
		\rueckrichtung via Widerspruch. Sei $\Q$ ein EMM und es existiere eine Arbitrage $\folge{\xi_n}{}$. Wegen $\Q \sim \P$ folgt \eqref{eq: Arb}:
		\begin{equation*}
			\begin{aligned}
				\Q(\schlange{\Pi}_N \ge 0) = 1 \\
				\Q(\schlange{\Pi}_N > 0) > 0
			\end{aligned}
		\end{equation*}
		Aus beiden Bedingungen folgt 
		\begin{equation*}
			\E^\Q\sqbrackets{\schlange{\Pi}_N} > 0
			\tag{$\star$} \label{eq: thm2.8-star}
		\end{equation*}
		Andererseits gilt $\schlange{\Pi}_N = 0 + \folge{\xi \bullet \schlange{S}}{N}$. Da $\schlange{S}$ ein $\Q$-Martingal ist, ist auch $(\xi \bullet S)$ ein $\Q$-Martingal. Dann gilt $\E^\Q \sqbrackets{\schlange{\Pi}_N} = \E^\Q\sqbrackets{\folge{\xi \bullet \schlange{S}}{N}} = 0$ im Widerspruch zu \eqref{eq: thm2.8-star}.
	\end{equivalence}
\end{proof}