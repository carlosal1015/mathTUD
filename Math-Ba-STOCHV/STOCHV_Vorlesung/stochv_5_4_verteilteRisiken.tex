\section{Verteilte Risiken}

\begin{description}
	\item[Hierarchischer Aufbau]
	% TODO Skizze mit Banken
	
	\begin{itemize}[nolistsep]
		\item Risiken werden im Allgemeinen auf mehrere Ebenen gemessen
		\item Risikokennzahlen müssen konsistent zwischen Ebenen übertragen werden
	\end{itemize}
	
	\item[Schematisch]
	\begin{itemize}[nolistsep]
		\item Gesamtverlust $L$
		\item Einzelverluste $L_1 + \cdots + L_n = L$
		\item Risikomaß $\rho$
	\end{itemize}
\end{description}

% TODO BILD schematisch

\subsection{Aggregation von Risiken}
\begin{description}
	\item[Economic Capital] Mit $EC_i$ bezeichnen wir das \enquote{Economic Capital}, d.h. die Kapitalreserve zur Absicherung von Risiko $L_i$. Diese wird aus dem Risikomaß $\rho(L_i)$ ermittelt.
	
	\item[Gegeben] seien $EC_1, \dots, EC_n$. 
	\item[Gesucht] ist nun $EC$.
	
	\item[Ziel:] Wir wollen die naive Summationsregel $EC = EC_1 + \dots + EC_n$ verbessern.
	
	\item[Annahme:] 
	\begin{itemize}[nolistsep]
		\item $\rho$ sei monetär, positiv homogen und verteilungsinvariant
		\item $(L_1, \dots, L_n) \sim \Normal(\mu, \Sigma)$
		\item $\sigma_i^2 = \Var[L_i]$ und $\sigma_L^2 = \Var[L]$
		\item $\rho_{ij} = \Corr{L_i}{L_j}$
	\end{itemize}
\end{description}


Wir setzen $EC_i = \rho(L_i) - \EW[L_i]$ und $EC = \rho(L) - \EW[L]$.

Sei $r \defeq \rho(Z)$ mit $Z \sim \Normal(0,1)$. Es gilt
\begin{equation*}
	\begin{aligned}
		EC_i &= \rho(L_i) - \EW[L_i] = \sigma_i * \rho\underbrace{\brackets{\frac{L_i - \mu_i}{\sigma_i}}}_{\sim \Normal(0,1)} + \mu_i - \mu_i = \sigma_i * r \\
		EC &= \rho(L) - \EW[L] = \sigma_L * \rho \brackets{\frac{L - \mu_L}{\sigma_L}} + \mu_L - \mu_L = \sigma_L * r \\
		&= \sqrt{\trans{\one} \Sigma \one} * r = \sqrt{\sum_{i,j = 1}^{n} \sigma_i \sigma_j \rho_{ij}} * r = \sqrt{\sum_{i,j = 1}^n \frac{EC_i * EC_j}{r^2} * \rho_{ij}} * r \\
		&= \sqrt{\sum_{i=1}^n EC_i * EC_j * \rho_{ij}}
	\end{aligned}
\end{equation*}

Somit erhalten wir die \enquote{Correlation-Adjusted Summation Rule}
\begin{equation*}
	EC = \sqrt{\sum_{i,j = 1}^n EC_i * EC_j * \Corr{L_i}{L_j}}
\end{equation*}

\begin{itemize}[nolistsep]
	\item bezieht Abhängigkeiten der Risiken und Diversifikationseffekte mit ein
	\item gilt auch in der Klasse der \enquote{elliptischen Verteilungen}
\end{itemize}



\subsection{Das Allokationsproblem}
\begin{description}
	\item[Gegeben] sei das gesamte Risikokapital $\rho(L)$.
	\item[Gesucht] ist das \enquote{Allocated Capital} $AC_i$ für den Verlust $L_i$ unter de Nebenbedingung $AC_1 + \dots + AC_n = \rho(L)$ (\enquote{volle Allokation})
	\item[Annahme:] $\rho$ positiv homogen
\end{description}

Als Hilfsmittel verwenden wir den Satz von \person{Euler} über homogene Funktionen.

\begin{*definition}
	Sei $X \subseteq \Rn$ nichtleer und $\abb{f}{X}{\R}$. Die Funktion $f$ heißt homogen vom Grad $k \in \R\setminus\menge{0}$, wenn
	\begin{equation*}
		f(\lambda x) = \lambda^k f(x) \text{ für alle } x \in X \und \lambda > 0  \mit \lambda x \in X
		\tag{H} \label{eq: h}
	\end{equation*}
\end{*definition}

\begin{theorem}[Satz von Euler über homogene Funktionen]
	Sei $\abb{f}{[0,\infty)^n}{\R}$ stetig auf $X$ und differenzierbar auf $X^\circ$. Die Funktion $f$ ist homogen vom Grad $k \in \R\setminus\menge{0}$ genau dann, wenn
	\begin{equation*}
		k * f(x) = \sum_{i = 1}^n \partdiff{x_i} f(x) * x_i \quad \forall x \in X^\circ
		\tag{EU} \label{eq: eu}
	\end{equation*}
\end{theorem}
\begin{proof_equiv}
	\hinrichtung Sei $f$ homogen vom Grad $k$. Fixiere $x \in X^\circ$ und definiere
	\begin{equation*}
		g_x(\lambda) \defeq f(\lambda x) - \lambda^k f(x) \qquad (\lambda > 0)
	\end{equation*}
	Ableiten liefert
	\begin{equation*}
		g_x'(\lambda) = \sum_{i = 1}^n \partdiff{x_i} f(\lambda x) * x_i - k \lambda^{k-1} f(x)
	\end{equation*}
	Da $f$ homogen ist, ist $g_x(\lambda) = 0$ für alle $\lambda > 0$ und insbesondere ist damit auch $g_x'(\lambda) = 0$ für alle $\lambda > 0$. Dies gilt folglich auch für $\lambda = 1$, d.h. $g_x'(1) = 0$, woraus schließlich \eqref{eq: eu} folgt.
	\rueckrichtung Es gelte \eqref{eq: eu}. Definiere wie oben $g_x(\lambda) \defeq f(\lambda x) - \lambda^k f(x)$ und es gilt $g_x(1) = f(x) - f(x)$. Leite analog ab, d.h. 
	\begin{equation*}
		g_x'(\lambda) = \sum_{i = 1}^n \partdiff{x_i} f(\lambda x) * x_i - k \lambda^{k-1} f(x) \overset{\eqref{eq: eu}}{=} \frac{1}{\lambda} k f(\lambda x) - k \lambda^{k-1} f(x) = \frac{k}{\lambda} g_x(\lambda)
	\end{equation*}
	Somit erhalten wir eine Differentialgleichung für $g_x(\lambda)$. Wir wollen zeigen, dass $g_x(\lambda) = 0$ für alle $\lambda > 0$. Falls $g_x(\lambda) \neq 0$ für ein $\lambda \in U$
	\begin{equation*}
		\frac{g_x'(\lambda)}{g_x(\lambda)} = \frac{k}{\lambda} \follows \brackets{\log g_x(\lambda)}' = \frac{k}{\lambda} \follows\log g_x(\lambda) = k * \log(\lambda) + c
	\end{equation*}
	Somit ist $g_x(\lambda) = c' * \lambda^k$ die allgemeine Lösung der Differentialgleichung. Wegen $g_x(1) = 0$ erhalten wir $c' = 0$ und folglich auch $g_x(\lambda) = 0$ für alle $\lambda > 0$. Daraus folgt nun \eqref{eq: h}.
\end{proof_equiv}

Anwendung auf das Allokationsproblem: Für $x \in [0,\infty)^n$ betrachten wir $L(x) = x_1 * L_1 + \dots + x_n * L_n$, d.h. 
\begin{equation*}
	L(\one) = L_1 + \dots + L_n
\end{equation*}
Definiere $r_\rho(x) = \rho(L(x))$. Es gilt unter Nutzung der positiven Homogenität von $\rho$
\begin{equation*}
	r_\rho(\lambda x) = \rho(L(\lambda x)) = \rho(\lambda * L(x)) = \lambda \rho(L(x)) = \lambda r_\rho(x)
\end{equation*}
Somit ist $r_\rho$ homogen vom Grad $1$ und der Satz von Euler liefert $r_\rho(x) = \sum_{i=1}^n \partdiff{x_i} r_\rho(x) * x_i$. Insbesondere gilt
\begin{equation*}
	\rho(L) = r_\rho(\one) = \sum_{i=1}^n \underbrace{\partdiff{x_i} r_\rho(\one)}_{\defqe AC_i}
\end{equation*}

\begin{*definition}
	Die Kapital-Allokation 
	\begin{equation*}
		AC_i = \partdiff{x_i} r_\rho(\one) \quad \mit \quad r_\rho(x) \defeq \rho(x_1 L_1 + \dots + x_n L_n)
	\end{equation*}
	heißt \begriff{Euler-Prinzip} der Allokation.
\end{*definition}

\begin{*beispiel}
	\begin{enumerate}[label=(\alph*), leftmargin=*]
		\item Betrachte die Standardabweichung $\rho(L) = \sqrt{\Var[L]}$ als positiv homogenes Risikomaß. Es gilt
		\begin{equation*}
			r_\rho(x) = \rho(x_1 L_1 + \dots + x_n L_n) = \sqrt{\trans{x} * \Sigma * x} \quad \mit \quad \Sigma_{ij} = \Cov{L_i}{L_j} 
		\end{equation*}
		Ableiten in Richtung $x_i$ liefert
		\begin{equation*}
			\partdiff{x_i} r_\rho(x) = \frac{1}{2} * \brackets{\trans{x} \Sigma  x}^{-\frac{1}{2}} * \brackets{\Sigma x}_i * 2 = \frac{\trans{e_i} \Sigma x}{\sqrt{\trans{x} \Sigma x}}
		\end{equation*}
		Somit gilt 
		\begin{equation*}
			AC_i = \partdiff{x_i} r_\rho(\one) = \frac{\trans{e_i} \Sigma x}{\sqrt{\trans{x} \Sigma x}} = \frac{\Cov{L_i}{L}}{\sqrt{\Var[L]}}
		\end{equation*}
		Man erhält hier das \enquote{Kovarianzprinzip} ähnlich dem \enquote{Beta} im CAPM.
		%
		\item Wir betrachten $r_\rho(L) = \VaR_\alpha(L)$ unter der Annahme $\EW[L] < \infty$ und $(L_1, \dots, L_n)$ haben gemeinsame Dichte. Dann gilt 
		\begin{equation*}
			AC_i = \EW[L_i \mid L = \VaR_\alpha(L)]
		\end{equation*}
		%
		\item Betrachte nun $\rho(L) = \ES_\alpha(L)$. Unter den gleichen Annahmen wie in (b) gilt
		\begin{equation*}
			AC_i = \EW[L_i \mid L \ge \VaR_\alpha(L)]
		\end{equation*}
	\end{enumerate}
\end{*beispiel}